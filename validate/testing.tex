%
%
%
\typeout{ABSOLUTELY UNFINISHED!!!!!}
%

\documentclass[final]{ltugboat}
\usepackage{shortvrb}
\MakeShortVerb{|}
\usepackage{calc}

% \verbfile{<name>} to input a possibly long file verbatim
%
\makeatletter
\def\verbfile#1{\begingroup
  \footnotesize
  \@verbatim
  \frenchspacing \@vobeyspaces
  \@input{#1}\endtrivlist\endgroup\@doendpe}
\makeatother

%\setcounter{secnumdepth}{-1}

\title{The \LaTeX{} regression test suite: concepts and
implementation}

\author{D. P. Carlisle \and  F. Mittelbach}

\begin{document}
\maketitle

\begin{abstract}
  This paper describes the concepts and the implementation of a
  regression test suite for \LaTeX{}. A general overview about the
  history behind this package is given in~\cite{tub:xxx}.

  As experience shows that there can't be enough test files in such a
  suite we make a plea to the \TeX{} community to help us making
  \LaTeX{} distributions even more reliable by joining a new volunteer
  group working on the task of updating and adding to this suite.
\end{abstract}

\section{Aims of this Package}
This set of files implements a scheme suggested by Frank Mittelbach
for ensuring that new releases of \LaTeX{} do not inadvertently modify
the behaviour of commands. This is definitely a non-trivial task, as
\LaTeX{} is a large system, and in `fixing' one bug, one often needs
to modify the definitions of several `internal' commands, which may
affect many other commands which have no obvious connection with the
original problem.

\section{The Basic System}
The idea is to have a suite of test files, each one exercising a
particular set of related commands. These should be called in such a
way that one can tell from the \texttt{.log} file (not the \texttt{.dvi}
file) that the command has `met its specification'. The \texttt{.log}
file is then edited to remove certain irrelevant information and will
then be stored, as a \texttt{.tlg} file. Before a new release is issued,
all the test files will be run through the new version, and the
resulting \texttt{.tlg} files will be automatically compared with the
saved original versions. Any tests which do not produce identical
results will then be notified to the maintainer of
\LaTeX, who can visually compare the \texttt{.tlg} files to see whether the
differences are due to an `improvement' or are the result of a newly
introduced bug!

To distinguish test files from other files they have to have the
extension \texttt{.lvt} (standing for \LaTeX{} validation test).


\section{The Format of a Test File}

If the functions to be tested need to be called inside the \LaTeX{}
\texttt{document} environment, then start the file as usual with:
\begin{verbatim}
\documentclass{article}
\begin{document}
\end{verbatim}
or any other class. In some cases you may also want to load some
packages if they are need for your test.

If you are testing some commands of the commands from \LaTeX`s kernel,
then these lines may be omitted.

Then start the test procedure with
\begin{verbatim}
% \iffalse meta-comment
%
% Copyright (C) 1992-1994 by David Carlisle, Frank Mittelbach.  
% Copyright (C) 2008 LaTeX3 project
% All rights reserved.
% 
% This file is part of the validate package.
% 
% IMPORTANT NOTICE:
% 
% You are not allowed to change this file.  In case of error
% write to the email address mentioned in the file readme.val.
% 
% \fi
%                  regression-test.tex
                   %%%%%%%%%%%%%%%%%%%

% David Carlisle
% Version 0.0,  28 May 1992
% Version 0.1,  18 Jun 1992 FMi small updates
% Version 1.0a, 28 Jun 1992 FMi small updates for distribution
% Version 1.0b, 1993/12/08  DPC update for LaTeX2e
% Version 1.0e, add config file.

% \def\fileversion{v1.0e}
% \def\filedate{1994/05/19}

% This file should not be used as a package or class file, 
% it should be \input.

% The scope of this \makeatletter will then be the rest of the
% document.  Put TeX into scroll mode, and stop it showing the
% implementation details of macros in error messages.
%
\makeatletter
\scrollmode
\errorcontextlines=-1

% Use the same \showbox settings as 2.09, unless they are changed in 
% the test file. (2e sets these to -1)
\showboxbreadth=5
\showboxdepth=3

% Start the test, after the optional \documentclass (or \documentstyle)
% \begin{document} commands with \START.  All lines in the .log file
% before this will be ignored. It also prints a docstrip-style
% character table in the .tlg file so the .tlg file can easily be
% checked for email translations.
%
\def\START{\typeout{START-TEST-LOG^^J^^J%
   This is a generated file for the LaTeX (2e + expl3) validation system.%
^^J^^JDon't change this file in any respect.%
^^J^^J\CTable^^J}}

\begingroup
\catcode`\^^\=0
\catcode`\^^A=\catcode`\%
^^\catcode`^^\ =11
^^\catcode`^^\%=11
^^\catcode`^^\#=11
^^\catcode`^^\~=11
^^\endlinechar=`^^\^^J
^^\catcode`^^\\=11^^A
^^\gdef^^\CTable{
%% \CharacterTable
%%  {Upper-case    \A\B\C\D\E\F\G\H\I\J\K\L\M\N\O\P\Q\R\S\T\U\V\W\X\Y\Z
%%   Lower-case    \a\b\c\d\e\f\g\h\i\j\k\l\m\n\o\p\q\r\s\t\u\v\w\x\y\z
%%   Digits        \0\1\2\3\4\5\6\7\8\9
%%   Exclamation   \!     Double quote  \"     Hash (number) \#
%%   Dollar        \$     Percent       \%     Ampersand     \&
%%   Acute accent  \'     Left paren    \(     Right paren   \)
%%   Asterisk      \*     Plus          \+     Comma         \,
%%   Minus         \-     Point         \.     Solidus       \/
%%   Colon         \:     Semicolon     \;     Less than     \<
%%   Equals        \=     Greater than  \>     Question mark \?
%%   Commercial at \@     Left bracket  \[     Backslash     \\
%%   Right bracket \]     Circumflex    \^     Underscore    \_
%%   Grave accent  \`     Left brace    \{     Vertical bar  \|
%%   Right brace   \}     Tilde         \~}
%%
}^^A
^^\endgroup{}%

% The test should end with
% \END or \end{document}
%
\let\@@@end\@@end
%\let\@ED=\enddocument
\def\END{\typeout{END-TEST-LOG}\@@@end}
\let\@@end\END


% After the \START should come declarations of the format and style
% options being used.
%
\def\FORMAT#1{\typeout{Format: #1}%
  \def\@tempa{#1}\ifx\@tempa\@EJ\else
   \OMIT\@warning{Declared format #1,^^JActual format \@EJ}\TIMO\fi}

% The old version got this information from everyjob, 
% but that does not work with LaTeX2e as \everyjob is cleared.
\edef\@EJ{\fmtname <\fmtversion>}

% Some author info:
\def\AUTHOR#1{\typeout{Author: #1}}
\def\ADDRESS#1{\typeout{Address: #1}}

% Not all packages declare themselves to the log file, and we can not
% rely on TeX`s output as it includes full path names, and does not
% include version numbers etc.  So for each package included give a
% declaration like: \PACKAGES{array v2.0d}
%
\def\STYLE#1{\typeout{Main Style: #1}}%
\def\STYLEOPTIONS#1{\typeout{Style Options: #1}}


% If The class or package is loaded with options, you may
% specify the options in the \CLASS (\PACKAGE) declaration. eg:
%
% \CLASS[german,a4page]{article v2.0 1994/01/02}
% \PACKAGE{ifthen v2.2 1993/11/12}
% \PACKAGE[dvips]{graphics v 3.8 1994/02/02}
%
\def\CLASS{\@ifnextchar[\OPTCLASS\XCLASS}
\def\OPTCLASS[#1]#2{%
  \typeout{Main Class: #2^^J\@spaces Options: #1}}
\def\XCLASS#1{%
  \typeout{Main Class: #1}}

\def\PACKAGE{\@ifnextchar[\OPTPACKAGE\XPACKAGE}
\def\OPTPACKAGE[#1]#2{%
  \typeout{Package: #2^^J\@spaces Options: #1}}
\def\XPACKAGE#1{%
  \typeout{Package: #1}}



% LaTeX2e always uses NFSS2 so new test files need not use 
% \FONTSELECTION but it is retained for compatibility for test files
% written for 209/NFSS1.
%
\def\FONTSELECTION#1{%
  \OMIT\@@warning{\noexpand\FONTSELECTION obsolete.^^J%
                 LaTeX2e always uses NFSS2}\TIMO
  \typeout{Font Selection: #1}}



% Surround commands which produce irrelevant lines in the .log file by
% \OMIT
% \TIMO
%
\def\OMIT{\typeout{OMIT}}
\def\TIMO{\typeout{TIMO}}

% After the above declarations, and before the main tests, you may
% optionally `declare' all the commands in the `module' that you are
% about to test. These commands will be registered as defined,
% undefined or relaxed (ie \let to \relax). You may wish to declare
% commands not currently implemented, so that if they are added at a
% later stage, the test will fail, reminding someone to document the
% fact that the user interface has changed. So if you are testing
% array and tabular environments, you may wish do declare
% \extrarowheight. This is undefined in the curent latex, but would
% become defined if Mittelbach's array.sty was incorporated into
% latex.tex.
%
\def\declare@command#1{%
  \ifx#1\@undefined\typeout{Undefined \string#1}\else
  \ifx#1\relax\typeout{Relaxed \space\space\string#1}\else
         \typeout{Defined \space\space\string#1}\fi\fi}


% To allow testing of possible changes, we allow extra code to be read
% in before the test starts. The necessary code should be placed in a
% file test2e.cfg.
%
\OMIT
\InputIfFileExists{regression-test.cfg}
      {\typeout{^^J***^^Jregression-test.cfg in operation^^J***^^J}}{}
\TIMO

%%%%%%%%% 

% We are not starved for space in the log file output, so let's make it as
% verbose as is useful when reading the .diff's.
\newcount \gTESTint
\def \SEPARATOR {%
  \typeout{%
    ============================================================}}
\newcommand \TEST [2] {%
  \advance \gTESTint 1
  \SEPARATOR
  \typeout{%
    TEST \the\gTESTint: \detokenize{#1}}%
  \SEPARATOR 
  \begingroup
    \let \TYPE \typeout
    #2%
  \endgroup
  \SEPARATOR \typeout{}%
}
\newcommand \TESTEXP [2] {%
  \advance \gTESTint 1
  \SEPARATOR
  \typeout{%
    TEST \the\gTESTint: \detokenize{#1}}%
  \SEPARATOR 
  \begingroup
    \let \TYPE \@firstofone
    \typeout{#2}%
  \endgroup
  \SEPARATOR \typeout{}%
}

\renewcommand \TESTMISSING [1] {%
  \begingroup
    \escapechar=-1
    \expandafter\ifx\csname\string#1\endcsname\relax
      \typeout{MISSING: \@backslashchar\string#1}%
    \fi
  \endgroup
}

\def \TRUE  {\TYPE{TRUE}}
\def \FALSE {\TYPE{FALSE}}
\def \YES   {\TYPE{YES}}
\def \NO    {\TYPE{NO}}

\def \NEWLINE {\TYPE{^^J}}

\endinput




\START
\end{verbatim}
Everything in the \texttt{.log} file before this will be omitted,
\verb|\START| also prints a character table in the \texttt{.log} file,
so that the \texttt{.tlg} file may be checked for errors caused be
faulty email connections.%
\footnote{In this day and age, this precaution is probably unnecessary in
its current implementation, but as we start to include tests for unicode
behaviour it will again be of importance.}
After \verb|\START|, the \texttt{@} sign is
set up to act as a letter, so you can use any commands that can
normally only be used in a package or class file.

If you like you can identify yourself as the original author of this
test file by issuing the following commands.
\begin{verbatim}
\AUTHOR{my name}
\ADDRESS{my email address}
\end{verbatim}
This will help us to get back to you later if necessary.

You should then declare the format that you are using, the class, and
any packages that you have input. This should include dates and/or
version numbers, so that if at some future time, the test fails, any
differences between the original versions and the new versions can be
checked, to find the cause of the failure.

This is important as
not all packages declare themselves to the log file, and we can not
rely on \TeX's output as it includes full path names, and does not
include version numbers etc.%
\footnote{Is this still true now that we have \texttt{\string\listfiles}?}
So for each package included give a
declaration like:
\begin{verbatim}
\FORMAT{LaTeX2e <1997/12/01> patch level 1}
\CLASS[a4paper]{article 1996/10/31 v1.3u}
\PACKAGE{ifthen 1996/08/02 v1.0m}
\PACKAGE[dvips]{graphics 1996/10/31 v1.0c}
\PACKAGES{array 1996/06/14 v2.3i}
\end{verbatim}

You may then optionally `declare' a collection of commands that make
up the `module' that you are testing.  \verb|\DECLARECOMMAND\foo| will
report whether \verb|\foo| is defined, undefined, or equivalent to
\verb|\relax|.  Note that you may also declare commands which are not
currently defined in \LaTeX, but which might be added. The system will
then pick up the addition of such a command, and remind the
maintainers to announce that there has been a change in the user
interface. Thus if you were testing the \texttt{array} environment,
you might want to declare \verb|\extrarowheight| (currently defined in
the \texttt{array} package) or \verb|\hhline| (currently defined in
the \texttt{hhline} package).  Note that you would not declare
\verb|\@tabularcr|, as this is an internal command, and the fact that
it is undefined in the \texttt{array.sty} version does not constitute
a change in the interface.

After that you may have any series of \LaTeX{} commands, these should
be called in such a way that any incorrect behaviour would show up in
the \texttt{.log} file. For example, an expandable command could be tested
by executing it inside a \verb|\typeout| command; an un-expandable command
could be tested storing its result in a macro and applying the \verb|\show|
command.

Two commands are provided to assist this process called |\TEST| and
|\TESTEXP|, although their use is, strictly speaking, optional. Both keep
track of a running test number, and print clear separators in the |.log|
between each test. They take two arguments: the first is printed verbatim
in the |.log| file; here you can describe the test and possibly duplicate
the expected outcome to make it easier to see where things are going wrong
when the test fails. The second argument executes (within a group) the test
itself. These commands will be described in more detail in
section~\ref{sec:testmacro}.

Any commands, such as those allocating registers, which produce lines
in the \texttt{.log} file, which you do not want to be considered as part
of the test, you should surround with the declarations \verb|\OMIT|
and \verb|\TIMO|.

Finally the file should finish with either \verb|\END|, or
\verb|\end{document}|. Only use the second form if you actually need to
close the \texttt{document} environment, say to get the \texttt{.aux} files
correct.

An example of a test file to demonstrate these commands is shown in
Figure~\ref{fig:minimaltest}, with the relevant parts of the |.log| file
due to the tests shown in Figure~\ref{fig:minimallog}.

\begin{figure}
\begin{verbatim}
\documentclass{minimal}
% \iffalse meta-comment
%
% Copyright (C) 1992-1994 by David Carlisle, Frank Mittelbach.  
% Copyright (C) 2008 LaTeX3 project
% All rights reserved.
% 
% This file is part of the validate package.
% 
% IMPORTANT NOTICE:
% 
% You are not allowed to change this file.  In case of error
% write to the email address mentioned in the file readme.val.
% 
% \fi
%                  regression-test.tex
                   %%%%%%%%%%%%%%%%%%%

% David Carlisle
% Version 0.0,  28 May 1992
% Version 0.1,  18 Jun 1992 FMi small updates
% Version 1.0a, 28 Jun 1992 FMi small updates for distribution
% Version 1.0b, 1993/12/08  DPC update for LaTeX2e
% Version 1.0e, add config file.

% \def\fileversion{v1.0e}
% \def\filedate{1994/05/19}

% This file should not be used as a package or class file, 
% it should be \input.

% The scope of this \makeatletter will then be the rest of the
% document.  Put TeX into scroll mode, and stop it showing the
% implementation details of macros in error messages.
%
\makeatletter
\scrollmode
\errorcontextlines=-1

% Use the same \showbox settings as 2.09, unless they are changed in 
% the test file. (2e sets these to -1)
\showboxbreadth=5
\showboxdepth=3

% Start the test, after the optional \documentclass (or \documentstyle)
% \begin{document} commands with \START.  All lines in the .log file
% before this will be ignored. It also prints a docstrip-style
% character table in the .tlg file so the .tlg file can easily be
% checked for email translations.
%
\def\START{\typeout{START-TEST-LOG^^J^^J%
   This is a generated file for the LaTeX (2e + expl3) validation system.%
^^J^^JDon't change this file in any respect.%
^^J^^J\CTable^^J}}

\begingroup
\catcode`\^^\=0
\catcode`\^^A=\catcode`\%
^^\catcode`^^\ =11
^^\catcode`^^\%=11
^^\catcode`^^\#=11
^^\catcode`^^\~=11
^^\endlinechar=`^^\^^J
^^\catcode`^^\\=11^^A
^^\gdef^^\CTable{
%% \CharacterTable
%%  {Upper-case    \A\B\C\D\E\F\G\H\I\J\K\L\M\N\O\P\Q\R\S\T\U\V\W\X\Y\Z
%%   Lower-case    \a\b\c\d\e\f\g\h\i\j\k\l\m\n\o\p\q\r\s\t\u\v\w\x\y\z
%%   Digits        \0\1\2\3\4\5\6\7\8\9
%%   Exclamation   \!     Double quote  \"     Hash (number) \#
%%   Dollar        \$     Percent       \%     Ampersand     \&
%%   Acute accent  \'     Left paren    \(     Right paren   \)
%%   Asterisk      \*     Plus          \+     Comma         \,
%%   Minus         \-     Point         \.     Solidus       \/
%%   Colon         \:     Semicolon     \;     Less than     \<
%%   Equals        \=     Greater than  \>     Question mark \?
%%   Commercial at \@     Left bracket  \[     Backslash     \\
%%   Right bracket \]     Circumflex    \^     Underscore    \_
%%   Grave accent  \`     Left brace    \{     Vertical bar  \|
%%   Right brace   \}     Tilde         \~}
%%
}^^A
^^\endgroup{}%

% The test should end with
% \END or \end{document}
%
\let\@@@end\@@end
%\let\@ED=\enddocument
\def\END{\typeout{END-TEST-LOG}\@@@end}
\let\@@end\END


% After the \START should come declarations of the format and style
% options being used.
%
\def\FORMAT#1{\typeout{Format: #1}%
  \def\@tempa{#1}\ifx\@tempa\@EJ\else
   \OMIT\@warning{Declared format #1,^^JActual format \@EJ}\TIMO\fi}

% The old version got this information from everyjob, 
% but that does not work with LaTeX2e as \everyjob is cleared.
\edef\@EJ{\fmtname <\fmtversion>}

% Some author info:
\def\AUTHOR#1{\typeout{Author: #1}}
\def\ADDRESS#1{\typeout{Address: #1}}

% Not all packages declare themselves to the log file, and we can not
% rely on TeX`s output as it includes full path names, and does not
% include version numbers etc.  So for each package included give a
% declaration like: \PACKAGES{array v2.0d}
%
\def\STYLE#1{\typeout{Main Style: #1}}%
\def\STYLEOPTIONS#1{\typeout{Style Options: #1}}


% If The class or package is loaded with options, you may
% specify the options in the \CLASS (\PACKAGE) declaration. eg:
%
% \CLASS[german,a4page]{article v2.0 1994/01/02}
% \PACKAGE{ifthen v2.2 1993/11/12}
% \PACKAGE[dvips]{graphics v 3.8 1994/02/02}
%
\def\CLASS{\@ifnextchar[\OPTCLASS\XCLASS}
\def\OPTCLASS[#1]#2{%
  \typeout{Main Class: #2^^J\@spaces Options: #1}}
\def\XCLASS#1{%
  \typeout{Main Class: #1}}

\def\PACKAGE{\@ifnextchar[\OPTPACKAGE\XPACKAGE}
\def\OPTPACKAGE[#1]#2{%
  \typeout{Package: #2^^J\@spaces Options: #1}}
\def\XPACKAGE#1{%
  \typeout{Package: #1}}



% LaTeX2e always uses NFSS2 so new test files need not use 
% \FONTSELECTION but it is retained for compatibility for test files
% written for 209/NFSS1.
%
\def\FONTSELECTION#1{%
  \OMIT\@@warning{\noexpand\FONTSELECTION obsolete.^^J%
                 LaTeX2e always uses NFSS2}\TIMO
  \typeout{Font Selection: #1}}



% Surround commands which produce irrelevant lines in the .log file by
% \OMIT
% \TIMO
%
\def\OMIT{\typeout{OMIT}}
\def\TIMO{\typeout{TIMO}}

% After the above declarations, and before the main tests, you may
% optionally `declare' all the commands in the `module' that you are
% about to test. These commands will be registered as defined,
% undefined or relaxed (ie \let to \relax). You may wish to declare
% commands not currently implemented, so that if they are added at a
% later stage, the test will fail, reminding someone to document the
% fact that the user interface has changed. So if you are testing
% array and tabular environments, you may wish do declare
% \extrarowheight. This is undefined in the curent latex, but would
% become defined if Mittelbach's array.sty was incorporated into
% latex.tex.
%
\def\declare@command#1{%
  \ifx#1\@undefined\typeout{Undefined \string#1}\else
  \ifx#1\relax\typeout{Relaxed \space\space\string#1}\else
         \typeout{Defined \space\space\string#1}\fi\fi}


% To allow testing of possible changes, we allow extra code to be read
% in before the test starts. The necessary code should be placed in a
% file test2e.cfg.
%
\OMIT
\InputIfFileExists{regression-test.cfg}
      {\typeout{^^J***^^Jregression-test.cfg in operation^^J***^^J}}{}
\TIMO

%%%%%%%%% 

% We are not starved for space in the log file output, so let's make it as
% verbose as is useful when reading the .diff's.
\newcount \gTESTint
\def \SEPARATOR {%
  \typeout{%
    ============================================================}}
\newcommand \TEST [2] {%
  \advance \gTESTint 1
  \SEPARATOR
  \typeout{%
    TEST \the\gTESTint: \detokenize{#1}}%
  \SEPARATOR 
  \begingroup
    \let \TYPE \typeout
    #2%
  \endgroup
  \SEPARATOR \typeout{}%
}
\newcommand \TESTEXP [2] {%
  \advance \gTESTint 1
  \SEPARATOR
  \typeout{%
    TEST \the\gTESTint: \detokenize{#1}}%
  \SEPARATOR 
  \begingroup
    \let \TYPE \@firstofone
    \typeout{#2}%
  \endgroup
  \SEPARATOR \typeout{}%
}

\renewcommand \TESTMISSING [1] {%
  \begingroup
    \escapechar=-1
    \expandafter\ifx\csname\string#1\endcsname\relax
      \typeout{MISSING: \@backslashchar\string#1}%
    \fi
  \endgroup
}

\def \TRUE  {\TYPE{TRUE}}
\def \FALSE {\TYPE{FALSE}}
\def \YES   {\TYPE{YES}}
\def \NO    {\TYPE{NO}}

\def \NEWLINE {\TYPE{^^J}}

\endinput




\begin{document}
\START
\AUTHOR{Frank Mittelbach}
\TEST{\g@addto@macro, expect "aaabbb"}{
  \def\ab{aaa}
  \g@addto@macro\ab{bbb}
  \show\ab
}
\TEXTEXP{\@car, expect "YES"}{
  \@car \YES \ERROR \@nil
}
\END
\end{verbatim}
\caption{An minimal example of a test suite document.}
\label{fig:minimaltest}
\end{figure}

\begin{figure}
\fontsize{7pt}{8pt}
\begin{verbatim}
============================================================
TEST 1: \g@addto@macro , expect "aaabbb"
============================================================
> \ab=macro:
->aaabbb.
<argument> ... \g@addto@macro \ab {bbb} \show \ab

l.10 }

============================================================

============================================================
TEST 2: \@car , expect "YES"
============================================================
YES
============================================================
\end{verbatim}
\caption{Truncated output from running the test file shown in
         Figure~\ref{fig:minimaltest}.}
\label{fig:minimallog}
\end{figure}

\section{Tips}
The test should be written to show that the command meets its
specification. Ideally it should not fail if the command is correctly
re-implemented in a different way. This means that the primitive tracing
facilities like \verb|\tracingmacros| and \verb|\tracingcommands| should
probably not be used, as these reveal the implementation details of the
command.

If the command involves visual formatting, one way to get information
into the \texttt{.log} file is to do all the commands inside
\verb|\setbox0=\vbox{...}|, and then \verb|\showbox0|. Before showing
the box you should set \verb|\showboxdepth| and \verb|\showboxbreadth|
to suitable values so that just the right amount of information
displayed. \LaTeX{} has a command called \verb=\showoutput= which can
be of some help here.

\verb|\tracingoutput| or \verb|\tracingpages| might also be used (with
care), say if you were testing section headings at the top and bottom
of pages.  Other useful commands are \verb|\typeout{\meaning...}|,
which is like
\verb|\show|, but puts less junk in the \texttt{.log} file, and
\verb|\typeout{\the...}|, which is like \verb|\showthe|.

Please remember that it isn't sufficient to produce a test file that
shows the desired behavior in the {\tt.dvi} file since this file will
not be inspected by the test system later on. Nevertheless, it is
often helpful to check the {\tt.dvi} when creating a test file but you
have to make sure that the relevant information is also displayed in
the {\tt.tlg} in the end.

\section{The \texttt{\string\TEST} and \texttt{\string\TESTEXP} macros}
\label{sec:testmacro}

These macros are recommended for structuring the test file into logical
pieces. As mentioned earlier, it is the second argument of these macros
that the test material is placed, the first is just to tag the test with
some meaningful text.

The difference between the two macros is that |\TEST| is for testing
unexpandable commands, whereas |\TESTEXP| is for expandable ones. The
consequence of this difference is that the \emph{entire} contents of
|\TESTEXP|s test is executed within a |\typeout| macro, automatically printing
its results to the |.log| file. This was demonstrated in the example shown
in Figure~\ref{fig:minimaltest}:
\begin{verbatim}
\TEXTEXP{\@car, expect "YES"}{
  \@car \YES \ERROR \@nil
}
\end{verbatim}
where |\YES| is provided by the |regression-test| package to be the string
``|YES|'' when within |\TESTEXP|. Within |\TEST|, by contrast, |\YES| will
print ``|YES|'' to the |.log| file instead. The commands |\NO|, |\TRUE|, and
|\FALSE| are also defined, which all behave in the same way.

Any non-expandable setup code that is required to execute a |\TESTEXP| can
always be placed outside of the command; alternatively, you could run the
test within a plain |\TEST| command and |\typeout| the results manually.

To emphasise this point, recall from the example that within any |\TEST|
macro, the result you're testing for \emph{must} be written to the file with
|\show| or |\typeout{\meaning...}| or similar. The |\YES| and |\NO| (etc.)
macros can be convenient for this purpose, also. The non-expandable example
shown earlier could also be written:
\begin{verbatim}
\TEST{\g@addto@macro, expect "aaabbb"}{
  \def\test{aaa}
  \g@addto@macro\test{bbb}
  \def\expect{aaabbb}
  \ifx\test\expect \YES \else \NO \fi
}
\end{verbatim}
In this case the `expected' value is coded into the test itself rather than
implicitly contained within the |.log| file.

\section{What format to use}

The format used to run the tests should not contain local specialties,
i.e., it should have been built from a standard distribution using the
no configuration files for fonts (i.e., no \texttt{fonttext.cfg} and
no \texttt{fontmath.cfg}) nor a configuration file for hyphenation
patterns (i.e., no \texttt{hyphen.cfg}). In particular, this means
that the format will have only standard American hyphenation patterns
included and that it will not support Babel extensions.%
\footnote{Er, completely wrong?!}

At some time in the future multi-lingual extensions should be added to
the scope of the test suite but we prefer to wait with that for the
release of a new multi-lingual interface to the kernel,
see~\cite{tub:MR97} for research results on that interface.


\section{Creating the .tlg File}


If you are on a UNIX machine, a supplied shell script will run {\tt
  latex} twice on your test file, then edit the \texttt{.log} to
produce a \texttt{.tlg} file automatically.  At present, on other
operating systems, you will need to do this by hand.\footnote{We are
  grateful if we can include scripts for other operating systems to do
  this task.} Run \LaTeX{} twice, afterwards delete all lines upto and
including \verb|START-TEST-LOG|, and all lines after, and including
\verb|END-TEST-LOG|.  Similarly remove lines \verb|OMIT| and
\verb|TIMO|, and everything in between.  In addition remove all lines
that refer to loading a font definition file (extension \texttt{.fd})
including the line containing the closing parentheses denoting that
the file was closed.  Finally replace the base part of the test file
name whenever it shows up with ``nothing'', e.g., turn
`\texttt{(tlen01.aux)}' into `{\tt(.aux)}'. In this way the files stay
valid if we have to rename them.

\section{Sending new files to the \LaTeX{} maintainers}

It is important to realize, that only completely documented and
checked files can be included into the automated checking system for
\LaTeX2.09. The documentation, i.e., information what is supposed to
show up in the {.tlg} file, etc., is very important, since if later
such a file shows a change we need to be able to determine the reason
for it. It is also important that we don't have to re-check the test
file, in other words we like to get both the test file and the
hand-check result file, i.e., the {\tt.tlg} file from you. Please send
them to the address mentioned in \texttt{readme.val} file.


\section{If You Find a Bug}
The basic idea of this test system is to ensure that future releases
do not introduce new bugs. That is, it will normally be assumed that
if the new release produces different results on a test to an old
release, then the new release has a bug. Because of this, if you
discover while producing the file, that \LaTeX{} acts incorrectly in
some context, still submit the file, but note clearly in the file, and
in the email message, that you consider the current behaviour
incorrect. Otherwise you may ensure that this behaviour is maintained
in all future releases!


\section{Call for volunteers}

The more test files there are within this test suite exists the more
reliable the \LaTeX{} system will become. We therefore call on you for
help!  What we hope to find is a new group of volunteers that is
interested in working on an extension of the \LaTeX{} test suite
system. There is no need to be an expert \TeX{} or \LaTeX{} programmer
for this task though some experience with \LaTeX{} and its inner
workings will be necessary.

\begin{center}
\fbox{\begin{minipage}{\linewidth-2em}
  \bfseries If you are interested in joining this effort, please
  contact Daniel Flipo at
\begin{center}
  \texttt{Daniel.Flipo@univ-lille1.fr}
\end{center}
  who kindly agreed to act as a coordinator between the individual
  volunteers.
\end{minipage}}
\end{center}
Further details about this effort can be found in~\cite{tub:xxx}.

\section{Sample files}

As a more complex example we show a test file for the ``length'' module of
\LaTeX2.09.
Please note that it is important to document the desired output and
also that only relevant information shows up in the {\tt.tlg} file.
For this reason the allocation information for \verb=\newlength= is
suppressed since it may change without making the interface invalid.

\subsection{The file \texttt{tlen01.lvt}}
\verbfile{tlen01.lvt}

\onecolumn

\subsection{The file \texttt{tlen01.tlg}}
This is the resulting output after applying the UNIX script {\tt
maketlg} to the file \texttt{tlen01.lvt}.
\verbfile{tlen01.tlg}



\begin{thebibliography}{1}
\bibitem{A-W:LLa94}
Leslie Lamport.
\newblock {\em {\LaTeX:} A Document Preparation System}.
\newblock Addison-Wesley, Reading, Massachusetts, second edition, 1994.

\bibitem{tub:xxx}
Frank Mittelbach.
\newblock \emph{A regression test suite for \LaTeXe}.
\newblock \TUB{} (?)\ldots

\bibitem{tub:MR97}
Frank Mittelbach and Chris Rowley.
\newblock \emph{Language Information in   Structured Documents:
  A Model for Mark-up and Rendering}.
\newblock \TUB{} (?)\ldots

\end{thebibliography}



\end{document}



% \iffalse
%% File: template.dtx (C) Copyright 1999-2001 David Carlisle, Frank Mittelbach
%%                    (C) Copyright 2004-2008 Frank Mittelbach, LaTeX3 Project
%%
%% It may be distributed and/or modified under the conditions of the
%% LaTeX Project Public License (LPPL), either version 1.3c of this
%% license or (at your option) any later version.  The latest version
%% of this license is in the file
%%
%%    http://www.latex-project.org/lppl.txt
%%
%% This file is part of the ``xbase bundle'' (The Work in LPPL)
%% and all files in that bundle must be distributed together.
%%
%% The released version of this bundle is available from CTAN.
%%
%% -----------------------------------------------------------------------
%%
%% The development version of the bundle can be found at
%%
%%    http://www.latex-project.org/cgi-bin/cvsweb.cgi/
%%
%% for those people who are interested.
%%
%%%%%%%%%%%
%% NOTE: %%
%%%%%%%%%%%
%%
%%   Snapshots taken from the repository represent work in progress and may
%%   not work or may contain conflicting material!  We therefore ask
%%   people _not_ to put them into distributions, archives, etc. without
%%   prior consultation with the LaTeX Project Team.
%%
%% -----------------------------------------------------------------------
%%
%<*driver|package>
\RequirePackage{l3names}
%</driver|package>
%\fi
\GetIdInfo$Id$
          {template}
%\iffalse
%<*driver>
%\fi
\ProvidesFile{\filename.\filenameext}
  [\filedate\space v\fileversion\space\filedescription]
%\iffalse
\documentclass{ltxdoc}
\makeatletter
 \newenvironment{decl}[1][]%
    {\par\small\addvspace{4.5ex plus 1ex}%
     \vskip -\parskip
     \ifx\relax#1\relax
        \def\@decl@date{}%
     \else
        \def\@decl@date{\NEWfeature{#1}}%
     \fi
     \noindent\hspace{-\leftmargini}%
     \begin{tabular}{|l|}\hline\ignorespaces}%
    {\\\hline\end{tabular}\nobreak\@decl@date\par\nobreak
     \vspace{2.3ex}\vskip -\parskip\@afterheading}
\makeatother

 \begin{document}
 \catcode`\_=11
 \catcode`\:=11
 \DocInput{template.dtx}
 \end{document}
%</driver>
%
%
% \fi
%
%
% \CheckSum{0}
%
% \title{The \textsf{template} package\thanks{This file
%         has version number \fileversion, last
%         revised \filedate.}}
% \author{DPC, FMi}
% \date{\filedate}
%  \maketitle
%
% \changes{0.06}{1998/12/17}{generic functions reborn as templates}
%
% \tableofcontents
%
% \section{Introduction}
% A \emph{Template} is a named `function' that has a fixed number
% of \emph{mandatory arguments} and an additional set of \emph{keys}
% or `named attributes' that are set in a `key value list'. That is,
% a comma separated  set of assignments of the form:\\
% \meta{key$_1$} |=| \meta{value$_1$} |,|
% \meta{key$_2$} |=| \meta{value$_2$} \ldots
%
% More specific instances of the template may be declared by
% specifying settings of the parameters. The key value list is parsed
% at the time the instance is declared, and an `internal' set of
% parameter assignments is passed to the template code. It is normally
% not parsed at run-time, though it is possible to enforce this
% behaviour.
%
% Templates have a \emph{type} and a \emph{name}. Templates
% of the same type have the same argument and parameter
% structure. That is, templates of the same type are expected to be
% exchangeable semantically. (However, except for checking that
% templates of the same type always have the same number of arguments
% this is not enforced by the code.)
%
% A template type is declared via the |\DeclareTemplateType|
% declaration which takes two arguments: the name for the type and the
% number of arguments a template of this type requires.
%
% Templates are declared via the |\DeclareTemplate|
% command which takes five arguments
% \begin{itemize}
% \item The \emph{type} of the template (no |\|).
% \item The \emph{name} of the template (no |\|).
% \item The number of arguments for the template (same as on type
% declaration).
% \item A list declaring the keys accepted by the template,
% with information about the action to take when the key is specified
% with a value.
% \item The code for the template. This may be arbitrary \TeX\
% code. At some point it should execute |\DoParameterAssignments|
% to run the parameter assigments.
%
% The mandatory arguments are accessed via |#1|, |#2|, \ldots
% \end{itemize}
%
%
% Each element of the key specification list is of the form:
% \begin{flushleft}
% \meta{key name} |=|
%       \meta{key type} \meta{optional default} \meta{internal code}
% \end{flushleft}
%
% The \meta{key types} are essentially specified by giving a symbolic
% representation of the assignment function to be used by \TeX.
%
% Currently the possibilities are
%
% \begin{center}\catcode`|=12
% \begin{tabular}{l|ll|l}
% key type& letter& internal code& argument form\\\hline
% Command & f\emph{n}&command name & command definition\\
% name    & n &       command name & command definition\\
% length  & l &       length register& calc length syntax\\
% fake length & L &  command name  & calc length syntax\\
% count   & c &       count register& calc count syntax\\
% fake count  & C &  command name  & calc count syntax\\
% boolean & b &       name of \verb|\newif| switch& true or false\\
% switch  & s &       \marg{true code}\marg{false code}& true or false\\
% instance& i\marg{type}& command name& instance of this \emph{type}\\
% direct  & x &       Internal code& any\\
% general & g &       general code & any
% \end{tabular}
% \end{center}
%
% In addition, any of these types may be prefixed by |+| to
% denote a global assignment (described below).
% |f| takes a digit from 0--9 to denote the number of arguments.
% |n| is in fact the same as |f0|. When an instance is declared
% The value assigned to the key should be the definition of
% the command, using |#1|\ldots|#9| to denote the specified arguments.
%
% |c| takes an internal form a count register,
%  not a \LaTeX\ counter name.
%
% For |f|, |n|, |l|, and |c|, the assignment is done twice, once
% at the time an instance is \emph{declared}. (This may involve
% using \textsf{calc} expresions. Then the `primitive assignment'
% of the value (not using calc) is copied to the internal parameter
% list, to be executed when an instance is run. Sometimes you need the
% expression to be evaluated at the time an instance is run rather
% than the time it is declared. For example it may be an expression
% involving some values that are not fixed throughout a document.
% In this case the instance declaration may give a value in the
% form |\DelayedEvaluation|\marg{calc expression}. In this case
% the value is not evaluated when the instance is declared, and
% instead the entire expression is copied to the `internal parameter
% list' and is evaluated whenever the instance is used.
%
% |L| and |C| take the same value types as |l| and |c| but the internal
% assignments are to macros not registers.
%
% Keys declared with |b| and |s| each take values either \emph{true} or
% \emph{false}. if the key zzz is declared with |b| then specifying
% |zzz=true| will essentially pass |\zzztrue| to the internal parameter
% list (although in fact |\zzztrue| need not be defined) . If instead
% zzz had been declared via |s|,  then |zzz=true| would pass the tokens
% of the \marg{true code} to the internal parameter list.
%
% If a key is specified as |x|, then when used the \emph{internal code}
% will be copied to the internal parameter lists. This code may use |#1|
% to denote the value supplied to the key in the instance declaration.
% Note that this code is \emph{only} copied at the time the instance is
% declared. It is not executed at this time. It is executed when the
% instance is executed.\footnote{Despite the question of whether or
% not x and g are still necessary these days, they have the wrong
% `names' since x is the one that is not executed during delcaration
% while g is.}
%
% If a key is declared with |g| then the code is run at the time the
% instance is declared. By default \emph{nothing} is passed to the
% internal parameter list. This code may use |#1| to denote the
% value that will be supplied when an instance is declared.
% Any code that should be run when an instance is executed should
% be explicitly passed to the internal parameter list using
% |\toks_add_right:Nn\l_TP_KV_assignments_toks{|\ldots|}|
%
% A key declared with |i|\marg{type} takes as value the name of a
% declared instance of that type. The command token associated with
% the key will store a command essentially equivalent to a call
% to |\UseInstance|\marg{type}\marg{name}, but in a slightly optimised
% internal form.
% As an exception to this rule the replacement code may be of the form
% |\UseTemplate| followed by the key settings for the template
% but without the mandatory arguments. In this case the `inner'
% instance declaration is `pre compiled' and the token assigned to
% the store the value assigned to this key will execute an instance
% of the template directly, it will not re-parse the keyword settings
% each time the instance is used.
%
% \section{Commands}
%
%
% \subsection{Template declaration commands}
%
% \begin{decl}
%  |\DeclareTemplateType| \marg{type}\marg{num}
% \end{decl}
% Declare a template type.
%
% \begin{decl}
%  |\DeclareTemplate|
%   \marg{type}\marg{tname}\marg{num}\marg{keyspec}\marg{code}
% \end{decl}
% Declare a template \meta{tname} of type \meta{type} with the set of
% keys as defined by \meta{keyspec}. From this template instances can
% be declared using |\DeclareInstance| At runtime such instances will run
% \meta{code} and expect \meta{num} mandatory arguments (same number
% for all templates of one type).\footnote{The \meta{num} argument is
% redundent as it can be deduced from the type. However, for practical
% reasons it seems better to keep that information with each
% individual template declaration.}
%
% \begin{decl}
% |\DeclareRestrictedTemplate|
% \marg{type}\marg{new-tname}\marg{old-tname}\marg{keyvals}
% \end{decl}
% Declare as new template \meta{new-tname} for type \meta{type} by
% taking template \meta{old-tname} as the basis and setting one or
% more of its keys to specific values.
%
% \begin{decl}
% |\DoParameterAssignments|
% \end{decl}
% The list of key value assignments made (and saved) during template
% declaration is evaluated at this point in the template code.
%
%
% \subsection{Instance declaration commands}
%
% \begin{decl}
% |\DeclareInstance| \marg{type}\marg{iname}\marg{tname}\marg{keyvals}
% \end{decl}
% Declare an instance of type \meta{type} named \meta{iname} build
% from using template \meta{tname} with key settings as given by
% \meta{keyvals}.
%
% \begin{decl}
% |\DeclareCollectionInstance|
% \marg{collection}\marg{type}\marg{iname}\marg{tname}\marg{keyvals}
% \end{decl}
% Same as |\DeclareInstance| except that this instance is only active
% when for the type \meta{type} the collection \meta{collection} was
% selected via |\UseCollection|. E.g., within the frontmatter one
% could make all headings behave differently by defining collection
% instances for template type `head'.
%
% \subsection{Key value commands}
%
% \begin{decl}
% |\DelayEvaluation|\marg{code}
% \end{decl}
% Used in the value spec for an instance to declare that the value
% \meta{code} should not be evaluated at declaration time but at
% run-time. Can also be used in the defaults for keys (given in square
% brackets) in the declaration of templates.
%
% \begin{decl}
% |\MultiSelection| \meta{counter} \marg{cases} \marg{else}
% \end{decl}
% Used in the value spec for an instance key to declare that the value of
% this key depends on the current setting of \meta{counter} at run-time.
% The \meta{cases} argument is a comma-separated list of ``values'',
% the \meta{else} argument a single ``value''. If at run-time
% \meta{counter} has the value $i$ then the $i$-th element of the
% \meta{cases} list is selected. If that does not exist the
% \meta{else} case is returned.
%
%
% \subsection{Processing commands}
%
% \begin{decl}
% |\UseTemplate| \marg{type}\marg{tname}\marg{keyval}
% \end{decl}
% Execute a template \meta{tname} of type  \meta{type} at run-time
% using \meta{keyvals} as the value assignments for its keys. In this
% case the keys are evaluated at run-time thus this method is far
% slower than using a predeclared instance of this template (see
% below). This command can also appear as the value for a key of type
% `i' in which case the evaluation happens at declaration time of the
% template that contains this key!
%
%
% \begin{decl}
% |\UseInstance| \marg{type}\marg{iname}
% \end{decl}
% Run the instance \meta{iname} of template type \meta{type}. If a
% collection is in force see if there is a collection instance of name
% \meta{iname} and if so run that instead.
%
% \begin{decl}
% |\UseCollection| \marg{type}\marg{collection}
% \end{decl}
% Declare that from now on (normal scoping rules) the collection
% \meta{collection} for template type \meta{type} is in force. This
% means that a call to |\UseInstance| will first check if there is a
% collection instance defined, and if so use that instance, otherwise
% use the normal instance.
%
%
% \subsection{Test commands}
%
% \begin{decl}
% |\IfExistsInstanceTF| \marg{type}\marg{iname}\marg{true}\marg{false}
% \end{decl}
% Test if for template type \meta{type} an instance with name
% \meta{iname} exists. Select \meta{true} or \meta{false} code
% accordingly.
%
%
%
% \section{Examples of template key types}
%
%  The general syntax for key specification in templates (fourth argument
%  of the command |\DeclareTemplate|) is:
%\begin{flushleft}
%| {| \\
%|  | \meta{key-name$_1$} |=|\meta{key-type$_1$}
%                            \meta{optional-default$_1$} \meta{storage-bin$_1$}|,|\\
%|  | \meta{key-name$_2$} |=|\meta{key-type$_2$}
%                            \meta{optional-default$_2$} \meta{storage-bin$_2$}|,|\\
%|     ...|\\
%| }|
%\end{flushleft}
% In this section we look at all possible key types and give examples
% for them.
%
%
% \subsection{Attributes that receive names as values}
%
%  The type |n| expects to receive a \LaTeX{} name as a value. Used,
%  for example, to specify the name of a \LaTeX{} counter to use.
%\begin{verbatim}
%  heading-id     =n                                  \heading@id,
%  counter-id     =n  [\DelayEvaluation{\heading@id}] \heading@counter,
%\end{verbatim}
% Notice the use of |\DelayEvaluation| in the default of
% |counter-id|. It is necessary to make the default the token
% |\heading@id| if we want to inherit the value from the |heading-id|
% key. Otherwise it gets value of |\heading@id| at the time the
% instance is declared.
%
%
% \subsection{Attributes that receive functions as values}
%
%
%  The type |f|\meta{num} expects a function with \meta{num} arguments
%  as a value. The arguments are denoted by |#1|, |#2|, etc. In most
%  cases either |f0| (for declarations) or |f1| (to format one argument) are
%  needed.
%\begin{verbatim}
%  initial-font   =f0            \initial@font,
%  initial-format =f1 [#1]       \initial@boxhandling,
%\end{verbatim}
%
% \subsection{Attributes that receive dimensions as values}
%
%  As far as specifying instances the |l| and |L| type behave
%  identically. They differ only in the type of internal storage-bin
%  they need: |l| expects a length register while |L| expects an
%  ordinary macro name and assigns its value via |\def:Npn |.
%\begin{verbatim}
%  pre-sep        =l   \topsep,
%  post-sep       =L   \botsep,
%\end{verbatim}
%
%
%
% \subsection{Attributes that receive integers as values}
%
%  The |c| and |C| type receive integers as values. Again either of
%  them can be transparently used. In case of |c| the
%  \meta{storage-bin} has to be a \TeX{} count register not a \LaTeX{}
%  counter name, i.e., set up via |\newcount|. (\LaTeX{} counters can
%  be used as well if they are accessed via their internal name, i.e.,
%  via |\c@|\meta{\LaTeX-counter})
%\begin{verbatim}
%  pre-penalty    =c   \@beginparpenalty,
%  penalty        =C   \hmaterial@penalty,
%\end{verbatim}
%
%
%
% \subsection{Attributes that receive template instances as values}
%
%  The type |i|\marg{type} takes as value the name of a declared
%  instance of that type. The \meta{storage-bin} associated with the
%  key will store a command essentially equivalent to a call to
%  |\UseInstance|\marg{type}\marg{name}, but in a slightly optimised
%  internal form.
%
%  As an exception to this rule the replacement code may be of the form
%  |\UseTemplate| followed by the key settings for the template but
%  without the mandatory arguments. In this case the `inner' instance
%  declaration is `pre compiled' and the token assigned to the store
%  the value assigned to this key will execute an instance of the
%  template directly, it will not re-parse the keyword settings each
%  time the instance is used.
%\begin{verbatim}
%  justification-setup =i{justification} \list@justification,
%\end{verbatim}
%
%  Usage within an instance declaration is either
%\begin{verbatim}
%  justification-setup = raggedright,
%\end{verbatim}
%  i.e., name of a declared instance or a call to |\UseTemplate|
%\begin{verbatim}
%  justification-setup = \UseTemplate{justification}{TeX}
%                            { startskip = 0pt, ... },
%\end{verbatim}
%
%
%
% \subsection{Attributes that receive true or false values}
%
%  The type |s| expects the strings |true| or |false| as values. In
%  this case the declaration has no \meta{storage-bin}. Instead the
%  declaration consists of two brace groups containing code. Depending
%  on the value one of the groups gets copied verbatim into the
%  internal parameter list of the instance and gets executed at
%  run-time at the point where |\DoParameterAssignments| is seen.
%\begin{verbatim}
%  item-implicit-boolean  =s
%      { \def:Npn \item@implicit@code{\item\relax} }{},
%  numbered-boolean       =b [true] @heading@nums,
%\end{verbatim}
%
%
% \subsection{Attributes that accept any value}
%
%  The type |g| is a low-level specification which contains arbitrary
%  code in place of the \meta{storage-bin}. This code is evaluated at
%  declaration time of the instance and by default \emph{nothing} is
%  passed to the internal parameter list (this has to happen explicitly
%  from within the code). |#1| may be used to access the value
%  specified.
%
%  The main purpose for this type is of historical nature (originally
%  most of the other types have been implemented internally using |g|).
%
%  The type |x| also requires code in place of the \meta{storage-bin}.
%  However with this type all of the code is copied unevaluated to the
%  internal parameter list. There are some applications for this type
%  when implementing customisable defaults. However, it is likely that
%  it will not survive a final release.
%\begin{verbatim}
%  generic-key   =g \typeout{#1},
%  extra-assigns =x \typeout{#1},
%\end{verbatim}
%
%
%
%
%
%
% \section{A complete example}
%
% The following example shows a sketch of a template for typesetting
% captions to be used as part of a larger mechanism setting whole
% floats.\footnote{I made it up while I went along so if you spot the
% ``missing brace'' or some other blunder tell me, FMi.}
%
%  We declare a template type \texttt{caption} then define an example
% template for that type and finally produce some instances from that.
%
% \subsection{Declaring the template type}
%
% To define the template type we first have to ask ourselves what
% information would be varying each time such a template is used?  A
% potential answer could be the following:
% \begin{itemize}
% \item
%   The float name, e.g., `Table' or `Fig.' etc.
% \item
%   The float
%   number e.g., `10' or `3--c' etc.
% \item
%   The actual caption text as specified in the document.
% \end{itemize}
% Since the above items would be differed in each instantiation of
% such a template we would pass them as mandatory argument to the
% template.
%
% Are there others? Possibly. Here are two more that seem to be
% useful, at least in a number of cases:
% \begin{itemize}
% \item
%   The text of the legend in document classes that distinguish
%   between caption text (heading to the figure/table) and legend
%   (explanatory material)
% \item
%   Measure to which the caption should be typeset.
% \end{itemize}
% The last one of these might need some extra explanation. Suppose a
% design requires that the caption width is decided depending on the
% width of the table of figure, e.g., the caption is supposed to
% typeset below some illustration and should not be wider than that
% illustration, or the caption is typeset aside to the illustration
% using the remaining space. In that case the process that formats the
% whole float needs to communicate with the current template to pass
% that (varying) information along. Of course, that could happen by
% using global variables, e.g., the outer process sets the measure as
% desired before calling the caption formatting template. What makes
% more sense is likely to be a matter of taste but it also has to do
% with the precise semantics of the template type. Staying with our
% example: if the the semantics of the template type \texttt{caption}
% is supposed to produce a formatted box (in \TeX{} terms) then we
% should pass the measure as an argument if we ever intend to allow
% for variations. If on the other hand the semantics are to format a
% certain set of text into the current galley (which has measure of
% its own) then a measure argument would not belong to this template
% type.
%
% Are there other variations sensible? Yes, for example, instead of
% passing a fixed string like ``Fig.'' as the first argument one could
% pass an abstract float type identifier and let the template worry to
% deduce from that information what fixed string to produce.
%
% Another question: why should we pass the fixed text (or an abstract
% identifier from which it can be deduced) and the number as separate
% arguments to the template instead of passing a combined string (like
% it is done in the |\@makecaption| command of \LaTeXe{})? Answer:
% because this allows to build templates that can individually
% manipulate both bits of information, e.g., to format the number in a
% different font, etc.
%
% So what are the conclusions of this discussion?  Defining the
% semantics of a template type is difficult and often needs several
% trials to come up with something that is covering the anticipated
% use. There is clearly not a cardinal way for defining template
% types; how the overall separation into smaller units is done is
% partly a matter of taste and partly a matter of the major layout
% characteristics that one tries to support.
%
% Returning to our example: let's assume we settle for the first four
% arguments, i.e., the calling template is responsible for setting the
% measure for the caption text if necessary.
%
% What we also have to do is to define (at least for ourselves) what
% data the arguments accept and what their semantics are. An informal
% summary of that could be the following:
% \begin{center}
% \begin{tabular}{rrl}
%  \textit{Arg} & \textit{Data Type}     & \textit{Description} \\[3pt]
%  1 & text                 & fixed float description
%  \\
%  2 & text/|\NoValue|      & float number
%  \\
%  3 & text                 & caption text
%  \\
%  4 & text/|\NoValue|      & legend text
%  \\
% \end{tabular}
% \end{center}
% The second and the fourth argument are allowed to be missing (i.e.,
% can get |\NoValue| passed as a value). Note that the empty string in
% case of a text argument is different from |\NoValue|.
%
% We further declare that it is permissible for a template of this
% type to ignore the information provided by all arguments except 3,
% i.e., the caption text.
%
% Finally the result of the template formatting should is to typeset
% text into a current galley (paragraph mode in \LaTeX{} lingua).
%
% All the above is semantic information that (at least right now) is
% not being enforced by declaring a template type (except for the
% number of arguments) but each template of a certain type is supposed
% to conform to this specification nonetheless.\footnote{To make this
% even clearer we are thinking of extending the template type
% declaration with another argument in which one has to formally or
% informally (?) specifies information like the one in the table
% above.}
%
% This finally leads to the following declaration:
%\begin{verbatim}
%\DeclareTemplateType{caption}{4}
%\end{verbatim}
%
% \subsection{Defining a first template}
%
% We start by defining a simple template of type \texttt{caption}
% which roughly formats a caption like those being presented in
% \LaTeXe{}'s article class, i.e., the caption is typeset as a
% paragraph if it is longer than a single line, otherwise it is
% centered. The legend even if present is ignored. Above and below we
% give the designer the possibility to add some space.
%
% In fact the examples is more or less identical in code to
% |\@makecaption| except that if the second argument (i.e., the
% number) is |\NoValue| it and its preceding space\footnote{For those
% who wonder: spaces are by default ignored within definitions when
% the new packages are used due to a command \texttt{\textbackslash
% InternalSyntaxOn}, do get a normal space one has to use
% \texttt{\textasciitilde} and to obtain an unbreakable space
% \texttt{\textbackslash nobreakspace}.} gets ignored.
%
% We start by declaring the template \texttt{toosimple} of type
% \texttt{caption} having four mandatory arguments (as described in
% the discussion of the template type).
%\begin{verbatim}
%\DeclareTemplate{caption}{toosimple}{4}
%\end{verbatim}
% The next argument of |\DeclareTemplate| lists all keys for the
% template. In this case we have keys for the vertical spaces above
% and below. We make them type |L| to save on registers but with a bit
% of care we could also have used scratch registers like |\@tempskipa|
% etc. Their default values are both zero.
%\begin{verbatim}
%  {
%    above-skip =L [0pt] \caption@above@skip ,
%    below-skip =L [0pt] \caption@below@skip ,
%  }
%\end{verbatim}
% The final argument of |\DeclareTemplate| contains the actual
% processing code. We start with looking at the second mandatory
% argument (caption number) to find out if it is |\NoValue| and
% depending on the result define a helper command |\caption@start|.
%\begin{verbatim}
%  {
%    \IfNoValueTF{#2}
%       { \def:Npn \caption@start{#1:~} }
%       { \def:Npn \caption@start{#1~#2:~} }
%\end{verbatim}
% Having dealt with the prelims we now run |\DoParameterAssigments| at
% which point the keys of the template are made available, e.g., at
% this point all those right hand containers such as
% |\caption@above@skip| get assigned the value specified in an
% instantiation of the template. (That scheme allows to do preliminary
% processing up front, e.g., defaults for the keys could be assigned
% prior to that point in which case they are overwritten if the
% template instance specifies a different value. the use of specifying
% defaults via the |[..]| syntax as done above is slightly faster at
% run-time but needs more memory.)
%\begin{verbatim}
%    \DoParameterAssigments
%\end{verbatim}
% The rest of the code should look familiar to anybody who ever looked
% at \texttt{article.cls}. The only point worth mentioning are the
% |\relax| commands after |\caption@above@skip| and
% |\caption@below@skip|. Since we have decided to use |L| as key type
% these commands are macros and not registers containing the
% dimensions as strings. This means that we have to be careful to
% ensure that \TeX{} knows where the dimension ends. In certain cases
% text following such a command might be mistaken as being part of the
% dimension (e.g., if followed by the word \texttt{plus}, etc.). In
% the code below this could only happen for the second |\vskip| but it
% is good practice to always add a terminating |\relax| to avoid such
% hidden traps.
%\begin{verbatim}
%    \vskip \caption@above@skip \relax
%    \sbox \@tempboxa {\caption@start #3}
%    \ifdim \wd\@tempboxa >\hsize
%      \caption@start #3\par
%    \else
%      \global \@minipagefalse
%      \hb@xt@\hsize{\hfil\box\@tempboxa\hfil}
%    \fi
%    \vskip \caption@below@skip \relax
%  }
%\end{verbatim}
%
% Why is the above template of not much use? Simply because it doesn't
% offer any flexibility to declare different designs. The only
% alteration offered to the designer is to modify the space above and
% below the caption, e.g., the following declaration would mimic the
% definition within the \texttt{article.cls}  class of \LaTeXe :
%\begin{verbatim}
%\DeclareInstance{caption}{article}{toosimple}
%  {
%    above-skip =  0pt,
%    below-skip = 10pt,
%  }
%\end{verbatim}
% And that's all that can be manipulated. All items that people asking
% to change, e.g., not having a colon after the number, using
% different fonts and font sizes, etc.\ are still hard-wired and thus
% inaccessible. So we have to do better if we want to make use of the
% power the template mechanism offers.
%
%
%
% \subsection{Defining a better template}
%
% First step in defining better templates is to ask ourselves a couple
% of questions:
% \begin{itemize}
% \item What are the main characteristics of the layout the template
%   is supposed to support?
% \item What are the elements that we want to allow (or can allow) the
%   designer to modify?
% \end{itemize}
%
% Take the first question first: the layout supported by the template
% of the previous section had as its main characteristics that it
% would center the caption if it would fit in a single line in the
% current measure. We could consider this being an unchangable
% characteristic of the layout this template produces (and a designer
% would need to use a different template of type \texttt{caption} if a
% design compatible with this restriction is desired) or we could try
% to make our template smarter by adding bells and wistles that allow
% the designer to say stuff like:
%\begin{verbatim}
%  one-line-format = \hfil #1 \hfil,
%\end{verbatim}
% or
%\begin{verbatim}
%  one-line-action = center,
%\end{verbatim}
% depending on how we intend to offer changing the behavior of the
% template. Like when trying to define sensible template types we have
% no single road to heaven (and probably as many to hell) --- it has a
% lot to do with how we think about design.
%
% My advice, after having tried to work with these concepts for a
% while, is to keep templates simple in so far as that most of not all
% attribute for a template should be relevant for the design. In other
% words, if you have attributes that, depending on their setting, make
% half of the other attributes not applicable then it may be
% appropriate to think about providing several templates instead. To
% give an example from \LaTeXe : instead of having |\@startsection|
% deal with both vertical heads and run-in heads provide individual
% templates. (|\@startsection| is this famous command where design
% switches are build in by making dimensions negative to signal
% something and afterwards use the absolute value.)
% Another way to look at this is to say that a template should
% normally not contain large amounts of code which is only selected in
% a subset of attribute settings.
%
% As said before there are no golden rules, it is perfectly possible
% to make hugely complicated templates that solve every possible
% aspect of layout one could think of in one go --- it is just that
% with keeping it more simple one can get the same functionality with
% less headaches for the template writer as well as the template user
% later on.
%
% Returning to our example: allowing to handle the case of a single
% line caption specially could well be considered part of the
% template. In contrast: layouts that would put the caption number
% sideways, i.e., which would need totally different internal coding
% should probably be coded as a separate template of type
% \texttt{caption}.
%
% So for our next example template we settle for the fixed caption
% text plus number (if any) being at the beginning of the variable
% caption text (coming from the document) and being together formatted
% as some sort of a pargraph. In case of the whole caption being a
% single line we allow the designer to specify how to lay it out
% (e.g., centered, flush left, etc.). If there is a legend it will get
% formatted by a vertical space followed by the legend formatted as
% another paragraph.
%
% More precisely we allow for the following bells and wistles:
%\begin{verbatim}
%\DeclareTemplate{caption}{lesssimple}{4}
%  {
%\end{verbatim}
% The designer can specify the space above and below the caption like
% we did in our first example.
%\begin{verbatim}
%   above-skip         =L           [0pt] \caption@above@skip ,
%   below-skip         =L           [0pt] \caption@below@skip ,
%\end{verbatim}
% Regarding the caption number we support the case where no number is
% present (the value being |\NoValue|) as well as the number being
% present. For both cases the designer has to specify what formatting
% should be attached. By default all is being typeset in the font the
% whole caption is presented but if there is a need for it the
% designer can use the following keys to attach special formatting
% devices to each particular item beside specifying special spacing
% information or replacing the default colon after the number with
% something else.
%\begin{verbatim}
%   number-format      =f2      [#1~#2:~] \caption@number@format,
%   nonumber-format    =f1         [#1:~] \caption@nonumber@format,
%\end{verbatim}
% If the caption is  fitting onto a single line we make it possible
% for the designer to specify how this single line should be
% positioned (the default is to center the line).
%\begin{verbatim}
%   single-line-format =f1 [\hfil#1\hfil] \caption@single@line@format,
%\end{verbatim}
% The font for the caption (including the fixed text and the number
% unless specified differently above) is going to be the one decided
% by the next key.
%\begin{verbatim}
%   caption-font       =f0  [\normalfont] \caption@font,
%\end{verbatim}
% The next attribute deserves some extra explanation: here we make use
% of an interface which is explained in more detail when we reveil the
% support for galley formatting.\footnote{Guess I have to apologize
% for the fact that i partly make use of that interface in this
% example while on other occasions (like the use of vertical spacing)
% within the example I do not---consistency around midnight is not my
% strength I fear (FMi).} In a nutshell the template type \texttt{hj}
% (hyphenation \& justification) allows one to define a) the
% justification concepts applied to the upcoming paragraphs, e.g.,
% whether they should be set flush left, adjusted, first line
% centered, etc.\ b) the linebreaking strategy used and c) the
% hyphenation rules which should apply. All this is done by selecting
% an appropriate (predefined) instance of this type as will hopefully
% become somewhat clearer in the example instances shown below.
%\begin{verbatim}
%   caption-hj-setup   =i {hj}  [default] \caption@hj@instance,
%\end{verbatim}
% In case there is a legend to format we give the designer the
% possibility to specify by how much vertical space it should be
% separated from the preceding paragraph (i.e., the caption text). The
% attributes for font and hj setup are comparable to those for the
% caption text itself (except that they will only apply to the
% legend). The only addition is the key \texttt{legend-text} which is
% allowed to take a fixed text (plus any formating and spacing for it)
% which will be added to the front of the legend in case it is
% provided at all (by default it is empty).
%\begin{verbatim}
%   legend-sep         =L           [0pt] \caption@legend@sep ,
%   legend-text        =f0             [] \caption@legend@text,
%   legend-font        =f0  [\normalfont] \caption@legend@font,
%   legend-hj-setup    =i {hj}  [default] \caption@legend@hj@instance,
%  }
%\end{verbatim}
% The actual code for the template should hold few if any
% surprises. In fact it is more or less identical to the one of the
% first template example, except that now we have now taken out some
% of the hardwired decisions and placed them into attributes.
%\begin{verbatim}
%  {
%    \IfNoValueTF{#2}
%       { \def:Npn \caption@start{\caption@number@format{#1}{#2}} }
%       { \def:Npn \caption@start{\caption@nonumber@format{#1}}   }
%    \DoParameterAssigments
%    \vskip \caption@above@skip \relax
%\end{verbatim}
% To properly measure the caption to determine if it fits a single
% line we have to set it in the right font, so here as well as below
% we have to apply |\caption@font|.
%\begin{verbatim}
%    \sbox \@tempboxa {\caption@font \caption@start #3}
%    \ifdim \wd\@tempboxa >\hsize
%      \begingroup
%        \caption@font \caption@hj@instance
%        \caption@start #3\par
%      \endgroup
%    \else
%      \global \@minipagefalse
%      \hb@xt@\hsize{\caption@single@line@format{\box\@tempboxa}}
%    \fi
%\end{verbatim}
% To decide whether or not we have to set any legend we have to test
% |#4| for being |\NoValue|. This part of the code was not present in
% the previous example but otherwise should be straight forward.
%\begin{verbatim}
%    \IfNoValueF{#4}
%      {
%       \vskip \caption@legend@sep \relax
%       \begingroup
%          \caption@legend@font \caption@legend@hj@instance
%          \caption@legend@text
%          #4\par
%       \endgroup
%      }
%    \vskip \caption@below@skip \relax
%  }
%\end{verbatim}
%
% I wouldn't claim the the above template is good or contains
% everything that would be desired and I'm sure that in the end we
% will have several such template for typesetting the caption part and
% perhaps decide on a different template type in the first place. So
% this is only to give a glimpse of how the template interface could
% be applied and I hope that reading it can see a) how they can apply
% it to other areas as well as see what is wrong with the example
% itself.
%
% To just note one point that i thought of being wrong after writing
% the above paragraphs: the key \texttt{single-line-format} was
% declared to be a function with one argument with the idea that
% besides specifying the single line should be centered (|\hfil|) on
% both sides, or flush left, or flush right (|\hfil| on one side) one
% could also specify something like
%\begin{verbatim}
%  single-line-format = \hspace{10pt}#1\hfil,
%\end{verbatim}
% that is a fixed indentation on the left in case where the caption is
% a single line. However, of course one can't. Or at least it is not
% safe to do so since our test in the code tests the width of the line
% without taking into account such a finite fixed space and guess what
% might happen? So in summary, flexibility needs some thought and
% often some afterthoughts as well --- happy thinking :-)
%
%
%
% \subsection{Defining a few instances}
%
% So let us conclude this example with a few sample instances. We
% start with one that repeats what current \LaTeXe{} provides in the
% article class. It shows all keys with values. However in fact only
% the first key is actually needed since all others are the same as
% the default values in the template (and of course a legend is not
% specifiable in standard \LaTeX{} coding so those settings simply do
% not apply anyway).
%\begin{verbatim}
%\DeclareInstance{caption}{article}{lesssimple}
%  {
%   above-skip         = 10pt,
%   below-skip         = 0pt,
%   number-format      = #1~#2:~,
%   nonumber-format    = #1:~,
%   single-line-format = \hfil#1\hfil,
%   caption-font       = \normalfont,
%   caption-hj-setup   = default,
%   legend-sep         = 0pt,
%   legend-text        = ,
%   legend-font        = \normalfont,
%   legend-hj-setup    = default,
%  }
%\end{verbatim}
%
% The next examples are taken from books on the shelf essentially a
% random selection I fear. This one is from \emph{Introduction to
% Database Design} by C.~J.~Date and it uses Helvetica for the caption
% text with the caption flush left, with the figure and the fixed
% string (e.g., `Fig.' in bold face) separated by a quad of space. No
% legend either so this is not set up. The \texttt{hj} instance
% \texttt{noindentflushleft} is supposed to produce a ragged right paragraph
% without any indentation. It would have to be set up elsewhere
% (instance to the template of type \texttt{hj}).
%\begin{verbatim}
%\DeclareInstance{caption}{DATE}{lesssimple}
%  {
%   above-skip         = 10pt,
%   below-skip         = 0pt,
%   number-format      = \textbf{#1~#2}\quad,
%   nonumber-format    = \textbf{#1}\quad,
%   single-line-format = #1\hfil,
%   caption-font       = \fontfamily{phv} \normalfont,
%   caption-hj-setup   = noindentflushleft,
%  }
%\end{verbatim}
%
% The final example is from the book ``Methods of Book Design'' by
% H.~Williamson which sets the caption centered if it fits a single
% line but adjusted as a paragraph without any indentation if longer
% than a single line. It uses old style numerals followed by a period
% for the number (though the example isn't quite right as i guess the
% text font used already has oldstyle numerals as default, so
% |\oldstylenums| is in fact not necessary).
%\begin{verbatim}
%\DeclareInstance{caption}{WILLIAMSON}{lesssimple}
%  {
%   above-skip         = 10pt,
%   below-skip         = 0pt,
%   number-format      = #1~\oldstylenums{#2}.~,
%   nonumber-format    = #1~,
%   single-line-format = \hfil#1\hfil,
%   caption-font       = \normalfont,
%   caption-hj-setup   = noindentadjusted,
%  }
%\end{verbatim}
%
%
%
% \section{Notes}
%
% \subsection{Note on multi-valued parameters}
%
%
%  The following code\footnote{docu taken from trial implementation in
%  xlists.dtx, FMi} implements for registers (ie L,l,C,c keys) and
%  for names (ie n key) a multi-selection mechanism of the following
%  form:
% \begin{verbatim}
%    key    = \MultiSelection \ListDepth {
%                   \DelayEvaluation {2.5em},
%                   20pt + 34pt }
%                  { \DelayEvaluation {1em} },
% \end{verbatim}
% where the first argument to |\MultiSelection| is a counter, the
% second argument is a comma separated list of values denoting the
% values for the cases 1, 2,\ldots, and the third argument contains
% the value for all other cases.
%
% The values are evaluated at declaration time in case of registers
% and therefore can contain calc expressions as well as
% |\DelayEvaluation|.
%
% Due to the implementation the case list is not allowed to have a
% trailing comma! And of course no checks are made whatsoever :-(
%
% A probably much nicer syntax would be something like this:
% \begin{verbatim}
%    key    = \MultiSelection {
%                selector = \ListDepth,
%                1        = \DelayEvaluation {2.5em},
%                2        = 20pt + 34pt,
%                else     = \DelayEvaluation {1em}
%              },
% \end{verbatim}
% but i found that too difficult to implement right now.
%
% I think it should also be considered if this kind of thing should be
% a generally available feature on all key types especially on the
% f\meta{number} ones.
%
% Anyway it is what i need for lists right now and as such it is
% sufficient.
%
%
% \subsection{Notes on template restriction}
%
% Possible semantics:
%
% a: just:-) changes the defaults  ie the new template has as
%    defaults those of its source as modified by the supplied
%    keyvals;
%
% b: similar to a: but also removes some keywords ie the new template
%    will not accept the keywods whose values are set by the suppied
%     keyvals;
%
% c: plan C.
%
%
% Towards an implementation of b: but without a restriction on what
%   keys appear where.
%
%
%
%
% \subsection{Open issues}
%
% In this section unresolved issues or ideas to think about and
% perhaps implement are collected. There is no particular order to
% them.
%
% \begin{itemize}
% \item The order of arguments in |\UseCollection| is illogical in my
%  eyes!  A collection typically modifies the behavior of several
%  types and thus should perhaps be first (as it is in the
%  |\DeclareCollectionInstance| case). Or not, or what?
%
% \item How should |\IfExistsInstanceTF| behave for Collection
%  instances? Do we need a special check for those or a default
%  action? Or do we need an additional test for the existence of
%  collection instances?
%
% \item It was suggested that the template type declaration should get
%  another argument in which (in?)formally the semantics for the
%  template types are described, e.g., data type of arguments,
%  resulting output, \ldots{} (somewhat like the description arguments
%  for functions and variables in Emacs-Lisp). The advantage being
%  that this helps employing the templates better as well as perhaps
%  guiding context sensitive editors to support the work with such
%  templates (e.g., providing help texts).
%
% \item The same might be of interest for the keys of individual
%  templates though here syntax support is already available to some
%  extend by the declaration of key types.
%
% \item There might be a need to distinguish between \TeX's dimen and
%  skip registers. Right now this is not done and both |l| and |L|
%  accepts what \LaTeX{} calls ``rubber length'' specifications.
%
% \item The type |b| can probably vanish. It is equivalent to
%  specifying the mutators of a |\newif| command in the brace groups,
%  e.g.
%\begin{verbatim}
%  numbered-boolean =b [true] @heading@nums,
%  numbered-boolean =s [true] {\@heading@numstrue}
%                             {\@heading@numsfalse},
%\end{verbatim}
%
% \item See issue raised about syntax (and semantics) for
% |\Multiselection|.
%
% \item |f0| keys should perhaps support |\UseTemplate| by replacing
% it with its internal form. or perhaps this is a rubbish idea?
%
% \item Marcin Wolinski suggested to use |\EvalOnUse| instead or in
% addition to |\DelayEvalutation|.
%
% \end{itemize}
%
% \StopEventually{}
%
% \section{Implementation}
%    \begin{macrocode}
%<*package>
\ProvidesExplPackage
  {\filename}{\filedate}{\fileversion}{\filedescription}
%    \end{macrocode}
%
%    \begin{macrocode}
\RequirePackage{ldcsetup,xparse}
\RequirePackage{l3toks,l3tlp,l3skip,l3int,l3clist,l3token,l3calc}
%    \end{macrocode}
%
%
% Declare a private token register for building parameter lists.
% Having the number saves a few expandafters
% (probably not needed in the end).
%    \begin{macrocode}
\toks_new:N\l_TP_KV_assignments_toks
\toks_new:N\l_TP_default_assignments_toks
%    \end{macrocode}
%
%
%
% \begin{macro}{\TP_declare_instance:Nnn}
% \begin{macro}{\TP_declare_instance:cnn}
% Declare a command name to be an instance of a template
% ie with a particular setting of the parameters.\\
% |#1| internal command name for instance to be (globally) declared\\
% |#2| template type/template name\\
% |#3| key value assignments for parameters of |#2|
%    \begin{macrocode}
\def_new:NNn \TP_declare_instance:Nnn 3{
  \group_begin:
    \TP_instdecl_generate_assignments:nn {#2}{#3}
    \gdef:Npx #1 {
      \tlp_if_eq:cNTF { TP>/#2 } \c_TP_doparameterassignments_tlp
%    \end{macrocode}
%    If the body of the template consists only of the token
%    |\DoParameterAssignments|, then we insert the list of parameter
%    assignments directly. Otherwise we have push them onto the stack
%    and prepare to execute the body code (which in turn will pop them
%    again when it reaches |\DoParameterAssignments| inside).
%    \begin{macrocode}
         { \toks_use:N \l_TP_KV_assignments_toks }
         {
           \exp_not:N \TP_push_assignments:n
              {\toks_use:N\l_TP_KV_assignments_toks}
            \exp_not:c {TP>/#2}
         }
      }
  \group_end:}
\def_new:Npn \TP_declare_instance:cnn{\exp_args:Nc\TP_declare_instance:Nnn}
%    \end{macrocode}
% \end{macro}
% \end{macro}
%
% \begin{macro}{\c_TP_doparameterassignments_tlp}
%    \begin{macrocode}
\tlp_set:Nn \c_TP_doparameterassignments_tlp {\DoParameterAssignments}
%    \end{macrocode}
% \end{macro}
%
%
% \begin{macro}{\UseTemplate}
% |{type}{templatename}{keyval}|
% Directly use a template with a particular parameter setting.
% This is also picked up if used in a nested fashion inside a parameter
% list.\\
% |#1| type of a template.\\
% |#2| name of a template.\\
% |#3| key value assignments for parameters of |#1|.
%    \begin{macrocode}
\def_new:NNn \UseTemplate 3{
  \TP_instdecl_generate_assignments:nn {#1/#2}{#3}
  \TP_push_assignments:
  \use:c { TP>/#1/#2 }
}
%    \end{macrocode}
% \end{macro}
%
% \begin{macro}{\DoParameterAssignments}
%    Access the parameter assignment list that was once stored in
%    |\l_TP_KV_assignments_toks| and then moved onto the
%    |\g_TP_assignments_stack_toks|.
%    \begin{macrocode}
\def_new:Npn \DoParameterAssignments{
  \exp_after:NN
    \TP_pop_and_execute_assignments:nw
       \toks_use:N \g_TP_assignments_stack_toks \q_stop
}
%    \end{macrocode}
%
% \end{macro}
%
% \begin{macro}{\TP_pop_and_execute_assignments:nw}
%    \begin{macrocode}
\def_new:Npn \TP_pop_and_execute_assignments:nw#1#2\q_stop{
  \toks_gset:Nn \g_TP_assignments_stack_toks {#2}
  #1}
%    \end{macrocode}
% \end{macro}
%
%
% \begin{macro}{\g_TP_assignments_stack_toks}
%    \begin{macrocode}
\toks_new:N   \g_TP_assignments_stack_toks
\toks_gset:Nn \g_TP_assignments_stack_toks {\scan_stop:}% avoid brace loss
%    \end{macrocode}
% \end{macro}
%
% \begin{macro}{\TP_push_assignments:n}
% \begin{macro}{\TP_push_assignments:}
%   Push a list of parameter assignments onto the
%   |\g_TP_assignments_stack_toks|. As it all happens in token
%   registers |#|s need no doubling. |\TP_push_assignments:| expects
%   it to be |\l_TP_KV_assignments_toks| (needs fixing).
%    \begin{macrocode}
\def_long_new:Npn \TP_push_assignments:n#1{
  \toks_gput_left:Nn\g_TP_assignments_stack_toks{{#1}}
}
\def_new:Npn \TP_push_assignments:{
  \toks_gset:No \g_TP_assignments_stack_toks
    {\exp_after:NN
     {\toks_use:N\exp_after:NN\l_TP_KV_assignments_toks\exp_after:NN}
      \toks_use:N\g_TP_assignments_stack_toks}
}
%    \end{macrocode}
% \end{macro}
% \end{macro}
%
% 
% \begin{macro}{\DeclareTemplateType}
% |{type}{nofarg}|
%    \begin{macrocode}
\def_new:NNn \DeclareTemplateType 2{
  \tlp_set:cn {TP@<#1>} {{}#2}}
%    \end{macrocode}
% \end{macro}
%
%
% \begin{macro}{\TP_get_csname_prefix:n}
%    |{type}| returns prefix for csnames for template type,
%    based on current collection.
%    \begin{macrocode}
\def_new:Npn \TP_get_csname_prefix:n#1{
  <\exp_after:NN\exp_after:NN\exp_after:NN
   \use_arg_i:nn
      \cs:w TP@<#1>\cs_end:>#1/
}
%    \end{macrocode}
%
% \end{macro}
%
% \begin{macro}{\TP_get_arg_count:n}
% |{type}| returns arg count for the template type.
%    \begin{macrocode}
\def_new:Npn \TP_get_arg_count:n#1{
  \exp_after:NN\exp_after:NN\exp_after:NN
  \use_arg_ii:nn
    \cs:w TP@<#1>\cs_end:
}
%    \end{macrocode}
% \end{macro}
%
%
% \begin{macro}{\DeclareTemplate}
% |{type}{templatename}{nofarg}{keywordspec}{code}|
%    \begin{macrocode}
\def_long_new:NNn\DeclareTemplate 5{
  \cs_if_free:cTF{TP@<#1>}
    {\undefinedtype\DeclareTemplateType{#1}#3}
    {
      \int_compare:nNnF{#3}={\TP_get_arg_count:n{#1}}
      { \BadArgCount }
    }
%    \end{macrocode}
%    Parse the key declaration, and execute the list with a suitable
%    definition of |\KV@elt|.
%    \begin{macrocode}
  \let:NN \KV_elt:nn \TP_templdecl_process_KV:nn
  \def:Npn \KV_default_elt:n##1{
    \PackageError{template}{Missing~ =~ after~ ##1}\@ehd}
  \let:NN\KV@elt\KV_elt:nn
  \let:NN\KV@default@elt\KV_default_elt:n
  \tlp_set:Nn \l_TP_curr_name_tlp {#1/#2}
  \toks_clear:N\l_TP_default_assignments_toks
%    \end{macrocode}
% At this point there should be a test for which catcode regime we are
% in. We just test if spaces are ignored.
%    \begin{macrocode}
  %\int_compare:nNnTF{\char_value_catcode:n{`\ }}=\c_nine
  %\KV_parse_picky_no_space_removal_no_sanitize:n
  %\KV_parse_picky_space_removal_no_sanitize:n
  \KV@parse{#4}
%    \end{macrocode}
%
%    Define the defaults: the setting for |TPD>/\l_TP_curr_name_tlp| is a
%    tricky since |\l_TP_default_assignments_toks| may contain |#|. We
%    have to use an |x| expansion rather than
%    |o| since that will hide those during the assignment.
%    \begin{macrocode}
  \tlp_set:cx { TPD>/\l_TP_curr_name_tlp }
              {\toks_use:N\l_TP_default_assignments_toks}

  \tlp_clear:c {TPR>/\l_TP_curr_name_tlp}

  \tlp_set_eq:cN {TPO>/\l_TP_curr_name_tlp}\l_TP_curr_name_tlp
%    \end{macrocode}
%
%    Define the template (using |\def_new:Npn| means that one can't
%    redefine a template easily).
%    \begin{macrocode}
  \def_long:cNn  {TP>/\l_TP_curr_name_tlp}{#3}{#5}
}
%    \end{macrocode}
% \end{macro}
%
% \begin{macro}{\TP_templdecl_process_KV:nn}
%    The list of undefined keys and values is put in the list of the form\\
%    |\KV_elt:nn|\marg{key1}\marg{val1}%^^A
%    |\KV_elt:nn|\marg{key2}\marg{val2}\ldots\\
%    So just need to give this macro a suitable definition. We just need
%    to look at the first token of the value, to see what sort of key
%    it is, so call a helper function to split that off.
%    \begin{macrocode}
\def_new:Npn \TP_templdecl_process_KV:nn#1#2{%
  \let:NN \TP_templdecl_add_global_or_nothing: \use_noop:
  \bool_set_false:N\l_TP_global_assignment_bool
  \tlp_set:Nn\l_TP_currkey_tlp{#1}
  \TP_templdecl_parse_KV:N#2\q_stop}
%    \end{macrocode}
% \end{macro}
%
% \begin{macro}{\TP_templdecl_parse_KV:N}
%    Case switch on the possible key types.
%    \begin{macrocode}
\def_new:Npn \TP_templdecl_parse_KV:N#1{
%    \end{macrocode}
% In |#1| we have key, in |#2| the first character after the equal
% sign and in |#3| the remainder of the line. We now have to parse
% that remainder to find out if it contains a default value (in
% brackets) and then set up the key declaration needed to parse
% instance declarations. The method is similar in most cases: we call
% |\TP_parse_optional_key_default:nw| which parses for the default
% and pass it already found key name as first argument, what to do in
% the end as second argument, and the remainder delimited by |\q_stop|
% so that it becomes parseable.
%
% Note that the code in the second argument to
% |\TP_parse_optional_key_default:nw| normally calls on a macro with
% one more argument than actually provided: the reason being that the
% missing argument will be the remainder of the line (added by
% |\TP_parse_optional_key_default:nw| after the default has be
% removed (if present)).
%    \begin{macrocode}
  \cs_if_free:cTF{TP_use_arg_type_#1:w}
  {\PackageError{template}{Unknown~key~type~ (#1)~for~\l_TP_currkey_tlp}\@eha}
  {\use:c{TP_use_arg_type_#1:w}}
%    \end{macrocode}
%    The |f| and |i| keys are somewhat different since there we first
%    have to parse for an additional argument (a digit in case of |f|
%    or an template type name in case of |i|):
%    \begin{macrocode}
%    \end{macrocode}
%    One more alternative: a |+| after the equal sign signals global
%    so we change |\TP_templdecl_add_global_or_nothing:| to append a
%    |\pref_global:D| to the assignment toks and then reparse the rest.
%    \begin{macrocode}
%     \def:Npn  \TP_templdecl_add_global_or_nothing:
%          {\toks_put_right:Nn \l_TP_KV_assignments_toks {\pref_global:D} }
%     \TP_templdecl_parse_KV:nw{#1}#3\TP_templdecl_parse_KV:nnw
}
%    \end{macrocode}
% \end{macro}
%
%
% \begin{macro}{\l_TP_global_assignment_bool}
% For keeping track of the assignments.
%    \begin{macrocode}
\bool_new:N \l_TP_global_assignment_bool
%    \end{macrocode}
% \end{macro}
%
% \begin{macro}{\TP_use_arg_type_+:w}
%   The |+| does two things: Sets a boolean true to be used by the
%   types that can't simply be prefixed with |\pref_global:D| and
%   defines |\TP_templdecl_add_global_or_nothing:| to put the prefix
%   onto the list. After that we simply call the big switch
%   again.\footnote{It should probably all be changed to not rely on
%     the prefix working}.
%    \begin{macrocode}
\def_new:cpn{TP_use_arg_type_+:w} {
  \bool_set_true:N\l_TP_global_assignment_bool
  \def:Npn  \TP_templdecl_add_global_or_nothing:
  {\toks_put_right:Nn \l_TP_KV_assignments_toks {\pref_global:D} }
  \TP_templdecl_parse_KV:N
}
%    \end{macrocode}
% \end{macro}
%
% \begin{macro}{\TP_use_arg_type_l:w}
%   The |l| sets a length register. We disable the prefix and insert
%   either |\gsetlength| or |\setlength| depending on whether a |+|
%   was seen or not.
%    \begin{macrocode}
\def_new:Npn\TP_use_arg_type_l:w {
  \TP_parse_optional_key_default:nw
  {
    \let:NN \TP_templdecl_add_global_or_nothing: \use_noop:
    \bool_if:NTF \l_TP_global_assignment_bool
    {\TP_templdecl_setup_register_key:Nn\gsetlength}
    {\TP_templdecl_setup_register_key:Nn\setlength}
  }
}
%    \end{macrocode}
% \end{macro}
%
% \begin{macro}{\TP_use_arg_type_L:w}
% The |L| sets a fake length register.
%    \begin{macrocode}
\def_new:Npn\TP_use_arg_type_L:w {
  \TP_parse_optional_key_default:nw
  {\TP_templdecl_setup_fakeregister_key:NNn\setlength\l_tmpa_skip}
}
%    \end{macrocode}
% \end{macro}
%
% \begin{macro}{\TP_use_arg_type_c:w}
% The |c| sets a count register.
%    \begin{macrocode}
\def_new:Npn\TP_use_arg_type_c:w {
  \TP_parse_optional_key_default:nw
  {
    \let:NN\TP_templdecl_add_global_or_nothing:\use_noop:
    \bool_if:NTF \l_TP_global_assignment_bool
    {\TP_templdecl_setup_register_key:Nn\GSetInternalCounter}
    {\TP_templdecl_setup_register_key:Nn\SetInternalCounter}
  }
}
%    \end{macrocode}
% \end{macro}
%
% \begin{macro}{\TP_use_arg_type_C:w}
% The |C| sets a fake count register.
%    \begin{macrocode}
\def_new:Npn\TP_use_arg_type_C:w {
  \TP_parse_optional_key_default:nw
  {\TP_templdecl_setup_fakeregister_key:NNn
    \SetInternalCounter\l_tmpa_int}
}
%    \end{macrocode}
% \end{macro}
%
% \begin{macro}{\TP_use_arg_type_n:w}
% The |n| sets a token list pointer.
%    \begin{macrocode}
\def_new:Npn\TP_use_arg_type_n:w {
  \TP_parse_optional_key_default:nw
  {\TP_templdecl_setup_n_key:N}        
}
%    \end{macrocode}
% \end{macro}
%
% \begin{macro}{\TP_use_arg_type_f:w}
% \begin{macro}{\TP_templdecl_parse_f_arg:nw}
% The |f| defines a command with between 0 and 9 arguments.
%    \begin{macrocode}
\def_new:Npn\TP_use_arg_type_f:w #1{
  %\TP_templdecl_parse_f_arg:nw {#1}
  \TP_parse_optional_key_default:nw{\TP_templdecl_setup_f_key:Nn{#1}}
}
%    \end{macrocode}
%    Helper for |\TP_templdecl_parse_KV:nnw|.
%    \begin{macrocode}
\def_new:Npn \TP_templdecl_parse_f_arg:nw#1#2{
%    \end{macrocode}
%    The third argument of |\TP_templdecl_setup_f_key:Nn|, i.e., the
%    macro name, is the remaining data up to |\q_stop| which is picked up
%    by |\TP_parse_optional_key_default:nw|.
%    \begin{macrocode}
  \TP_parse_optional_key_default:nw{\TP_templdecl_setup_f_key:Nn{#1}{#2}}
}
%    \end{macrocode}
% \end{macro}
% \end{macro}
%
%
% \begin{macro}{\TP_use_arg_type_b:w}
% \begin{macro}{\TP_templdecl_setup_b_key:nn}
%   The |b| uses access to the |\if_true:| and |\if_false:|
%   primitives. Needed?
%    \begin{macrocode}
\def_new:Npn\TP_use_arg_type_b:w {
  \TP_parse_optional_key_default:nw
  {\TP_templdecl_setup_b_key:n}
}
%    \end{macrocode}
%
%    \begin{macrocode}
\def_new:Npn \TP_templdecl_setup_b_key:n#1{
  \let:cN { if#1 } \if_true:
  \TP_templdecl_define_key:n
  { \TP_templdecl_eval_b_key_value:nn {#1}{##1} }
}
%    \end{macrocode}
% \end{macro}
% \end{macro}
%
% \begin{macro}{\TP_templdecl_eval_b_key_value:nn}
%    Modify so the boolean does not need to have been
%    declared with |\newif|
%    \begin{macrocode}
\def_new:Npn \TP_templdecl_eval_b_key_value:nn#1#2{
  \cs_if_free:cTF {if#2}
    { \PackageError{template}{Bad~boolean~setting~#1=#2}\@eha }
    { \tlp_set_eq:cc {if_#1:}{if_#2:}
      \toks_put_right:Nf \l_TP_KV_assignments_toks
         { \tlp_set_eq:cc {if_#1:}{if_#2:} }
    }
}
%    \end{macrocode}
%
% \end{macro}
%
% \begin{macro}{\TP_use_arg_type_s:w}
% \begin{macro}{\TP_templdecl_setup_s_key:n}
% The |s| chooses between true and false.
%    \begin{macrocode}
\def_new:Npn\TP_use_arg_type_s:w {
  \TP_parse_optional_key_default:nw
  {\TP_templdecl_setup_s_key:n}
}
%    \end{macrocode}
%    \begin{macrocode}
\def_new:Npn \TP_templdecl_setup_s_key:n #1 {
  \TP_templdecl_define_key:n
    { \TP_templdecl_eval_s_key_value:nnn{##1}#1 }
}
%    \end{macrocode}
% \end{macro}
% \end{macro}
%
%
%
%
%
%
%
% \begin{macro}{\TP_use_arg_type_i:w}
% The |i| expects an instance.
%    \begin{macrocode}
\def_new:Npn\TP_use_arg_type_i:w #1{
  \TP_parse_optional_key_default:nw{\TP_templdecl_setup_i_key:nnn{#1}}
}
%    \end{macrocode}
%
%      declaration |hd =i{head} \fooo| \\
%      use         |hd = mine|\\
%      makes |\fooo| shorthand for |\UseInstance{head}{mine}|
%
%      also allowed: |hd = \UseTemplate{head}{...}{...}|\\
% in case you want to use an unnamed instance of type |head|
% in this place.
% \end{macro}
%
% \begin{macro}{\TP_templdecl_setup_i_key:nnn}
% MH change: do either local or global.
%    \begin{macrocode}
\def_new:Npn \TP_templdecl_setup_i_key:nnn#1#2{
  \TP_templdecl_define_key:n
  {
    \TP_templdecl_eval_i_key_value:Nnn #2 {#1}{##1} 
  }
}
%    \end{macrocode}
% \end{macro}
%
% \begin{macro}{\TP_templdecl_eval_i_key_value:Nnn}
% MH change: Add extra argument for local or global.
%    \begin{macrocode}
\def_new:Npn \TP_templdecl_eval_i_key_value:Nnn #1#2#3 {
  \tlist_if_head_eq_meaning:nNTF {#3.}\UseTemplate
  {
    \io_put_term:x{\token_to_string:N\UseTemplate\space seen}
%    \end{macrocode}
%
%    Code below from |\TP_templdecl_setup_f_key:Nn| (should be
%    combined and cleaned up)
%    at this point one should also check if first arg of |\UseTemplate|
%    corresponds to |#2| and if not complain (not done)
%
%    \begin{macrocode}
    {\TP_templdecl_declare_tmp_instance:nnnn #3 }
    \toks_put_right:No \l_TP_KV_assignments_toks
          { \exp_after:NN \KV@toks \exp_after:NN {\g_tmpa_tlp} }
    %\TP_templdecl_add_global_or_nothing:    
    %\toks_put_right:Nn \l_TP_KV_assignments_toks
    %      { \def:Npx #1{ \toks_use:N \KV@toks} }
    \bool_if:NTF \l_TP_global_assignment_bool
    {\toks_put_right:Nn \l_TP_KV_assignments_toks 
      {\gdef:Npx #1 { \toks_use:N \KV@toks}}
    }
    {\toks_put_right:Nn \l_TP_KV_assignments_toks 
      {\def:Npx #1 { \toks_use:N \KV@toks}}
    }
  }
  {
    \TP_let_instance:Nnn#1{#2}{#3}
%    \end{macrocode}
% We want the |\let:Nc| hiding in |\TP_let_instance:Nnn| to expand
% fully to two csnames so we put a |\tex_romannumeral:D 0| (which in
% itself expands to nothing) in front. This expands the |\let:Nc| fully
% before finding out that |\let:NwN| is not expandable.
%    \begin{macrocode}
      \toks_put_right:Nf  \l_TP_KV_assignments_toks
        { \TP_let_instance:Nnn#1{#2}{#3} }
  }
}
%    \end{macrocode}
% \end{macro}
%
%
%
%
%
%
%
%
% \begin{macro}{\TP_use_arg_type_x:w}
% \begin{macro}{\TP_templdecl_setup_x_key:nn}
% The |x| runs internal code.
%    \begin{macrocode}
\def_new:Npn\TP_use_arg_type_x:w {
  \TP_parse_optional_key_default:nw
  {\TP_templdecl_setup_x_key:n}
}
%    \end{macrocode}
%
%    \begin{macrocode}
\def_new:Npn \TP_templdecl_setup_x_key:n#1{
  \TP_templdecl_define_key:n
  { \toks_put_right:Nn\l_TP_KV_assignments_toks{#1} }
}
%    \end{macrocode}
% \end{macro}
% \end{macro}
%
%
%
%
%
% \begin{macro}{\TP_use_arg_type_g:w}
% \begin{macro}{\TP_templdecl_setup_g_key:nn}
% The |g| runs any code.
%    \begin{macrocode}
\def_new:Npn\TP_use_arg_type_g:w {
  \TP_parse_optional_key_default:nw
  {\TP_templdecl_setup_g_key:n}
}
%    \end{macrocode}
%
%    \begin{macrocode}
\def_new:Npn \TP_templdecl_setup_g_key:n #1 {
  \TP_templdecl_define_key:n{#1}}
%    \end{macrocode}
%
% \end{macro}
% \end{macro}
%
%
%
%
% \begin{macro}{\TP_templdecl_define_key:n}
% Here we define the key in the current template.
%    \begin{macrocode}
\def_new:Npn \TP_templdecl_define_key:n#1{
  \tlp_set:Nx \l_tmpa_tlp {
    \exp_not:N \TP_templdecl_remove_from_default_assignments:N
    \exp_not:c{KV@\l_TP_curr_name_tlp @\l_TP_currkey_tlp}
    \exp_not:o {\TP_templdecl_add_global_or_nothing: }
  }
  \exp_args:NcNo \def:NNn {KV@\l_TP_curr_name_tlp @\l_TP_currkey_tlp} 1
  { \l_tmpa_tlp #1 }
}
%    \end{macrocode}
% \end{macro}
%
%
%
%
%
%
%
%
%
%
% \begin{macro}{\TP_parse_optional_key_default:nw}
%   Look for default value. The |t| argument type here is one we
%   define for template so we can change it easily. Currently I just
%   set it to be a regular optional argument in brackets. |#1| is what
%   we are carrying over, |#2| the optional argument.
%    \begin{macrocode}
\def_long:Npn \TP_ignore_leading_space_in_arg_ii:nn#1#2{
  \exp_args:Nf\TP_ignore_leading_space_in_arg_ii_aux:nn
  {\exp_not:N #2}{#1}
}
\def_long:Npn \TP_ignore_leading_space_in_arg_ii_aux:nn#1#2{#2{#1}}


\DeclareArgumentType t[{meaning}{}{\NoValue}{#1[#2]}{#2}
\DeclareDocumentCommand\TP_parse_optional_key_default:nw{mt}{
  \IfNoValueTF{#2}
    {\TP_templdecl_finish_key_setup:nw{#1}}
    {\TP_templdecl_finish_key_setup_with_default:nnw{#1}{#2}}
}
%\show\TP_parse_optional_key_default:nw
%\exp_args:Nc\show{\string\TP_parse_optional_key_default:nw}
%    \end{macrocode}
% \end{macro}
%
% \begin{macro}{\TP_templdecl_finish_key_setup:nw}
%    After having parsed the line and not found any default value it
%    remains to actually define the key for the instance parsing by
%    executing the setup code (in |#1|) giving it |#2| (i.e., the
%    remainder of the line) as an argument.
%    \begin{macrocode}
\def_new:Npn \TP_templdecl_finish_key_setup:nw#1#2\q_stop{
  \TP_ignore_leading_space_in_arg_ii:nn{#1}{#2}
  %%%#1{#2}
}
%    \end{macrocode}
% \end{macro}
%
% \begin{macro}{\TP_templdecl_finish_key_setup_with_default:nnw}
%    If there is a default the situation is more complicated since we
%    not only have to set up the key for the instance but also have to
%    add the default value to |\l_TP_default_assignments_toks| in an
%    appropriate way.
%
%    First set up the the key itself:
%    \begin{macrocode}
\def_new:Npn \TP_templdecl_finish_key_setup_with_default:nnw#1#2#3\q_stop{
  \TP_ignore_leading_space_in_arg_ii:nn{#1}{#3}
  %%%  #1 {#3}
%    \end{macrocode}
% Now we run the new key code (which is stored in |\KV@...| hopefully)
% and give it the default found. By doing this in a group and by
% locally emptying |\l_TP_KV_assignments_toks| we will get the
% resulting assignment code into that register.
%
% (We set |\TP_templdecl_remove_from_default_assignments:N| to
% |\use_none:n| since this is a temporary operation and we don't want
% to change the default really.)
%    \begin{macrocode}
  \group_begin:
    \toks_clear:N \l_TP_KV_assignments_toks
    \let:NN \TP_templdecl_remove_from_default_assignments:N \use_none:n
    \use:c{KV@\l_TP_curr_name_tlp @\l_TP_currkey_tlp}{#2}
%    \end{macrocode}
% And now for a final trick: before closing the group again and losing
% our local changes we run |\exp_after:NN| several times to get the
% value of |\l_TP_KV_assignments_toks| into
% |\l_TP_default_assignments_toks| outside that group!
%    \begin{macrocode}
  \exp_after:NN
  \group_end:
  \exp_after:NN
  \toks_set:Nn
  \exp_after:NN
  \l_TP_default_assignments_toks
  \exp_after:NN
    { \cs:w KV@\l_TP_curr_name_tlp @\l_TP_currkey_tlp \exp_after:NN \cs_end:
  \exp_after:NN
      { \toks_use:N \exp_after:NN \l_TP_KV_assignments_toks
  \exp_after:NN
      }
      \toks_use:N\l_TP_default_assignments_toks
  }
}
%    \end{macrocode}
% \end{macro}
%
%
%
% \begin{macro}{\c_TP_true_tlp}
%    \begin{macrocode}
\tlp_new:Nn \c_TP_true_tlp {true}
%    \end{macrocode}
% \end{macro}
%
% \begin{macro}{\TP_templdecl_eval_s_key_value:nnn}
%    \begin{macrocode}
\def_new:Npn \TP_templdecl_eval_s_key_value:nnn#1#2#3 {
% no error check on this yet.
  \tlp_set:Nn \l_tmpa_tlp {#1}
  \tlp_if_eq:NNTF \l_tmpa_tlp \c_TP_true_tlp
     { \toks_put_right:Nn \l_TP_KV_assignments_toks {#2} }
     { \toks_put_right:Nn \l_TP_KV_assignments_toks {#3} }
}
%    \end{macrocode}
% \end{macro}
%
% \begin{macro}{\TP_templdecl_setup_register_key:Nnn}
% This is normally called automatically by |\DeclareTemplate|.
%
% Command for setting a template attribute whose name
% corresponds directly to a \TeX{} count or length register\\
% |#1| the function to set the value eg \setlength.\\
% |#2| key name.\\
% |#3| the register to set.
%
% This command \emph{fully evaluates} the argument at declare time,
% and assigns the value to the register. It also passes an assignment
% of the register to the final value into the parameter list for the
% template.
%
% If the value is a call to |\DelayEvaluation|, don't evaluate it now,
% just pass the whole assignment to the template. Remember to remove
% the |\DelayEvaluation|.
%    \begin{macrocode}
\def_new:Npn \TP_templdecl_setup_register_key:Nn #1#2{
  \TP_templdecl_define_key:n{
    \tlist_if_head_eq_meaning:nNTF{##1}\DelayEvaluation
    {
%    \end{macrocode}
% Old line commented out. Remove |\DelayEvaluation| and also remove
% the braces surrounding its argument.
%    \begin{macrocode}
      \toks_put_right:Nn \l_TP_KV_assignments_toks {#1#2{##1}}
      %\toks_set:No\l_tmpa_toks{\use_arg_ii:nn ##1}
      %\toks_put_right:Nx \l_TP_KV_assignments_toks 
      %  {\exp_not:n{#1#2}{\toks_use:N \l_tmpa_toks}}
    }
%    \end{macrocode}
%
%
% check for |\MultiSelection| creeping up and if so add something like
%
% \begin{verbatim}
%  \setlength\register{\ifcase\selector \or value1 \or value2
%                       ... \else valueotherwise \fi}
% \end{verbatim}
% to |\l_TP_KV_assignments_toks|.
%
%    \begin{macrocode}
    {
      \tlist_if_head_eq_meaning:nNTF{##1..}\MultiSelection
      {
        \group_begin:
          \TP_multiselection_add:nnnnnn #1#2##1
        \group_end:
%    \end{macrocode}
%
%
% there are probably better ways to do this (:-)
%
%    \begin{macrocode}
        \tlist_if_in:onTF{\toks_use:N\g_TP_multiselection_toks}\DelayEvaluation
        {
          \toks_put_right:No\l_TP_KV_assignments_toks
          {
            \exp_after:NN#1\exp_after:NN#2\exp_after:NN
            {\toks_use:N\g_TP_multiselection_toks}
          }
        }
        {
          \toks_put_right:No\l_TP_KV_assignments_toks
          {
            \exp_after:NN  #2
            \exp_after:NN= \toks_use:N\g_TP_multiselection_toks\scan_stop:
          }
        }
%    \end{macrocode}
%
%
% otherwise do as before
%
%    \begin{macrocode}
      }
      {
        #1#2{##1}
        \toks_put_right:No\l_TP_KV_assignments_toks {
          \exp_after:NN #2 \exp_after:NN = \tex_the:D #2\scan_stop:
        }
      }
    }
  }
}
%    \end{macrocode}
% \end{macro}
%
%
%
%
%
%
% \begin{macro}{\DelayEvaluation}
% \begin{macro}{\MultiSelection}
%   Since we are testing explicitly for |\DelayEvaluation| and
%   |\MultiSelection| a few places we better give them unique
%   meanings!
%    \begin{macrocode}
\def_new:NNn\DelayEvaluation 1{\use_none:n{\DelayEvaluation}#1}
\def_new:NNn\MultiSelection  1{\use_none:n{\MultiSelection}#1}
%    \end{macrocode}
% \end{macro}
% \end{macro}
%
% \begin{macro}{\TP_templdecl_remove_from_default_assignments:N}
%    Note: the toks register is more or less a plists and should
%    perhaps be implemented as such as this would make far more
%    readable code.
%    \begin{macrocode}
\def_new:Npn \TP_templdecl_remove_from_default_assignments:N#1{
  \def:Npn \tmp:w ##1#1##2##3#1##4\q_stop{
    \l_TP_default_assignments_toks{##1##3}
  }
  \exp_after:NN \tmp:w
     \toks_use:N\l_TP_default_assignments_toks #1\scan_stop:#1\q_stop}
%    \end{macrocode}
% \end{macro}
%
% \begin{macro}{\TP_templdecl_setup_f_key:Nn}
%       Same for macro names. Again usually called automatically when
%       declaring a new template.\\
% |#1|   Determines how many arguments the function should have.\\
% |#2| The macro to be defined.
%
%    If the `|##1|`, the value passed as the argument of the key
%    to the macro |#2| is invoked starts with |\FunctionInstance|, then a
%    special procedure is taken. Instead of defining a macro with the
%    specified number of arguments, the paramater list of the nested
%    function instance is parsed, and |#2| is defined to be a macro
%    expanding to that instance. In this case the specified template
%    is responsible for picking up the requested number of
%    arguments. (This is \emph{not} checked.)
%    \begin{macrocode}
\def_new:Npn \TP_templdecl_setup_f_key:Nn#1#2{
%    \end{macrocode}
%
% |##1| can either be arbitrary inline code, in which case it will be
% defined with something similar to |\newcommand[val]| so it needs to
% use |#1| -- |#val|.
%
%    define it locally here (why this, David???)
%    \begin{macrocode}
  \TP_templdecl_define_key:n
  { \TP_templdecl_define_function:NNn#1#2{##1} }
}
%    \end{macrocode}
% \end{macro}
%
%
% \begin{macro}{\TP_templdecl_define_function:NNn}
%    |\def:Npn | setup with a latex style `number of arguments' argument.
%    \begin{macrocode}
\def_new:Npn \TP_templdecl_define_function:NNn#1#2#3{
  \def:NNn #2 #1 {#3}
  \toks_put_right:Nf \l_TP_KV_assignments_toks  { \def:NNn #2 #1 {#3} }
}
%    \end{macrocode}
% \end{macro}
%
%
%
% \begin{macro}{\TP_templdecl_setup_n_key:N}
% Here is the extended version that tries to deal with
% |\MultiSelection|.
%
% In case of `n' keys there is no evaluation at declaration time so it
% is not sensible to look for |\DelayEvaluation|. For this reason as
% well as for the fact that |\TP_multiselection_add:nnnnnn| above
% assumes that it deals
% with registers that can be accessed via |\toks_use:N| we have to use a
% different command to handle the |\MultiSelection| args but
% it is essentially doing the same.
%
%    \begin{macrocode}
\def_new:Npn \TP_templdecl_setup_n_key:N#1{
  \TP_templdecl_define_key:n{
    \tlist_if_head_eq_meaning:nNTF{##1..}\MultiSelection
    {
      \group_begin:
        \TP_templdecl_multiselection:nnnn ##1
      \group_end:
%    \end{macrocode}
% Extracting the correct item from the |\if_case:w| we are building
% requires a bit of care. Firstly we want to expand the appropirate
% number of times to get to the item but we also want to ensure we do
% not have any unwanted leftover |\fi:|s or other junk which is bound
% to cause errors later on. Therefore we start an |f| expansion (so we
% don't have to count |\exp_after:NN|s and then stop it again when we
% want to.
%    \begin{macrocode}
      \toks_put_right:Nx\l_TP_KV_assignments_toks {
        \exp_not:n{\tlp_set:Nf #1} { \toks_use:N \g_TP_multiselection_toks}
      }
    }
    {
      \def:Npn #1{##1}                      % setting it?
      \toks_put_right:Nn \l_TP_KV_assignments_toks
                          { \tlp_set:Nn #1{##1} }
    }
  }
}
%    \end{macrocode}
% \end{macro}
%
% \begin{macro}{\TP_templdecl_multiselection:nnnn}
%   Start the |\if_case:w|.  When the item is retrieved using an |f|
%   type expansion we better stop it at the right place once we have
%   emerged on the other side of the conditional.
%    \begin{macrocode}
\def_new:Npn \TP_templdecl_multiselection:nnnn #1#2#3#4{
  \toks_gset:Nn \g_TP_multiselection_toks {\if_case:w #2}
  \clist_map_inline:nn {#3}{ 
    \TP_multiselection_add_or_case:n {\exp_stop_f:##1} 
  }
  \toks_gput_right:Nn\g_TP_multiselection_toks {
    \else: \use_arg_i_after_fi:nw { \exp_stop_f: #4} \fi: 
  }
}
%    \end{macrocode}
% \end{macro}
%
%
%
%
%
%
%
%
% \begin{macro}{\DeclareInstance}
% |{type}{instname}{templatename}{keyval}|
%    \begin{macrocode}
\def_new:Npn \DeclareInstance { \DeclareCollectionInstance{} }
%    \end{macrocode}
% \end{macro}
%
% \begin{macro}{\DeclareCollectionInstance}
%    |{collection}{type}{instname}{templatename}{keyval}|
%    The fifth argument is picked up implicitly.
%    \begin{macrocode}
\def_long_new:Npn \DeclareCollectionInstance#1#2#3#4{
  \TP_declare_instance:cnn { <#1>#2/#3 }{ #2/#4 }
}
%    \end{macrocode}
% \end{macro}
%
% \begin{macro}{\UseCollection}
% |{type}{collection}|
%    \begin{macrocode}
\def_new:Npn \UseCollection#1#2{
  \tlp_set:cx { TP@<#1> }
    { {#2} \TP_get_arg_count:n{#1} }
}
%    \end{macrocode}
% \end{macro}
%
% \begin{macro}{\TP_let_instance:Nnn}
%    |\internalcommand{type}{instname}|
%
%    The way this macro is used, it must result in |\let:NwN| |<csname1>|
%    |<csname2>| after exactly two expansions as it is used this way
%    in |\TP_templdecl_eval_i_key_value:nnn|!
%    \begin{macrocode}
\def_new:Npn \TP_let_instance:Nnn#1#2#3{
  \let:Nc #1
  {
    \cs_if_free:cTF { \TP_get_csname_prefix:n{#2} #3 }
      { <>#2/ }
      { \TP_get_csname_prefix:n{#2} }
    #3
  }
}
%    \end{macrocode}
% \end{macro}
%
% \begin{macro}{\UseInstance}
% |{type}{instname}|
%    \begin{macrocode}
\def_new:Npn \UseInstance#1#2{
  \TP_let_instance:Nnn \l_tmpa_tlp {#1}{#2}
  \tlp_if_eq:NNTF \l_tmpa_tlp \scan_stop:
     \INSTANCEundefined
     \l_tmpa_tlp
}
%    \end{macrocode}
% \end{macro}
%
% \begin{macro}{\TP_templdecl_declare_tmp_instance:nnnn}
%    This macro is called when we have seen a |\UseTemplate|
%    declaration as part of an |i| key value. Therefore the
%    first argument will be dropped (it contains the token
%    |\UseTemplate|) the second and third will be combined to refer to
%    the template and the fourth argument will be implictly picked up
%    by |\TP_declare_instance:Nnn|.
%    \begin{macrocode}
\def_long_new:Npn \TP_templdecl_declare_tmp_instance:nnnn#1#2#3{%
  \TP_declare_instance:Nnn \g_tmpa_tlp {#2/#3} }
%    \end{macrocode}
% \end{macro}
%
% \begin{macro}{\ShowTemplate}
% Some extension to |\ShowTemplate| so that we also get to see the
% restrictions if any
%    \begin{macrocode}
\def_new:Npn \ShowTemplate#1#2{
   \io_put_term:x{*******~ Template:~ #1/#2~ *******}
   \io_put_term:x{*}
   \io_put_term:x{*~ Defaults:}
   \io_put_term:x{*}
   \io_put_term:x{\token_to_string:N\TPD>/#1/#2=
      \cs_meaning:c{TPD>/#1/#2}}
   \io_put_term:x{*}
   \io_put_term:x{*~ Restrictions:}
   \io_put_term:x{*}
   \io_put_term:x{\token_to_string:N\TPR>/#1/#2=
      \cs_meaning:c{TPR>/#1/#2}}
   \io_put_term:x{*}
   \io_put_term:x{*~ Body:}
   \io_put_term:x{*}
   \cs_show:c {TP>/#1/#2}}
%    \end{macrocode}
% \end{macro}
%
%
% \begin{macro}{\ShowCollectionInstance}
%    \begin{macrocode}
\def_new:Npn \ShowCollectionInstance#1#2#3{
   \io_put_term:x{*******~ Instance:~ <#1>#2/#3~ *******}
   \io_put_term:x{*}
   \cs_show:c {<#1>#2/#3}}
\def_new:Npn \ShowInstance{\ShowCollectionInstance{}}
%    \end{macrocode}
% \end{macro}
%
% \begin{macro}{\TP_templdecl_setup_fakeregister_key:NNn}
% |{setcomand}{privateregister}{key}{internalcode}|
%    \begin{macrocode}
\def_new:Npn \TP_templdecl_setup_fakeregister_key:NNn#1#2#3{
  \TP_templdecl_define_key:n{
    \tlist_if_head_eq_meaning:nNTF{##1..}\DelayEvaluation
    {
%    \end{macrocode}
%    In the v0.08 version of \texttt{template.dtx} a
%    |\DelayEvaluation| for a faked register would simply be equiv to
%    a |\def:Npn | (code is below commented out). The negative side effect
%    of this is that something like |=L| used with |\DelayEvaluation|
%    would not allow for calc syntax since it would end up as
%    |\def:Npn |\allowbreak|\foo|\allowbreak|{a+b}|. The code below changes
%    this to first assign to a scratch register (at runtime) and then
%    do an |\edef|. Could be coded differently to save space (at cost
%    of time)
%    \begin{macrocode}
%       \toks_put_right:Nn \l_TP_KV_assignments_toks {\def:Npn #3{##1}}
%      \toks_put_right:Nn \l_TP_KV_assignments_toks
%                         {#1#2{##1}\def:Npx #3{\toks_use:N#2}}
      \toks_set:No \l_tmpa_toks {\use_arg_ii:nn ##1}
      \toks_put_right:Nx \l_TP_KV_assignments_toks {
        \exp_not:n{#1#2}{\toks_use:N \l_tmpa_toks}
          \exp_not:n{ \def:Npx #3{\toks_use:N#2} }
        
      }
    }
%    \end{macrocode}
%
%
%    Otherwise same game for fake registers except that instead of
%    passing the register to |\TP_multiselection_add:nnnnnn| we pass a
%    temp fake one and
%    doing a def instead of using |\setlength| or |\setcounter|
%
%    and i haven't done the |\DelayEvaluation| bit for that case! as
%    i'm not sure what the best approach is for those
%    things\footnote{we might disallow it for that case in general ---
%    not a nice rule but an explainable one}
%
%    \begin{macrocode}
    {
      \tlist_if_head_eq_meaning:nNTF{##1..}\MultiSelection
      {
        \group_begin:
          \TP_multiselection_add:nnnnnn#1#2##1
        \group_end:
        \toks_put_right:Nx\l_TP_KV_assignments_toks
          {\exp_not:n{\def:Npn #3} {\toks_use:N\g_TP_multiselection_toks}}
      }
      {
        #1#2{##1}
        \toks_put_right:Nx\l_TP_KV_assignments_toks
                    {\exp_not:n{\def:Npn#3} {\toks_use:N#2}}
      }
    }
  }
}
%    \end{macrocode}
% \end{macro}
%
%
% \begin{macro}{\g_TP_multiselection_toks}
%    \begin{macrocode}
\toks_new:N \g_TP_multiselection_toks
%    \end{macrocode}
% \end{macro}
%
% \begin{macro}{\TP_multiselection_add:nnnnnn}
%  \marg{operation} \marg{register} |\MultiSelection|
%  \marg{selector} \marg{case-list} \marg{else-case}
%
%  This command builds up the |\if_case:w| code from the three
%  arguments of |\MultiSelection| and stores it in
%  |\g_TP_multiselection_toks|. This code is supposed to be run in a
%  group so a) we don't have to initialise |\g_TP_multiselection_toks|
%  and b) all changes to the used registers not affecting the
%  outside. 
%
%    \begin{macrocode}
\def_new:Npn \TP_multiselection_add:nnnnnn #1#2#3#4#5#6{
  \toks_gset:Nn \g_TP_multiselection_toks {\if_case:w #4}
  \clist_map_inline:nn {#5}{
    \tlist_if_head_eq_meaning:nNTF{##1..}\DelayEvaluation
    {
      \TP_multiselection_add_or_case:n {##1}
    }
    {
      #1#2{##1}
      \TP_multiselection_add_or_case:o { \toks_use:N #2 }
    }
  }
  \toks_gput_right:Nn \g_TP_multiselection_toks {
    \else: \use_arg_i_after_fi:nw{#6}\fi: 
  }
}
%    \end{macrocode}
%
% \end{macro}
%
% \begin{macro}{\TP_multiselection_add_or_case:o}
%   No need to worry about where |\or:| is allowed to be added since
%   all loops in \LaTeX3 process the item outside conditionals.
%    \begin{macrocode}
\def_new:Npn \TP_multiselection_add_or_case:n #1 {
  \toks_gput_right:Nn \g_TP_multiselection_toks { 
    \or: \use_arg_i_after_orelse:nw{#1} 
  }
}
\def_new:Npn \TP_multiselection_add_or_case:o  {
  \exp_args:No \TP_multiselection_add_or_case:n
}
%    \end{macrocode}
% \end{macro}
%
% Since i like to set things like |item-label-text| using this
% mechanism i need to handle the `n' key specially.
%
% Actually i could have probably extended
% |\TP_templdecl_setup_f_key:nnN| thus making
% this generally available to all f\meta{number} keys but was too lazy
% (or too stupid) to get it right first time so settled for the simple
% solution.
%
% So |\TP_templdecl_parse_KV:nnw| now calls
% |\TP_templdecl_setup_n_key:nN| for the `n' key. looks like
% this thus be fixed some time soon
%
%
%
%
%
%
%
% \begin{macro}{\IfExistsInstanceTF}
%   tests that there is a \emph{default} definition
%    taken from xinitials.dtx:
%    \begin{macrocode}
\def_new:Npn \IfExistsInstanceTF#1#2{
  \cs_if_exist:cTF{<>#1/#2}
}
%    \end{macrocode}
%
%     FMi: what happens if we are in collection FOO and there exists an
%     instance I for type T within this collection but there doesn't exist
%     an instance in the empty collection?
%
%     What would happen if \ldots\  --- not clear to me what the sematics
%     really should be. The code below is not better only different( and
%     slower).\footnote{fix semantics}
%    \begin{macrocode}
\def:Npn \IfExistsInstanceTF#1#2{
  \TP_let_instance:Nnn \l_tmpa_tlp {#1}{#2}
% next is not \tlp_if_eq:NNTF but ...FT so done manually
  \if_meaning:NN\l_tmpa_tlp\scan_stop:
    \exp_after:NN\use_arg_ii:nn
  \else:
    \exp_after:NN\use_arg_i:nn
  \fi:}
%    \end{macrocode}
% \end{macro}
%
% \begin{macro}{\DeclareRestrictedTemplate}
% Setting it up:
%
%| \DeclareRestrictedTemplate|\\
%     |{T-type} {new-T-name} {source-T-name} {keyvals}|
%
% This uses the same code as {T-type} {source-T-name} but adds
% settings from {keyvals}
%    \begin{macrocode}
\def_new:Npn \DeclareRestrictedTemplate#1#2#3#4{
  % CCC do we need a group here??
  \tlp_set_eq:cc { TPD>/#1/#2 } { TPD>/#1/#3 }
  \tlp_set_eq:cc {  TP>/#1/#2 } {  TP>/#1/#3 }

  \toks_set:Nd \l_TP_KV_assignments_toks
           {\cs:w TPR>/#1/#3\cs_end:}

% adds stuff to \l_TP_KV_assignments_toks
  \setkeys {\cs:w TPO>/#1/#3\cs_end:}{#4}

  \tlp_set:cx { TPR>/#1/#2 }
              { \toks_use:N \l_TP_KV_assignments_toks }
  \cs_if_free:cTF { TPO>/#1/#3 }
    { \tlp_set:cn {TPO>/#1/#2}{#1/#3}       }
    { \tlp_set_eq:cc {TPO>/#1/#2}{TPO>/#1/#3}  }
}
%    \end{macrocode}
% \end{macro}
%
%
% Internals:
%
% \begin{macro}{\TP_instdecl_generate_assignments:nn}
%    These could probably be inlined, even when they do something!
%
%    Assumption: setkeys fully expands its first argument.
%
%    \begin{macrocode}
\def_new:Npn \TP_instdecl_generate_assignments:nn#1#2 {
                               % Returns to \l_TP_KV_assignments_toks
                               % the restrictions
                               % stored in the TP-structure (at present
                               % in YAM) of the template #1

    \toks_set:Nd \l_TP_default_assignments_toks
                 {\cs:w TPD>/#1\cs_end:\scan_stop:\scan_stop:}

    \toks_set:Nd \l_TP_KV_assignments_toks
                 {\cs:w TPR>/#1\cs_end:}

    \setkeys { \cs:w TPO>/#1 \cs_end: }
             { #2 }            % adds stuff to \l_TP_KV_assignments_toks

% prepends stuff to \l_TP_KV_assignments_toks :
    \exp_after:NN\TP_instdecl_add_default_recurse:nn
        \toks_use:N\l_TP_default_assignments_toks

}
%    \end{macrocode}
% \end{macro}
%
%
% \begin{macro}{\TP_instdecl_add_default_recurse:nn}
%    [ 2001/06/10 Think about doing this properly with explicit plists! --- but
%    this means that one has to think about whether or not plists should be
%    implemented as token registers and not as tlps as they are now. ]
%    \begin{macrocode}
\def_new:Npn \TP_instdecl_add_default_recurse:nn#1#2{
  \token_if_eq_meaning:NNF #1\scan_stop:
  {
    \l_tmpa_toks{#2}
    \tlp_set:Nx \l_tmpa_tlp {
      {\toks_use:N \l_tmpa_toks \toks_use:N \l_TP_KV_assignments_toks}
    }
    \l_TP_KV_assignments_toks \l_tmpa_tlp
    \TP_instdecl_add_default_recurse:nn
  }
}
%    \end{macrocode}
% \end{macro}
%
%
%
% |TPD>/type/name| stores the default parameter assignments.
%
% |TPR>/type/name| stores the parameter assignments that have been
% made for a restricted template otherwise it is undefined (or |\scan_stop:|).
%
%
% |TPO>/type/name| stores the full name (i.e. as |type/name|)
% of the template a restricted
% template is coming from originally.
%
%
%
%
%  \begin{macro}{\TP_split_finite_skip_value:nnNN}
%  This macro is for use in error checking template values like
%  "text-float-sep" that can't contain infinite glue and needs the
%  shrink and/or stretch components. First argument is the skip
%  register (which is likely to be user input), second is a template
%  key name, and the last two are the \meta{dimen} registers that
%  stores the stretch and shrink components. Assignments are global.
%    \begin{macrocode}
\def_new:Npn \TP_split_finite_skip_value:nnNN #1#2{
  \skip_split_finite_else_action:nnNN {#1} {
    \PackageError{template}{Value~ for~ key~ #2~ contains~ `fil(ll)'}
    {Only~ finite~ minus~ or~ plus~ parts~ are~ allowed~ for~ this~ key.}
  }
}
%    \end{macrocode}
%  \end{macro}
%
%
%
%
%    \begin{macrocode}
%</package>
%    \end{macrocode}
%
% \Finale
%

% \iffalse
%%
%% (C) Copyright 2000,2001,2004 Frank Mittelbach
%% All rights reserved.
%%
%% Not for general distribution. In its present form it is not allowed
%% to put this package onto CD or an archive without consulting the
%% the authors.
%% 
%    \begin{macrocode}
\def\next#1: #2.dtx,v #3 #4 #5 #6 #7$#8{
%<*dtx>
  \ProvidesFile{#2.dtx}
%</dtx>
%<package>\ProvidesPackage{#2}
%<driver>\ProvidesFile{#2.drv}
  [#4 #3 #8 (#6)]}
\next$Id$
       {xtools}
%    \end{macrocode}
%
%<*driver>
 \documentclass{ltxdoc}
%
 \begin{document}
 \catcode`\_=11
 \catcode`\:=11
 \DocInput{xtools.dtx}
 \end{document}
%</driver>
%
% \fi
%
%
% \GetFileInfo{xtools.dtx}
%
% \title{The \textsf{xtools} package\thanks{This file
%         has version number \fileversion, last
%         revised \filedate.}}
% \author{FMi}
% \date{\filedate}
%
%  \maketitle
%
% \tableofcontents
%
%
%
% \section{Intro}
%
%
% The implementation below is more or less a straight adaption of code
% from l3expl, so it still needs a rewrite one day. It is based on
% modules implementing ``queues'', ``property lists'', and ``quarks''
% most of which have been published as experimental code with a
% slightly different surface syntax, i.e., as \texttt{l3seq.sty},
% \texttt{l3prop.sty}, and \texttt{l3quark.sty}.
%
%    \begin{macrocode}
\RequirePackage{ldcsetup}
\InternalSyntaxOn
%    \end{macrocode}
%
%    \begin{macrocode}
\long\def\@firstofthree#1#2#3{#1}
\long\def\@secondofthree#1#2#3{#2}
\long\def\@thirdofthree#1#2#3{#3}
%    \end{macrocode}
%
%
%
%
% \subsection{Commands for manipulating queues}
%
% \begin{macro}{\trace@queue}
% \begin{macro}{\trace@queue@internal}
% \begin{macro}{\tracingqueues}
%    \begin{macrocode}
%<*trace>
\def\trace@queue#1{\ifnum \tracingqueues > \z@
      \typeout{Queues:~ #1~ \on@line}\fi}
\def\trace@queue@internal#1{\ifnum\tracingqueues>\@ne
      \typeout{Queues:~ #1~ \on@line}\fi}
\newcount\tracingqueues
%</trace>
%    \end{macrocode}
% \end{macro}
% \end{macro}
% \end{macro}
%
% \begin{macro}{\queue_new:N}
% \begin{macro}{\queue_new:c}
%    \begin{macrocode}
\def\queue_new:N#1{\let#1\@empty}
\def\queue_new:c#1{\@namedef{#1}{}}
%    \end{macrocode}
% \end{macro}
% \end{macro}
%
%    \begin{macrocode}
\let \queue_clear:N \queue_new:N
\def \queue_gclear:N {\global\queue_clear:N}
%    \end{macrocode}
%
% \begin{macro}{\queue_gadd:Nn}
%    \begin{macrocode}
\def\queue_gadd:Nn#1#2{\expandafter\gdef\expandafter#1\expandafter
    {#1\queue_elt#2\queue_eelt}
%<*trace>
  \trace@queue@internal{add~ `#2'~ to~ queue~ \string#1}
%</trace>
}
%    \end{macrocode}
% \end{macro}
%
%
% \begin{macro}{\queue_gadd:cn}
% \begin{macro}{\queue_gadd:No}
% \begin{macro}{\queue_gadd:co}
%    \begin{macrocode}
\def\queue_gadd:cn#1{\expandafter\queue_gadd:Nn\csname#1\endcsname}
\def\queue_gadd:No#1#2{\expandafter\queue_gadd:Nn
                       \expandafter#1\expandafter{#2}}
\def\queue_gadd:co#1{\expandafter\queue_gadd:No\csname#1\endcsname}
%    \end{macrocode}
% \end{macro}
% \end{macro}
% \end{macro}
%
% \begin{macro}{\queue_top:NN}
%    \begin{macrocode}
\def\queue_top:NN#1#2{
  \queue_empty_err:N#1
  \expandafter\queue_top_split:w#1\q_stop{\def#2}
%<*trace>
  \trace@queue@internal{top~ of~ queue~\string#1:~`\tlp_to_str:N#2'}
%</trace>
}
% \end{macro}
%
% \begin{macro}{\queue_top_split:w}
\def\queue_top_split:w\queue_elt#1\queue_eelt#2\q_stop#3{#3{#1}}
%    \end{macrocode}
% \end{macro}
%
% \begin{macro}{\queue_top:cN}
%    \begin{macrocode}
\def\queue_top:cN#1{\expandafter\queue_top:NN\csname#1\endcsname}
%    \end{macrocode}
% \end{macro}
%
%
%
% \begin{macro}{\queue_pop_aux:nnNN}
% \begin{macro}{\queue_pop_aux:w}
%    \begin{macrocode}
\def \queue_pop_aux:nnNN #1#2#3{
  \queue_empty_err:N #3
  \expandafter\queue_pop_aux:w #3\q_stop #1#2#3}
\def \queue_pop_aux:w \queue_elt#1\queue_eelt
                #2\q_stop #3#4#5#6{#3#5{#2}#4#6{#1}}
%    \end{macrocode}
% \end{macro}
% \end{macro}
%
%
% \begin{macro}{\queue_gpop:NN}
%    \begin{macrocode}
\def \queue_gpop:NN #1#2{\queue_pop_aux:nnNN \gdef \gdef #1 #2
%<*trace>
  \trace@queue@internal{pop~ of~ queue~\string#1:~`\tlp_to_str:N #2'}
%</trace>
}
%    \end{macrocode}
% \end{macro}
%
%
% \begin{macro}{\queue_gpop:cN}
%    \begin{macrocode}
\def \queue_gpop:cN #1{\expandafter\queue_gpop:NN\csname#1\endcsname}
%    \end{macrocode}
% \end{macro}
%
%
% \begin{macro}{\queue_empty_err:N}
%    \begin{macrocode}
\def\queue_empty_err:N #1{\ifx#1\@empty \ERROR \fi}
%    \end{macrocode}
% \end{macro}
%
%
%
% \begin{macro}{\queue_if_in:NnTF}
% \begin{macro}{\queue_if_in:NnF}
%    |\queue_if_in:NnTF| \meta{queue}\meta{item} \meta{true~case}
%    \meta{false~case} will check whether \meta{item} is in
%    \meta{queue} and then either execute the \meta{true~case} or the
%    \meta{false~case}.
%    \begin{macrocode}
\def \queue_if_in:NnTF #1#2{
  \def \tmp:w
      ##1\queue_elt #2\queue_eelt ##2##3\q_stop{
%    \end{macrocode}
%    Note that |##2| contains exactly one token which we can compare
%    with |\q_no_value|.
%    \begin{macrocode}
        \ifx\q_no_value##2
          \expandafter\@secondoftwo
        \else
          \expandafter\@firstoftwo
        \fi
      }
  \expandafter
  \tmp:w #1\queue_elt #2\queue_eelt \q_no_value \q_stop}
%    \end{macrocode}
%    
%    \begin{macrocode}
\def \queue_if_in:NnF #1#2{
  \def \tmp:w
      ##1\queue_elt #2\queue_eelt ##2##3\q_stop{
        \ifx\q_no_value##2
          \expandafter\@firstofone
        \else
          \expandafter\@gobble
        \fi
      }
  \expandafter
  \tmp:w #1\queue_elt #2\queue_eelt \q_no_value \q_stop}
%    \end{macrocode}
% \end{macro}
% \end{macro}
%
%
%
%    \begin{macrocode}
\def \queue_if_in:NoTF #1#2{\expandafter \queue_if_in:NnTF
                            \expandafter #1 \expandafter {#2}}
%    \end{macrocode}
%    
%    \begin{macrocode}
\def \queue_map:NN #1#2{
  \let \queue_map_funct:n #2
  \expandafter\queue_map_aux:w #1\queue_elt\q_stop\queue_eelt}
\def \queue_map_aux:w \queue_elt#1\queue_eelt{
%    \end{macrocode}
%    |#1| can compare arbitrary tokens so for testing against
%    |\q_stop| we need to wrap it into a macro.
%    \begin{macrocode}
  \def\tmp:w{#1}
  \ifx \tmp:w\q_stop \else
    \queue_map_funct:n {#1}
    \expandafter\queue_map_aux:w
  \fi}
%    \end{macrocode}
%    Inline version where the code is specified as second argument
%    with |##1| marking the mapping argument within the code.
%    \begin{macrocode}
\def \queue_map:Nn #1#2{
  \def \queue_map_funct:n ##1{#2}
  \expandafter\queue_map_aux:w #1\queue_elt\q_stop\queue_eelt}
%    \end{macrocode}
%    
%    \begin{macrocode}
\def \queue_empty:NTF #1 {
  \ifx#1\@empty
    \expandafter\@firstoftwo
  \else
    \expandafter\@secondoftwo
  \fi
}
%    \end{macrocode}
%
%
%
%
%
%
%
%
% \subsection{Commands for manipulating property lists}
%
%
% \begin{macro}{\prop_new:N}    
% \begin{macro}{\prop_new:c}    
%    \begin{macrocode}
\let \prop_new:N \queue_new:N
\let \prop_new:c \queue_new:c
%    \end{macrocode}
% \end{macro}
% \end{macro}
%
%    \begin{macrocode}
\let \prop_clear:N \queue_clear:N
\let \prop_gclear:N \queue_gclear:N
%    \end{macrocode}
%
% \begin{macro}{\prop_put:NNn}    
% \begin{macro}{\prop_gput:NNn}    
% \begin{macro}{\prop_gput:NNo}    
% \begin{macro}{\prop_gput:Nco}    
% \begin{macro}{\prop_gput:cNn}    
% \begin{macro}{\prop_gput:cNo}    
% \begin{macro}{\prop_gput:cco}    
% \begin{macro}{\prop_gput:ccn}    
%    \begin{macrocode}
\long\def \prop_put:NNn #1#2{\prop_split_aux:NNn
                             #1#2{\prop_put_aux:w {\def #1}#2}}
\long\def \prop_gput:NNn #1#2{\prop_split_aux:NNn
                                #1#2{\prop_put_aux:w {\gdef #1}#2}}
%  missing commands for galley stuff...
\def \prop_gput:NNo #1#2#3{
  \expandafter\prop_gput:NNn \expandafter #1 \expandafter 
                          #2 \expandafter { #3 } }
\def \prop_gput:Nco #1#2#3{
  \expandafter\prop_gput:NNn \expandafter #1 
                          \csname #2 \expandafter \endcsname
                          \expandafter { #3 } }
\def \prop_gput:cNn #1{
  \expandafter\prop_gput:NNn \csname #1\endcsname
}
\def \prop_gput:cNo #1{
  \expandafter\prop_gput:NNo \csname #1\endcsname
}
\def \prop_gput:cco #1#2#3{
  \expandafter\prop_gput:NNn \csname #1\expandafter\endcsname
                             \csname #2\expandafter\endcsname
                             \expandafter { #3 } }
\def \prop_gput:ccn #1#2{
  \expandafter\prop_gput:NNn \csname #1\expandafter\endcsname
                             \csname #2\endcsname
                              }
%    \end{macrocode}
% \end{macro}
% \end{macro}
% \end{macro}
% \end{macro}
% \end{macro}
% \end{macro}
% \end{macro}
% \end{macro}
%
%
% \begin{macro}{\prop_put_aux:w}    
%    \begin{macrocode}
\long\def \prop_put_aux:w #1#2#3#4#5#6{
  \quark_if_no_value:nTF {#4}
    {#1{#2{#6}#3}}
    {\def\tmp:w ##1#2\q_no_value {#1{#3#2{#6}##1}}
     \tmp:w #5}}
%    \end{macrocode}
% \end{macro}
%    
% \begin{macro}{\prop_split_aux:NNn}    
%    \begin{macrocode}
\long\def \prop_split_aux:NNn #1#2#3{
  \def\tmp:w ##1#2##2##3\q_stop {#3{##1}{##2}{##3}}
%                                       ^   ^ needed!
  \expandafter\tmp:w #1#2\q_no_value \q_stop}
%    \end{macrocode}
% \end{macro}
%    
%
% \begin{macro}{\prop_get:NNN}    
% \begin{macro}{\prop_get:NcN}    
% \begin{macro}{\prop_get:cNN}    
% \begin{macro}{\prop_gget:NNN}    
% \begin{macro}{\prop_gget:NcN}    
% \begin{macro}{\prop_gget:cNN}    
%    \begin{macrocode}
\def \prop_get:NNN #1#2{\prop_split_aux:NNn
                                    #1#2\prop_get_aux:w}
\def \prop_get:NcN #1#2 {
 \expandafter \prop_get:NNN \expandafter #1
 \csname #2 \endcsname 
}
\def \prop_get:cNN #1{
  \expandafter\prop_get:NNN \csname #1\endcsname
}
%    \end{macrocode}
%    
%    \begin{macrocode}
\def \prop_gget:NNN #1#2{\prop_split_aux:NNn
                                    #1#2\prop_gget_aux:w}
\def \prop_gget:NcN #1#2 {
 \expandafter \prop_gget:NNN \expandafter #1
 \csname #2 \endcsname 
}
\def \prop_gget:cNN #1{
  \expandafter\prop_gget:NNN \csname #1\endcsname
}
%    \end{macrocode}
% \end{macro}
% \end{macro}
% \end{macro}
% \end{macro}
% \end{macro}
% \end{macro}
%    
%
% \begin{macro}{\prop_get_aux:w}    
%    \begin{macrocode}
\long\def \prop_get_aux:w #1#2#3#4{\def#4{#2}}
\long\def \prop_gget_aux:w #1#2#3#4{\gdef#4{#2}}
%    \end{macrocode}
% \end{macro}
%    
%
% \begin{macro}{\prop_map_funct:Nn}    
%    \begin{macrocode}
\let \prop_map_funct:Nn \@gobbletwo
%    \end{macrocode}
% \end{macro}
%    
%
% \begin{macro}{\prop_map:NN}    
%    \begin{macrocode}
\def \prop_map:NN #1#2{
  \let \prop_map_funct:Nn #2
  \expandafter\prop_map_aux:w #1\q_stop \q_stop}
%    \end{macrocode}
% \end{macro}
%    
%
% \begin{macro}{\prop_map_aux:w}    
%    \begin{macrocode}
\def \prop_map_aux:w #1#2{
  \ifx #1\q_stop \else
    \prop_map_funct:Nn #1{#2}
    \expandafter\prop_map_aux:w
  \fi}
%    \end{macrocode}
% \end{macro}
%
% \begin{macro}{\prop_map:cN}    
% \begin{macro}{\prop_map:cc}    
%    \begin{macrocode}
\def \prop_map:cN #1{
  \expandafter \prop_map:NN \csname #1\endcsname }
\def \prop_map:cc #1#2{
  \expandafter \prop_map:NN \csname #1 \expandafter\endcsname 
   \csname #2\endcsname }
%    \end{macrocode}
% \end{macro}
% \end{macro}
%
%
%
%
% \subsection{Stuff unsorted}
%

% \begin{macro}{\tlp_gset:cn}
%    \begin{macrocode}
\def \tlp_gset:cn #1{\global\@namedef{#1}}
\def \tlp_gput_right:cn { \exp_args:Nc  \tlp_gput_right:Nn }
\def \tlp_gput_right:co { \exp_args:Nco \tlp_gput_right:Nn }
%    \end{macrocode}
% \end{macro}
%
% \begin{macro}{\_for}
%    \begin{macrocode}
\long\def\_for #1 = #2\do#3{%
  \expandafter\def\expandafter\@fortmp\expandafter{#2}%
  \ifx\@fortmp\@empty \else
    \expandafter\@forloop#2,\@nil,\@nil\@@#1{#3}\fi}
%    \end{macrocode}
% \end{macro}
%

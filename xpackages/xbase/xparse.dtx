% \iffalse
%% File: xparse.dtx (C) Copyright 1999 Frank Mittelbach, Chris Rowley, 
%%                      David Carlisle
%%                  (C) Copyright 2004-2008 Frank Mittelbach, 
%%                      LaTeX3 Project
%%                  (C) Copyright 2009-2010 LaTeX3 Project
%%
%% It may be distributed and/or modified under the conditions of the
%% LaTeX Project Public License (LPPL), either version 1.3c of this
%% license or (at your option) any later version.  The latest version
%% of this license is in the file
%%
%%    http://www.latex-project.org/lppl.txt
%%
%% This file is part of the ``xbase bundle'' (The Work in LPPL)
%% and all files in that bundle must be distributed together.
%%
%% The released version of this bundle is available from CTAN.
%%
%% -----------------------------------------------------------------------
%%
%% The development version of the bundle can be found at
%%
%%    http://www.latex-project.org/svnroot/experimental/trunk/
%%
%% for those people who are interested.
%%
%%%%%%%%%%%
%% NOTE: %%
%%%%%%%%%%%
%%
%%   Snapshots taken from the repository represent work in progress and may
%%   not work or may contain conflicting material!  We therefore ask
%%   people _not_ to put them into distributions, archives, etc. without
%%   prior consultation with the LaTeX Project Team.
%%
%% -----------------------------------------------------------------------
%% 
%
%<*driver|package>
\RequirePackage{l3names}
%</driver|package>
%\fi
\GetIdInfo$Id$
  {Generic document command parser}
%\iffalse
%<*driver>
%\fi
\ProvidesFile{\filename.\filenameext}
  [\filedate\space v\fileversion\space\filedescription]
%\iffalse
\documentclass{l3doc}
\usepackage{amstext}
\begin{document}
  \DocInput{xparse.dtx}
\end{document}
%</driver>
% \fi
%
% \title{The \textsf{xparse} package\thanks{This file
%         has version number \fileversion, last
%         revised \filedate.}\\
% Generic document command parser}
% \author{\Team}
% \date{\filedate}
% \maketitle
%
%\begin{documentation}
%
%\section{Creating document commands}
%
% The \pkg{xparse} package provides a high-level interface for
% producing document-level commands. In that way, it is intended as
% a replacement for the \LaTeXe\ \cs{newcommand} macro. However,
% \pkg{xparse} works so that the interface to a function (optional
% arguments, stars and mandatory arguments, for example) is separate
% from the internal implementation. \pkg{xparse} provides a normalised
% input for the internal form of a function, independent of the 
% document-level argument arrangement.
% 
% At present, the functions in \pkg{xparse} which are regarded as
% `stable' are:
%\begin{itemize}
%  \item \cs{DeclareDocumentCommand}
%  \item \cs{NewDocumentCommand}
%  \item \cs{RenewDocumentCommand}
%  \item \cs{ProvideDocumentCommand}
%  \item \cs{DeclareDocumentEnvironment}
%  \item \cs{NewDocumentEnvironment}
%  \item \cs{RenewDocumentEnvironment}
%  \item \cs{ProvideDocumentEnvironment}
%  \item \cs{IfNoValue(TF)} (the need for \cs{IfValue(TF)} is currently
%    an item of active discussion)
%  \item \cs{IfBoolean(TF)}
%\end{itemize}
% with the other functions currently regarded as `experimental'. Please
% try all of the commands provided here, but be aware that the 
% experimental ones may change or disappear.
% 
%\subsection{Specifying arguments}
%
% Before introducing the functions used to create document commands,
% the method for specifying arguments with \pkg{xparse} will be 
% illustrated. In order to allow each argument to be defined 
% independently, \pkg{xparse} does not simply need to know the 
% number of arguments for a function, but also the nature of each
% one. This is done by constructing an \emph{argument specification},
% which defines the number of arguments, the type of each argument
% and any additional information needed for \pkg{xparse} to read the
% user input and properly pass it through to internal functions. 
% 
% The basic form of the argument specifier is a list of letters, where
% each letter defines a type of argument. As will be described below,
% some of the types need additional information, such as default values.
% The argument types can be divided into two, those which define 
% arguments that are mandatory (potentially raising an error if not 
% found) and those which define optional arguments. The mandatory types
% are:
%\begin{itemize}[font=\ttfamily]
%  \item[m] A standard mandatory argument, which can either be a single
%    token alone or multiple tokens surrounded by curly braces.
%    Regardless of the input, the argument will be passed to the
%    internal code surrounded by a brace pair. This is the \pkg{xparse}
%    type specifier for a normal \TeX\ argument.
%  \item[l] An argument which reads everything up to the first
%    open group token: in standard \LaTeX\ this is a left brace. 
%  \item[u] Reads an argument `until' \meta{tokens} are encountered,
%    where the desired \meta{tokens} are given as an argument to the
%    specifier: \texttt{u}\marg{tokens}.
%\end{itemize}
% The types which define optional arguments are:
%\begin{itemize}[font=\ttfamily]
%  \item[o] A standard \LaTeX\ optional argument, surrounded with square 
%    brackets, which will supply
%    the special \cs{NoValue} token if not given (as described later).
%  \item[d] An optional argument which is delimited by \meta{token1}
%    and \meta{token2}, which are given as arguments: 
%    \texttt{d}\meta{token1}\meta{token2}. As with \texttt{o}, if no
%    value is given the special token \cs{NoValue} is returned.
%  \item[O] As for \texttt{o}, but returns \meta{default} if no
%    value is given.  Should be given as \texttt{O}\marg{default}.
%  \item[D] As for \texttt{d}, but returns \meta{default} if no
%    value is given: \texttt{D}\meta{token1}\meta{token2}\marg{default}.
%    Internally, the \texttt{o}, \texttt{d} and \texttt{O} types are
%    short-cuts to an appropriated-constructed \texttt{D} type argument.
%  \item[s] An optional star, which will result in a value 
%    \cs{BooleanTrue} if a star is present and \cs{BooleanFalse}
%    otherwise (as described later).
%  \item[t] An optional \meta{token}, which will result in a value 
%    \cs{BooleanTrue} if \meta{token} is present and \cs{BooleanFalse}
%    otherwise. Given as \texttt{t}\meta{token}.
%  \item[g] An optional argument given inside a pair of \TeX\ group
%    tokens (in standard \LaTeX, |{| \ldots |}|), which returns 
%    \cs{NoValue} if not present.
%  \item[G] As for \texttt{g} but returns \meta{default} if no value
%    is given: \texttt{G}\marg{default}.
%\end{itemize}
%
% Using these specifiers, it is possible to create complex input syntax
% very easily. For example, given the argument definition 
% `|s o o m O{default}|', the input `|*[Foo]{Bar}|' would be parsed as:
%\begin{itemize}[nolistsep]
%  \item |#1| = |\BooleanTrue|
%  \item |#2| = |{Foo}|
%  \item |#3| = |\NoValue|
%  \item |#4| = |{Bar}|
%  \item |#5| = |{default}|
%\end{itemize}
% whereas `|[One][Two]{}[Three]|' would be parsed as:
%\begin{itemize}[nolistsep]
%  \item |#1| = |\BooleanFalse|
%  \item |#2| = |{One}|
%  \item |#3| = |{Two}|
%  \item |#4| = |{}|
%  \item |#5| = |{Three}|
%\end{itemize}
% Note that after parsing the input there will be always exactly the 
% same number of \meta{balanced text} arguments as the number of letters
% in the argument specifier. The \cs{BooleanTrue} and \cs{BooleanFalse}
% tokens are passed without braces; all other arguments are passed as
% brace groups.
% 
% Two more tokens have a special meaning when creating an argument 
% specifier. First, \texttt{+} is used to make an argument long (to 
% accept paragraph tokens). In contrast to \LaTeXe's \cs{newcommand}, 
% this applies on an argument-by-argument basis. So modifying the 
% example to `|s o o +m O{default}|' means that the mandatory argument 
% is now \cs{long}, whereas the optional arguments are not.
% 
% Secondly, the token \texttt{>} is used to declare so-called
% `argument processors', which can be used to modify the contents of an
% argument before it is passed to the macro definition. The use of 
% argument processors is a somewhat advanced topic, (or at least a less 
% commonly used feature) and is covered in Section~\ref{sec:processors}.
% 
%\subsection{Spacing and optional arguments}
%
% TeX will find the first argument after a function name irrespective
% of any intervening spaces. This is true for both mandatory and
% optional arguments. So "\foo[arg]" and \verb*"\foo   [arg]" are
% equivalent. Spaces are also ignored when collecting arguments up
% to the last mandatory argument to be collected (as it must exist).
% So after
%\begin{verbatim}
%  \DeclareDocumentCommand \foo { m o m } { ... }
%\end{verbatim}
% the user input "\foo{arg1}[arg2]{arg3}" and 
% \verb*"\foo{arg1}  [arg2]   {arg3}" will both be parsed in the same
% way. However, spaces are \emph{not} ignored when parsing optional
% arguments after the last mandatory argument. Thus with
%\begin{verbatim}
%  \DeclareDocumentCommand \foo { m o } { ... }
%\end{verbatim}
% "\foo{arg1}[arg2]" will find an optional argument but
% \verb*"\foo{arg1} [arg2]" will not. This is so that trailing optional
% arguments are not picked up `by accident' in input.
% 
%\subsection{Declaring commands and environments}
% 
% With the concept of an argument specifier defined, it is now 
% possible to describe the methods available for creating both
% functions and environments using \pkg{xparse}.
% 
% The interface-building commands are the preferred method for 
% creating document-level functions in \LaTeX3. All of the functions 
% generated in this way are naturally robust (using the \eTeX\ 
% \cs{protected} mechanism).
% 
%\begin{function}{
%  \DeclareDocumentCommand|
%  \NewDocumentCommand|
%  \RenewDocumentCommand|
%  \ProvideDocumentCommand
%}
%  \begin{syntax}
%    "\DeclareDocumentCommand" <function> \Arg{arg spec} \Arg{code}
%  \end{syntax}
%  This family of commands are used to create a document-level
%  <function>. The argument specification for the function is
%  given by <arg spec>, and the function will execute <code>.
%  
%  As an example:
%  \begin{verbatim}
%    \DeclareDocumentCommand \chapter { s o m } {
%      \IfBooleanTF {#1} {
%        \typesetnormalchapter {#2} {#3} 
%      }{
%        \typesetstarchapter {#3}
%      }
%    }
%  \end{verbatim}
%  would be a way to define a \cs{chapter} command which would
%  essentially behave like the current \LaTeXe\ command (except that it
%  would accept an optional argument even when a \texttt{*} was parsed).
%  The \cs{typesetnormalchapter} could test its first argument for being
%  \cs{NoValue} to see if an optional argument was present.
% 
%  The difference between the \cs{Declare\ldots}, \cs{New\ldots}
%  \cs{Renew\ldots} and \cs{Provide\ldots} versions is the behaviour
%  if <function> is already defined. 
%  \begin{itemize}
%    \item \cs{DeclareDocumentCommand} will always create the new 
%      definition, irrespective of any existing <function> with the 
%      same name. 
%   \item \cs{NewDocumentCommand} will issue an error if <function>
%     has already been defined.
%   \item \cs{RenewDocumentCommand} will issue an error if <function>
%     has not previously been defined.
%   \item \cs{ProvideDocumentCommand} creates a new definition for
%     <function> only if one has not already been given.
%  \end{itemize}  
%  
%  \begin{texnote}
%     Unlike \LaTeXe's \cs{newcommand} and relatives, the 
%     \cs{DeclareDocumentCommand} function do not prevent creation of 
%     functions with names starting \cs{end\ldots}.
%  \end{texnote}
%\end{function} 
%
%\begin{function}{%
%  \DeclareDocumentEnvironment|
%  \NewDocumentEnvironment|
%  \RenewDocumentEnvironment|
%  \ProvideDocumentEnvironment
%}
%  \begin{syntax}
%    "\DeclareDocumentEnvironment" \Arg{environment} \Arg{arg spec}
%    ~~~~\Arg{start code} \Arg{end code}
%  \end{syntax}
%  These commands work in the same way as \cs{DeclareDocumentCommand},
%  etc., but create environments (\cs{begin}|{|<function>|}| \ldots
%  \cs{end}|{|<function>|}|). Both the <start code> and <end code>
%  may access the arguments as defined by <arg spec>.
%  
%  \begin{texnote}
%    When loaded as part of a \LaTeX3 format, these, these commands do
%    not create a pair of macros \cs{<environment>} and 
%    \cs{end<environment>}. Thus \LaTeX3 environments have to be 
%    accessed using the \cs{begin} \ldots \cs{end} mechanism. When 
%    \pkg{xparse} is loaded as a \LaTeXe\ package, \cs{<environment>}
%    and \cs{end<environment>} are defined, as this is necessary to 
%    allow the new environment to work!
%  \end{texnote}
%\end{function}
%
%\subsection{Testing special values}
%
% Optional arguments created using \pkg{xparse} make use of dedicated
% variables to return information about the nature of the argument
% received.
%
%\begin{variable}{\NoValue} 
%  \cs{NoValue} is a special marker returned by \pkg{xparse} if no 
%  value is given for an optional argument. If typeset (which should
%  not happen),  it will print the value \texttt{-NoValue-}.
%\end{variable}
%
%\begin{function}{\IfNoValue / (TF) (EXP)}
%  \begin{syntax}
%    "\IfNoValueTF" \Arg{argument} \Arg{true code} \Arg{false code}
%  \end{syntax}
%  The \cs{IfNoValue} tests are used to check if <argument> (|#1|,
%  |#2|, etc.) is the special \cs{NoValue} token. For example
%  \begin{verbatim}
%    \DeclareDocumentCommand \foo { o m } {
%      \IfNoValueTF {#1} {
%        \DoSomethingJustWithMandatoryArgument {#2}
%      }{
%        \DoSomethingWithBothArguments {#1} {#2}
%      }
%    }
%  \end{verbatim}
%  will use a different internal function if the optional argument
%  is given than if it is not present.
%  
%  As the \cs{IfNoValue(TF)} tests are expandable, it is possible to
%  test these values later, for example at the point of typesetting or
%  in an expansion context.
%\end{function}
%
%\begin{function}{\IfValue / (TF) (EXP)}
%  \begin{syntax}
%    "\IfValueTF" \Arg{argument} \Arg{true code} \Arg{false code}
%  \end{syntax}
%  The reverse form of the \cs{IfNoValue(TF)} tests are also available
%  as \cs{IfValue(TF)}. The context will determine which logical
%  form makes the most sense for a given code scenario.
%\end{function}  
%
%\begin{variable}{
%  \BooleanFalse|
%  \BooleanTrue
%} 
%  The \texttt{true} and \texttt{false} flags set when searching for
%  an optional token (using \texttt{s} or \texttt{t}<token>) have
%  names which are accessible outside of code blocks. 
%\end{variable}
%
%\begin{function}{\IfBoolean / (TF) (EXP)}
%  \begin{syntax}
%    "\IfBooleanTF" <argument> \Arg{true code} \Arg{false code}
%  \end{syntax}
%  Used to test if <argument> (|#1|, |#2|, etc.) is \cs{BooleanTrue}
%  or \cs{BooleanFalse}. For example
%  \begin{verbatim}
%    \DeclareDocumentCommand \foo { s m } {
%      \IfBooleanTF #1 {
%        \DoSomethingWithStar {#2} 
%      }{
%        \DoSomethingWithoutStar {#2}
%      }  
%    }
%  \end{verbatim}
%  checks for a star as the first argument, then chooses the action to
%  take based on this information.
%\end{function}
%
%\subsection{Argument processors}
%\label{sec:processors}
%
% \pkg{xparse} introduces the idea of an argument processor, which is
% applied to an argument \emph{after} it has been grabbed by the 
% underlying system but before it is passed to \meta{code}. An argument 
% processor can therefore be used to regularise input at an early stage, 
% allowing the internal functions to be completely independent of input 
% form. Processors are applied to user input and to default values for
% optional arguments, but \emph{not} to the special \cs{NoValue} marker.
% 
% Each argument processor is specified by the syntax 
% \texttt{>}\marg{processor} in the argument specification. Processors
% are applied from right to left, so that 
%\begin{verbatim}
%  >{\ProcessorB} >{\ProcessorA} m
%\end{verbatim}
% would apply \cs{ProcessorA} 
% followed by \cs{ProcessorB} to the tokens grabbed by the \texttt{m} 
% argument. 
%
%\begin{variable}{\ProcessedArgument}
%  \pkg{xparse} defines a very small set of processor functions. In the
%  main, it is anticipated that code writers will want to create their
%  own processors. These need to accept one argument, which is the
%  tokens as grabbed (or as returned by a previous processor function).
%  Processor functions should return the processed argument as the
%  variable \cs{ProcessedArgument}. 
%\end{variable}
% 
%\begin{function}{\xparse_process_to_str:n}
%  \begin{syntax}
%    "\xparse_process_to_str:n" \Arg{grabbed argument}
%  \end{syntax}
%  The \cs{xparse_process_to_str:n} processor applies the \LaTeX3
%  \cs{tl_to_str:n} function to the <grabbed argument>. For example
%  \begin{verbatim}
%    \DeclareDocumentCommand \foo { >{\xparse_arg_to_str:n} m } {
%      #1 % Which is now detokenized
%    }
%  \end{verbatim}
%\end{function}
%
%\begin{function}{ \ReverseBoolean }
%  \begin{syntax}
%    \cs{ReverseBoolean}
%  \end{syntax}
%  This processor reverses the logic of \cs{BooleanTrue} and 
%  \cs{BooleanFalse}, so that the the example from earlier would become 
%  \begin{verbatim}
%    \DeclareDocumentCommand \foo { > { \ReverseBoolean } s m } {
%      \IfBooleanTF #1 
%        { \DoSomethingWithoutStar {#2} }
%        { \DoSomethingWithStar {#2} }
%    }
%  \end{verbatim}
%\end{function}
%
%\begin{function}{ \SplitArgument }
%  \begin{syntax}
%    \cs{SplitArgument} \Arg{number} \Arg{token}
%  \end{syntax}
%  This processor splits the argument given at each occurrence of the
%  \meta{token} up to a maximum of \meta{number} tokens (thus
%  dividing the input into \( \text{\meta{number}} + 1 \) parts).
%  An error is given if too many \meta{tokens} are present in the
%  input. The processed input is places inside 
%  \( \text{\meta{number}} + 1 \) sets of braces for further use.
%  If there are less than \Arg{number} of \Arg{tokens} in the argument
%  then empty brace groups are added at the end of the processed
%  argument.
%  \begin{verbatim}
%    \DeclareDocumentCommand \foo 
%      { > { \SplitArgument { 2 } { ; } } m } 
%      { \InternalFunctionOfThreeArguments #1 }
%  \end{verbatim}
%  Any category code \( 13 \) (active) \meta{tokens} will be replaced
%  before the split takes place.
%\end{function}
%
%\begin{function}{ \SplitList }
%  \begin{syntax}
%    \cs{SplitList} \Arg{token}
%  \end{syntax}
%  This processor splits the argument given at each occurrence of the
%  \meta{token} where the number of items is not fixed. Each item is
%  then wrapped in braces within "#1". The result is that the
%  processed argument can be further processed using a mapping function.
%  \begin{verbatim}
%    \DeclareDocumentCommand \foo 
%      { > { \SplitList { ; } } m } 
%      { \MappingFunction #1 }
%  \end{verbatim}
%  Any category code \( 13 \) (active) \meta{tokens} will be replaced
%  before the split takes place.
%\end{function}
%
%\subsection{Separating interface and implementation}
%
% One \emph{experimental} idea implemented in \pkg{xparse} is to 
% separate out document command interfaces (the argument specification)
% from the implementation (code).  This is carried out using a 
% pair of functions, \cs{DeclareDocumentCommandInterface} and
% \cs{DeclareDocumentCommandImplementation}
%
%\begin{function}{\DeclareDocumentCommandInterface}
%  \begin{syntax}
%    "\DeclareDocumentCommandInterface" <function>
%    ~~~~\Arg{implementation} \Arg{arg spec}
%  \end{syntax}
%  This declares a <function>, which will take arguments as detailed
%  in the <arg spec>. When executed, the <function> will look for 
%  code stored as an <implementation>. 
%\end{function}  
%
%\begin{function}{\DeclareDocumentCommandImplementation}
%  \begin{syntax}
%    "\DeclareDocumentCommandImplementation" 
%    ~~~~\Arg{implementation} <args> \Arg{code}
%  \end{syntax}
%  Declares the <implementation> for a function to accept <args> 
%  arguments and expand to <code>. An implementation must take the
%  same number of arguments as a linked interface, although this is not
%  enforced by the code.
%\end{function}  
% 
%\subsection{Fully-expandable document commands}
%
% There are \emph{very rare} occasion when it may be useful to create
% functions using a fully-expandable argument grabber. To support this, 
% \pkg{xparse} can create expandable functions as well as the usual 
% robust ones. This imposes a number of restrictions on the nature of 
% the arguments accepted by a function, and the code it implements. 
% This facility should only be used when \emph{absolutely necessary};
% if you do not understand when this might be, \emph{do not use these
% functions}!
% 
%\begin{function}{\DeclareExpandableDocumentCommand}
%  \begin{syntax}
%    "\DeclareExpandableDocumentCommand"
%    ~~~~<function> \Arg{arg spec} \Arg{code}
%  \end{syntax}
%  This command is used to create a document-level <function>, 
%  which will grab its arguments in a fully-expandable manner. The 
%  argument specification for the function is given by <arg spec>, 
%  and the function will execute <code>. In  general, <code> will 
%  also be fully expandable, although it is possible that this will
%  not be the case (for example, a function for use in a table might
%  expand so that \cs{omit} is the first non-expandable token).
%  
%  Parsing arguments expandably imposes a number of restrictions on 
%  both the type of arguments that can be read and the error checking 
%  available:
%  \begin{itemize}
%    \item The function must have at least one mandatory argument, and
%      in particular the last argument must be one of the mandatory
%      types (\texttt{l}, \texttt{m} or \texttt{u}).
%    \item All arguments are either short or long: it is not possible
%      to mix short and long argument types.
%    \item The `optional group' argument types \texttt{g} and
%      \texttt{G} are not available.
%    \item It is not possible to differentiate between, for example
%      |\foo[| and |\foo{[}|: in both cases the \texttt{[} will be
%      interpreted as the start of an optional argument. As a result
%      result, checking for optional arguments is less robust than 
%      in the standard version.
%  \end{itemize} 
%  \pkg{xparse} will issue an error if an argument specifier is given
%  which does not conform to the first three requirements. The last
%  item is an issue when the function is used, and so is beyond the
%  scope of \pkg{xparse} itself.
%\end{function}
%
%\subsection{Access to the argument specification}
%
% The argument specifications for document commands and environments are
% available for examination and use.
% 
%\begin{function}{ 
%  \GetDocumentCommandArgSpec |
%  \GetDocumentEnvironmentArgSpec |
%}
%  \begin{syntax}
%    \cs{GetDocumentCommandArgSpec} \meta{function} 
%    \cs{GetDocumentEnvironmentArgSpec} \meta{environment}
%  \end{syntax}
%  These functions transfer the current argument specification for the
%  requested \meta{function} or \meta{environment} into the token list
%  variable \cs{ArgumentSpecification}. If the \meta{function} or
%  \meta{environment} has no known argument specification then an error
%  is issued. The assignment to \cs{ArgumentSpecification} is local to
%  the current \TeX\ group.
%\end{function} 
%
%\begin{function}{ 
%  \ShowDocumentCommandArgSpec |
%  \ShowDocumentEnvironmentArgSpec |
%}
%  \begin{syntax}
%    \cs{ShowDocumentCommandArgSpec} \meta{function}
%    \cs{ShowDocumentEnvironmentArgSpec} \meta{environment}
%  \end{syntax}
%  These functions show the current argument specification for the
%  requested \meta{function} or \meta{environment} at the terminal. If
%  the \meta{function} or \meta{environment} has no known argument
%  specification then an error is issued.
%\end{function}  
%
%\end{documentation}
% 
%\begin{implementation}
%  
%\subsection{Variables and constants}
%
%\begin{variable}{\c_xparse_shorthands_prop}
%  Shorthands and replacement text: set up at the start of the package,
%  and not be be altered later!
%\end{variable}
%
%\begin{variable}{\l_xparse_arg_tl}
%  Variable used as internal representation of 
%  \cs{ProcessedArgument}. Unlike the later, this register should not
%  be used directly when creating new processors.
%\end{variable}
%
%\begin{variable}{\l_xparse_args_tl}
%  Token list variable for arguments as they are picked up for passing 
%  on to user functions.
%\end{variable}
%
%\begin{variable}{\l_xparse_environment_args_tl}
%  Token list register to pass arguments to the end of an environment 
%  from the beginning.
%\end{variable}
%
%\begin{variable}{\l_xparse_environment_bool}
%  When creating functions, a short cut can be taken if all of the
%  arguments are of \texttt{m} type. The code for environments cannot
%  do that, and so a flag is needed.
%\end{variable}
%
%\begin{variable}{\l_xparse_error_bool}
%  For flagging up errors when making expandable commands.
%\end{variable}
%
%\begin{variable}{\l_xparse_function_tl}
%  Needed to pass along the function name when creating in an expandable
%  manner. This is needed as a series of functions have to be created
%  when making expandable functions. (In contrast, standard robust 
%  functions need at most two functions.)
%\end{variable}
%
%\begin{variable}{\l_xparse_last_arg_tl}
%  The last argument type added. As this must be mandatory when creating
%  expandable commands, this variable is needed to enforce this 
%  behaviour.
%\end{variable}
%
%\begin{variable}{\l_xparse_long_bool}
%  Flag used to indicate creation of \cs{long} arguments.
%\end{variable}
%
%\begin{variable}{\l_xparse_m_args_int}
%  Used to enumerate the \texttt{m} arguments with no modifications
%  (i.e., neither long nor processed after grabbing).
%\end{variable}
%
%\begin{variable}{\l_xparse_m_only_bool}
%  Flag used to indicate that all arguments are of type \texttt{m},
%  with no no modifications.
%\end{variable}
%
%\begin{variable}{\l_xparse_mandatory_args_int}
%  For counting up all mandatory arguments so that the code can tell 
%  when optional arguments come after the last mandatory one. Counts 
%  down again as mandatory arguments are added to the signature, so will
%  be zero for any trailing optional arguments.
%\end{variable}
%
%\begin{variable}{\l_xparse_processor_bool}
%  When converting an argument specification into a signature there is
%  a need to know if there are any argument processors set up. This is
%  used to tell if \texttt{m} arguments can simply be counted up or need
%  handling on a one-off basis.
%\end{variable}
%
%\begin{variable}{\l_xparse_processor_int}
%  Each time a processor is set up in the grabber routine, it is stored
%  and the total number of processors is recorded here. Later, the 
%  variable is counted back down to use the processors in reverse order
%  to the collection order.
%\end{variable}
%
%\begin{variable}{\l_xparse_signature_tl}
%  For constructing the signature of the function defined. As 
%  \pkg{xparse} works through an argument specification, grabber 
%  functions are added to this variable for each argument.
%\end{variable}
%
%\begin{variable}{\l_xparse_tmp_tl}
%  Scratch space, used for example to convert shorthand argument types
%  into the full versions.
%\end{variable}
%
%\begin{variable}{\l_xparse_total_args_int}
%  Used to enumerate the total number of arguments (i.e., the number of
%  letters in the argument specification).
%\end{variable}
%
%\subsection{Internal functions}
%
%\begin{function}{
%  \xparse_add_arg:n |
%  \xparse_add_arg:V
%}
%  \begin{syntax}
%    "\xparse_add_arg:n" <grabbed arg>
%  \end{syntax}
%  Adds <grabbed arg> to the output \pkg{xparse} supplies to the
%  defined <code>, applying any post-processing that is needed.
%\end{function}
%
%\begin{function}{
%  \xparse_add_grabber_mandatory:N|
%  \xparse_add_grabber_optional:N
%}
%  \begin{syntax}
%    "\xparse_add_grabber_mandatory:N" <grabber type>
%  \end{syntax}
%  Adds appropriate grabber for <grabber type> to the signature
%  being constructed, making it long if necessary. The 
%  \texttt{optional} version includes a second check to see if space
%  skipping should be on or off.
%\end{function}
%
%\begin{function}{
%  \xparse_add_type_+:w|
%  \xparse_add_type_>:w|
%  \xparse_add_type_d:w|
%  \xparse_add_type_D:w|
%  \xparse_add_type_g:w|
%  \xparse_add_type_G:w|
%  \xparse_add_type_l:w|
%  \xparse_add_type_m:w|
%  \xparse_add_type_t:w|
%  \xparse_add_type_u:w
%}
%  \begin{syntax}
%    "\xparse_add_type_u:w"
%  \end{syntax}
%  Carry out necessary processes to add given <type> of argument to 
%  the signature being constructed. Depending on the argument type
%  being added, one or more arguments will be absorbed.
%\end{function}
%
%\begin{function}{\xparse_check_and_add:N}
%  \begin{syntax}
%    "\xparse_check_and_add:N" <arg spec>
%  \end{syntax}
%  Ensures that <arg spec> is valid, and if so adds it to the signature
%  being constructed.
%\end{function}
%
%\begin{function}{
%  \xparse_count_mandatory:n|
%  \xparse_count_mandatory:N
%}
%  \begin{syntax}
%    "\xparse_count_mandatory:N" <arg spec>
%  \end{syntax}
%  Used to count how many mandatory arguments an argument specification
%  contains. The \texttt{n} function carries out the set up, before
%  handing of to the \texttt{N} function. This reads one token, and
%  calls the appropriate counter function.
%\end{function}
%
%\begin{function}{
%  \xparse_count_type_>:w|
%  \xparse_count_type_+:w|
%  \xparse_count_type_d:w|
%  \xparse_count_type_D:w|
%  \xparse_count_type_g:w|
%  \xparse_count_type_G:w|
%  \xparse_count_type_l:w|
%  \xparse_count_type_m:w|
%  \xparse_count_type_t:w|
%  \xparse_count_type_u:w|
%}
%  \begin{syntax}
%    "\xparse_count_type_D:w" 
%  \end{syntax}
%  Used to count up mandatory arguments: one function for each argument
%  type so that a simple loop can be used. Only the functions for
%  mandatory arguments do any more than call the loop again.
%\end{function}
%
%\begin{function}{\xparse_declare_cmd:Nnn}
%  \begin{syntax}
%    "\xparse_declare_cmd:Nnn" <function> \Arg{signature}
%    ~~~~\Arg{code}
%  \end{syntax}
%  Declares <function> using <signature> for argument definition and
%  <code> as expansion.
%  \begin{texnote}
%    This is the internal name for \cs{DeclareDocumentCommand}.
%  \end{texnote}
%\end{function}
%
%\begin{function}{\xparse_declare_cmd_interface:Nnn}
%  \begin{syntax}
%    "\xparse_declare_cmd_interface:Nnn" <function> 
%    ~~~~\Arg{implementation} \Arg{signature}
%  \end{syntax}
%  Declares <function> using <signature>, which should have code stored
%  as <implementation>.
%  \begin{texnote}
%    This is the internal name for \cs{DeclareDocumentCommandInterface}.
%  \end{texnote}
%\end{function}
%
%\begin{function}{\xparse_declare_cmd_implementation:nNn}
%  \begin{syntax}
%    "\xparse_declare_cmd_implementation:nNn"
%    ~~~~\Arg{implementation} <number> \Arg{code} 
%  \end{syntax}
%  Declares <code> taking <number> arguments as an <implementation>,
%  to be accessed using an interface defined elsewhere.
%  \begin{texnote}
%    This is the internal name for 
%    \cs{DeclareDocumentCommandImplementation}.
%  \end{texnote}
%\end{function}
%
%\begin{function}{\xparse_declare_env:nnnn}
%  \begin{syntax}
%    "\xparse_declare_env:nnnn" \Arg{env} \Arg{arg spec}
%    ~~~~\Arg{start code} \Arg{end code}
%  \end{syntax}
%  Declares <env> as an environment taking <arg spec> arguments 
%  at \cs{begin}\{<env>\}. The <start code> is executed at the beginning
%  of the environment, and the <end code> at the end. Both parts may
%  use the arguments defined by <arg spec>.
%  \begin{texnote}
%    This is the internal name for \cs{DeclareDocumentEnvironment}.
%  \end{texnote}
%\end{function}
%
%\begin{function}{\xparse_flush_m_args:}
%  \begin{syntax}
%    "\xparse_flush_m_args:" 
%  \end{syntax}
%  Adds an outstanding \texttt{m} arguments to the signature.
%\end{function}
%
%\begin{function}{\xparse_grab_arg:w}
%  \begin{syntax}
%    "\xparse_grab_arg:w"  <args>
%  \end{syntax}
%  Function re-defined each time an argument is grabbed to actually
%  do the grabbing. It is this function which will raise an error if
%  an argument runs away.
%\end{function}
%
%\begin{function}{
%  \xparse_grab_D:w              |
%  \xparse_grab_D_long:w         |
%  \xparse_grab_D_trailing:w     |
%  \xparse_grab_D_long_trailing:w|
%  \xparse_grab_G:w              |
%  \xparse_grab_G_long:w         |
%  \xparse_grab_G_trailing:w     |
%  \xparse_grab_G_long_trailing:w|
%  \xparse_grab_l:w              |
%  \xparse_grab_l_long:w         |
%  \xparse_grab_m:w              |
%  \xparse_grab_m_long:w         |
%  \xparse_grab_m_1:w            |
%  \xparse_grab_m_2:w            |
%  \xparse_grab_m_3:w            |
%  \xparse_grab_m_4:w            |
%  \xparse_grab_m_5:w            |
%  \xparse_grab_m_6:w            |
%  \xparse_grab_m_7:w            |
%  \xparse_grab_m_8:w            |
%  \xparse_grab_t:w              |
%  \xparse_grab_t_long:w         |
%  \xparse_grab_t_trailing:w     |
%  \xparse_grab_t_long_trailing:w|
%  \xparse_grab_u:w              |
%  \xparse_grab_u_long:w         |
%}
%  \begin{syntax}
%    "\xparse_grab_D:w" <arg data> "\l_xparse_args_tl"
%  \end{syntax}
%  Argument grabbing functions, which re-arrange other <arg data>
%  so that the argument is read correctly. The \texttt{trailing} 
%  versions do not skip spaces when searching for optional arguments.
%  For each argument type, the various versions feed the appropriate
%  information to a common auxiliary function which then sets up
%  \cs{xparse_grab_arg:w} to actually carry out the argument absorption.
%\end{function}
%
%\begin{function}{\xparse_if_no_value:n / (TF) (EXP)}
%  \begin{syntax}
%    "\xparse_if_no_value:nTF" \Arg{arg}
%    ~~~~\Arg{true code} \Arg{false code}
%  \end{syntax}
%  Executes <true code> if <arg> is equal to the special 
%  \cs{NoValue} marker and <false code> otherwise.  Provided that
%  the primitive \cs{(pdf)strcmp} is available, this function is
%  expandable.
%\end{function}
%
%\begin{function}{
%  \xparse_prepare_signature:n|
%  \xparse_prepare_signature:N
%}
%  \begin{syntax}
%    "\xparse_prepare_signature:n"  \Arg{arg specs}
%  \end{syntax}
%  Parse one or more <arg specs> and convert to an output <signature>.  
%\end{function}
%
%\begin{function}{\xparse_process_arg:n}
%  \begin{syntax}
%    "\xparse_process_arg:n" \Arg{processor}
%  \end{syntax}
%  Sets up code to apply <processor> to next grabbed argument.
%\end{function}
%
%\subsection{Creating expandable commands}
%
%\begin{function}{
%  \xparse_exp_add_type_d:w|
%  \xparse_exp_add_type_D:w|
%  \xparse_exp_add_type_l:w|
%  \xparse_exp_add_type_m:w|
%  \xparse_exp_add_type_t:w|
%  \xparse_exp_add_type_u:w
%}
%  \begin{syntax}
%    "\xparse_exp_add_type_u:w" \Arg{delimiter}
%  \end{syntax}
%  Carry out necessary processes to add given <type> of argument for
%  an expandable command. Depending on the argument type being added,
%  one or more arguments will be absorbed.
%\end{function}
%
%\begin{function}{\xparse_exp_check_and_add:N}
%  \begin{syntax}
%    "\xparse_exp_check_and_add:N" <arg spec>
%  \end{syntax}
%  Ensures that <arg spec> is valid, and if so adds it to expandable
%  function being constructed.
%\end{function}
%
%\begin{function}{\xparse_exp_declare_cmd:Nnn}
%  \begin{syntax}
%    "\xparse_exp_declare_cmd:Nnn" <function> \Arg{signature}
%    ~~~~\Arg{code}
%  \end{syntax}
%  Declares <function> using <signature> for argument definition and
%  <code> as expansion, and creating an expandable command.
%  \begin{texnote}
%    This is the internal name for 
%    \cs{DeclareExpandableDocumentCommand}.
%  \end{texnote}
%\end{function}
%
%\begin{function}{
%  \xparse_exp_prepare_function:n|
%  \xparse_exp_prepare_function:N
%}
%  \begin{syntax}
%    "\xparse_exp_prepare_function:n"  \Arg{arg specs}
%  \end{syntax}
%  Parse one or more <arg specs> and convert to an expandable
%  function. 
%\end{function}
%
%\begin{function}{\xparse_exp_set:cpx}
%  \begin{syntax}
%    "\xparse_exp_set:cpx" <csname> <parameters> \Arg{code}
%  \end{syntax}
%  An alias for either \cs{cs_set:cpx} or \cs{cs_set_nopar:cpx},
%  depending on the \cs{long} status of the expandable function.
%\end{function}
%
%\section{\pkg{xparse} implementation}
%
% The usual lead-off: only needed for the package, of course (one day we
% may have a \LaTeX3 kernel).
%    \begin{macrocode}
%<*package>
\ProvidesExplPackage
  {\filename}{\filedate}{\fileversion}{\filedescription}
\RequirePackage{expl3}
%</package>
%<*initex|package>
%    \end{macrocode}
%    
%\subsection{Variables and constants}
%
%\begin{macro}{\c_xparse_shorthands_prop}
% Shorthands are stored as a property list: this is set up here as it
% is a constant.
%    \begin{macrocode}
\prop_new:N \c_xparse_shorthands_prop
\prop_put:Nnn \c_xparse_shorthands_prop { o } { d[] }
\prop_put:Nnn \c_xparse_shorthands_prop { O } { D[] }
\prop_put:Nnn \c_xparse_shorthands_prop { s } { t* }
%    \end{macrocode}
%\end{macro}
%
%\begin{macro}{\l_xparse_arg_tl}
% Token registers for single grabbed argument when post-processing.
%    \begin{macrocode}
\tl_new:N \l_xparse_arg_tl
%    \end{macrocode}
%\end{macro}
%
%\begin{macro}{\l_xparse_args_tl}
% Token registers for grabbed arguments.
%    \begin{macrocode}
\tl_new:N \l_xparse_args_tl
%    \end{macrocode}
%\end{macro}
%
%\begin{macro}{\l_xparse_environment_bool}
% Generating environments uses the same mechanism as generating
% functions. However, full processing of arguments is always needed
% for environments, and so the function-generating code needs to
% know this.
%    \begin{macrocode}
\bool_new:N \l_xparse_environment_bool
%    \end{macrocode}
%\end{macro}
%
%\begin{macro}{\l_xparse_error_bool}
% Used to signal an error when creating expandable functions.
%    \begin{macrocode}
\bool_new:N \l_xparse_error_bool
%    \end{macrocode}
%\end{macro}
%
%\begin{macro}{\l_xparse_function_tl}
% When creating expandable functions, the current function name needs
% to be passed along.
%    \begin{macrocode}
\tl_new:N \l_xparse_function_tl
%    \end{macrocode}
%\end{macro}
%
%\begin{macro}{\l_xparse_last_arg_tl}
% Used when creating expandable arguments.
%    \begin{macrocode}
\tl_new:N \l_xparse_last_arg_tl
%    \end{macrocode}
%\end{macro}
%
%\begin{macro}{\l_xparse_long_bool}
% A flag for \cs{long} arguments.
%    \begin{macrocode}
\bool_new:N \l_xparse_long_bool
%    \end{macrocode}
%\end{macro}
%
%\begin{macro}{\l_xparse_m_args_int}
% The number of simple \texttt{m} arguments is tracked so they can be
% dumped \emph{en masse}.
%    \begin{macrocode}
\int_new:N \l_xparse_m_args_int
%    \end{macrocode}
%\end{macro}
%
%\begin{macro}{\l_xparse_m_only_bool}
% A flag to indicate that only \texttt{m} arguments have been found.
%    \begin{macrocode}
\bool_new:N \l_xparse_m_only_bool
%    \end{macrocode}
%\end{macro}
%
%\begin{macro}{\l_xparse_mandatory_args_int}
% So that trailing optional arguments can be picked up, a count has to
% be taken of all mandatory arguments. This is then decreased as 
% mandatory arguments are added to the signature, so will be zero
% only if there are no more mandatory arguments to add.
%    \begin{macrocode}
\int_new:N \l_xparse_mandatory_args_int
%    \end{macrocode}
%\end{macro}
%
%\begin{macro}{\l_xparse_processor_bool}
% When reading through the argument specifier, a flag is needed to 
% show that a processor has been found for the current argument. This
% is used when checking how to handle \texttt{m} arguments.
%    \begin{macrocode}
\bool_new:N \l_xparse_processor_bool 
%    \end{macrocode}
%\end{macro}
%
%\begin{macro}{\l_xparse_processor_int}
% In the grabber routine, each processor is saved with a number 
% recording the order it was found in. The total is then used to work 
% back through the grabbers so they apply to the argument right to left.
%    \begin{macrocode}
\int_new:N \l_xparse_processor_int
%    \end{macrocode}
%\end{macro}
%
%\begin{macro}{\l_xparse_signature_tl}
% Token registers for constructing signatures.
%    \begin{macrocode}
\tl_new:N \l_xparse_signature_tl
%    \end{macrocode}
%\end{macro}
%
%\begin{macro}{\l_xparse_tmp_tl}
% A general purpose token list variable.
%    \begin{macrocode}
\tl_new:N \l_xparse_tmp_tl
%    \end{macrocode}
%\end{macro}
%
%\begin{macro}{\l_xparse_total_args_int}
% Thje total number of arguments is used to create the internal function
% which has a fixed number of arguments.
%    \begin{macrocode}
\int_new:N \l_xparse_total_args_int
%    \end{macrocode}
%\end{macro}
%    
%\subsection{Turning the argument specifier into grabbers}
%
%\begin{macro}{\xparse_add_grabber_mandatory:N}
%\begin{macro}{\xparse_add_grabber_optional:N}
% To keep the various checks needed in one place, adding the grabber to
% the signature is done here. For mandatory arguments, the only question
% is whether to add a long grabber. For optional arguments, there is 
% also a check to see if any mandatory arguments are still to be added.
% This is used to determine whether to skip spaces or not where 
% searching for the argument.
%    \begin{macrocode}
\cs_new_nopar:Npn \xparse_add_grabber_mandatory:N #1 {
  \tl_put_right:Nx \l_xparse_signature_tl {
    \exp_not:c { 
      xparse_grab_ #1 \bool_if:NT \l_xparse_long_bool { _long } :w
    }
  }
  \bool_set_false:N \l_xparse_long_bool
  \int_decr:N \l_xparse_mandatory_args_int
}
\cs_new_nopar:Npn \xparse_add_grabber_optional:N #1 {
  \tl_put_right:Nx \l_xparse_signature_tl {
    \exp_not:c { 
      xparse_grab_ #1 
      \bool_if:NT \l_xparse_long_bool { _long } 
      \int_compare:nF {
        \l_xparse_mandatory_args_int > \c_zero
      } { _trailing }
      :w
    }
  }
  \bool_set_false:N \l_xparse_long_bool
}
%    \end{macrocode}
%\end{macro}
%\end{macro}
%
% All of the argument-adding functions work in essentially the same
% way, except the one for \texttt{m} arguments. Any collected \texttt{m}
% arguments are added to the signature, then the appropriate grabber
% is added to the signature. Some of the adding functions also pick up
% one or more arguments, and are also added to the signature. All of the
% functions then call the loop function \cs{xparse_prepare_signature:N}.
%
%\begin{macro}{\xparse_add_type_+:w}
% Making the next argument \cs{long} means setting the flag and
% knocking one back off the total argument count. The \texttt{m}
% arguments are recorded here as this has to be done for every case
% where there is then a \cs{long} argument.
%    \begin{macrocode}
\cs_new_nopar:cpn { xparse_add_type_+:w } {
  \xparse_flush_m_args:
  \bool_set_true:N \l_xparse_long_bool 
  \bool_set_false:N \l_xparse_m_only_bool
  \int_decr:N \l_xparse_total_args_int
  \xparse_prepare_signature:N
}
%    \end{macrocode}
%\end{macro}
%
%\begin{macro}{\xparse_add_type_>:w}
% When a processor is found, the function \cs{xparse_process_arg:n}
% is added to the signature along with the processor code itself. When
% the signature is used, the code will be added to an execution list by
% \cs{xparse_process_arg:n}. Here, the loop calls 
% \cs{xparse_prepare_signature_aux:N} rather than 
% \cs{xparse_prepare_signature:N} so that the flag is not reset.
%    \begin{macrocode}
\cs_new:cpn { xparse_add_type_>:w } #1 {
  \bool_set_true:N \l_xparse_processor_bool
  \xparse_flush_m_args:
  \int_decr:N \l_xparse_total_args_int
  \tl_put_right:Nn \l_xparse_signature_tl {  
    \xparse_process_arg:n {#1}
  }
  \xparse_prepare_signature_aux:N
}
%    \end{macrocode}
%\end{macro}
%
%\begin{macro}{\xparse_add_type_d:w}
% To save on repeated code, \texttt{d} is actually turned into
% the same grabber as is used by \texttt{D}, by putting the
% \cs{NoValue} default in the correct place. So there is some
% simple argument re-arrangement to do. Remember that |#1| and |#2|
% should be single tokens.
%    \begin{macrocode}
\cs_new:Npn \xparse_add_type_d:w #1#2 {
  \xparse_add_type_D:w #1 #2 { \NoValue }
}
%    \end{macrocode}
%\end{macro}
%
%\begin{macro}{\xparse_add_type_D:w}
% All of the optional delimited arguments are handled internally by
% the \texttt{D} type. At this stage, the two delimiters are stored
% along with the default value.
%    \begin{macrocode}
\cs_new:Npn \xparse_add_type_D:w #1#2#3 {
  \xparse_flush_m_args:
  \xparse_add_grabber_optional:N D
  \tl_put_right:Nn \l_xparse_signature_tl { #1 #2 {#3} }
  \xparse_prepare_signature:N
}
%    \end{macrocode}
%\end{macro}
%
%\begin{macro}{\xparse_add_type_g:w}
% The \texttt{g} type is simply an alias for \texttt{G} with the 
% correct default built-in.
%    \begin{macrocode}
\cs_new_nopar:Npn \xparse_add_type_g:w {
  \xparse_add_type_G:w { \NoValue }
}
%    \end{macrocode}
%\end{macro}
%
%\begin{macro}{\xparse_add_type_G:w}
% For the \texttt{G} type, the grabber and the default are added to 
% the signature.
%    \begin{macrocode}
\cs_new:Npn \xparse_add_type_G:w #1 {
  \xparse_flush_m_args:
  \xparse_add_grabber_optional:N G
  \tl_put_right:Nn \l_xparse_signature_tl { {#1} }
  \xparse_prepare_signature:N
}
%    \end{macrocode}
%\end{macro}
%
%\begin{macro}{\xparse_add_type_l:w}
% Finding \texttt{l} arguments is very simple: there is nothing to do
% other than add the grabber.
%    \begin{macrocode}
\cs_new_nopar:Npn \xparse_add_type_l:w {
  \xparse_flush_m_args:
  \xparse_add_grabber_mandatory:N l
  \xparse_prepare_signature:N
}
%    \end{macrocode}
%\end{macro}
%
%\begin{macro}{\xparse_add_type_m:w}
% The \texttt{m} type is special as short arguments which are not 
% post-processed are simply counted at this stage. Thus there is a check
% to see if either of these cases apply. If so, a one-argument grabber
% is added to the signature. On the other hand, if a standard short 
% argument is required it is simply counted at this stage, to be
% added later using \cs{xparse_flush_m_args:}.
%    \begin{macrocode}
\cs_new_nopar:Npn \xparse_add_type_m:w {
  \bool_if:nTF {
    \l_xparse_long_bool || \l_xparse_processor_bool
  } {
    \xparse_flush_m_args:
    \xparse_add_grabber_mandatory:N m
  }{
    \int_incr:N \l_xparse_m_args_int
  }
  \xparse_prepare_signature:N
}
%    \end{macrocode}
%\end{macro}
%
%\begin{macro}{\xparse_add_type_t:w}
% Setting up a \texttt{t} argument means collecting one token for the
% test, and adding it along with the grabber to the signature.
%    \begin{macrocode}
\cs_new:Npn \xparse_add_type_t:w #1 {
  \xparse_flush_m_args:
  \xparse_add_grabber_optional:N t
  \tl_put_right:Nn \l_xparse_signature_tl { #1 }
  \xparse_prepare_signature:N
}
%    \end{macrocode}
%\end{macro}
%
%\begin{macro}{\xparse_add_type_u:w}
% At the set up stage, the \texttt{u} type argument is identical to the
% \texttt{G} type except for the name of the grabber function.
%    \begin{macrocode}
\cs_new:Npn \xparse_add_type_u:w #1 {
  \xparse_flush_m_args:
  \xparse_add_grabber_mandatory:N u
  \tl_put_right:Nn \l_xparse_signature_tl { {#1} }
  \xparse_prepare_signature:N
}
%    \end{macrocode}
%\end{macro}
%
%\begin{macro}{\xparse_check_and_add:N}
% This function checks if the argument type actually exists and gives
% an error if it doesn't.
%    \begin{macrocode}
\cs_new_nopar:Npn \xparse_check_and_add:N  #1 {
  \cs_if_free:cTF { xparse_add_type_ #1 :w } {
    \msg_kernel_error:nnx { xparse } { unknown-argument-type } {#1}
    \xparse_add_type_m:w
  }{
    \use:c { xparse_add_type_ #1 :w }
  }
}
%    \end{macrocode}
%\end{macro}
%
%\begin{macro}{\xparse_count_mandatory:n}
%\begin{macro}{\xparse_count_mandatory:N}
%\begin{macro}[aux]{\xparse_count_mandatory_aux:N}
% To count up mandatory arguments before the main parsing run, the
% same approach is used. First, check if the current token is a 
% short-cut for another argument type. If it is, expand it and loop
% again. If not, then look for a `counting' function to check the 
% argument type. No error is raised here if one is not found as one
% will be raised by later code.
%    \begin{macrocode}
\cs_new:Npn \xparse_count_mandatory:n #1 {
  \int_zero:N \l_xparse_mandatory_args_int
  \xparse_count_mandatory:N #1 \q_nil
}
\cs_new:Npn \xparse_count_mandatory:N #1 {
  \quark_if_nil:NF #1 {
    \prop_if_in:NnTF \c_xparse_shorthands_prop {#1} {
      \prop_get:NnN \c_xparse_shorthands_prop {#1} \l_xparse_tmp_tl
      \exp_last_unbraced:NV \xparse_count_mandatory:N \l_xparse_tmp_tl
    }{
      \xparse_count_mandatory_aux:N #1
    }
  }
}
\cs_new:Npn \xparse_count_mandatory_aux:N #1 {
  \cs_if_free:cTF { xparse_count_type_ #1 :w } {
    \xparse_count_type_m:w
  }{
    \use:c { xparse_count_type_ #1 :w }
  }
}
%    \end{macrocode}
%\end{macro}
%\end{macro}
%\end{macro}
%
%\begin{macro}{\xparse_count_type_>:w}
%\begin{macro}{\xparse_count_type_+:w}
%\begin{macro}{\xparse_count_type_d:w}
%\begin{macro}{\xparse_count_type_D:w}
%\begin{macro}{\xparse_count_type_g:w}
%\begin{macro}{\xparse_count_type_G:w}
%\begin{macro}{\xparse_count_type_l:w}
%\begin{macro}{\xparse_count_type_m:w}
%\begin{macro}{\xparse_count_type_t:w}
%\begin{macro}{\xparse_count_type_u:w}
% For counting the mandatory arguments, a function is provided for 
% each argument type that will mop any extra arguments and call the
% loop function. Only the counting functions for mandatory arguments
% actually do anything: the rest are simply there to ensure the loop 
% continues correctly.
%    \begin{macrocode}
\cs_new:cpn { xparse_count_type_>:w } #1 {
  \xparse_count_mandatory:N
}
\cs_new_nopar:cpn { xparse_count_type_+:w } {
  \xparse_count_mandatory:N
}
\cs_new:Npn \xparse_count_type_d:w #1#2 {
  \xparse_count_mandatory:N
}
\cs_new:Npn \xparse_count_type_D:w #1#2#3 {
  \xparse_count_mandatory:N
}
\cs_new_nopar:Npn \xparse_count_type_g:w {
  \xparse_count_mandatory:N
}
\cs_new:Npn \xparse_count_type_G:w #1 {
  \xparse_count_mandatory:N
}
\cs_new_nopar:Npn \xparse_count_type_l:w {
  \int_incr:N \l_xparse_mandatory_args_int
  \xparse_count_mandatory:N
}
\cs_new_nopar:Npn \xparse_count_type_m:w {
  \int_incr:N \l_xparse_mandatory_args_int
  \xparse_count_mandatory:N
}
\cs_new:Npn \xparse_count_type_t:w #1 {
  \xparse_count_mandatory:N
}
\cs_new:Npn \xparse_count_type_u:w #1 {
  \int_incr:N \l_xparse_mandatory_args_int
  \xparse_count_mandatory:N
}
%    \end{macrocode}
%\end{macro}
%\end{macro}
%\end{macro}
%\end{macro}
%\end{macro}
%\end{macro}
%\end{macro}
%\end{macro}
%\end{macro}
%\end{macro}
%
%\begin{macro}{\xparse_declare_cmd:Nnn}
%\begin{macro}[aux]{\xparse_declare_cmd_aux:Nnn}
%\begin{macro}[aux]{\xparse_declare_cmd_aux:cnn}
%\begin{macro}[aux]{\xparse_declare_cmd_all_m:Nn}
%\begin{macro}[aux]{\xparse_declare_cmd_mixed:Nn}
% First, the signature is set up from the argument specification. There
% is then a check: if only \texttt{m} arguments are needed (which 
% includes functions with no arguments at all) then the definition is
% simple. On the other hand, if the signature is more complex then an 
% internal function actually contains the code with the user function
% as a simple wrapper.
%    \begin{macrocode}
\cs_new:Npn \xparse_declare_cmd:Nnn #1#2 {
  \cs_if_exist:NTF #1 
    { 
      \msg_kernel_warning:nnxx { xparse } { redefine-command } 
        { \token_to_str:N #1 } { \exp_not:n {#2} }
    }
    { 
      \msg_kernel_info:nnxx { xparse } { define-command } 
        { \token_to_str:N #1 } { \exp_not:n {#2} }
    }
  \xparse_declare_cmd_aux:Nnn #1 {#2}
}
\cs_new:Npn \xparse_declare_cmd_aux:Nnn #1#2#3 {    
  \xparse_count_mandatory:n {#2}
  \xparse_prepare_signature:n {#2}
  \bool_if:NTF \l_xparse_m_only_bool {
    \xparse_declare_cmd_all_m:Nn #1 {#3}
  }{
    \xparse_declare_cmd_mixed:Nn #1 {#3}
  }
}
\cs_generate_variant:Nn \xparse_declare_cmd_aux:Nnn { cnn }
\cs_new:Npn \xparse_declare_cmd_all_m:Nn #1#2 {
  \cs_generate_from_arg_count:NNnn 
    #1 \cs_set_protected_nopar:Npn \l_xparse_total_args_int {#2}
}
\cs_new:Npn \xparse_declare_cmd_mixed:Nn #1#2 {
  \cs_set_protected_nopar:Npx #1 {
    \exp_not:n { 
      \int_zero:N \l_xparse_processor_int
      \tl_set:Nn \l_xparse_args_tl 
    } { \exp_not:c { \token_to_str:N #1 } }
    \exp_not:V \l_xparse_signature_tl
    \exp_not:N \l_xparse_args_tl
  }
  \cs_generate_from_arg_count:cNnn 
    { \token_to_str:N #1 } \cs_set:Npn \l_xparse_total_args_int {#2}
}
%    \end{macrocode} 
%\end{macro}
%\end{macro}
%\end{macro}
%\end{macro}
%\end{macro}
%
%\begin{macro}{\xparse_declare_cmd_implementation:nNn}
% Creating a stand-alone implementation using the `two-part' mechanism
% is quite easy as this is just a wrapper for
% \cs{cs_generate_from_arg_count:cNnn}.
%    \begin{macrocode}
\cs_new:Npn \xparse_declare_cmd_implementation:nNn #1#2#3 {
  \cs_generate_from_arg_count:cNnn { implementation_ #1 :w } 
    \cs_set:Npn {#2} {#3}
}
%    \end{macrocode}
%\end{macro}
%
%\begin{macro}{\xparse_declare_cmd_interface:Nnn}
%\begin{macro}[aux]{\xparse_declare_cmd_interface_all_m:Nn}
%\begin{macro}[aux]{\xparse_declare_cmd_interface_mixed:Nn}
% As with the basic function \cs{xparse_declare_cmd:Nnn}, there are
% three things to do here. First, generate a signature from the 
% argument specification. Then use that to create a function which
% will call the implementation part. Finally, a holder implementation
% is created. As before, there is a short-cut for functions which only
% have \texttt{m} type arguments.
%    \begin{macrocode}
\cs_new:Npn \xparse_declare_cmd_interface:Nnn #1#2#3 {
  \xparse_prepare_signature:n {#3}
  \bool_if:NTF \l_xparse_m_only_bool {
    \xparse_declare_cmd_interface_all_m:Nn #1 {#2}
  }{
    \xparse_declare_cmd_interface_mixed:Nn #1 {#2}
  }
  \cs_generate_from_arg_count:cNnn { implementation_ #2 :w } 
    \cs_set:Npn \l_xparse_total_args_int  { ``#2'' }
}
\cs_new:Npn \xparse_declare_cmd_interface_all_m:Nn #1#2 {
  \cs_generate_from_arg_count:NNnn 
    #1 \cs_set_protected_nopar:Npn \l_xparse_total_args_int 
    { \use:c { implementation_ #2 :w } }
}
\cs_new:Npn \xparse_declare_cmd_interface_mixed:Nn #1#2 {
  \cs_set_protected_nopar:Npx #1 {
    \exp_not:n { 
      \int_zero:N \l_xparse_processor_int
      \tl_set:Nn \l_xparse_args_tl 
    } { \exp_not:c { \token_to_str:N #1 } }
    \exp_not:V \l_xparse_signature_tl
    \exp_not:N \l_xparse_args_tl
  }
  \cs_generate_from_arg_count:cNnn 
    { \token_to_str:N #1 } \cs_set:Npn \l_xparse_total_args_int 
    { \use:c { implementation_ #2 :w } }
}
%    \end{macrocode}
%\end{macro}
%\end{macro}
%\end{macro}
%
%\begin{macro}{\xparse_declare_env:nnnn}
% The idea here is to make sure that the end of the environment has the
% same arguments available as the beginning.
%    \begin{macrocode}
\cs_new:Npn \xparse_declare_env:nnnn #1#2#3#4 {
  \bool_set_true:N \l_xparse_environment_bool
%</initex|package>
%<*initex>
  \cs_if_exist:cTF { environment_begin_ #1 :w } 
%</initex>
%<*package>
  \cs_if_exist:cTF {#1}  
%</package>
%<*initex|package> 
    { 
      \msg_kernel_warning:nnxx { xparse } { redefine-environment } 
        {#1} { \exp_not:n {#2} }
    }
    { 
      \msg_kernel_info:nnxx { xparse } { define-environment } 
        {#1} { \exp_not:n {#2} }
    }
  \xparse_declare_cmd_aux:cnn { environment_begin_ #1 :w } {#2} {
    \group_begin:
      \cs_set_protected_nopar:cpx { environment_end_ #1 :w }
        {
            \exp_not:c { environment_end_ #1 _aux:N }
            \exp_not:V \l_xparse_args_tl 
          \group_end:
        }
      #3
  }
 \cs_set_protected_nopar:cpx { environment_end_ #1 : } 
   { \exp_not:c { environment_end_ #1 :w } }
  \bool_set_false:N \l_xparse_environment_bool
  \cs_set_nopar:cpx { environment_end_ #1 _aux:N } ##1 {
    \exp_not:c { environment_end_ #1 _aux :w }
  }
  \cs_generate_from_arg_count:cNnn 
    { environment_end_ #1 _aux :w } \cs_set:Npn
    \l_xparse_total_args_int {#4}
%</initex|package>
%<*package>
  \cs_set_eq:cc {#1} { environment_begin_ #1 :w } 
  \cs_set_eq:cc { end #1 } { environment_end_ #1 : }
%</package>
%<*initex|package>
}
%    \end{macrocode}
%\end{macro}
%
%\begin{macro}{\xparse_flush_m_args:}
% As \texttt{m} arguments are simply counted, there is a need to add 
% them to the token register in a block. As this function can only
% be called if something other than \texttt{m} turns up, the flag can 
% be switched here. The total number of mandatory arguments added to
% the signature is also decreased by the appropriate amount.
%    \begin{macrocode}
\cs_new_nopar:Npn \xparse_flush_m_args: {
  \cs_if_exist:cT { 
    xparse_grab_m_ \int_use:N \l_xparse_m_args_int :w 
  } {
    \tl_put_right:Nx \l_xparse_signature_tl {
       \exp_not:c { xparse_grab_m_ \int_use:N \l_xparse_m_args_int :w }
    }
    \int_set:Nn \l_xparse_mandatory_args_int {
      \l_xparse_mandatory_args_int - \l_xparse_m_args_int
    }
  }
  \int_zero:N \l_xparse_m_args_int
  \bool_set_false:N \l_xparse_m_only_bool
}
%    \end{macrocode}
%\end{macro}
%
%
%\begin{macro}[TF]{\xparse_if_no_value:n}
% Tests for \cs{NoValue}.
%    \begin{macrocode}
\prg_new_conditional:Nnn \xparse_if_no_value:n { TF,T,F } {
  \str_if_eq:nnTF {#1} { \NoValue } {
    \prg_return_true:
  }{
    \prg_return_false:
  }
}
%    \end{macrocode}
%\end{macro}
%
%\begin{macro}{\xparse_prepare_signature:n}
% Creating the signature is a case of working through the input and
% turning into the output in \cs{l_xparse_signature_tl}. A track is
% also kept of the total number of arguments. This function sets 
% everything up then hands off to the parser.
%    \begin{macrocode}
\cs_new:Npn \xparse_prepare_signature:n #1 {
  \bool_set_false:N \l_xparse_long_bool
  \int_zero:N \l_xparse_m_args_int
  \bool_if:NTF \l_xparse_environment_bool {
    \bool_set_false:N \l_xparse_m_only_bool
  }{
    \bool_set_true:N \l_xparse_m_only_bool
  }
  \bool_set_false:N \l_xparse_processor_bool
  \tl_clear:N \l_xparse_signature_tl
  \int_zero:N \l_xparse_total_args_int
  \xparse_prepare_signature:N #1 \q_nil
}
%    \end{macrocode}
%\end{macro}
%
%\begin{macro}{\xparse_prepare_signature:N}
%\begin{macro}[aux]{\xparse_prepare_signature_aux:N}
% The main signature-preparation loop is in two parts, to keep the code 
% a little clearer. Most of the checks here is pretty clear, with a key
% point to watch what is next on the stack so that the loop continues
% correctly. 
%    \begin{macrocode}
\cs_new:Npn \xparse_prepare_signature:N #1 {
  \bool_set_false:N \l_xparse_processor_bool
  \xparse_prepare_signature_aux:N #1
}
\cs_new:Npn \xparse_prepare_signature_aux:N #1 {
  \quark_if_nil:NTF #1 {
    \bool_if:NF \l_xparse_m_only_bool {
      \xparse_flush_m_args:
    }
  }{
    \prop_if_in:NnTF \c_xparse_shorthands_prop {#1} {
      \prop_get:NnN \c_xparse_shorthands_prop {#1} \l_xparse_tmp_tl
      \exp_last_unbraced:NV \xparse_prepare_signature:N \l_xparse_tmp_tl
    }{
      \int_incr:N \l_xparse_total_args_int
      \xparse_check_and_add:N #1
    }
  }
}
%    \end{macrocode}
%\end{macro}
%\end{macro}
%
%\begin{macro}{\xparse_process_arg:n}
% Processors are saved for use later during the grabbing process.
%    \begin{macrocode}
\cs_new:Npn \xparse_process_arg:n #1 {
  \int_incr:N \l_xparse_processor_int
  \cs_set:cpn { 
    xparse_processor_ \int_use:N \l_xparse_processor_int :n 
  } ##1
    { #1 {##1} }
}
%    \end{macrocode}
%\end{macro}
%
%\subsection{Grabbing arguments}
%
%\begin{macro}{\xparse_add_arg:n}
%\begin{macro}{\xparse_add_arg:V}
%\begin{macro}[aux]{\xparse_add_arg_aux:n}
%\begin{macro}[aux]{\xparse_add_arg_aux:V}
% The argument-storing system provides a single point for interfacing
% with processors. They are done in a loop, counting downward. In this 
% way, the processor which was found last is executed first. The result
% is that processors apply from right to left, as intended. Notice that
% a set of braces are added back around the result of processing so that
% the internal function will correctly pick up one argument for each
% input argument.
%    \begin{macrocode}
\cs_new:Npn \xparse_add_arg:n #1 {
  \int_compare:nTF { \l_xparse_processor_int = \c_zero } {
    \tl_put_right:Nn \l_xparse_args_tl { {#1} }
  }{
    \xparse_if_no_value:nTF {#1} {
      \int_zero:N \l_xparse_processor_int
      \tl_put_right:Nn \l_xparse_args_tl { {#1} }
    }{
      \xparse_add_arg_aux:n {#1}
    }
  }
}
\cs_generate_variant:Nn \xparse_add_arg:n { V }
\cs_new:Npn \xparse_add_arg_aux:n #1 {
  \tl_set_eq:NN \ProcessedArgument \l_xparse_arg_tl
  \use:c { xparse_processor_ \int_use:N \l_xparse_processor_int :n } 
    {#1}
  \int_decr:N \l_xparse_processor_int  
  \int_compare:nTF { \l_xparse_processor_int = \c_zero } {
    \tl_put_right:Nx \l_xparse_args_tl {
      { \exp_not:V \ProcessedArgument }
    }
  }{
    \xparse_add_arg_aux:V \ProcessedArgument
  }  
}
\cs_generate_variant:Nn \xparse_add_arg_aux:n { V }
%    \end{macrocode}
%\end{macro}
%\end{macro}
%\end{macro}
%\end{macro}
%
% All of the grabbers follow the same basic pattern. The initial 
% function sets up the appropriate information to define 
% \cs{parse_grab_arg:w} to grab the argument. This means determining
% whether to use \cs{cs_set:Npn} or \cs{cs_set_nopar:Npn}, and for 
% optional arguments whether to skip spaces. In all cases, 
% \cs{xparse_grab_arg:w} is then called to actually do the grabbing.
%
%\begin{macro}{\xparse_grab_arg:w}
%\begin{macro}[aux]{\xparse_grab_arg_aux_i:w}
%\begin{macro}[aux]{\xparse_grab_arg_aux_ii:w}
% Each time an argument is actually grabbed, \pkg{xparse} defines a
% function to do it. In that way, long arguments from previous functions
% can be included in the definition of the grabber function, so that
% it does not raise an error if not long. The generic function used
% for this is reserved here. A couple of auxiliary functions are also
% needed in various places.
%    \begin{macrocode}
\cs_new:Npn \xparse_grab_arg:w { }
\cs_new:Npn \xparse_grab_arg_aux_i:w { }
\cs_new:Npn \xparse_grab_arg_aux_ii:w { }
%    \end{macrocode}
%\end{macro}
%\end{macro}
%\end{macro}
%
%\begin{macro}{\xparse_grab_D:w}
%\begin{macro}{\xparse_grab_D_long:w}
%\begin{macro}{\xparse_grab_D_trailing:w}
%\begin{macro}{\xparse_grab_D_long_trailing:w}
% The generic delimited argument grabber. The auxiliary function does
% a peek test before calling \cs{xparse_grab_arg:w}, so that the 
% optional nature of the argument works as expected.
%    \begin{macrocode}
\cs_new:Npn \xparse_grab_D:w #1#2#3#4 \l_xparse_args_tl {
  \xparse_grab_D_aux:NNnnNn #1 #2 {#3} {#4} \cs_set_nopar:Npn 
    { _ignore_spaces }
}
\cs_new:Npn \xparse_grab_D_long:w #1#2#3#4 \l_xparse_args_tl {
  \xparse_grab_D_aux:NNnnNn #1 #2 {#3} {#4} \cs_set:Npn 
    { _ignore_spaces }
}
\cs_new:Npn \xparse_grab_D_trailing:w #1#2#3#4 \l_xparse_args_tl {
  \xparse_grab_D_aux:NNnnNn #1 #2 {#3} {#4} \cs_set_nopar:Npn { }
}
\cs_new:Npn \xparse_grab_D_long_trailing:w #1#2#3#4 
  \l_xparse_args_tl {
  \xparse_grab_D_aux:NNnnNn #1 #2 {#3} {#4} \cs_set:Npn { }
}
%    \end{macrocode}
%\begin{macro}[aux]{\xparse_grab_D_aux:NNnnNn}
% This is a bit complicated. The idea is that, in order to check for
% nested optional argument tokens (\texttt{[[...]]} and so on) the
% argument needs to be grabbed without removing any braces at all. If
% this is not done, then cases like |[{[}]| fail. So after testing for
% an optional argument, it is collected piece-wise. First, the opening
% token is removed, then a check is made for a group. If it looks like
% the entire argument is a group, then an extra set of braces are 
% added back in. The closing token is then used to collect everything
% else. There is then a test to see if there is nesting, by looking
% for a `spare' open-argument token. If that is found, things hand off
% to a loop to deal with that.
%    \begin{macrocode}
\cs_new:Npn \xparse_grab_D_aux:NNnnNn #1#2#3#4#5#6 {
  #5 \xparse_grab_arg:w #1 {
    \peek_meaning:NTF \c_group_begin_token {
      \xparse_grab_arg_aux_i:w
    }{
      \xparse_grab_arg_aux_ii:w
    }
  }
  #5 \xparse_grab_arg_aux_i:w ##1 {
    \peek_charcode:NTF #2 {
      \xparse_grab_arg_aux_ii:w { {##1} }
    }{
      \xparse_grab_arg_aux_ii:w {##1}
    }
  }
  #5 \xparse_grab_arg_aux_ii:w ##1 #2 {
    \tl_if_in:nnTF {##1} {#1} {
      \xparse_grab_D_nested:NNNnn #1 #2 {##1} {#4} #5
    }{ 
      \xparse_add_arg:n {##1}
      #4 \l_xparse_args_tl
    }
  }
  \use:c { peek_charcode #6 :NTF } #1 { 
    \xparse_grab_arg:w 
  }{
    \xparse_add_arg:n {#3}
    #4 \l_xparse_args_tl
  }
}
%    \end{macrocode}
%\end{macro}
%\end{macro}
%\end{macro}
%\end{macro}
%\end{macro}
%\begin{macro}[aux]{\xparse_grab_D_nested:NNnNn}
%\begin{macro}{\_l_xparse_nesting_a_tl}
%\begin{macro}{\_l_xparse_nesting_b_tl}
%\begin{macro}{\q_xparse}
% Catching nested optional arguments means more work. The aim here is
% to collect up each pair of optional tokens without \TeX\ helping out,
% and without counting anything. The code above will already have
% removed the leading opening token and a closing token, but the
% wrong one. The aim is then to work through the the material grabbed
% so far and divide it up on each opening token, grabbing a closing
% token to match (thus working in pairs). Once there are no opening
% tokens, then there is a second check to see if there are any
% opening tokens in the second part of the argument (for things 
% like "[][]"). Once everything has been found, the entire collected
% material is added to the output as a single argument.
%    \begin{macrocode}
\tl_new:N \_l_xparse_nesting_a_tl
\tl_new:N \_l_xparse_nesting_b_tl
\quark_new:N \q_xparse
\cs_new_protected:Npn \xparse_grab_D_nested:NNNnn #1#2#3#4#5 {
  \tl_clear:N \_l_xparse_nesting_a_tl
  \tl_clear:N \_l_xparse_nesting_b_tl
  #5 \xparse_grab_arg:w ##1 #1 ##2 \q_xparse ##3 #2
    { 
      \tl_put_right:Nn \_l_xparse_nesting_a_tl { ##1 #1 }
      \tl_put_right:Nn \_l_xparse_nesting_b_tl { #2 ##3 }
      \tl_if_in:nnTF {##2} {#1}
        { \xparse_grab_arg:w ##2 \q_xparse }
        { 
          \tl_if_in:NnTF \_l_xparse_nesting_b_tl {#1}
            {
              \tl_set_eq:NN \l_xparse_tmp_tl \_l_xparse_nesting_b_tl
              \tl_clear:N \_l_xparse_nesting_b_tl
              \exp_last_unbraced:NV \xparse_grab_arg:w
                \l_xparse_tmp_tl \q_xparse
            }
            {
              \tl_put_right:Nn \_l_xparse_nesting_a_tl {##2}
              \tl_put_right:NV \_l_xparse_nesting_a_tl 
                \_l_xparse_nesting_b_tl
              \xparse_add_arg:V \_l_xparse_nesting_a_tl
              #4 \l_xparse_args_tl
          }
        }
    }
  \xparse_grab_arg:w #3 \q_xparse
}
%    \end{macrocode}
%\end{macro}
%\end{macro}
%\end{macro}
%\end{macro}
%
%\begin{macro}{\xparse_grab_G:w}
%\begin{macro}{\xparse_grab_G_long:w}
%\begin{macro}{\xparse_grab_G_trailing:w}
%\begin{macro}{\xparse_grab_G_long_trailing:w}
%\begin{macro}[aux]{\xparse_grab_G_aux:nnNn}
% Optional groups are checked by meaning, so that the same code will 
% work with, for example, Con\TeX{}t-like input.
%    \begin{macrocode}
\cs_new:Npn \xparse_grab_G:w #1#2 \l_xparse_args_tl {
  \xparse_grab_G_aux:nnNn {#1} {#2} \cs_set_nopar:Npn { _ignore_spaces }
}
\cs_new:Npn \xparse_grab_G_long:w #1#2 \l_xparse_args_tl {
  \xparse_grab_G_aux:nnNn {#1} {#2} \cs_set:Npn { _ignore_spaces }
}
\cs_new:Npn \xparse_grab_G_trailing:w #1#2 \l_xparse_args_tl {
  \xparse_grab_G_aux:nnNn {#1} {#2} \cs_set_nopar:Npn { }
}
\cs_new:Npn \xparse_grab_G_long_trailing:w #1#2 \l_xparse_args_tl {
  \xparse_grab_G_aux:nnNn {#1} {#2} \cs_set:Npn { }
}
\cs_set:Npn \xparse_grab_G_aux:nnNn #1#2#3#4 {
  #3 \xparse_grab_arg:w ##1 {
    \xparse_add_arg:n {##1}
    #2 \l_xparse_args_tl
  }
  \use:c { peek_meaning #4 :NTF } \c_group_begin_token { 
    \xparse_grab_arg:w 
  }{
    \xparse_add_arg:n {#1}
    #2 \l_xparse_args_tl
  }
}
%    \end{macrocode}
%\end{macro}
%\end{macro}
%\end{macro}
%\end{macro}
%\end{macro}
%
%\begin{macro}{\xparse_grab_l:w}
%\begin{macro}{\xparse_grab_l_long:w}
%\begin{macro}[aux]{\xparse_grab_l_aux:nN}
% Argument grabbers for mandatory \TeX\ arguments are pretty simple.
%    \begin{macrocode}
\cs_new:Npn \xparse_grab_l:w #1 \l_xparse_args_tl {
  \xparse_grab_l_aux:nN {#1} \cs_set_nopar:Npn
}
\cs_new:Npn \xparse_grab_l_long:w #1 \l_xparse_args_tl { 
  \xparse_grab_l_aux:nN {#1} \cs_set:Npn
}
\cs_new:Npn \xparse_grab_l_aux:nN #1#2 {
  #2 \xparse_grab_arg:w ##1## {
    \xparse_add_arg:n {##1}
    #1 \l_xparse_args_tl
  }
  \xparse_grab_arg:w
}
%    \end{macrocode}
%\end{macro}
%\end{macro}
%\end{macro}
%
%\begin{macro}{\xparse_grab_m:w}
%\begin{macro}{\xparse_grab_m_long:w}
% Collecting a single mandatory argument is quite easy.
%    \begin{macrocode}
\cs_new:Npn \xparse_grab_m:w #1 \l_xparse_args_tl {
  \cs_set_nopar:Npn \xparse_grab_arg:w ##1 {
    \xparse_add_arg:n {##1}
    #1 \l_xparse_args_tl
  }
  \xparse_grab_arg:w
}
\cs_new:Npn \xparse_grab_m_long:w #1 \l_xparse_args_tl {
  \cs_set:Npn \xparse_grab_arg:w ##1 {
    \xparse_add_arg:n {##1}
    #1 \l_xparse_args_tl
  }
  \xparse_grab_arg:w
}
%    \end{macrocode}
%\end{macro}
%\end{macro}
%
%\begin{macro}{\xparse_grab_m_1:w}
%\begin{macro}{\xparse_grab_m_2:w}
%\begin{macro}{\xparse_grab_m_3:w}
%\begin{macro}{\xparse_grab_m_4:w}
%\begin{macro}{\xparse_grab_m_5:w}
%\begin{macro}{\xparse_grab_m_6:w}
%\begin{macro}{\xparse_grab_m_7:w}
%\begin{macro}{\xparse_grab_m_8:w}
% Grabbing 1--8 mandatory arguments. We don't need to worry about 
% nine arguments as this is only possible if everything is 
% mandatory. Each function has an auxiliary so that \cs{par} tokens
% from other arguments still work.
%    \begin{macrocode}
\cs_new:cpn { xparse_grab_m_1:w } #1 \l_xparse_args_tl {
  \cs_set_nopar:Npn \xparse_grab_arg:w ##1 {
    \tl_put_right:Nn \l_xparse_args_tl { {##1} }
    #1 \l_xparse_args_tl
  }
  \xparse_grab_arg:w
}
\cs_new:cpn { xparse_grab_m_2:w } #1 \l_xparse_args_tl {
  \cs_set_nopar:Npn \xparse_grab_arg:w ##1##2 {
    \tl_put_right:Nn \l_xparse_args_tl { {##1} {##2} }
    #1 \l_xparse_args_tl
  }
  \xparse_grab_arg:w
}
\cs_new:cpn { xparse_grab_m_3:w } #1 \l_xparse_args_tl {
  \cs_set_nopar:Npn \xparse_grab_arg:w ##1##2##3 {
    \tl_put_right:Nn \l_xparse_args_tl { {##1} {##2} {##3} }
    #1 \l_xparse_args_tl
  }
  \xparse_grab_arg:w
}
\cs_new:cpn { xparse_grab_m_4:w } #1 \l_xparse_args_tl {
  \cs_set_nopar:Npn \xparse_grab_arg:w ##1##2##3##4 {
    \tl_put_right:Nn \l_xparse_args_tl { {##1} {##2} {##3} {##4} }
    #1 \l_xparse_args_tl
  }
  \xparse_grab_arg:w
}
\cs_new:cpn { xparse_grab_m_5:w } #1 \l_xparse_args_tl {
  \cs_set_nopar:Npn \xparse_grab_arg:w ##1##2##3##4##5 {
    \tl_put_right:Nn \l_xparse_args_tl { 
      {##1} {##2} {##3} {##4} {##5} 
    }
    #1 \l_xparse_args_tl
  }
  \xparse_grab_arg:w
}
\cs_new:cpn { xparse_grab_m_6:w } #1 \l_xparse_args_tl {
  \cs_set_nopar:Npn \xparse_grab_arg:w ##1##2##3##4##5##6 {
    \tl_put_right:Nn \l_xparse_args_tl { 
      {##1} {##2} {##3} {##4} {##5} {##6}
    }
    #1 \l_xparse_args_tl
  }
  \xparse_grab_arg:w
}
\cs_new:cpn { xparse_grab_m_7:w } #1 \l_xparse_args_tl {
  \cs_set_nopar:Npn \xparse_grab_arg:w ##1##2##3##4##5##6##7 {
    \tl_put_right:Nn \l_xparse_args_tl { 
      {##1} {##2} {##3} {##4} {##5} {##6} {##7} 
    }
    #1 \l_xparse_args_tl
  }
  \xparse_grab_arg:w
}
\cs_new:cpn { xparse_grab_m_8:w } #1 \l_xparse_args_tl {
  \cs_set_nopar:Npn \xparse_grab_arg:w ##1##2##3##4##5##6##7##8 {
    \tl_put_right:Nn \l_xparse_args_tl { 
      {##1} {##2} {##3} {##4} {##5} {##6} {##7} {##8} 
    }
    #1 \l_xparse_args_tl
  }
  \xparse_grab_arg:w
}
%    \end{macrocode}
%\end{macro}
%\end{macro}
%\end{macro}
%\end{macro}
%\end{macro}
%\end{macro}
%\end{macro}
%\end{macro}
%
%\begin{macro}{\xparse_grab_t:w}
%\begin{macro}{\xparse_grab_t_long:w}
%\begin{macro}{\xparse_grab_t_trailing:w}
%\begin{macro}{\xparse_grab_t_long_trailing:w}
%\begin{macro}[aux]{\xparse_grab_t_aux:NnNn}
% Dealing with a token is quite easy. Check the match, remove the 
% token if needed and add a flag to the output.
%    \begin{macrocode}
\cs_new:Npn \xparse_grab_t:w #1#2 \l_xparse_args_tl {
  \xparse_grab_t_aux:NnNn #1 {#2} \cs_set_nopar:Npn { _ignore_spaces }
}
\cs_new:Npn \xparse_grab_t_long:w #1#2 \l_xparse_args_tl {
  \xparse_grab_t_aux:NnNn #1 {#2} \cs_set:Npn { _ignore_spaces }
}
\cs_new:Npn \xparse_grab_t_trailing:w #1#2 \l_xparse_args_tl {
  \xparse_grab_t_aux:NnNn #1 {#2} \cs_set_nopar:Npn { }
}
\cs_new:Npn \xparse_grab_t_long_trailing:w #1#2 \l_xparse_args_tl {
  \xparse_grab_t_aux:NnNn #1 {#2} \cs_set:Npn { }
}
\cs_new:Npn \xparse_grab_t_aux:NnNn #1#2#3#4 {
  #3 \xparse_grab_arg:w {
    \use:c { peek_charcode_remove #4 :NTF } #1 { 
      \xparse_add_arg:n { \BooleanTrue }
      #2 \l_xparse_args_tl
    }{
      \xparse_add_arg:n { \BooleanFalse }
      #2 \l_xparse_args_tl
    }
  }
  \xparse_grab_arg:w
}
%    \end{macrocode}
%\end{macro}
%\end{macro}
%\end{macro}
%\end{macro}
%\end{macro}
%
%\begin{macro}{\xparse_grab_u:w}
%\begin{macro}{\xparse_grab_u_long:w}
%\begin{macro}[aux]{\xparse_grab_u_aux:NnN}
% Grabbing up to a list of tokens is quite easy: define the grabber,
% and then collect.
%    \begin{macrocode}
\cs_new:Npn \xparse_grab_u:w #1#2 \l_xparse_args_tl {
  \xparse_grab_u_aux:NnN {#1} {#2} \cs_set_nopar:Npn
}
\cs_new:Npn \xparse_grab_u_long:w #1#2 \l_xparse_args_tl {
  \xparse_grab_u_aux:NnN {#1} {#2} \cs_set:Npn
}
\cs_new:Npn \xparse_grab_u_aux:NnN #1#2#3 {
  #3 \xparse_grab_arg:w ##1 #1 {
    \xparse_add_arg:n {##1}
    #2 \l_xparse_args_tl
  }
  \xparse_grab_arg:w
}
%    \end{macrocode}
%\end{macro}
%\end{macro}
%\end{macro}
%
%\subsection{Argument processors}
%
%\begin{macro}{\xparse_process_to_str:n}
% A basic argument processor: as much an example as anything else.
%    \begin{macrocode}
\cs_new:Npn \xparse_process_to_str:n #1 {
  \tl_set:Nx \ProcessedArgument {
    \tl_to_str:n {#1}
  }
}
%    \end{macrocode}
%\end{macro}
%
%\begin{macro}{\_parse_bool_reverse:N}
% A simple reversal.
%    \begin{macrocode}
\cs_new_protected:Npn \_xparse_bool_reverse:N #1 {
  \bool_if:NTF #1
    { \tl_set:Nn \ProcessedArgument { \c_false_bool } }
    { \tl_set:Nn \ProcessedArgument { \c_true_bool } }
}
%    \end{macrocode}
%\end{macro}
%
%\begin{macro}{\_l_xparse_split_argument_tl}
%\begin{macro}{\_xparse_split_argument:nnn}
%\begin{macro}[aux]{\_xparse_split_argument_aux_i:w}
%\begin{macro}[aux]{\_xparse_split_argument_aux_ii:w}
%\begin{macro}[aux]{\_xparse_split_argument_aux_iii:w}
% The idea of this function is to split the input \( n + 1 \) times using
% a given token.
%    \begin{macrocode}
\tl_new:N \_l_xparse_split_argument_tl
\group_begin:
  \char_make_active:N \@
  \cs_gset_protected:Npn \_xparse_split_argument:nnn #1#2#3 
    {
      \tl_set:Nn \_l_xparse_split_argument_tl {#3}
      \group_begin:
      \char_set_lccode:nn { `\@ } { `#2} 
      \tl_to_lowercase:n
        {
          \group_end:
          \tl_replace_all_in:Nnn \_l_xparse_split_argument_tl { @ } {#2}
        }
      \cs_set_protected:Npn \_xparse_split_argument_aux_i:w
        ##1 \q_mark ##2 #2 ##3 \q_stop
        {
          \tl_put_right:Nn \ProcessedArgument { {##2} }
          ##1 \q_mark ##3 \q_stop
        }
      \cs_set_protected:Npn \_xparse_split_argument_aux_iii:w
        ##1 #2 ##2 \q_stop
        {
          \IfNoValueF {##1} 
            { 
              \msg_kernel_error:nnxxx { xparse } { split-excess-tokens }
                { \exp_not:n {#2} } {#1} { \exp_not:n {#3} }
            }
        }
      \tl_set:Nx \l_xparse_tmp_tl
        {
          \prg_replicate:nn { #1 + 1 }
            { \_xparse_split_argument_aux_i:w }
          \_xparse_split_argument_aux_ii:w
          \exp_not:N \q_mark
          \exp_not:V \_l_xparse_split_argument_tl
          \prg_replicate:nn {#1} { \exp_not:n {#2} \NoValue }
          \exp_not:n { #2 \q_stop }
        }  
      \l_xparse_tmp_tl  
    }
\group_end:
\cs_set_protected:Npn \_xparse_split_argument_aux_i:w { }
\cs_set_protected:Npn \_xparse_split_argument_aux_ii:w 
  #1 \q_mark #2 \q_stop 
  {
    \tl_if_empty:nF {#2}
      { \_xparse_split_argument_aux_iii:w #2 \q_stop } 
  }
\cs_set_protected:Npn \_xparse_split_argument_aux_iii:w { }
%    \end{macrocode}
%\end{macro}
%\end{macro}
%\end{macro}
%\end{macro}
%\end{macro}
%
%\begin{macro}{\_l_xparse_split_list_tl}
%\begin{macro}{\_xparse_split_list:nn}
%\begin{macro}{\_xparse_split_list_aux:w}
% Splitting a list is done again by first dealing with active
% characters, then looping over the list using the same method as the
% \cs{clist_map_\ldots} functions. 
%    \begin{macrocode}
\tl_new:N \_l_xparse_split_list_tl
\group_begin:
  \char_make_active:N \@
  \cs_gset_protected:Npn \_xparse_split_list:nn #1#2
    {
      \tl_set:Nn \_l_xparse_split_list_tl {#2}
      \group_begin:
      \char_set_lccode:nn { `\@ } { `#1 }
      \tl_to_lowercase:n
        {
          \group_end:
          \tl_replace_all_in:Nnn \_l_xparse_split_list_tl { @ } {#1}
        }
      \cs_set:Npn \_xparse_split_list_aux:w ##1 #1 
        {
          \quark_if_recursion_tail_stop:n {##1}
          \tl_put_right:Nn \ProcessedArgument { {##1} }
          \_xparse_split_list_aux:w
        }
      \tl_if_empty:NTF \_l_xparse_split_list_tl
        { \tl_set:Nn \ProcessedArgument { { } } }  
        {
          \tl_clear:N \ProcessedArgument
          \exp_last_unbraced:NV \_xparse_split_list_aux:w
            \_l_xparse_split_list_tl #1 
            \q_recursion_tail #1 \q_recursion_stop
        }
    }
\group_end:
\cs_set:Npn \_xparse_split_list_aux:w { }
%    \end{macrocode}
%\end{macro}
%\end{macro}
%\end{macro}
%
%\subsection{Creating expandable functions}
%
% The trick here is to pass each grabbed argument along a chain of
% auxiliary functions. Each one ultimately calls the next in the chain,
% so that all of the arguments are passed along delimited using
% \cs{q_xparse}. At the end of the chain, the marker is removed
% so that the user-supplied code can be passed the correct number
% of arguments. All of this is done by expansion!
%
%\begin{macro}{\xparse_exp_add_type_d:w}
% As in the standard case, the trick here is to slot in the default
% and treat as type \texttt{D}.
%    \begin{macrocode}
\cs_new:Npn \xparse_exp_add_type_d:w #1#2 {
  \xparse_exp_add_type_D:w #1 #2 { \NoValue }
}
%    \end{macrocode}
%\end{macro}
%
%\begin{macro}{\xparse_exp_add_type_D:w}
% The most complex argument to grab in an expandable manner is the 
% general delimited one. First, a short-cut is set up in 
% \cs{l_xparse_tmp_tl} for the name of the current grabber function.
% This is then created to grab one argument and test if it is equal
% to the opening delimiter. If the test fails, the code adds the default
% value and closing delimiter before `recycling' the argument. In either
% case, the second auxiliary function is called. It finds the closing
% delimiter and so the optional argument (if any). The function then
% calls the next one in the chain, passing along the argument(s)
% grabbed thus-far using \cs{q_xparse} as a marker.
%    \begin{macrocode}
\cs_new:Npn \xparse_exp_add_type_D:w #1#2#3 {
  \tl_set:Nx \l_xparse_tmp_tl {
    \exp_after:wN \token_to_str:N \l_xparse_function_tl 
    \int_use:N \l_xparse_total_args_int
  }
  \xparse_exp_set:cpx { \l_xparse_tmp_tl } ##1 \q_xparse ##2 {
    \exp_not:N \tl_if_head_eq_charcode:nNTF {##2} #1 {
      \exp_not:c { \l_xparse_tmp_tl aux } 
        ##1 \exp_not:N \q_xparse  
    }{
      \exp_not:c { \l_xparse_tmp_tl aux } 
        ##1 \exp_not:N \q_xparse  #3 #2 {##2}  
    } 
  }
  \xparse_exp_set:cpx { \l_xparse_tmp_tl aux} ##1 \q_xparse ##2 #2 {
    \exp_not:c {
      \exp_after:wN \token_to_str:N \l_xparse_function_tl 
      \int_eval:n { \l_xparse_total_args_int + 1 }
    } ##1 {##2} \exp_not:N \q_xparse
  }
  \xparse_exp_prepare_function:N
}
%    \end{macrocode}
%\end{macro}
%
%\begin{macro}{\xparse_exp_add_type_l:w}
%\begin{macro}{\xparse_exp_add_type_m:w}
% Gathering \texttt{l} and \texttt{m} arguments is almost the same. 
% The grabber for the current argument is created to simply get the 
% necessary argument and pass it along with any others through to the 
% next function in the chain.
%    \begin{macrocode}
\cs_new_nopar:Npn \xparse_exp_add_type_l:w {
  \xparse_exp_set:cpx { 
    \exp_after:wN \token_to_str:N \l_xparse_function_tl 
    \int_use:N \l_xparse_total_args_int
  } ##1 \q_xparse ##2## {
    \exp_not:c {
      \exp_after:wN \token_to_str:N \l_xparse_function_tl 
      \int_eval:n { \l_xparse_total_args_int + 1 }
    }
    ##1 {##2} \exp_not:N \q_xparse
  }
  \xparse_exp_prepare_function:N
}
\cs_new_nopar:Npn \xparse_exp_add_type_m:w {
  \int_incr:N \l_xparse_m_args_int
  \xparse_exp_set:cpx { 
    \exp_after:wN \token_to_str:N \l_xparse_function_tl 
    \int_use:N \l_xparse_total_args_int
  } ##1 \q_xparse ##2 {
    \exp_not:c {
      \exp_after:wN \token_to_str:N \l_xparse_function_tl 
      \int_eval:n { \l_xparse_total_args_int + 1 }
    }
    ##1 {##2} \exp_not:N \q_xparse
  }
  \xparse_exp_prepare_function:N
}
%    \end{macrocode}
%\end{macro}
%\end{macro}
%
%\begin{macro}{\xparse_exp_add_type_t:w}
% Looking for a single token is a simpler version of the \texttt{D}
% code. The same idea of picking up one argument is used, but there is
% no need for a second function as there is no closing token to find. So
% either \cs{BooleanTrue} or \cs{BooleanFalse} are added to the list of
% arguments. In the later case, the grabber argument must be `recycled'.
%    \begin{macrocode}
\cs_new:Npn \xparse_exp_add_type_t:w #1 {
  \tl_set:Nx \l_xparse_tmp_tl {
    \exp_after:wN \token_to_str:N \l_xparse_function_tl 
    \int_eval:n { \l_xparse_total_args_int + 1 }
  }
  \xparse_exp_set:cpx {
    \exp_after:wN \token_to_str:N \l_xparse_function_tl 
    \int_use:N \l_xparse_total_args_int
  } ##1 \q_xparse ##2 {
    \exp_not:N \tl_if_head_eq_charcode:nNTF {##2} #1 {
      \exp_not:c { \l_xparse_tmp_tl }
      ##1 \exp_not:n { { \BooleanTrue } \q_xparse }
    }{
      \exp_not:c { \l_xparse_tmp_tl }
      ##1 \exp_not:n { { \BooleanFalse } \q_xparse {##2} }
    } 
  }
  \xparse_exp_prepare_function:N
}
%    \end{macrocode}
%\end{macro}
%
%\begin{macro}{\xparse_exp_add_type_u:w}
% Setting up for a \texttt{u} argument is a case of defining the 
% grabber for the current argument in a delimited fashion. The rest of
% the process is as the other grabbers: add to the chain and call the 
% next function.
%    \begin{macrocode}
\cs_new:Npn \xparse_exp_add_type_u:w #1 {
  \xparse_exp_set:cpx { 
    \exp_after:wN \token_to_str:N \l_xparse_function_tl 
    \int_use:N \l_xparse_total_args_int
  } ##1 \q_xparse ##2 #1 {
    \exp_not:c {
      \exp_after:wN \token_to_str:N \l_xparse_function_tl 
      \int_eval:n { \l_xparse_total_args_int + 1 }
    }
    ##1 {##2} \exp_not:N \q_xparse
  }
  \xparse_exp_prepare_function:N
}
%    \end{macrocode}
%\end{macro}
%
%\begin{macro}{\xparse_exp_check_and_add:N}
% Virtually identical to the normal version, except calling the 
% expandable \texttt{add} functions rather than the standard versions.
%    \begin{macrocode}
\cs_new_nopar:Npn \xparse_exp_check_and_add:N  #1 {
  \cs_if_free:cTF { xparse_exp_add_type_ #1 :w } {
    \msg_kernel_error:nnx { xparse } { unknown-argument-type } {#1}
    \tl_set:Nn \l_xparse_last_arg_tl { m }
    \xparse_exp_add_type_m:w
  }{
    \tl_set:Nn \l_xparse_last_arg_tl {#1}
    \use:c { xparse_exp_add_type_ #1 :w }
  }
}
%    \end{macrocode}
%\end{macro}
%
%\begin{macro}{\xparse_exp_declare_cmd:Nnn}
%\begin{macro}[aux]{\xparse_exp_declare_cmd_all_m:Nn}
%\begin{macro}[aux]{\xparse_exp_declare_cmd_mixed:Nn}
%\begin{macro}[aux]{\xparse_exp_declare_cmd_mixed_aux:Nn}
% The overall scheme here is very different from the standard method.
% For each argument, an internal function is created to grab an argument
% and pass along previous ones. Each `daisy chains' to call the next 
% one in the sequence. Thus at the end of the chain, an extra `argument'
% function is included to unwind the chain and pass data to the the
% internal function containing the actual code. If all of the arguments
% are type \texttt{m}, then the same tick is used as in the standard
% version. The \texttt{x} in the lead-off and mop-up functions makes 
% sure that the braces around the first argument are not lost.
%    \begin{macrocode}
\cs_new:Npn \xparse_exp_declare_cmd:Nnn #1#2#3 {
  \cs_if_exist:NTF #1 
    { 
      \msg_kernel_warning:nnxx { xparse } { redefine-command } 
        { \token_to_str:N #1 } { \exp_not:n {#2} }
    }
    { 
      \msg_kernel_info:nnxx { xparse } { define-command } 
        { \token_to_str:N #1 } { \exp_not:n {#2} }
    }
  \tl_set:Nn \l_xparse_function_tl {#1}
  \xparse_exp_prepare_function:n {#2}
  \int_compare:nTF { 
    \l_xparse_total_args_int = \l_xparse_m_args_int 
  } {
    \xparse_exp_declare_cmd_all_m:Nn #1 {#3}
  }{
    \xparse_exp_declare_cmd_mixed:Nn #1 {#3}
  }
}
\cs_new:Npn \xparse_exp_declare_cmd_all_m:Nn #1#2 {
  \bool_if:NTF \l_xparse_long_bool {
    \cs_generate_from_arg_count:NNnn 
      #1 \cs_set:Npn \l_xparse_total_args_int {#2}
  }{
    \cs_generate_from_arg_count:NNnn 
      #1 \cs_set_nopar:Npn \l_xparse_total_args_int {#2}
  }
}
\cs_new:Npn \xparse_exp_declare_cmd_mixed:Nn #1#2 {
  \exp_args:NnV \tl_if_in:nnTF { l m u } \l_xparse_last_arg_tl {
    \xparse_exp_declare_cmd_mixed_aux:Nn #1 {#2}
  }{
    \msg_kernel_error:nn { xparse } { expandable-ending-optional }
  }
}
\cs_new:Npn \xparse_exp_declare_cmd_mixed_aux:Nn #1#2 {  
  \cs_set_nopar:Npx #1 {
    \exp_not:c { \token_to_str:N #1 1 } x \exp_not:N \q_xparse
  }
  \cs_set_nopar:cpx { 
    \token_to_str:N #1 \int_eval:n { \l_xparse_total_args_int + 1  }
  } x ##1 \q_xparse {
    \exp_not:c { \token_to_str:N #1 } ##1
  }
  \cs_generate_from_arg_count:cNnn 
    { \token_to_str:N #1 } \cs_set:Npn \l_xparse_total_args_int {#2}
}
%    \end{macrocode}
%\end{macro}
%\end{macro}
%\end{macro}
%\end{macro}
%
%\begin{macro}{\xparse_exp_prepare_function:n}
%\begin{macro}[aux]{\xparse_exp_prepare_function_aux:n}
% A couple of early validation tests. Processors are forbidden, as are
% \texttt{g}, \texttt{l} and \texttt{u} arguments (the later more for
% ease than any technical reason).
%    \begin{macrocode}
\cs_new:Npn \xparse_exp_prepare_function:n  #1 {
  \bool_set_false:N \l_xparse_error_bool
  \tl_if_in:nnT {#1} { > } {
    \msg_kernel_error:nnx { xparse } { processor-in-expandable } {#1}
    \bool_set_true:N \l_xparse_error_bool
  }
  \tl_if_in:nnT {#1} { g } {
    \msg_kernel_error:nnx { xparse } { grouped-in-expandable } 
      { g } {#1}
    \bool_set_true:N \l_xparse_error_bool
  }
  \tl_if_in:nnT {#1} { G } {
    \msg_kernel_error:nnx { xparse } { grouped-in-expandable } 
      { G } {#1}
    \bool_set_true:N \l_xparse_error_bool
  }
  \bool_if:NF \l_xparse_error_bool {
    \xparse_exp_prepare_function_aux:n {#1}
  }
}
\cs_new:Npn \xparse_exp_prepare_function_aux:n  #1 {
  \cs_set_eq:NN \xparse_prepare_next:w \xparse_exp_prepare_function:N
  \cs_set_eq:NN \xparse_exp_set:cpx \cs_set_nopar:cpx
  \bool_set_false:N \l_xparse_long_bool
  \int_zero:N \l_xparse_m_args_int
  \int_zero:N \l_xparse_total_args_int
  \tl_if_in:nnT {#1} { + } {
    \bool_set_true:N \l_xparse_long_bool
    \cs_set_eq:NN \xparse_exp_set:cpx \cs_set:cpx
  }
  \xparse_exp_prepare_function:N #1 \q_nil
}
%    \end{macrocode}
%\end{macro}
%\end{macro}
%
%\begin{macro}{\xparse_exp_prepare_function:N}
%\begin{macro}[aux]{\xparse_exp_prepare_function_long:N}
%\begin{macro}[aux]{\xparse_exp_prepare_function_short:N}
% Preparing functions is a case of reading the signature, as in the
% normal case. However, everything has to be either short or long, and
% so there is an extra step to make sure that once one \texttt{+} has
% been seen everything has one. That detour then takes us back to 
% a standard looping concept.
%    \begin{macrocode}
\cs_new:Npn \xparse_exp_prepare_function:N #1 {
  \bool_if:NTF \l_xparse_long_bool {
    \xparse_exp_prepare_function_long:N #1
  }{
    \xparse_exp_prepare_function_short:N #1
  }
}
\cs_new:Npn \xparse_exp_prepare_function_long:N #1 {
  \quark_if_nil:NF #1 {
    \str_if_eq:nnTF {#1} { + } {
      \xparse_exp_prepare_function_short:N
    }{
      \msg_kernel_error:nn { xparse } { expandable-inconsistent-long }
      \xparse_exp_prepare_function_short:N #1
    }
  }
}
\cs_new:Npn \xparse_exp_prepare_function_short:N #1 {
  \quark_if_nil:NF #1 {
    \prop_if_in:NnTF \c_xparse_shorthands_prop {#1} {
      \prop_get:NnN \c_xparse_shorthands_prop {#1} \l_xparse_tmp_tl
      \bool_if:NT \l_xparse_long_bool {
        \tl_put_left:Nn \l_xparse_tmp_tl { + }
      }
      \exp_last_unbraced:NV \xparse_exp_prepare_function:N 
        \l_xparse_tmp_tl
    }{
      \int_incr:N \l_xparse_total_args_int
      \xparse_exp_check_and_add:N #1
    }
  }
}
%    \end{macrocode}
%\end{macro}
%\end{macro}
%\end{macro}
%
%\begin{macro}{\xparse_exp_set:cpx}
% A short-cut to save constantly re-testing \cs{l_xparse_long_bool}.
%    \begin{macrocode}
\cs_new_eq:NN \xparse_exp_set:cpx \cs_set_nopar:cpx
%    \end{macrocode}
%\end{macro}
%
%\subsection{Access to the argument specification}
%
%\begin{macro}{\parse_get_arg_spec:N}
%\begin{macro}{\parse_get_arg_spec:n}
%\begin{macro}{\ArgumentSpecification}
% Recovering the argument specification is also trivial, using the
% \cs{tl_set_eq:cN} function.
%    \begin{macrocode}
\cs_new_protected:Npn \parse_get_arg_spec:N #1 {
  \cs_if_exist:cTF { l_parse_ \token_to_str:N #1 _arg_spec_tl }
    {
      \tl_set_eq:Nc \ArgumentSpecification
        { l_parse_ \token_to_str:N #1 _arg_spec_tl }
    }
    {
      \msg_kernel_error:nx { unknown-document-command }
        { \token_to_str:N #1 }
    }
}
\cs_new_protected:Npn \parse_get_arg_spec:n #1 {
  \cs_if_exist:cTF { l_parse_ #1 _arg_spec_tl }
    { \tl_set_eq:Nc \ArgumentSpecification { l_parse_ #1 _arg_spec_tl } }
    { \msg_kernel_error:nx { unknown-document-environment } {#1} }
}
\tl_new:N \ArgumentSpecification
%    \end{macrocode}
%\end{macro}
%\end{macro}
%\end{macro}
%
%\begin{macro}{\parse_show_arg_spec:N}
%\begin{macro}{\parse_show_arg_spec:n}
% Showing the argument specification simply means finding it and then
% calling the \cs{tl_show:c} function.
%    \begin{macrocode}
\cs_new_protected:Npn \parse_show_arg_spec:N #1 {
  \cs_if_exist:cTF { l_parse_ \token_to_str:N #1 _arg_spec_tl }
    {
      \tl_set_eq:Nc \ArgumentSpecification
        { l_parse_ \token_to_str:N #1 _arg_spec_tl }
      \tl_show:N \ArgumentSpecification
    }
    {
      \msg_kernel_error:nx { unknown-document-command }
        { \token_to_str:N #1 }
    }
}
\cs_new_protected:Npn \parse_show_arg_spec:n #1 {
  \cs_if_exist:cTF { l_parse_ #1 _arg_spec_tl }
    {
      \tl_set_eq:Nc \ArgumentSpecification { l_parse_ #1 _arg_spec_tl } 
      \tl_show:N \ArgumentSpecification
    }
    { \msg_kernel_error:nx { unknown-document-environment } {#1} }
}
%    \end{macrocode}
%\end{macro}
%\end{macro}
%    
%\subsection{Messages}
%    
% Some messages intended as errors.
%    \begin{macrocode}
\msg_kernel_new:nnnn { xparse } { command-already-defined } 
  { Command~'#1'~already~defined! }
  {
    You~have~used~\token_to_str:N \NewDocumentCommand \\
    with~a~command~that~already~has~a~definition.\\
    Perhaps~you~meant~\token_to_str:N \RenewDocumentCommand.
  }
\msg_kernel_new:nnnn { xparse } { command-not-yet-defined } 
  { Command ~'#1'~not~yet~defined! }
  {
    You~have~used~\token_to_str:N \RenewDocumentCommand \\
    with~a~command~that~was~never~defined.\\
    Perhaps~you~meant~\token_to_str:N \NewDocumentCommand.
  }
\msg_kernel_new:nnnn { xparse } { environment-already-defined } 
  { Environment~'#1'~already~defined! }
  {
    You~have~used~\token_to_str:N \NewDocumentEnvironment \\
    with~a~command~that~already~has~a~definition.\\
    Perhaps~you~meant~\token_to_str:N \RenewDocumentEnvironment.
  }
\msg_kernel_new:nnnn { xparse } { environment-not-yet-defined } 
   {Environment~'#1'~not~yet~defined! }
  {
    You~have~used~\token_to_str:N \RenewDocumentEnvironment \\
    with~a command~that~was~never~defined.\\
    Perhaps~you~meant~\token_to_str:N \NewDocumentEnvironment.
  }
\msg_kernel_new:nnnn { xparse } { expandable-ending-optional } 
  {
    Signature~for~expandable~command~ends~with \\
    optional~argument~\msg_line_context:.
  }
  {
    Expandable~commands~must~have~a~final~mandatory~argument \\
    (or~no arguments~at all).~You~cannot~have~a~terminal~optional \\
    argument~with~expandable~commands.
  } 
\msg_kernel_new:nnnn { xparse } { expandable-inconsistent-long } 
  {
    Inconsistent~handling~of~long~arguments~for \\
    expandable~command~\msg_line_context:.
  }
  {
    The~arguments~for~an~expandable~command~must~either~all~be \\
    short~or~all~be~long.~You~have~tried~to~mix~the~two~types.
  }
\msg_kernel_new:nnnn { xparse } { grouped-in-expandable } 
  {%
    Argument~specifier~'#1'~forbidden~in~expandable~commands~
    \msg_line_context:.
  }
  {
    Argument~specification~'#2'~contains~the~optional~grouped~
    argument~#1':\\
    this~is only~supported~for~standard~robust~functions.
  }
\msg_kernel_new:nnnn { xparse } { processor-in-expandable } 
  {
    Argument~processors~cannot~be~used \\
    with~expandable~functions~\msg_line_context:.
  }
  {
    Argument~specification~'#1'~contains~a~processor~function:\\
    this~is~only~supported~for~standard~robust~functions.
  }
\msg_kernel_set:nnnn { xparse } { split-excess-tokens } 
  { Too~many~'#1'~tokens~when~trying~to~split~argument. }
  {
    LaTeX~was~asked~to~split~the~input~'#3'\\
    at~each~occurrence~of~the~token~'#1',~up~to~a~maximum~of~#2~tokens.\\
    There~were~too~many~'#1'~tokens.
  }
\msg_kernel_new:nnnn { xparse } { unknown-argument-type } 
  { Unknown~argument~type~'#1'~replaced~by~'m'.~Fingers~crossed~... }
  {
    The~letter~#1'~does~not~specify~a~known~argument~type.\\
    I'm~assuming~you~want~a~standard~mandatory~argument~(type~'m').
  }
%    \end{macrocode}
%    
% Intended more for information.
%    \begin{macrocode}
\msg_kernel_new:nnn { xparse } { define-command } 
  {
    Defining~document~command~#1\\
    with~arg.~spec.~'#2'~\msg_line_context:.
  }
\msg_kernel_new:nnn { xparse } { define-environment } 
  {
    Defining~document~environment~'#1'\\
    with~arg.~spec.~'#2'~\msg_line_context:.
  }
\msg_kernel_new:nnn { xparse } { redefine-command } 
  {
    Redefining~document~command~#1\\
    with~arg.~spec.~'#2'~\msg_line_context:.
  }
\msg_kernel_new:nnn { xparse } { redefine-environment } 
  {
    Redefining~document~environment~'#1'\\
    with~arg.~spec.~'#2'~\msg_line_context:.
  }
%    \end{macrocode}
%    
%\subsection{User functions}
%
% The user functions are more or less just the internal functions 
% renamed. 
% 
%\begin{macro}{\BooleanFalse}
%\begin{macro}{\BooleanTrue}
% Design-space names for the Boolean values.
%    \begin{macrocode}
\cs_new_eq:NN \BooleanFalse \c_false_bool
\cs_new_eq:NN \BooleanTrue  \c_true_bool
%    \end{macrocode}
%\end{macro}
%\end{macro}
% 
%\begin{macro}{\DeclareDocumentCommand}
%\begin{macro}{\NewDocumentCommand}
%\begin{macro}{\RenewDocumentCommand}
%\begin{macro}{\ProvideDocumentCommand}
% The user macros are pretty simple wrappers around the internal ones.
%    \begin{macrocode}
\cs_new_protected:Npn \DeclareDocumentCommand #1#2#3 {
  \xparse_declare_cmd:Nnn #1 {#2} {#3}
}
\cs_new_protected:Npn \NewDocumentCommand #1#2#3 {
  \cs_if_exist:NTF #1 { 
    \msg_kernel_error:nnx { xparse } { command-already-defined } 
      { \token_to_str:N #1 }
  }{
    \xparse_declare_cmd:Nnn #1 {#2} {#3}
  }
}
\cs_new_protected:Npn \RenewDocumentCommand #1#2#3 {
  \cs_if_exist:NTF #1 {
    \xparse_declare_cmd_aux:Nnn #1 {#2} {#3}
    \msg_kernel_info:nnxx { xparse } { redefine-command }
      { \token_to_str:N #1 } { \exp_not:n {#2} }
  }{  
    \msg_kernel_error:nnx { xparse } { command-not-yet-defined } 
      { \token_to_str:N #1 }
  }
}
\cs_new_protected:Npn \ProvideDocumentCommand #1#2#3 {
  \cs_if_exist:NF #1 { 
    \xparse_declare_cmd:Nnn #1 {#2} {#3}
  }
}
%    \end{macrocode}
%\end{macro}
%\end{macro}
%\end{macro}
%\end{macro}
%
%\begin{macro}{\DeclareDocumentCommandImplementation}
%\begin{macro}{\DeclareDocumentCommandInterface}
% The separate implementation/interface system is again pretty simple
% to create at the outer layer.
%    \begin{macrocode}
\cs_new_protected:Npn \DeclareDocumentCommandImplementation #1#2#3 {
  \xparse_declare_cmd_implementation:nNn {#1} #2 {#3}
}
\cs_new_protected:Npn \DeclareDocumentCommandInterface #1#2#3 {
  \xparse_declare_cmd_interface:Nnn #1 {#2} {#3}
}
%    \end{macrocode}
%\end{macro}
%\end{macro}
%
%\begin{macro}{\DeclareDocumentEnvironment}
%\begin{macro}{\NewDocumentEnvironment}
%\begin{macro}{\RenewDocumentEnvironment}
%\begin{macro}{\ProvideDocumentEnvironment}
% Very similar for environments.
%    \begin{macrocode}
\cs_new_protected:Npn \DeclareDocumentEnvironment #1#2#3#4 {
  \xparse_declare_env:nnnn {#1} {#2} {#3} {#4}
}
\cs_new_protected:Npn \NewDocumentEnvironment #1#2#3#4 {
%</initex|package>
%<*initex>
  \cs_if_exist:cTF { environment_begin_ #1 :w } { 
%</initex>
%<*package>
  \cs_if_exist:cTF {#1} { 
%</package>
%<*initex|package>
    \msg_kernel_error:nnx { xparse } 
      { environment-already-defined } {#1}
  }{
    \xparse_declare_env:nnnn {#1} {#2} {#3} {#4}
  }
}
\cs_new_protected:Npn \RenewDocumentEnvironment #1#2#3#4 {
%</initex|package>
%<*initex>
  \cs_if_exist:cTF { environment_begin_ #1 :w } { 
%</initex>
%<*package>
  \cs_if_exist:cTF {#1} { 
%</package>
%<*initex|package>
    \xparse_declare_env:nnnn {#1} {#2} {#3} {#4}
  }{  
    \msg_kernel_error:nnx { xparse } 
      { environment-not-yet-defined } {#1}
  }
}
\cs_new_protected:Npn \ProvideDocumentEnvironment #1#2#3#4 {
%</initex|package>
%<*initex>
  \cs_if_exist:cF { environment_begin_ #1 :w } { 
%</initex>
%<*package>
  \cs_if_exist:cF { #1 } { 
%</package>
%<*initex|package>
    \xparse_declare_env:nnnn {#1} {#2} {#3} {#4}
  }
}
%    \end{macrocode}
%\end{macro}
%\end{macro}
%\end{macro}
%\end{macro}
%
%\begin{macro}{\DeclareExpandableDocumentCommand}
% The expandable version of the basic function is essentially the same.
%    \begin{macrocode}
\cs_new_protected:Npn \DeclareExpandableDocumentCommand #1#2#3 {
  \xparse_exp_declare_cmd:Nnn #1 {#2} {#3}
}
%    \end{macrocode}
%\end{macro}
%
%\begin{macro}[TF]{\IfBoolean}
% The logical \meta{true} and \meta{false} statements are just the 
% normal \cs{c_true_bool} and \cs{c_false_bool}, so testing for them is
% done with the \cs{bool_if:NTF} functions from \textsf{l3prg}.
%    \begin{macrocode}
\cs_new_eq:NN \IfBooleanTF \bool_if:NTF
\cs_new_eq:NN \IfBooleanT  \bool_if:NT
\cs_new_eq:NN \IfBooleanF  \bool_if:NF
%    \end{macrocode}
%\end{macro}
%
%\begin{macro}[TF]{\IfNoValue}
% Simple re-naming. 
%    \begin{macrocode}
\cs_new_eq:NN \IfNoValueF  \xparse_if_no_value:nF 
\cs_new_eq:NN \IfNoValueT  \xparse_if_no_value:nT 
\cs_new_eq:NN \IfNoValueTF \xparse_if_no_value:nTF 
%    \end{macrocode}
%\end{macro}
%\begin{macro}[TF]{\IfValue}
% Inverted logic.
%    \begin{macrocode}
\cs_set:Npn \IfValueF { \xparse_if_no_value:nT }
\cs_set:Npn \IfValueT { \xparse_if_no_value:nF }
\cs_set:Npn \IfValueTF #1#2#3 { 
  \xparse_if_no_value:nTF {#1} {#3} {#2}
}
%    \end{macrocode}
%\end{macro}
%
%\begin{macro}{\NoValue}
% The marker for no value being give: this can be typeset safely.
% This is coded by hand as making it \cs{protected} ensures that it
% will not turn into anything else by accident.
%    \begin{macrocode}
\cs_new_protected:Npn \NoValue { -NoValue- }
%    \end{macrocode}
%\end{macro}
%
%\begin{macro}{\ProcessedArgument}
% Processed arguments are returned using this name, which is reserved
% here although the definition will change.
%    \begin{macrocode}
\cs_new:Npn \ProcessedArgument { }
%    \end{macrocode}
%\end{macro}
%
%\begin{macro}{\ReverseBoolean}
% A processor to reverse the logic for token detection.
%    \begin{macrocode}
\cs_new_eq:NN \ReverseBoolean \_xparse_bool_reverse:N
%    \end{macrocode}
%\end{macro}
%
%\begin{macro}{\SplitArgument}
%\begin{macro}{\SplitList}
% Another simple copy.
%    \begin{macrocode}
\cs_new_eq:NN \SplitArgument \_xparse_split_argument:nnn
\cs_new_eq:NN \SplitList     \_xparse_split_list:nn
%    \end{macrocode}
%\end{macro}
%\end{macro}
%
%\begin{macro}{\GetDocumentCommandArgSpec}
%\begin{macro}{\GetDocumentEnvironmentArgSpec}
%\begin{macro}{\ShowDocumentCommandArgSpec}
%\begin{macro}{\ShowDocumentEnvironmentArgSpec}
% More simple mappings.
%    \begin{macrocode}
\cs_new_eq:NN \GetDocumentCommandArgSpec      \parse_get_arg_spec:N
\cs_new_eq:NN \GetDocumentEnvironmmentArgSpec \parse_get_arg_spec:n
\cs_new_eq:NN \ShowDocumentCommandArgSpec     \parse_show_arg_spec:N
\cs_new_eq:NN \ShowDocumentEnvironmentArgSpec \parse_show_arg_spec:n
%    \end{macrocode}
%\end{macro}
%\end{macro}
%\end{macro}
%\end{macro}
%
%    \begin{macrocode}
%</initex|package>
%    \end{macrocode}
%
% \end{implementation}
% 
% \PrintIndex

% \iffalse
%% File xo-trace.dtx (C) Copyright 1999-2000 Frank Mittelbach, David Carlisle, Chris Rowley
%%                   (C) Copyright 2004-2009 Frank Mittelbach, LaTeX3 Project
%%
%% It may be distributed and/or modified under the conditions of the
%% LaTeX Project Public License (LPPL), either version 1.3c of this
%% license or (at your option) any later version.  The latest version
%% of this license is in the file
%%
%%    http://www.latex-project.org/lppl.txt
%%
%% This file is part of the ``xor bundle'' (The Work in LPPL)
%% and all files in that bundle must be distributed together.
%%
%% The released version of this bundle is available from CTAN.
%%
%% -----------------------------------------------------------------------
%%
%% The development version of the bundle can be found at
%%
%%    http://www.latex-project.org/svnroot/experimental/trunk/
%%
%% for those people who are interested.
%%
%%%%%%%%%%%
%% NOTE: %%
%%%%%%%%%%%
%%
%%   Snapshots taken from the repository represent work in progress and may
%%   not work or may contain conflicting material!  We therefore ask
%%   people _not_ to put them into distributions, archives, etc. without
%%   prior consultation with the LaTeX Project Team.
%%
%% -----------------------------------------------------------------------
%%
\RequirePackage{l3names}
\GetIdInfo $Id$
          {xo-trace}
\ProvidesExplPackage{\filename}
  {\filedate}{\fileversion}{\filedescription}
% \fi
%
%
%
% \begin{macro}{\showmarks}
%    Show the current meaning of the different marks on the page.
%    \begin{macrocode}
\cs_set_nopar:Npn \showmarks {
  \toks_set:No \l_tmpa_toks{\topmark}
  \typeout{topmark:~ \toks_use:N\l_tmpa_toks}
  \toks_set:No \l_tmpa_toks{\firstmark}
  \typeout{firstmark:~ \toks_use:N\l_tmpa_toks}
  \toks_set:No \l_tmpa_toks{\botmark}
  \typeout{botmark:~ \toks_use:N\l_tmpa_toks}
}
%    \end{macrocode}
% \end{macro}
%
% \begin{macro}{\showfloatlists}
%
%    \begin{macrocode}
\cs_new_nopar:Npn \showfloatlists{{
  \typeout{free:~  \meaning\g_xor_floats_free_seq}
  \typeout{active:~\meaning\g_xor_floats_active_seq}
  \typeout{mvl:~   \meaning\g_xor_floats_mvl_seq}
  \typeout{defer:~ \meaning\g_xor_area_DDD_float_seq}
  \typeout{here:~ \meaning\g_xor_floats_here_seq }

  \clist_map_function:NN \g_xor_areas_used_clist
                \showfloatarea

  \edef\@tempa{currbox\expandafter
               \strip@prefix\meaning\g_xor_curr_float_box_tl>\space
              this@float\expandafter
               \strip@prefix\meaning\g_xor_this_float_box_tl>}
  \show\@tempa}}
%    \end{macrocode}
% \end{macro}
%
%
% \begin{macro}{\showfloatarea}
%    Show the contents of a single area for debugging purposes.
%    \begin{macrocode}
\cs_new_nopar:Npn \showfloatarea#1{
  \cs_if_free:cT{g_xor_area_#1_float_seq}
     {
      \seq_if_empty:cF {g_xor_area_#1_float_seq}
          { \typeout{~\space\space #1:~
%FMi tmp
                    \cs_meaning:c {g_xor_area_#1_float_seq}} }
     }
}
%    \end{macrocode}
%

%  \begin{macro}{\trace_display:n}
%    This macro unconditionally displays some tracing information on the
%    screen.
%    \begin{macrocode}
%<*trace>
\cs_new_nopar:Npn \trace_display:n #1
  {
%    \end{macrocode}
%    In case we want to show the content of some sequence we locally redefine
%    their data structures (may have to be done for other types as well).
%    \begin{macrocode}
    \seq_push_item_def:n { ##1 \space }
    \typeout { :~\g_trace_prefix_tl~#1 }
    \seq_pop_item_def:
  }
%    \end{macrocode}
%  \end{macro}
%
% 
%  \begin{macro}{g_trace_prefix_tl}
%    \begin{macrocode}
\tl_new:N\g_trace_prefix_tl
%    \end{macrocode}
%  \end{macro}
%
%  \begin{macro}{\trace_push:n}
%    \begin{macrocode}
\cs_new_nopar:Npn \trace_push:n #1 {
  \trace:n{entering~#1}
  \tl_gput_right:Nn \g_trace_prefix_tl{-}}
%    \end{macrocode}
%  \end{macro}
%
%
%  \begin{macro}{\trace_pop:n}
%    \begin{macrocode}
\cs_new_nopar:Npn \trace_pop:n #1 {
%    \end{macrocode}
%    To get rid of one |-| inside |\g_trace_prefix_tl| we expand it grab one
%    token and reassign the rest. Kind of a ``poor man's'' pop.
%    \begin{macrocode}
  \tl_gset:Nx \g_trace_prefix_tl {\exp_after:wN \use_none:n \g_trace_prefix_tl}
  \trace:n{leaving~#1}
}
%    \end{macrocode}
%  \end{macro}
%
%
%  \begin{macro}{\tracefloats}
%  \begin{macro}{\notrace}
%    \begin{macrocode}
\cs_new_nopar:Npn \tracefloats{\cs_gset_eq:NN \trace:n \trace_display:n}
\cs_new_nopar:Npn \notrace    {\cs_gset_eq:NN \trace:n \use_none:n}
%    \end{macrocode}
%    The default is not to trace.
%    \begin{macrocode}
\notrace
%    \end{macrocode}
%  \end{macro}
%  \end{macro}
%
%
%
%
%  \begin{macro}{\traceonly}
%    Tracing only some (at the moment exactly one) function and its subcomponents.
%    \begin{macrocode}
\cs_new_nopar:Npn \traceonly#1{
  \notrace
  \tl_set:Nn \l_trace_only_tl {#1}
%    \end{macrocode}
%    |\trace_push:n| is redefined to compare the current function on the stack
%    (its argument) with the function to be traced. If the desired function is
%    entered the current |\g_trace_prefix_tl| is stored away in
%    |\g_trace_this_prefix_tl| and tracing is started.\footnote{fix this is
%    rubbish ... the idea should be to allow for recursive entering the same
%    function  ... fix some other time :-) }
%    \begin{macrocode}
  \cs_gset_nopar:Npn \trace_push:n ##1 {
    \tl_set:Nn \l_trace_this_only_tl {##1}
    \if_meaning:w \l_trace_this_only_tl \l_trace_only_tl
      \tl_gset_eq:NN \g_trace_this_prefix_tl \g_trace_prefix_tl
      \tracefloats
      \trace:n{entering~##1}
    \fi:
    \tl_gset:Nn \g_trace_prefix_tl {-\g_trace_prefix_tl}}
%    \end{macrocode}
%    
%    \begin{macrocode}
  \cs_set_nopar:Npn \trace_pop:n ##1 {
    \trace:n{leaving~##1}
    \tl_gset:Nn \g_trace_prefix_tl{\exp_after:wN \use_none:n \g_trace_prefix_tl}
    \if_meaning:w \g_trace_this_prefix_tl \g_trace_prefix_tl
      \tl_gset_eq:NN \g_trace_this_prefix_tl \c_empty_tl
      \notrace
    \fi:}
}
%    \end{macrocode}
%  \end{macro}
%
%    \begin{macrocode}
\tracefloats
%    \end{macrocode}
%
%    \begin{macrocode}
%</trace>
%<-trace>\cs_set_eq:NN \trace:n \use_none:n
%    \end{macrocode}
%


%  \begin{macro}{\xor_progress:n}
%  \begin{macro}{\xor_progress_newline:n}
%  \begin{macro}{\xor_progress_failed:n}
%  \begin{macro}{\xor_progress_separator:}
%    
%    \begin{macrocode}
%<*progress>
\cs_new_nopar:Npn \xor_progress:n #1{\message{#1}}
\cs_new_nopar:Npn \xor_progress_newline:n #1{\xor_progress:n{~ #1^^J}}
\cs_new_nopar:Npn \xor_progress_failed:n  #1{\xor_progress_newline:n{ ->~ failed:~ #1}}
\cs_new_nopar:Npn \xor_progress_separator:{
       \xor_progress:n{===============================
                      ========================================^^J}}
%</progress>
%    \end{macrocode}
%  \end{macro}
%  \end{macro}
%  \end{macro}
%  \end{macro}

% \iffalse
%%
%% (C) Copyright 1999-2000 Frank Mittelbach, David Carlisle, Chris Rowley
%% All rights reserved.
%%
%% Not for general distribution. In its present form it is not allowed
%% to put this package onto CD or an archive without consulting the
%% the authors.
%% 
% \fi
%

%    \begin{macrocode}
\def\@tempa#1: #2.dtx,v #3 #4 #5 #6 #7${
  \ProvidesPackage{#2}[#4 #3 #5 #6]}
\@tempa$Id$
%    \end{macrocode}
%
%
%
% Ignore white space in this package.
%    \begin{macrocode}
\IgnoreWhiteSpace
%    \end{macrocode}
%
%
%
% \subsection{Selecting the next area to try}
%
%
% \begin{macro}{\std@try@this@area}
%    The |\std@try@this@area| command is called when we have
%    selected a float from the |\g_xor_active_floats_seq| and want to place it
%    in one of the open areas to see if it fits there. If we succeed
%    in finding a potential candidate place we will exit using
%    |\contruct@trial| which will run a trial cutting the galley to
%    see if each column gets the right kind of text and contains the
%    right kind of callouts.
%    The areas are tried in a defined order.
%
%    If we don't find any open areas we return putting the current
%    float onto the |\g_xor_area_DDD_seq| and then calling |\@trynextfloat|.
%
%    If we find an open area but for some reason the float is not
%    allowed to go there we close that area for floats of this type
%    and recurse (i.e. try to use the next open area for this type).
%
%    The reasons for failure to place the float into the first open
%    area are numerous. First there are restrictions on the number of
%    floats on a page (if we reach this all areas get closed) there
%    there are similar restrictions for each individual float area.
%
%    If those tests are passed we trial typeset the float with its
%    caption (or rather typeset and attach its caption) to determine
%    the size needed by the float. That will give us a dimension to
%    test against restrictions for the amount of space floats are
%    allowed to occupy etc.
%
%    \begin{macrocode}
\def\std@try@this@area{
%<*trace>
  \@tracepush{std@try@this@area}
%</trace>
%    \end{macrocode}
%    If all areas for the float type are closed we have to defer this float.
%    \begin{macrocode}
  \ifx\this@open@areas\@empty
%<*trace>
        \tr@ce{defer:~no~open~area~ available}
%</trace>
        \do@next\defer@and@try@next@float
  \else
%    \end{macrocode}
%    Otherwise the next area to try is the first of |\this@open@areas|:
%    \begin{macrocode}
    \setup@this@area{\expandafter\@carcube\this@open@areas\@nil}
%<*trace>
    \tr@ce{area~trial:~ \this@area}
%</trace>
%<*progress>
    \progress{~ area~trial:~ \this@area}
%</progress>
%    \end{macrocode}
%    Next test needs cleanup once the span has a decent data
%    structure.\footnote{FIX!!}
%    \begin{macrocode}
    \ifnum \this@area@span@number = \if!\this@span@number! 1\else  % big hack
                                    \this@span@number \fi
                                    \relax
%<*trace>
      \tr@ce{span~ count~ okay:~ \this@area \space =~ 
             \if!\this@span@number! 1\else  % big hack
                                    \this@span@number \fi}
%</trace>
%    \end{macrocode}
%
%    \begin{macrocode}
      \ifnum \pagesetup@max@float@num > \page@float@count
%    \end{macrocode}
%
%    \begin{macrocode}
        \ifnum\csname g_xor_area_ \this@area _max_float_num \endcsname =
              \csname g_xor_area_ \this@area     _float_int \endcsname
%
%    \end{macrocode}
%    Current area doesn't accept any more floats, so try next one (if any)
%    \begin{macrocode}
%<*trace>
          \tr@ce{close~area:~\this@area\space float~num~reached ~
                 (\csname g_xor_area_ \this@area _max_float_num\endcsname)}
%</trace>
%<*progress>
          \progress@failed{\this@area\space float~num~reached ~
                 (\csname g_xor_area_ \this@area _max_float_num\endcsname)}
%</progress>
          \do@next\try@next@area
%    \end{macrocode}
%
%    \begin{macrocode}
        \else
          \xin@\this@area
               \g_xor_curr_page_closed_areas_clist
          \ifin@
%<*trace>
            \tr@ce{area~ closed~ for~ all~ types;~ member~ of~
                   (\g_xor_curr_page_closed_areas_clist)}
%</trace>
%<*progress>
            \progress@failed{area~ closed~ for~ all~ types}
%</progress>
            \do@next\try@next@area
          \else
%<*trace>
            \tr@ce{area~ not~ closed~ for~ all~ types:~ not~ member~ of~
                   (\g_xor_curr_page_closed_areas_clist)}
%</trace>
            \xin@\this@area
                 \this@closed@areas
            \ifin@
%<*trace>
              \tr@ce{area~ closed~ for~ class~ \this@class;~ member~ of~
                     (\this@closed@areas)}
%</trace>
%<*progress>
              \progress@failed
                    {area~ closed~ for~ class~ \this@class}
%</progress>
              \do@next\try@next@area
            \else
%<*trace>
              \tr@ce{area~ open~ for~ class~ \this@class;~ not~ member~ of~
                     (\this@closed@areas)}
%</trace>
              \xin@\this@area\this@fps
              \ifin@
%
                \append@caption@to@float
                \construct@and@test@col@hts
                \if@test
                  \do@next\try@next@area
                \else
                  \do@next\pretests@success@action
                \fi

              \else
%<*trace>
                \tr@ce{close~area:~\this@area\space float~not~allowed~
                       by~ user~ control~ (\this@fps)}
%</trace>
%<*progress>
                \progress@failed{\this@area\space float~not~allowed~
                       by~ user~ control~ (\this@fps)}
%</progress>
                \do@next\try@next@area
              \fi
            \fi
          \fi
        \fi
      \else
% defer
        \global\let\this@open@areas\@empty
%<*trace>
        \tr@ce{defer:~max~float~num~reached ~(\pagesetup@max@float@num)}
%</trace>
%<*progress>
        \progress@failed{max~float~num~reached ~(\pagesetup@max@float@num)}
%</progress>
        \do@next\defer@and@try@next@float
      \fi
    \else
%<*trace>
      \tr@ce{span~ count~ unsuitable:~ \this@area \space /=~ 
             \if!\this@span@number! 1\else  % big hack
                                    \this@span@number \fi}
%</trace>
%<*progress>
      \progress@failed{span~ count~ \this@area \space /=~ 
             \if!\this@span@number! 1\else  % big hack
                                    \this@span@number \fi}
%</progress>
      \do@next\try@next@area
    \fi
  \fi
%<*trace>
  \@tracepop{std@try@this@area}
%</trace>
  \do@continue
}
%    \end{macrocode}
% \end{macro}
%
%
%
%
% \begin{macro}{\do@next}
%    Emergency macro to reduce the number of input levels (as the code
%    got past the internal default of 300). This is because we have a
%    lot of recursion going on without being tail
%    recursion. Essentially this means some of the code needs slightly
%    different implementation.\footnote{fix}
%    \begin{macrocode}
\def\do@next{\let\do@continue}
%    \end{macrocode}
% \end{macro}
%
% \begin{macro}{\relaxed@try@this@area}
%    \begin{macrocode}
\def\relaxed@try@this@area{
%<*trace>
  \@tracepush{relaxed@try@this@area}
%</trace>
%    \end{macrocode}
%    If all areas for the float type are closed we have to defer this float.
%    \begin{macrocode}
  \ifx\this@open@areas\@empty
%<*trace>
        \tr@ce{defer:~no~open~area~ available}
%</trace>
        \do@next\defer@and@try@next@float
  \else
%    \end{macrocode}
%    Otherwise the next area to try is the first of |\this@open@areas|:
%    \begin{macrocode}
    \setup@this@area{\expandafter\@carcube\this@open@areas\@nil}
%<*trace>
    \tr@ce{area~trial:~ \this@area}
%</trace>
%<*progress>
    \progress{~ area~trial:~ \this@area}
%</progress>
%    \end{macrocode}
%    Next test needs cleanup once the span has a decent data
%    structure.\footnote{FIX!!}
%    \begin{macrocode}
    \ifnum \this@area@span@number = \if!\this@span@number! 1\else  % big hack
                                    \this@span@number \fi
                                    \relax
%<*trace>
      \tr@ce{span~ count~ okay:~ \this@area \space =~ 
             \if!\this@span@number! 1\else  % big hack
                                    \this@span@number \fi}
%</trace>
%    \end{macrocode}
%
%    \begin{macrocode}
      \xin@\this@area\g_xor_curr_page_closed_areas_clist
      \ifin@
%<*trace>
        \tr@ce{area~ closed~ for~ all~ types;~ member~ of~
               (\g_xor_curr_page_closed_areas_clist)}
%</trace>
%<*progress>
            \progress@failed{area~ closed~ for~ all~ types}
%</progress>
        \do@next\try@next@area
      \else
%<*trace>
        \tr@ce{area~ not~ closed~ for~ all~ types:~ not~ member~ of~
               (\g_xor_curr_page_closed_areas_clist)}
%</trace>
        \xin@\this@area
             \this@closed@areas
        \ifin@
%<*trace>
          \tr@ce{area~ closed~ for~ class~ \this@class;~ member~ of~
                 (\this@closed@areas)}
%</trace>
%<*progress>
          \progress@failed
                    {area~ closed~ for~ class~ \this@class}
%</progress>
          \do@next\try@next@area
        \else
%<*trace>
          \tr@ce{area~ open~ for~ class~ \this@class;~ not~ member~ of~
                 (\this@closed@areas)}
%</trace>
          \append@caption@to@float
          \construct@and@test@col@hts
          \if@test
            \do@next\try@next@area
          \else
            \do@next\pretests@success@action
          \fi
        \fi
      \fi
    \else
%<*trace>
      \tr@ce{span~ count~ unsuitable:~ \this@area \space /=~ 
             \if!\this@span@number! 1\else  % big hack
                                    \this@span@number \fi}
%</trace>
%<*progress>
      \progress@failed{span~ count~ \this@area \space /=~ 
             \if!\this@span@number! 1\else  % big hack
                                    \this@span@number \fi}
%</progress>
      \do@next\try@next@area
    \fi
  \fi
%<*trace>
  \@tracepop{relaxed@try@this@area}
%</trace>
  \do@continue
}
%    \end{macrocode}
% \end{macro}
%
%
%
%
% \begin{macro}{\construct@and@test@col@hts}
%    The |\construct@and@test@col@hts| loops over all columns
%    affected by the float area we want to place our float into
%    and reduces the column size as
%    needed. It sets the switch |@test| to true in case the float
%    doesn't fit into area for some reason. It is up to the calling
%    macro to take proper action in this case (including resetting
%    column heights to their former values).
%    \begin{macrocode}
\def\construct@and@test@col@hts {
%    \end{macrocode}
%    We use the information from the area not from the float (which
%    allows us to put small floats into larger areas.
%    \begin{macrocode}
  \update@this@area@columns
     {
      \expandafter
      \construct@and@test@col@ht
         \csname g_xor_ht_col_ \the\count@ _dim \endcsname
         {\the\count@}
%    \end{macrocode}
%    We need to break out of the updating loop if we found a column that
%    doesn't work, a) to save time and b) since the
%    |\construct@and@test@col@ht| resets the switch again to false
%    (in the current implementation).
%    \begin{macrocode}
      \if@test
         \count@\z@   % break out of loop
      \fi
     }
}
%    \end{macrocode}
% \end{macro}
%
%
%
% \begin{macro}{\construct@and@test@col@ht}
%    The |\construct@and@test@col@ht| reduces the height of one column
%    by the size of the current float (plus some suitable extra space)
%    and tests if this reduction is possible/allowed. It sets the
%    switch |@test| to true if it encounters a problem.
%    \begin{macrocode}
\def\construct@and@test@col@ht#1#2{
%<*trace>
  \tr@ce{col height~before:~ \the#1 (\string#1)}
%</trace>
%    \end{macrocode}
%    In |\@tempdima| we calculate the amount of space we need if we
%    place the float. This is the size of the float, i.e., its height
%    and depth, plus either |\pagesetup@float@text@sep| if this is the
%    very first float in this column (on top or on bottom), or
%    |\pagesetup@float@area@sep| if it is the first float in the
%    current area, or |\pagesetup@float@float@sep| if we already have
%    floats in this area.
%    \begin{macrocode}
  \@tempdima          \ht\this@captioned@float
  \advance \@tempdima \dp\this@captioned@float
  \ifnum \csname g_xor_ \this@area@type
                 _floats_col_ #2 _num 
         \endcsname
          = \z@
%<*trace>
       \tr@ce{first~ float~ in~ any~ \this@area@type\space 
              in~ column~ #2:~
              adding~ \string\pagesetup@float@text@sep
              =\the\pagesetup@float@text@sep}
%</trace>
    \advance \@tempdima \pagesetup@float@text@sep
  \else
    \advance \@tempdima
    \seq_if_empty:cTF {g_xor_area_ \this@area _seq}
        {
         \pagesetup@float@area@sep
%<*trace>
         \tr@ce{first~ float~ in~ \this@area :~
                adding~ \string\pagesetup@float@area@sep
                =\the\pagesetup@float@area@sep}
%</trace>
        }
        {
         \pagesetup@float@float@sep 
%<*trace>
         \tr@ce{additional~ float~ in~ \this@area :~
                adding~ \string\pagesetup@float@float@sep
                = \the\pagesetup@float@float@sep}
%</trace>
        }
  \fi
%    \end{macrocode}
%    Now we make the area size fully fall into the page grid (if there
%    is one). To do this we first substract any delta that has been
%    accumulated for the column (to get back to the real size) and
%    then run |\snap@to@grid| to get the next grid point.
%    \begin{macrocode}
  \advance \@tempdima 
          -\csname g_xor_ \this@area@type _delta_col_ #2 _tlp \endcsname \relax
%    \end{macrocode}
%    The |\g_xor_|\meta{area-type}|_delta_col_|\meta{col}|_tlp| are macros but
%    fortunately low-level \TeX{} supports register assignments of the
%    form |--3pt| so the above will work even if the macro contains a
%    negative value.
%    \begin{macrocode}
  \snap@to@grid  \@tempdima \pagesetup@grid@point@sep
%    \end{macrocode}
%
%    If the distance to the next grid point is larger than the
%    distance to the previous one, and if the space between floats and
%    text is allowed to shrink by the needed amount we will use the
%    previous grid point.
%    \begin{macrocode}
  \ifdim \returned@lower@delta@size < \returned@delta@size \relax
    \ifdim \returned@lower@delta@size < \pagesetup@float@text@shrink
%<*trace>
      \tr@ce{GRID:~ column~#2:~ choosing~ lower~ grid~ point}
%</trace>
%    \end{macrocode}
%    We do this by copying the |\returned@lower@...| to the macros
%    used below. Note that for the delta size we have to use the
%    negation since we want to backup by this amount and delta sizes
%    are always given in absolute values.
%    \begin{macrocode}
      \global\let \returned@size \returned@lower@size
      \xdef \returned@delta@size {-\returned@lower@delta@size }
    \fi
  \fi
%<*trace>
  \tr@ce{GRID:~ column~#2:~ \the\@tempdima\space ->~ \returned@size}
%</trace>
  \@tempdima \returned@size
%    \end{macrocode}
%    
%    After updating |\@tempdima| in this way we now have to check if
%    the space remaining for the text column is large enough.
%    \begin{macrocode}
  \@tempdimb#1
  \advance \@tempdimb -\@tempdima
  \@testfalse
  \ifdim \textminlines\baselineskip > \@tempdimb 
%    \end{macrocode}
%    If not we bail out
%    \begin{macrocode}
     \@testtrue
%<*trace>
     \tr@ce{close~area:~\this@area\space not~ enough~ text~ lines~
       left~
       (\textminlines x\the\baselineskip> \the\@tempdimb)}
%</trace>
%<*progress>
     \progress@failed{not~ enough~ text~ space~
       (\textminlines x\the\baselineskip\space
        >~ \the\@tempdimb)}
%</progress>
  \else
%    \end{macrocode}
%    Otherwise we have to update the column heights (and their saved
%    version in case we back out later) as well as storing the new
%    delta (which is still in |\returned@delta@size|).
%    \begin{macrocode}
%<*trace>
     \tr@ce{saved@col@ht@ #2~ <-~\the#1}
%</trace>
     \expandafter\xdef
        \csname saved@col@ht@ #2\endcsname{\the#1}
     \global#1\@tempdimb
     \tlp_gset_eq:cc
        {g_xor_saved_ \this@area@type _delta_col_ #2 _tlp}
        {g_xor_       \this@area@type _delta_col_ #2 _tlp}
     \tlp_gset_eq:cN
        {g_xor_ \this@area@type _delta_col_ #2 _tlp}
        \returned@delta@size
%<*trace>
     \tr@ce{GRID (delta):~ column~#2\this@area@type :~
        \tlp_use:c{g_xor_saved_ \this@area@type _delta_col_ #2 _tlp}
        \space ->~ \returned@delta@size}
%</trace>
  \fi
%<*trace>
  \tr@ce{col height~after:~ \the#1}
%</trace>
}
%    \end{macrocode}
% \end{macro}
%
%    
% \begin{macro}{\textminlines}
%    Tmp definition; should be interfaced to pagesetup.
%    \begin{macrocode}
\def\textminlines{4}
%    \end{macrocode}
% \end{macro}
%
%    
%
% \subsection{Supporting grid based designs}
%
%
%
% \begin{macro}{\pagesetup@grid@point@sep}
%    |\pagesetup@grid@point@sep| is the distance between grid points
%    for page design or 0pt if we are not doing grid design.
%    \begin{macrocode}
\let\pagesetup@grid@point@sep\ERROR
%    \end{macrocode}
% \end{macro}
%
%
% \begin{macro}{\snap@to@grid}
%    |\snap@to@grid| takes a dimension as first argument and a
%    ``grid-size'' as second argument and from those two values
%    calculates a new dimension which is a multiple of the grid-size
%    and equal or larger than the first argument.
%
%    It returns this calculated value globally in |\returned@size|. In
%    addition it will return in |\returned@delta@size| the delta
%    between the original and the new dimension.
%
%    It also returns a lower or equal grid point in
%    |\returned@lower@size| and the delta to it (absolute value) in
%    |\returned@lower@delta@size|. 
%
%    Thus the first argument lies on a grid point if and only if
%    |\returned@size| equals |\returned@lower@size|.
%
%    If we are not doing grid typesetting then |\snap@to@grid| is
%    actually a dummy which will return always the first argument in
%    |\returned@size| and |\returned@lower@size|, and |0pt| in the
%    deltas.
%
%    The actual definition is assigned within the page setup.
%    \begin{macrocode}
\let\snap@to@grid\ERROR
%    \end{macrocode}
% \end{macro}
%
%
% \begin{macro}{\dummy@snap@to@grid}
%    In case we don't do grids we simply pretend that all sizes lie on
%    grid points.
%    \begin{macrocode}
\def \dummy@snap@to@grid #1#2{
   \begingroup
%    \end{macrocode}
%    Definitions below can be optimised for speed but the ones used
%    give better tracing.
%    \begin{macrocode}
     \@tempdima #1 \relax
     \xdef \returned@size { \the\@tempdima }
     \global\let \returned@lower@size \returned@size
     \gdef \returned@delta@size {0pt}
     \global\let \returned@lower@delta@size  \returned@delta@size
   \endgroup
}
%    \end{macrocode}
% \end{macro}
%
%
% \begin{macro}{\std@snap@to@grid}
%    But if we do grid typesetting we have to do a bit more work
%    here.
%    \begin{macrocode}
\def \std@snap@to@grid #1#2{
%    \end{macrocode}
%    Everything is done in a group so that the register change do not
%    affect other parts of the code.
%    \begin{macrocode}
  \begingroup
%    \end{macrocode}
%    First we calculate in |\@tempdimc| the nearest grid point which
%    is smaller or equal to the given size in |#1| by using scaled
%    point arithmetic.
%    \begin{macrocode}
     \@tempdima #1\relax
     \@tempdimb #2\relax
     \@tempcnta \@tempdima      % orig size in sp
     \@tempcntb \@tempdimb      % grid size in sp
     \divide \@tempcnta \@tempcntb
     \@tempdimc \@tempcnta\@tempdimb
%    \end{macrocode}
%    If the calculated grid point is smaller than the original
%    dimension use the next larger one. But record the other in
%    |\returned@lower@size|.
%    \begin{macrocode}
     \ifdim \@tempdimc < \@tempdima
       \xdef \returned@lower@size { \the\@tempdimc }
       \advance\@tempdimc \@tempdimb 
     \fi
%    \end{macrocode}
%    The |\returned@size| will be whatever the result of the above is
%    and |\returned@delta@size| the difference to the given value in
%    |#1|. Note that |\returned@size| will always be larger or equal
%    to this value so the difference computed below will also be
%    non-negative!
%    \begin{macrocode}
     \xdef \returned@size { \the\@tempdimc }
     \advance \@tempdimc -\@tempdima
     \xdef \returned@delta@size { \the\@tempdimc }
%    \end{macrocode}
%    Now if this difference turns out to be zero then we actually
%    started with a dimension exactly on a grid point. In this case we
%    better define both |\returned@lower@size| and
%    |\returned@lower@delta@size| to equal their counterparts.
%    \begin{macrocode}
     \ifdim \@tempdimc = \z@
       \global \let \returned@lower@size \returned@size
       \global \let \returned@lower@delta@size \returned@delta@size
%    \end{macrocode}
%    Otherwise we only have to compute |\returned@lower@delta@size|
%    since in that case we already recorded |\returned@lower@size|
%    above (and we better had since by now we lost this information as
%    we reused |\@tempdimc|). Now the value for this macro should be
%    the difference between |\returned@delta@size| and our grid size
%    both of which are already stored in some temp registers,
%    thus\ldots
%    \begin{macrocode}
     \else
       \advance\@tempdimb-\@tempdimc
       \xdef \returned@lower@delta@size { \the\@tempdimb }
     \fi
  \endgroup
%<*trace>
  \tr@ce{GRID:~ \string\returned@lower@size=\returned@lower@size}
  \tr@ce{GRID:~ \string\returned@lower@delta@size=\returned@lower@delta@size}
  \tr@ce{GRID:~ \string\returned@size=\returned@size}
  \tr@ce{GRID:~ \string\returned@delta@size=\returned@delta@size}
%</trace>
}
%    \end{macrocode}
% \end{macro}
%
%
%
%
%
% \subsection{Deferring a float}
%
%
%
% \begin{macro}{\really@defer@and@try@next@float}
%    \begin{macrocode}
\def\really@defer@and@try@next@float{
%<*progress>
        \progress@failed{-->~ defer}
        \progress@nl{}
%</progress>
      \seq_gput_right:No \g_xor_area_DDD_seq \this@float@box
%<*trace>
      \tr@ce{g_xor_curr_page_closed_areas_clist~ <-~ "\g_xor_curr_page_closed_areas_clist"~ +~
             "\g_xor_DDD_all_close_clist"}
%</trace>
      \clist_gput_right:No \g_xor_curr_page_closed_areas_clist
                           \g_xor_DDD_all_close_clist

      \if_meaning:NN \g_xor_DDD_class_close_clist \pagesetup@area@list
%<*trace>
        \tr@ce{this@closed@areas~ <-~ 
               "\g_xor_DDD_class_close_clist"~ (all~ closed)}
%</trace>
        \clist_gset_eq:NN \this@closed@areas \g_xor_DDD_class_close_clist
      \else
%<*trace>
        \tr@ce{this@closed@areas~ <-~ "\this@closed@areas"~ +~
               "\g_xor_DDD_class_close_clist"}
%</trace>
        \xdef\this@closed@areas{\this@closed@areas,
                   \g_xor_DDD_class_close_clist}
      \fi

      \global\expandafter\let\csname closed@\this@class
                                     @areas\endcsname % FMi tmp
                             \this@closed@areas
      \try@next@float
}
%    \end{macrocode}
% \end{macro}
% 
%    
%
% \begin{macro}{\defer@and@try@next@float}
%
%    This code either defers the current float and restarts by calling
%    |\try@next@float| or recurses after relaxing the placement
%    conditions (in case the current float should be flushed).
%
%    If the placement conditions are already relaxed we move the first
%    flush point that affects the current float one column forward and
%    then retry. This might result in succeding to place the float, if
%    not it will ultiamtely result in the flush point being moved from
%    the page at which point the float can be deferred.
%
%    \begin{macrocode}
\def\defer@and@try@next@float{
%<*trace>
  \@tracepush{defer@and@try@next@float}
%</trace>
%    \end{macrocode}
%    If there have been no flush points or if the flush points seen do
%    not affect the current float we can immediately defer it.
%    \begin{macrocode}
  \if@flushseen
    \xin@\this@class\flush@classes@list@max
    \ifin@
%<*trace>
      \tr@ce{flush:~ class~ \this@class\space in~
             \flush@classes@list@max}
%</trace>
%    \end{macrocode}
%    First test is just there to avoid doing any of the loops 
%    (might not be worth having)
%    \begin{macrocode}
      \ifnum \csname flush@min@col@1\endcsname > \g_curr_column_int
%<*trace>
        \tr@ce{flush:~ no~ flush~ points~ on~ page;~ defer}
%</trace>
        \do@next\really@defer@and@try@next@float
      \else
%    \end{macrocode}
%    
%    The code is often invoked before there is an attempt to split the
%    galley; this means we don't know which flush point affects the
%    current float and thus have to manually get at it.
%
%    So we loop through the |\flush@last@float@|\meta{num} values to
%    find the first one that a) affects the current float type and b)
%    holds a sequence number that is higher or equal to the sequence
%    number of the current float.
%    \begin{macrocode}
        \count@ \@ne
        \loop
%    \end{macrocode}
%    If |\count@| is greater than |\flush@seq@num| we have exhausted
%    all possible flush points without finding a candidate.
%    \begin{macrocode}
          \ifnum \flush@seq@num < \count@ 
            \in@false
          \else
%    \end{macrocode}
%    Otherwise, if |\g_xor_this_flseq_num| is greater than
%    |\flush@last@float@|\meta{num} the current flush point is before
%    the call-out to the current float.
%    \begin{macrocode}
%<*trace>
            \tr@ce{flush:~  \g_xor_this_flseq_num \space ???~
                   \csname flush@last@float@ \the\count@ \endcsname}
%</trace>
            \ifnum \g_xor_this_flseq_num >
                   \csname flush@last@float@ \the\count@ \endcsname
                   \relax
%    \end{macrocode}
%    We set |\in@true| to get a repeat below with a new value of
%    |\count@|.
%    \begin{macrocode}
              \in@true
            \else
%    \end{macrocode}
%    Otherwise we have a flush point that is later in the galley in
%    comparison to the call-out of the current float thus potentially
%    affects it. Therefore we now have to check if the sequence class
%    of the current float is affected by the flush point found:
%    \begin{macrocode}
              \expandafter\xin@\expandafter\this@class
              \csname flush@classes@list@ \the\count@ \endcsname
%    \end{macrocode}
%    Now we could make good use of the |\unless| primitive of
%    e\TeX. But since we want to run with standard \TeX{} and don't
%    have |\xnotin@| available we invert the result.
%    \begin{macrocode}
              \ifin@ 
                \in@false 
              \else 
                \in@true
%<*trace>
                 \tr@ce{flush:~ class~ \this@class\space not~ in~ 
                   \csname flush@classes@list@ \the\count@ \endcsname
                   \space (ignored)
                   }
%</trace>
              \fi
            \fi
          \fi
        \ifin@
          \int_incr:N \count@
        \repeat
%<*trace>
        \tr@ce{flush:~ first~ float~ point~ after~ float~ =~
               \the\count@}
%</trace>
        \ifnum \flush@seq@num < \count@
%<*trace>
          \tr@ce{flush:~ this~ float~ past~all~flush~ points;~ defer}
%</trace>
          \do@next\really@defer@and@try@next@float
        \else
%    \end{macrocode}
%    Once a |flush@min@col| is larger than |\g_curr_column_int| it has
%    moved to the next page. In particular this is true for the
%    artificial value |\maxdimen| to which it is sometimes set to
%    denote this fact. If so we can defer our float.
%    \begin{macrocode}
          \ifnum \csname flush@min@col@\the\count@\endcsname 
               > \g_curr_column_int
%<*trace>
            \tr@ce{flush:~ float~ past~ all~ flush~ points~
                   on~ current~ page;~ defer}
%</trace>
            \do@next\really@defer@and@try@next@float
          \else
%    \end{macrocode}
%    If we have already relaxed the conditions and we are still in
%    conflict with a flush point on the current page we nearly ran out
%    of options. In this case we are forced to move the flush point,
%    but rather than moving it to the next page and defering the float
%    we can move it one column further (which might make it fall off
%    the page though) and then retry to place the float since we have
%    now more space available.
%    \begin{macrocode}
            \ifx\try@this@area\relaxed@try@this@area
%<*trace>
              \tr@ce{flush:~ defer~ forced;~ move~ 
                     flush~ point~ columns}
%</trace>
%<*progress>
              \progress@nl{}
              \progress@nl{Flushing~ impossible~ -->~ breaking~
                           before~ flush~ point~ and~ retry}
%</progress>
%    \end{macrocode}
%    We should only increment the |flush@min@col@| value of the first
%    flush point that affects our float (this will automatically move
%    later flush points if necessary). Incrementing the later ones is
%    incorrect as they might not need moving!
%    \begin{macrocode}
              \num_gincr:c {flush@min@col@ \the\count@}
%<*trace>
              \tr@ce{flush@min@col@\the\count@ \space <-~ 
                     \csname flush@min@col@ \the\count@ \endcsname
                     }
%</trace>
%<*progress>
              \progress@nl{}
              \progress@nl{Defer~ impossible~ -->~ moving~
                           flush~ point~ to~ column~
                           \csname flush@min@col@ \the\count@\endcsname
                           \space
                           and~ retry}
%</progress>
%    \end{macrocode}
%    Not clear that it is really necessary to check all areas again,
%    but it is late (after midnight) and it is definitely not
%    wrong.\footnote{Check!}
%    \begin{macrocode}
              \relax@float@placement@conditions  % needed to reset open areas
              \do@next\try@this@area

            \else
%    \end{macrocode}
%
%    \begin{macrocode}
              \relax@float@placement@conditions
              \do@next\try@this@area
            \fi
          \fi
        \fi
      \fi
    \else
%<*trace>
      \tr@ce{flush:~ class~ \this@class\space not~ in~
             \flush@classes@list@max;~ defer}
%</trace>
      \do@next\really@defer@and@try@next@float
    \fi

  \else
%<*trace>
    \tr@ce{flush:~ no~ flush~ point~ seen; defer}
%</trace>
    \do@next\really@defer@and@try@next@float
  \fi

%<*trace>
  \@tracepop{defer@and@try@next@float}
%</trace>
  \do@continue
}
%    \end{macrocode}
% \end{macro}
%
%
% \begin{macro}{\relax@float@placement@conditions}
%    \begin{macrocode}
\def \relax@float@placement@conditions {
%<*progress>
    \progress@failed{-->~ retry~ with~ relaxed~ conditions}
    \progress@nl{}
%</progress>
%<*trace>
    \tr@ce{flush:~relax~placement~conditions}
%</trace>
    \global\let\try@this@area\relaxed@try@this@area
% next line perhaps external?
    \global\let\this@open@areas\saved@this@open@areas

    \global\let\check@some@constraints\relaxed@check@some@constraints
    \global\let\check@callout@constraints\relaxed@check@callout@constraints
    \global\let\calculate@target@fl@column\relaxed@calculate@target@fl@column
}
%    \end{macrocode}
% \end{macro}
%    
% \begin{macro}{\tighten@float@placement@conditions}
%    \begin{macrocode}
\def\tighten@float@placement@conditions {
%<*trace>
    \tr@ce{flush:~tighten~placement~conditions}
%</trace>
    \global\let\try@this@area\std@try@this@area

    \global\let\check@some@constraints\std@check@some@constraints
    \global\let\check@callout@constraints\std@check@callout@constraints
    \global\let\calculate@target@fl@column\std@calculate@target@fl@column
}
%    \end{macrocode}
% \end{macro}
%
%
%    
% \begin{macro}{\partly@tighten@float@placement@conditions}
%    This is used when we have placed the last float that is affected
%    by a certain flush point. We don't want to tighten up completely
%    since this would essentially mean that we can't place any further
%    floats since callout relationships etc might be violated for
%    them. Good question is whether or not the last restriction should
%    be used again since this too might prevent any further placements.
%    \begin{macrocode}
\def\partly@tighten@float@placement@conditions {
%<*trace>
    \tr@ce{flush:~partly~tighten~placement~conditions}
%</trace>
    \global\let\try@this@area\std@try@this@area

%    \global\let\check@some@constraints\std@check@some@constraints
%    \global\let\check@callout@constraints\std@check@callout@constraints
    \global\let\calculate@target@fl@column\std@calculate@target@fl@column
}
%    \end{macrocode}
% \end{macro}
%
%
%
% \subsection{Checking float placement during trial}
%
% The interface to checking float placement in relation to its callout
% is the following piece of code in |\grab@column@or|:
%\begin{verbatim}
%    \seq_map:NN \g_xor_float_classes_seq
%                \check@callout@constraints
%\end{verbatim}
% This loops through all float types and executes
% |\check@callout@constraints| with the float type as an argument. The output
% routine |\grab@column@or| itself is called for each column of a trial
% configuration, thus the above loop is called for each column
% individually (|\g_curr_column_int| can be used to determine the current
% column number).
%
% By giving |\check@callout@constraints| an appropriate definition a
% pagesetup template can implement different relationships between
% callout and float.
%
% Possible tests (getting stronger):
% \begin{itemize}
%
% \item
%   Don't check, i.e., add the float when you find it and it fits
%   according to other criteria (like number of floats in the area,
%   etc.). This is implemented in |\check@callout@none|.
%
%
% \item
%  Check if callouts for all floats on the page (not column) are
%  either on the same page or on an earlier page; i.e., callout can be
%  late as long as the float is visible from the callout.
%
%   Fail if for last column and all float types:
%     last callout number for float type is smaller than maximum last float
%     of type put into any column.\footnote{This description is
%     probably wrong}
%
%   This is implemented in |\check@callout@page|.
%
% \item
%  Check if callouts for all floats in the column are either on the
%  same column or on an earlier column.
%
%  Fail if for any column and any float type:
%  last callout number for float type is smaller last float sequence
%  number for type recorded for this column.
%
%  This case consists in fact of two subcases depending on how we
%  interpret to which column a spanning float belongs. If we claim
%  that a spanning float is placed into its starting column, then we
%  fail if its call-out is in a later column even though this column
%  might still be spanned by the float area.
%
%  This is implemented in |\check@callout@column|.
%
% \item
%  Check if callouts for all floats in the column are on an earlier
%   column or if on the same column the float was added to the bottom
%   (or marginal) area; i.e. strict float/callout order
%
%   Fail if for each type and for any column:
%     `top' callout number  less than `top' float number  
%      (at top of text column)
%\begin{verbatim}    
%      [Corectness proof: 
%       TRUE  => first callout's float comes before bottom
%                  and so comes too early 
%             => FAIL
%       FALSE => first callout's float comes in bottom and all other
%                  callouts come later and so are in bottom or beyond 
%             => OK
%       ]
%\end{verbatim}
%   The proof above is in fact only valid if you look at the whole
%   document and not only at a single page since a float in the bottom
%   area of the last column with its callout on the next page will
%   only be detected when testing the next page. Therefore one needs
%   an additional check of type |\check@callout@column|.
%
%   This is implemented in |\check@callout@after|.
%
%
% \end{itemize}
%
%
% \begin{macro}{\check@callout@none}
%    This test is an easy one: just do nothing, i.e., gobble the argument.
%    \begin{macrocode}
\let\check@callout@none\@gobble
%    \end{macrocode}
% \end{macro}
%
% \begin{macro}{\check@callout@page}
%    We are are only interested in the callout/float relation per page
%    we only have to do a check when producing the last column, i.e.,
%    when |\g_xor_curr_col_int| is |\g_column_int|.
%    \begin{macrocode}
\def\check@callout@page#1{
%<*trace>
  \@tracepush{check@callout@page}
%</trace>
  \ifnum\g_xor_curr_col_int=\g_column_int
%    \end{macrocode}
%    We store in |\g_xor_flseq_max_num| the highest sequence number for floats of
%    the current type up to the end of the page. For this we have to
%    find the maximum of |\g_xor_flseq_type_|\meta{type}|_col_0_num| (highest float number on
%    previous pages) and those for the columns, e.g.,  
%    |\g_xor_flseq_type_|\meta{type}|_col_|\meta{column}|_num|; this is done by the following
%    code:
%    \begin{macrocode}
    \num_gset_eq:Nc \g_xor_flseq_max_num
                    {g_xor_flseq_type_#1_col_0_num}
    \xor_forall_columns:n {
      \ifnum \num_use:c{g_xor_flseq_type_#1_col_ \the\g_xor_curr_col_int _num}
           > \g_xor_flseq_max_num \relax
         \num_gset_eq:Nc \g_xor_flseq_max_num {g_xor_flseq_type_#1_col_ \the\g_xor_curr_col_int _num}
      \fi
    }
%    \end{macrocode}
%    We then store the number of the last callout in |\count@|; the
%    |0| will take care of the potential problem that there was never any
%    callout so far. And we better have a |\relax| afterwards since
%    otherwise we will expand the |\ifnum| before we have finished
%    assigning the |\count@|.
%    \begin{macrocode}
    \count@0\LastMark{#1}\relax
%<*trace>
    \tr@ce{Last~callout~ (#1)~ =~ \the\count@}
%</trace>
%    \end{macrocode}
%    Now we have to compare those two numbers to find out if that
%    trial has failed:
%    \begin{macrocode}
    \ifnum\count@<\g_xor_flseq_max_num\relax
      \@failtrue
%<*progress>
    \progress@failed{last~callout~
       \the\count@\space~<~\g_xor_flseq_max_num
       \space last~float~put~on~page~or~ earlier}
%</progress>
%<*trace>
    \tr@ce{Failed:~(#1)~ last~callout~
       \the\count@\space~<~\g_xor_flseq_max_num
       \space last~float~put~on~page~or~ earlier}
    \else
    \tr@ce{OK:~(#1)~ last~callout~
       \the\count@\space~>=~\g_xor_flseq_max_num
       \space last~float~put~on~page~or~ earlier}
%</trace>
    \fi
  \fi
%<*trace>
  \@tracepop{check@callout@page}
%</trace>
}
%    \end{macrocode}
% \end{macro}
%
%
%
% \begin{macro}{\g_xor_flseq_max_num}
%    The macro |\g_xor_flseq_max_num| is used within |\check@callout@page| to hold
%    the highest sequence number of any allocated float on the current
%    page for the type under testing.  It will be recalculated on
%    each pass.\footnote{One could generate this value while running through
%    the trials --- this would perhaps be a bit more time efficient.}
%    \begin{macrocode}
\let\g_xor_flseq_max_num\ERROR
%    \end{macrocode}
% \end{macro}
%
%
% \begin{macro}{\check@callout@column}
%    Checking each column separately means we have to compare for each
%    type the last callout on this or previous columns (i.e., as
%    returned by |\LastMark|) with the highest sequence number for
%    floats of this type in the current column (as stored in
%    |\g_xor_flseq_type_|\meta{type}|_col_|\meta{column}|_num|). So we first store the callout
%    info in |\count@|.
%    \begin{macrocode}
\def\check@callout@column#1{
%<*trace>
  \@tracepush{check@callout@column}
%</trace>
  \count@0\LastMark{#1}\relax
%    \end{macrocode}
%    Then we do the test. If there are no floats of the current type
%    in the current column |\g_xor_flseq_type_|\meta{type}|_col_|\meta{column}|_num| will be zero
%    and thus the following test will come out true. This is the
%    correct behaviour as any callouts that might be present will be
%    correctly evaluated in tests on neighboring columns with floats. 
%    \begin{macrocode}
  \ifnum\count@< \num_use:c{g_xor_flseq_type_#1_col_ \the\g_xor_curr_col_int _num}
                 \relax
%<*progress>
  \progress@failed{last~ callout~
         \the\count@\space <~
         \num_use:c {g_xor_flseq_type_#1_col_ \the\g_xor_curr_col_int _num}
         \space last~ float~ placed~ in~ column~ \the\g_xor_curr_col_int}
%</progress>
%<*trace>
  \tr@ce{Failed:~(#1)~ last~ callout~
         \the\count@\space <~
         \num_use:c{g_xor_flseq_type_#1_col_ \the\g_xor_curr_col_int _num}
         \space last~float~placed}
%</trace>
%    \end{macrocode}
%    If the test fails we abort this trial by setting the switch
%    |@fail| to true. We also set
%    |\g_xor_curr_col_int| to |\g_column_int| which will save us any further
%    iteration (in case this wasn't the last column). Finally we
%    locally set |\@elt| to |\@gobble| which will essentially about
%    the current loop through the |\g_xor_float_classes_seq|.
%    \begin{macrocode}
    \@failtrue
    \global\g_xor_curr_col_int\g_column_int
%FMi tmp
    \def\seq_elt:w ##1\seq_elt_end:{}
%    \end{macrocode}
%    Otherwise the constraints for callout of current float type are
%    met so we report this fact if we do tracing.
%    \begin{macrocode}
%<*trace>
  \else
  \tr@ce{OK:~(#1)~ last~ callout~
         \the\count@\space >=~
         \num_use:c {g_xor_flseq_type_#1_col_ \the\g_xor_curr_col_int _num}
         \space last~float~placed}
%</trace>
  \fi
%<*trace>
  \@tracepop{check@callout@column}
%</trace>
}
%    \end{macrocode}
% \end{macro}
%
%
% \begin{macro}{\check@callout@after}
%
%    \begin{macrocode}
\def\check@callout@after#1{
%<*trace>
  \@tracepush{check@callout@after}
%</trace>
%    \end{macrocode}
%
%    We first check if all floats are either on the same column than
%    their callout or on an later column by calling
%    |\check@callout@column|. If this returns |@fail| we immediately
%    abort any further processing.
%    \begin{macrocode}
  \check@callout@column{#1}
  \if@fail\else
%    \end{macrocode}
%
%    If the above check found no violation we have to check for top
%    area floats that might preceed their callouts.
%
%    For this we look at the value of the
%    |\g_xor_flseq_areas_top_type_|\meta{float-type}|_col_|\meta{col}|_num| macro which holds the
%    highest float sequence number for floats of this type allocated
%    in that column.
%    \begin{macrocode}
     \num_gset_eq:Nc
            \g_xor_flseq_returned_num
            {g_xor_flseq_areas_top_type_#1_col_
                               \the\g_xor_curr_col_int _num }
%    \end{macrocode}
%    
%    If there was no float in the top area we get |0| back and in
%    that case we are done since all our constraints are met.
%    \begin{macrocode}
     \ifnum \g_xor_flseq_returned_num = \z@
%<*trace>
     \tr@ce{OK:~(#1)~ top~ areas~ have~ no~ floats}
%</trace>
    \else
%    \end{macrocode}
%
%    Otherwise we have to look at the top callout number and compare
%    it to the last float number of the top area.
%    \begin{macrocode}
      \count@ 0\PreviousMark{#1}\relax
%    \end{macrocode}
%    If that callout number is less than the last float number we
%    failed since this means that there is a float in the top area
%    which callout has not yet been seem. So we abort the
%    trial. Otherwise our constraints are met.
%    \begin{macrocode}
      \ifnum \count@ < \g_xor_flseq_returned_num \relax
%    \end{macrocode}
%    But before aborting we give some information on why we
%    failed. This has to come first since we change |\g_xor_curr_col_int|
%    below.
%    \begin{macrocode}
%<*progress>
        \progress@failed{top~ callout~
              \the\count@\space <~ \g_xor_flseq_returned_num
              \space last~ float~ put~ in~ top~ 
              of~ column~ \the\g_xor_curr_col_int}
%</progress>
        \@failtrue
        \global\g_xor_curr_col_int\g_column_int
%FMi tmp
%        \let\@elt\@gobble
        \def\seq_elt:w ##1\seq_elt_end:{}
%<*trace>
        \tr@ce{Failed:~(#1)~ top~ callout~
              \the\count@\space <~ \g_xor_flseq_returned_num
              \space last~float~put~in~top}
      \else
        \tr@ce{OK:~(#1)~ top~ callout~
               \the\count@\space >=~ \g_xor_flseq_returned_num
              \space last~float~put~in~top}
%</trace>
      \fi
    \fi
  \fi
%<*trace>
  \@tracepop{check@callout@after}
%</trace>
}
%    \end{macrocode}
% \end{macro}
%
%
%
%
% \subsection{Checking bottom float footnote constraints}
%
% In |\grab@column@or|, the output routine that grabs a single column
% during a trial, we have the command |\check@some@constraints| which
% allows to check some constraints and if they aren't met is supposed
% to set the switch |@fail|.
%
% This can, for example be used to implement a constraint that bottom
% floats are only allowed if there are no footnotes in the current
% column.\footnote{At the moment this is restricted to single column
%    floats. Spanning floats are always allowed.}
%
% \begin{macro}{\check@float@footnote@forbidden}
%    The command |\check@float@footnote@forbidden| is the
%    implementation for |\check@some@constraints| that prevents bottom
%    floats if the current column contains footnotes.
%    \begin{macrocode}
\def\check@float@footnote@forbidden{
%<*trace>
  \@tracepush{check@float@footnote@forbidden}
%</trace>
  \ifvoid\footins
  \else
%    \end{macrocode}
%    If the current column contains footnotes and we have bottom
%    floats we fail. First we check if there is a bottom area for this
%    column defined if not we pretend there is an empty one.
%    \begin{macrocode}
    \expandafter
    \ifx \csname g_xor_area_ 
                 b\int_use:N\g_xor_curr_col_int 1
                 _seq
         \endcsname
         \relax
         \seq_gclear:c { g_xor_area_ 
                         b\int_use:N\g_xor_curr_col_int 1
                         _seq}
    \fi
%    \end{macrocode}
%    Then we check if the bottom area is empty and fail otherwise
%    \begin{macrocode}
%<+trace> \seq_if_empty:cTF
%<-trace> \seq_if_empty:cF
        {g_xor_area_ 
         b\int_use:N\g_xor_curr_col_int 1
         _seq}
%<*trace>
        { \tr@ce{OK:~ no~ old~ bottom~ floats} }
%</trace>
        {
%<*progress>
         \progress@failed{old~bottom~floats:~ \expandafter\meaning
            \csname g_xor_area_ b\int_use:N\g_xor_curr_col_int 1 _seq\endcsname}
%</progress>
%<*trace>
         \tr@ce{Failed:~ old~bottom~floats:~ \expandafter\meaning
            \csname g_xor_area_ b1\int_use:N\g_xor_curr_col_int _seq\endcsname}
%</trace>
         \@failtrue
        }
%    \end{macrocode}
%    Finally we test if the float we are try to place is going onto a
%    bottom area.
%
%    However we only do this if the float doesn't span.\footnote{Extend?}
%    \begin{macrocode}
    \ifnum \this@area@span@number = \@ne
%<*trace>
      \tr@ce{this@area,column:~ \this@area,~\the\g_xor_curr_col_int}
%</trace>
      \if b \this@area@type
         \ifnum \this@area@col@number = \g_xor_curr_col_int
           \@failtrue
         \fi
      \fi
    \fi
    \if@fail
%<*progress>
      \progress@failed{column~ \the\g_xor_curr_col_int\space
                       contains~ footnotes~ and~ bottom~ floats}
%</progress>
%<*trace>
      \tr@ce{Failed:~ column~ \the\g_xor_curr_col_int\space
                      contains~ footnotes~ and~ bottom~ floats}
%</trace>
      \global\g_xor_curr_col_int\g_column_int
    \fi
  \fi
%<*trace>
  \@tracepop{check@float@footnote@forbidden}
%</trace>
}
%    \end{macrocode}
% \end{macro}
%
%
% \begin{macro}{\check@float@footnote@none}
%
%    This is the implementation for |\check@some@constraints| if we
%    don't care about mixing footnotes and bottom floats. The reason
%    |\@empty| rather than |\relax| is used is that otherwise the
%    |\ifx| test in the template would fail.
%    \begin{macrocode}
\let\check@float@footnote@none\@empty
%    \end{macrocode}
% \end{macro}
%
%
%
%
%
%
% \endinput
\endinput
%
% $Log$
% Revision 1.2  2004/09/27 20:06:31  mittelba
% in the middle of normalizing to expl3 syntax
%
% Revision 1.1  2001/07/26 19:55:12  latex3
% original web distrib
%
% Revision 1.37  2000/08/11 07:14:28  latex3
% added header
%
% Revision 1.36  2000/08/11 06:49:47  latex3
% untabify
%
% Revision 1.35  2000/08/11 06:45:03  latex3
% only partially tighten float constraints after passing a flush point
%
% Revision 1.34  2000/07/23 21:03:19  latex3
% removed obsolete commentary
%
% Revision 1.33  2000/07/21 13:53:17  latex3
% more renaming for float sequence class concept
%
% Revision 1.32  2000/07/18 21:06:36  latex3
% introduced float sequence classes
%
% Revision 1.31  2000/07/12 17:24:53  latex3
% more support for grids
%
% Revision 1.30  2000/07/10 21:00:57  latex3
% support for grid typesetting
%
% Revision 1.29  2000/07/04 19:42:05  latex3
% fix tracing for GRID stuff
%
% Revision 1.28  2000/07/01 15:57:25  latex3
% use \update@this@area@columns
% integrate code to support grid design (first draft)
%
% Revision 1.27  2000/06/29 17:18:14  latex3
% introduced \setup@this@area
%
% Revision 1.26  2000/06/26 15:03:40  latex3
% making prototype for \pagesetup@float@area@sep
%
% Revision 1.25  2000/06/22 20:08:09  latex3
% renamed some macros to get them more uniform
%
% Revision 1.24  2000/06/22 11:01:24  latex3
% When substracting heights from columns do this for the columns spanned
% by the area. Don't use the span number from the float (in the future
% this might be smaller than the span number of the area)
%
% Revision 1.23  2000/06/19 21:38:04  latex3
% fixed flush point handling when deferring
%
% Revision 1.22  2000/06/18 19:03:53  latex3
% added "Checking bottom float footnote constraints" from xo-page
%
% Revision 1.21  2000/06/16 11:20:15  latex3
% rename \construct@and@test@col@height to \construct@and@test@col@ht
% rename \construct@and@test@col@heights to \construct@and@test@col@hts
% rename \cl@height1 to \@col@ht@1 (etc)
%
% Revision 1.20  2000/06/16 11:06:37  latex3
% support float-callout-span-constraint
%
% Revision 1.19  2000/06/16 07:49:49  latex3
% improve progress info
%
% Revision 1.18  2000/06/15 17:49:02  latex3
% moved macros from xo-page into here since they deal with placement
%
% Revision 1.17  2000/06/15 15:23:32  latex3
% implemented new semantics for area names
%
% Revision 1.16  2000/06/13 20:48:18  latex3
% docu update
%
% Revision 1.15  2000/06/06 12:50:40  latex3
% before attempting fuzzy flushpoints
%
% Revision 1.14  2000/05/03 18:57:56  latex3
% provide \construct@and@test@col@heights subroutine
%
% fix problem that deferred floats would not stop other floats going
% into earlier areas
%
% Revision 1.13  2000/03/31 17:08:50  latex3
% avoid using up too much input stack by using tail recursion in
% strategic places (tmp fix)
%
% Revision 1.12  2000/03/26 21:04:03  latex3
% some renaming of macros
%
% Revision 1.11  2000/03/24 15:34:27  latex3
% version that starts supporting spans (still a hack yet)
%
% Revision 1.10  2000/03/22 15:52:37  latex3
% some normalisations of names
% first go at spans (tmp version)
%
% Revision 1.9  2000/03/17 20:23:11  latex3
% more fixes to flushing (looks good now)
%
% Revision 1.8  2000/03/16 10:28:29  latex3
% partial and full flush working for the first time
%
% Revision 1.7  2000/03/05 19:31:56  latex3
% some renaming
% extend to multiple columns (6 max right now)
% fix defer logic: when flushseen don't defer immediately
%
% Revision 1.6  2000/02/27 15:12:45  david
% *** empty log message ***
%
% Revision 1.5  2000/02/27 11:27:45  david
% first attempt at flush floats
%
% Revision 1.4  2000/02/26 18:24:21  david
% renaming, and using \initialise@areas
%
% Revision 1.3  2000/02/16  13:40:09  latex3
% added 3col support
%
% Revision 1.2  2000/02/16  10:10:43  latex3
% added documentation
% added test for float area size after adding float
%
% Revision 1.1  2000/02/13  22:27:39  latex3
% Initial revision
%

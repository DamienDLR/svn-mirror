% Copyright 2000 Frank Mittelbach
% All rights reserved
%
%
% NOTE: the text contains reference to the fact that it is typeset
% using the new OR!
%
\def\rcsdata$#1: #2,v #3${
  \def\rcsdata{#2 #3}}
\rcsdata$Id$

% next lines needs pdflatex 0.14f+ !!!!
\ifx\rpcode\undefinedcsname
\else
  \input protcode.tex
  \pdfprotrudechars=2
%  \pdfoutput=0             
\fi

\tolerance=6000
\emergencystretch=1cm
\errorcontextlines=\maxdimen

%\tracingonline=1
%\tracingpages=1
%\showoutput

\newcount\ProcessControl

% 0 means std LaTeX, 1 new OR, 2 new OR + galley2 (not functional)

\ProcessControl=1

%\typein[\ProcessControl]
%       {^^J^^J0 means std LaTeX, 1 new OR
%, 2 new OR + galley2 (needs *new* LaTeX)
%       }


\newcommand\ORspecial[1]{\ifnum\ProcessControl>0\relax#1\fi}



\ifnum\ProcessControl=0
  \documentclass[twocolumn]{article}
%  \documentclass[preprint]{ltugproc}
  \renewcommand\floatpagefraction{0.9}
  \setlength\belowcaptionskip{10pt}
\else
  \documentclass{article}
\fi

\parskip=0pt
\widowpenalty=5000

\usepackage{array,url,varioref}
\usepackage{times}

\usepackage{shortvrb}
\MakeShortVerb\|

\ifcase\ProcessControl
%  \usepackage{hyperref}
  \usepackage{calc}
\or
  \usepackage{trace}
  \usepackage{xoutput}
  \usepackage{xo-grid}
  \ShowGridfalse
\else
  \usepackage{trace,
            galley2,xcontents,xlists-samples,
            xoutput}
  \usepackage{xo-grid}
  \ShowGridfalse

  \let\thetable\thefigure % needed to be able to use old xcontents.sty
\fi


\ifnum\ProcessControl=2

\makeatletter
\def\l@v@lalgorithms{0}
\makeatother

\DeclareInstance{contentsobject}{section}{std}{
  pre-v-action     = \DelayEvaluation{\addpenalty{10000}
                     \addvspace{1em plus 1pt}},
  right-margin-sep = 15pt,
  title-format     = \textbf{#1},
  number-format    = \textbf{#1\hfil},
  pnum-format      = \textbf{\hfil#1},
  leaders-action   = ~,
  number-width     = \DelayEvaluation{1.5em}}

\DeclareInstance{contentsobject}{subsection}{std}{
  object-indent = \DelayEvaluation{1.5em},
  number-width  = \DelayEvaluation{2.3em}}

\DeclareInstance{contentsobject}{subsubsection}{std}{
  object-indent = \DelayEvaluation{3.8em},
  number-width  = \DelayEvaluation{3.2em}}


\DeclareInstance{contentsobject}{figure}{std}{
  pre-v-action     = ,
  right-margin-sep = 20pt,
  title-format     = \textit{#1},
  number-format    = \textrm{\hfil#1},
  pnum-format      = \textbf{\hfil#1},
  leaders-action   = ~,
  number-width     = \DelayEvaluation{1.5em}}


\DeclareInstance{contentsobject}{table}{std}{
  pre-v-action     = ,
  right-margin-sep = 15pt,
  title-format     = \textrm{#1},
  number-format    = \textbf{#1\hfil},
  pnum-format      = \textbf{\hfil#1},
  leaders-action   = ~,
  number-width     = \DelayEvaluation{1.5em}}

\fi



\ifnum\ProcessControl> 0


\DeclareFloatSequenceClass{floats}  % this will vanish again

\DeclareFloatType{figure}
  {
   sequence-class-id  = floats,
   toc-extension      = lof,
   caption-text       = \figurename,
   numbered-boolean   = true,
   numbered-id        = figure,
   numbered-within-id = ,
   numbered-action    = \arabic{figure},
   body-decls         = 
  }




\DeclareFloatType{table}
  {
   sequence-class-id  = floats,
   toc-extension      = lot,
   caption-text       = \tablename,
   numbered-boolean   = true,
   numbered-id        = figure,
   body-decls        = \footnotesize,
  }

\DeclareFloatArea   {t12}
   {
     class-close-list = ,
     all-close-list  = ,
     max-float-num   = 1,
   }


\DeclareFloatArea   {b12}
   {
     class-close-list = {t12,t11,b11},
     all-close-list  = ,
     max-float-num   = 1,
   }


\DeclareFloatArea   {t11}
   {
     class-close-list = ,
     all-close-list  = {b11},
     max-float-num   = 1,
   }


\DeclareFloatArea   {b11}
   {
     class-close-list = {t12,t11},
     all-close-list  = ,
     max-float-num   = 1,
   }


\DeclareFloatArea   {t21}
   {
     class-close-list = {t12,t11,b11},
     all-close-list  = ,
     max-float-num   = 1,
   }


\DeclareFloatArea   {b21}
   {
     class-close-list = {t12,t11,b11},
     all-close-list  = ,
     max-float-num   = 1,
   }


\topskip=12pt

\DeclareInstance{pagesetup2}{twocol}{std}{
   column-num    = 2,
   column-width  = 229.5pt, % this strange value is what LaTeX2e
                            % uses for two column article
   column-height = \DelayEvaluation 50\baselineskip + \topskip,
   column-sep    = 20pt,
%
   float-callout-constraint = after,
   float-callout-span-constraint = flexible,
   bottom-float-footnote-constraint=forbidden,
%   bottom-float-footnote-constraint=none,
   max-float-num =2,
%
   area-list = {t12,t11,b11,b12,t21},
   area-list = {b12,b11,b21},
%
%   footnote-setup = \UseTemplate{footnotesetup}{ftnright}{},
%   footnote-setup = \UseTemplate{footnotesetup}{margin}{},
   footnote-setup = \UseTemplate{footnotesetup}{std}{},
%
   float-text-sep  = 36pt minus 8pt,
   float-area-sep  = 20pt,
   float-float-sep = 20pt,
%
   grid-point-sep = 12pt,     
  }

\fi


\newcommand\meta[1]{\mbox{$\langle$\textit{#1}$\rangle$}}

\title{Formatting documents with floats\\
       A new algorithm for \LaTeXe$^*$}
\author{\copyright{} Frank Mittelbach}
\date{\rcsdata}



\begin{document}

\ORspecial{\UseInstance{pagesetup2}{twocol}}


\maketitle

\tableofcontents

\ORspecial\vfill

\begin{abstract}
This paper describes an approach to placement of floats in
multicolumn documents.

The current version of \LaTeX{} was originally written for
single-column documents and extended to support two-column documents by
essentially building each column independently from the other. As a
result the current system shows severe limitations in two column mode,
such as the fact that spanning floats are always deferred  to at least the
next page or that numbering between column floats and spanning floats
can get out of sequence.

The new algorithm is intended to overcome these limitations and at the
same time extend the supported class of document layouts to multiple
columns with floats spanning an arbitrary number of columns.
\end{abstract}

%\ORspecial\vfill
\ORspecial\newpage

\section{Introduction}

One problem with formatting documents containing floats is the number
of potential formatting solutions that need to be checked out. The
number of trials grows combinatorially in the number of floats and
areas which can receive them. If we have $n$ floats waiting to be
placed and $m$ areas in which we can place them on the current page
being built (not counting the ``deferred area'') then the number of
different placements is given by
\begin{equation}
   \#\mbox{trials} =
   { n+m \atopwithdelims() m} =
   \frac{(n+m)!}{n! \, m!}
   \label{eq:binom}
\end{equation}
assuming that the order of floats has to be preserved, i.e., if the
call-out of float $f_i$ is before the call-out of $f_j$ in the text
stream then the float $f_i$ will be placed in earlier than float $f_j$
where ``earlier'' is a defined relation of float areas.


For example, if we have 8 floats waiting to be distributed among 12
areas (which corresponds to a three column page with float areas at the
top and bottom allowing for partial spans) then we have to check
125970 possible distributions; if two additional floats appear we end
up with 646646 trials.

Even though a large number of these distributions would be
unacceptable and discardable straight away, after some initial test,
the resulting running time of the algorithm would clearly be beyond
any acceptable speed. (Assuming we could do 1000 trials per second,
which is ridiculous high since many of them would require
trial-typesetting the whole page, then the case of 646646 trials would
still take roughly 10 minutes to form a decision.)

 Thus it is important to find algorithms with
complexity that is at worst linear in both the number of floats on the
trial list and the number of possible float areas, even if this means
that in a few cases a relatively good layout will not be found.
It is even better if they have minimal redundancy.


Note that assessing the actual running time of \TeX{} code is not
straightforward since some activities are very much faster than
others.  For example, performing a test by using a reasonable number
of macro expansions and register assignments may be very much slower
than running through a long typeset list and then doing a simple test.
 

The algorithm we have implemented fulfills the requirement of being
(essentially) linear in the number of floats and the number of float
areas.


\section{The document source model}

The document source is a single stream of continuous text containing
call-outs to floating objects. (At the moment the call-outs are marked
by placing the objects into the stream but it would be possible to
provide them as separate objects.) 
Floating objects (as of today) come in three incarnations:
\begin{itemize}
\item Objects where the call-out and the placement requires a strict
spatial relationship, e.g., same line in the margin. An example would
be marginal notes as implemented by |\marginpar| in \LaTeXe.
\item Objects where the call-out and the placement are required to fall
onto the same column/\allowbreak page/\allowbreak spread, e.g.,
footnotes.
\item Objects where there is a defined relation between call-out and
object placement, e.g., ``not in an earlier column'', or ``on the same
page or later'', etc. These are the traditional floats.
\end{itemize}
Float objects in the last group are typed where the type is defined by
the logical content of the object, e.g., ``figure'', ``table'', and so on.


The document formatting is achieved using a minimal but customizable
lookahead (typically the considered galley material is the equivalent
of one page/\allowbreak spread of textual material ignoring the
additional size taken up by embedded float objects).

While making up pages the main ``quality'' guidance for the algorithm
is to try to place each float as early as possible without violating
defined constraints.


\ORspecial\ShowGridtrue


\section{The document layout model}

\subsection{Page layout grid}

The algorithm supports the specification of a page grid on which it
will align text columns and other elements.  This will allow (if
suitable parameters for various elements are chosen) to have text
lines of different columns all lying on grid points.\footnote{On the
current page lines are drawn to highlight the grid. Note that
headings, lists, and other ``display'' objects are not aligned.}

\subsection{Columns}

The page layouts which are supported by the new algorithm support
an arbitrary number of text columns of equal width.

The number of columns per page as well as their width can be changed
at forced page breaks such as the start of chapters.

\subsection{Balanced columns}

Balancing columns (as done by the \texttt{multicol} package) is planned
but not implemented. The major problem in that area is the handling of
column floats during the balancing process.


\subsection{Float areas}\label{overlap}

Float objects are distributed into float areas which are
rectangular in shape. Float areas span one or more text columns, their
horizontal size is therefore given by the following formula (where
$c$ is the number of columns spanned):
\[
   \meta{area-width} =  c \times 
                        \bigl( \meta{col-width} + \meta{col-sep}\bigr)
                        - \meta{col-sep}
\]
The naming conventions for float areas is as follows:
\[
\meta{identifier}\meta{start-column}\meta{span-count}.
\]
The \meta{identifier} is a single letter denoting the type of area,
e.g., |t| for top, |b| for bottom. The \meta{span-count} is a single
digit denoting the number of columns to span. The
\meta{start-column} is a single digit\footnote{With a bit of care in
the code this could be extended to allow more than one digit.}
denoting the start column of the area. Thus |t23| is a top area
starting at column two and spanning three columns, i.e., two, three 
and four.
A restriction due to the naming scheme is that currently no more than
9 columns are possible.\footnote{The scheme is different from the
original one used, where \texttt{t23} would have denoted an area
starting at column two and spanning until column three.}

Only a subset of the float areas is allowed to be populated on a
page. In essence the new algorithm does not support placements
that result in ``splitting'' the text of a column due to a float (other
than column `here' floats).\footnote{Perhaps this restriction will be
lifted one day.} This means that population of some float areas
must be prevented, namely those satisfying these conditions
when $pcs$ (where $p=\textrm{pos}$, 
$c=\textrm{column}$, $s=\textrm{span}$) has just been populated:
\begin{figure}[h]
\begin{minipage}{\columnwidth}
\vspace*{-2\baselineskip}
\tiny % to fit on page
\[
   pij  \textrm{ with }   i < c \leq i+j < c+s
\]
or
\[
   pij  \textrm{ with }   i \leq c+s < i+j \leq 
                            \mbox{\meta{number-of-columns}}
\]
\end{minipage}
\caption{Testing ``here'' floats}
\end{figure}


The first formula describes the areas which partly overlap from the
left, the second formula describes those that partly overlap from
the right. Areas which are sub- or super-areas, e.g., \texttt{t13}
and \texttt{t22} do not affect each other. The above restriction is
necessary to prevent situations like the one shown in
figure~\vref{fig:overlap}
%
\begin{figure}
\footnotesize
\caption{Overlapping float areas}\label{fig:overlap}
\begin{minipage}{.6\columnwidth}
\begin{verbatim}
        aaaaaaaaaaa 444
        aaaaaaaaaaa 444
        aaaaaaaaaaa 444
        111 222
        111 222 bbbbbbb
        111 222 bbbbbbb
        111 222 bbbbbbb
        111 222
        111 222 333 444
        111 222 333 444
        111 222 333 444
\end{verbatim}
\end{minipage}
\end{figure}
%
i.e., where the float area \texttt{t32} (represented as b's) would
result in splitting the fourth column into two independent text areas.

%\showfloatlists

The possibilities, as well as the restrictions, are equal for both top
and bottom areas. This means that the new scheme in particular
supports spanning bottom areas.

\ORspecial\ShowGridfalse

\subsection{Float pages and columns}

Float pages, i.e., pages consisting only of floats, will be supported
as well as float columns.


\subsection{Float types}

The type of float influences the formatting, e.g., where the caption
is placed in relation to the float body, how it is formatted, what
kind of fixed strings are added, etc. It also restricts the placement
algorithm in respect to which float areas can be populated as
explained below.


\subsection{Margins}

The marginal areas can receive marginal notes which are aligned with
the corresponding text line. In documents with more than two columns
marginal notes are currently not supported though one could envision
allowing them even there. If marginals have to compete for space the
later marginal will be moved downwards if there is enough space on the
page, otherwise the line containing the marginal will be moved to the
next column/\allowbreak page.\footnote{This is not yet implemented ---
right now they overprint each other.}

An alternative usage of the margin is to place footnotes into it. A
prototype version of this is provided already, see
section~\vref{footnotes}.

Another potential use of the margin areas is to use them (or parts
thereof) as float areas in their own right. The problem with this
would be that these float areas would have a horizontal width which is
different from the column width, thus allowing only a limited class of
floats to appear therein.

Another potential extension would be to allow float areas that border
on a margin to use the marginal space as part of the float area,
thereby allowing the filling of such an area with floats which are
wider than the nominal float area. A special case of this, the
placement of the caption in the margin beside the float body, is
already provided by choosing a suitable caption formatting instance.




\subsection{Footnotes}\label{footnotes}

%\traceon

Footnotes can be regarded as a special type of floats. They are objects
which are associated with lines of text (their call-out) but in
contrast to normal floats such as ``figures'' or ``tables'' their placement
constraints are stronger, e.g., they typically have to appear at the
bottom of the column which contains their call-outs, or at least they
have to appear on the same page as their call-outs.

In its current version, the model supports footnotes beneath the
call-out column (normal behavior); all footnotes in the last column
(as with the \texttt{ftnright} package for two-column mode); all
footnotes in the outer (or inner) margin. 

Without an extension to the page makeup algorithm (but instead with a
suitable redefinition of the footnote commands) they could be processed
as marginal notes or alternatively as ``end-notes''.


\subsection{Headers and footers}

The header and footer areas may use data received from individual
columns. An extended version of \TeX's mark mechanism is made
available which allows the definition of arbitrarily many independent
classes of marks. Within each mark class information about the top
mark (i.e., the mark active at the top of the column) the first mark
and the last mark is made available for retrieval.

This allows the production of correct running headers and footers for
various types of applications such as dictionaries, manuals, etc.



\section{The processing model}

\subsection{Float placement concepts}\label{sec:float-place}

To build a page(-spread) the algorithm first assembles enough textual
material to be able to fill the page without placing any floats.
During this process all floats that have their call-outs within the
assembled galley are collected. They form, together with unplaced
floats from previous pages, an ordered trial list of floats.

The allowed float areas on the page under construction are totally
ordered as well.

The algorithm proceeds by taking the first float from the trial list
and trying to place it into the first float area from the area
list. It then checks if all constraints (see below) are met and if not
the algorithm
will try to place the float into the next area until either all
constraints are met or the areas in the float area list are exhausted.
A trial that does not fail means that
this distribution of floats becomes the best solution so far and all
further trials will be based on adding to this solution (no
backtracking). If the algorithm fails to place the float into any
area it means that the float will be deferred to a later page.

As floats are added to areas, the constraints for further trials 
are changed. There are several reasons for this:
on one hand, the call-out positions of
various floats move
since the float will occupy space on the page; on the other hand,
placing a float in some area might result in disallowing the placement
of other floats in the same or in other areas.


\subsection{Float pages and columns}

At the moment there is only rudimentary support for float pages
available: at the start of each page the algorithm will try to form a
float page out of all floats that have been deferred from previous
pages. However there is no layout control available to define the
conditions under which such a trial will succeed.


\subsection{Float storage}

Float bodies are typeset into boxes at the point of `call-out', as
with the \texttt{figure} and \texttt{table} environments in the standard
\LaTeX; it may also be possible to specify at the call-out point a
logical pointer to a float whose typesetting is specified elsewhere
(e.g., an external file).

However, text sub-elements such as the caption, etc., (e.g., from
|\caption|) are not typeset at this stage but are stored as token
lists; this allows for trying different possible layout specifications,
e.g., for its measure, during the float-positioning trials.  At present
this is confined to at most a single caption element per float.

\subsection{Caption processing}

When a float is placed into an area the caption is trial formatted and
mounted onto the float body. This process can take into account
various information about the float positioning trial, such as the
area to format it into, the fact that it formats onto a verso or recto
page, etc. It might try several possibilities before making a decision,
e.g., if one formatting of the float results in violating some
constraint(s) it might try a different formatting at this point.



\subsection{Flushing floats}

It is possible to mark points in the source document as boundaries beyond
which floats whose call-outs are prior to the boundary cannot pass. In
other words a ``flush point'' directs the algorithm to place all
affected floats into areas which are ``before'' the flush point.

If due to other constraints the float could not be placed in such an
area the algorithm first retries all potential areas using a less
rigid set of constraints (for example, restrictions on the number of
allowed floats per area are dropped) and if this still doesn't enable
the algorithm to place the float properly it will as a last resort
move the flush point to a later column, which means breaking the
column text before the flush point.

Flushing of floats can be done either for all floats or on a
per float type basis, e.g., it is possible to flush only floats of
type ``figure''.

A flush point can be given an additional attribute which controls the
``fuzziness'' used by the algorithm. By default the flush point
algorithm uses |strict| flushing as described above. The attribute
|column| modifies the algorithms behaviour by enabling a float to move
past the flush point as long as it will be placed on the same
column. Similarily the attribute values |page| and |spread| will
enforce that the float will not be deferred further than the current
page or the current spread. This way it can be guaranteed that a float
is always visible from its call-out.


\subsection{Float sequence classes}

Float sequence classes are collections of float types; each float type
belongs to exactly one float sequence class. Within each sequence
class the call-out order in the document is always preserved by the
float placement algorithm, e.g., if $c_1,c_2,\ldots,c_n$ are the
call-outs of all floats of a float sequence class then the
corrsponding floats will be placed such that $f_i$ will be placed
before $f_j$ whenever $i<j$.  Thus by putting all float types into a
single float sequence class all floats are placed in the order of
their call-outs. As the other extreme, if each float type has its own
sequence class\footnote{This is the \LaTeXe{} default.} then floats
from one type might move before floats of other types even though the
corresponding call-outs are in a different order.



\subsection{Float and call-out relations} \label{fl-callout-relations}

The algorithm also keeps track of the relation between an individual
float and its call-out. This allows one to define constraints which
guide the algorithm during the float placement phase. It is always
permissible to place a float ``after'' its call-out, e.g., in a later
column/\allowbreak page. At the moment the following constraints can
be specified:
\begin{description}
\item[none] which means that the relation between call-out and float
placement is not relevant for placing floats.
\item[page] which means that the float can be placed anywhere on the
page with the call-out (it is visible from the call-out).
\item[column] which means that the float can be placed before the
call-out as long as it is placed in the same column.
\item[after] which means that the float has to be placed strictly
after the call-out.
\end{description}
When extending the algorithm to directly support spreads the
above list is going to be extended by an option that allows floats to
move backwards on the whole spread.




\subsection{Spanning float and call-out relations} \label{fl-span-relation}

For floats that span two or more columns there are several
possibilities to interpret the spatial relationship between call-out
and float areas. For example, if a float, whose call-out is in the
second column, has been placed into area |b12|, is this float
``before'' or ``after'' its call-out? The answer to this question
depends on whether we consider the float being placed into the first
or the second column, both of which are valid interpretations.

At the moment the following behaviour can be specified:
\begin{description}
\item[strict] which means that the leftmost column spanned
  by the float is being regarded as the column in which the float was placed.
\item[flexible] which means that the rightmost column spanned 
  by the float is being regarded as the column in which the float was placed.
\end{description}
These settings are only relevant if the main float/call-out relations
are set to |column| or |after|.


\subsection{Float and footnote relations}

It is possible to direct the algorithm to check on each column if
there are footnotes, and if so to prevent it from placing floats in
the bottom 
area. In theory it might be possible that a forbidden constellation
might resolve itself once the algorithm has added further floats,
e.g., it could be the case that by adding additional floats the
offending footnote gets moved to a different column. However, checking
for this would mean potentially large backtracking so the algorithm
uses a conservative approach and simply considers a trial as failed if
footnotes and bottom areas collide.

It is planned to allow a designer the choice of specifying where the
footnotes should be placed in relation to any bottom floats (if the
combination is allowed). Right now this is not implemented and
column footnotes will always appear below the text column, i.e., above
any bottom floats.




\subsection{Area statuses}

For each area the algorithm keeps track about whether or not it is
closed for individual float types, e.g., is not accepting any more
floats of type ``figure'' or closed for all types. The status of an
area can change due to floats being placed into other areas (this
might, for example, close earlier areas, or areas that overlap) or it
can change due to the fact that the area became too full in some way
(e.g., a size constraint or a number of floats constraint).

Some of these constraints can be ``relaxed'' in certain situations,
e.g., if the algorithm is directed to flush out remaining floats prior
to a certain point in the galley it will drop constraints related to
number of floats per area or size restrictions. However, if an area
was closed due to a different float being placed into some other area,
this area will stay closed in all circumstances to ensure proper
sequential placement of floats and to ensure that overlapping areas
that are forbidden as explained in section~\vref{overlap} will not
receive floats at the same time.


%\showfloatlists


\subsection{Area constraints}

The algorithm offers several possibilities for the designer to
specify how and under what circumstances a float is allowed to be
added to a certain area on the page.

As explained above all areas on a page are tried in a specific
order. This order can be specified and changed for specific parts of
the document. Areas that are closed for the current type will be
bypassed as well as areas which do not span the right number of
columns to fit the horizontal size of the float. If these initial
tests succeed the float may still fail to be placed into a certain
area if it doesn't fulfill the following set of constraints:
\begin{itemize}
\item There is an upper limit on the total number of floats that can
be placed on an individual page.
\item Each area has an upper limit of floats that can go into it.
\item After placing the float the remaining space in the text column
must be larger than a specified value.
\end{itemize}
All such constraints are customizable.

Additional constraints will probably be implemented
once there has been some experience of what controls are actually
needed to allow the specification for a reasonable number of layouts.

For example, \LaTeXe{} allows the designer to restrict the maximum size of
an area, but should one provide this or should there be a constraint
on the size of all stacked areas? Or should there be both?



\subsection{To ``Here'' or not to ``Here''}

\LaTeXe{} allows the user to control the placement of an individual float by
specifying one or more areas into which the float would be
allowed to move using single letters. As a special notation an
\texttt{h} would denote a so-called ``here'' float. Its advertised
semantics is to try placing the float ``at the position in the text
where the environment appears'' \cite[p.~197]{A-W:LLa94}.
If this is not feasible \LaTeXe{} would try the remaining
allowed possibilities on the next page, thus a float with a
\texttt{ht} specification  would either appear within the text or
at the top of the next or a later
page.\footnote{\clubpenalty=100 In two-column mode
this can in fact result in a placement on the top of the second column
even though the call-out position finally falls into the middle of
that column.}

In many cases people however prefer a ``here'' which always means
``here''. The latter form is implemented in some add-on packages for
\LaTeXe{}, however usually at the cost of allowing floats to appear out
of order.

The new model supports only the absolute ``here'' form for floats;
however, correct ordering of floats in the output is guaranteed (if
the tag generating the here float issues flushing of floats for the
current type). If
there is not enough space to place the float in a column, the float
plus the preceding text line\footnote{More precisely the column is
broken at the last breakpoint preceding the current position which is
normally one line above but could be more (or less).} is moved to the
next column/\allowbreak page.


\subsection{Grid layout}

To produce layouts with elements placed on an underlying grid
(typically with grid points vertically separated by |\baselineskip|)
the algorithm assumes that certain parts of the text column, e.g.,
normal text will automatically align on the grid as long as the first
line is positioned on the grid. A further assumption is that such
parts of the column do not contain stretchable amounts of vertical
glue so that they are not subject to stretching or shrinking if the
material is adjusted to fit a given size.

Given these assumptions, the algorithm proceeds by ensuring that the
space taking up by floats (including their separating white spaces) is
always of a size such that the remaining space for the text part of the
columns is allowing for an integral number of grid lines. This is
achieved by stretching or shrinking the space separating the areas
from the text appropriately while building the page as explained in
section\vref{sec:float-place}.

Within the text column there are typically a number of ``display
objects'' such as headings, equations, quotations, lists, etc\@. which
should not be aligned on the grid. Instead, typically the text before
and after is supposed to lie on the grid\footnote{In some cases,
depending on the design, parts of the structure might be supposed to
aligned as well.} This is supported by allowing to mark lines of text
(or more generally points in the galley) to ``snap to the nearest grid
point''. One can think of the implementation working by taking the
column material up to the marked line and putting it into a vertical
box of the size of the nearest possible grid point. By this approach
stretchable glue around such a display object will allow the text line
that should snap to the grid move into the correct position. This box
is then given back to the page builder to assemble more material for
the column. Since this way the preceding part of the column becomes
rigid; thus a later request for snapping to the grid will only stretch
or shrink material further down the column.

A prototype implementation that makes most standard \LaTeX{} objects,
like headings, displays, etc.\ support grid design is available with
the package \texttt{xo-grid}. It is used for typesetting this
document.


\section{User control}


\subsection{Column and page breaks}

Breaking of columns and pages can be controlled from the source
document by placing special tags into it. The |\columnbreak| command
ends the current column after the current line (if used in horizontal
mode). Similarly the |\pagebreak| command ends the current
page.\footnote{At the moment these commands force a break; there is no
possibility, as in \LaTeXe{} to suggest only that the current point
is a good  or bad break.}


\subsection{Manual float flushing}

The flush float functionality is available within the source document
via the command |\flushfloats|. This command takes two optional arguments
which, if present, denote the float type to flush (by default all) and the
``fuzziness'' of the flush (by default |strict|). Other allowed values
for the fuzziness are |column|, |page|, or |spread|. If a type is
specified for flushing, effectively all types with the same float
sequence class are flushed to preserve the ordering.



\subsection{Specifying preferred areas}

At the time of writing the document source interface for specifying
the group of areas into which a float is allowed to move is not yet
decided. One could envision keeping the original \LaTeX{} interface to
float environments with optional argument. In that case something like
|[t]| could be internally interpreted as ``any top area that exists''
and translated into a list such as |t12 t11 t21|. But other interfaces
are conceivable as well.



\subsection{Manually position all floats}

Any algorithm that automatically places all floats may fail to produce
adequate results in some situations. In \LaTeXe{} the user was offered
only the optional arguments of the float environments and by this
method and by moving floats slightly in the source document one was
finally able to change the formatting as needed.

This was a time consuming and error prone manual task and any slight
change in the source document text was likely to result in making this
work obsolete.

To improve on this situation the new algorithm can be directed to
write out a file containing all of its float\footnote{Floats in this
contect mean ``traditional'' floats, not footnotes or marginpars.}
selections (an example is
shown in table~\vref{fig:fpl}).
%
\begin{table}
\begin{minipage}{\columnwidth-24pt}
\footnotesize
\begin{verbatim}
Page: 1 (1)
    Area: t13
        Float: 4 (figure 4) []
    Area: b21
        Float: 2 (figure 2) [mylab:fig1]
    Area: t31
        Float: 3 (figure 3) [mylab:fig2]

    Area: hhh
        Float: 11 (table 1) []

Page: 2 (2)
    Area: t13
        Float: 8 (figure 8) []
    Area: t22
        Float: 5 (figure 5) []
    Area: b11
        Float: 6 (figure 6) [mylab:fig3]
    Area: b31
        Float: 7 (figure 7) [mylab:fig4]
\end{verbatim}
\end{minipage}
\caption{An example \texttt{fpl} file}\label{fig:fpl}
\end{table}
%
By simple drag and drop the user can produce alterations to this
selection. If such a modified file is stored as |\jobname.fpc| then
the algorithm will use these selections without attempting to apply
any of its internal rules. Thus the formatting will happen exactly as
specified.\footnote{If the floats are stored within the source
document at the point of their call-outs, the algorithm will be able
to position a float only if it has already encountered the float in the
source document. This means that one can move a float arbitrarily
forward but only to a limited extent before its call-out position. If
the floats are stored externally to the source document this
restriction does not apply.}

Beside moving floats between float areas it will be possible to move
floats in and out of the special area called |hhh| which represents a
list of all ``here'' floats on the page. If a float is moved into the
``here'' area it means that it will be positioned as a here float at
the point of its call-out.


As an extension to this method we are experimenting with restricting
the manual control only to parts of the document, e.g., allowing the
user to manually fix a single chapter but have the algorithm determine
the remainder. We also plan to integrate column length control in this
way, so that it becomes easily possible to run a page or double-spread
long or short by specifying this externally rather than via tags in
the source document.



\subsection[Tracing the algorithm's behavior]
           {Tracing the algorithm's\\ behavior}

In contrast to the \LaTeXe{} output routine, which is a black box as far
as the user is concerned, the new algorithm tries hard to make its
decision process comprehensible. Table~\vref{tab:progress} shows
\ifnum\ProcessControl=0
  \begin{table*}
  \caption{Progress output of the algorithm}\label{tab:progress}
\else
%
% FMi: people reading the source will notice that there isn't yet any
%      interface for spanning floats ... so we have to do this by hand
%
  \def\SPANCNT{2}%
  \begin{table}[t12,b12]
  \caption{Progress output of the algorithm}\label{tab:progress}
  \begin{minipage}{1.3\columnwidth}
\fi
%
\footnotesize
\begin{verbatim}
=======================================================================
 STATS: floats waiting = 2 on page 13
=======================================================================
Float: \bx@E   {5}  {table} (floats)  {5}   {Statistics from the algorithm}
 area trial: b12  -> failed: span count b12 /= 1
 area trial: b11  -> accepted
Float: \bx@F   {6}  {table} (floats)  {6}   {Running times of the algorithm}
 area trial: b12  -> failed: span count b12 /= 1
 area trial: b11  -> failed: b11 float num reached (1)
 area trial: b21  -> failed: area below flush point (2=2, b21)
 -> failed: --> retry with relaxed conditions
 
 area trial: b12  -> failed: span count b12 /= 1
 area trial: b11  -> accepted
 STATS: trials = 7
\end{verbatim}
\ifnum\ProcessControl=0
  \end{table*}
\else
  \end{minipage}
  \end{table}\def\SPANCNT{}%
\fi
%
a sample output produced by it. It shows for each float which areas
have been tried, why they were rejected etc. There is also an option
which produces about 1000 times as much information but the latter is
probably useful only for debugging the system in case there are errors
in the code.



\subsection{Manually aligning text in grid layout}

If the algorithm produces grid layout it automatically aligns certain
text lines on the underlying grid. For manual control this
functionality is also provided with the command |\TextAlignGrid| which
will align the current text line on the grid. By issuing a
|\IgnoreAlignToGrid| command grid alignment will be temporarily
disabled, while |\ObeyAlignToGrid| will reestablish automatic grid
processing.


\section{Layout Specification}

In the class file the designer is given control over the algorithm's
behavior in all the aspects described above (and several more). 

The layout specifications are done through the new template and
instance concept, see~\cite{tub:MCR99-3}. 
Additional information such as experimental code, further
documentation, etc.\ can
be found on the \LaTeX{} project web site at:
\begin{quote}
\url{http://www.latex-project.org}
\end{quote}

In contrast to the algorithm itself, which in its basic functionality
now seems to be stable and reliable, the design interface is far
more experimental. Thus the example declarations given below represent
only the current state of thought (or of implementation) 
and are likely to be modified at any moment.


\subsection{Float type declarations}

Float types are declared using the  command |\DeclareFloatType| which
takes two arguments: the name of the type which is declared and in the
second argument a list of key value pair which describe the
properties of the float type, e.g.,
\begin{verbatim}
  \DeclareFloatType{figure}
    {
     sequence-class-id  = floats,
     toc-extension      = lof,
     caption-text       = \figurename,
     numbered-boolean   = true,
     numbered-id        = figure,
     numbered-within-id = section,
     numbered-action    = 
          \thesection.\arabic{figure},
     body-decls         = ,
    }
\end{verbatim}
The \texttt{sequence-class-id} key defines to which float sequence
class the type belongs to. If it is absent a sequence class with the
same name as the type is assumed. The sequence class will be
automatically initialized if not referenced before.

The \texttt{toc-extension} key defines the extension to be used to
write the caption to when generating ``List of floats'' listings. By
using the same extension with different types it is possible to
generate combined listings, such as ``List of table and figures''.

The \texttt{caption-text} key defines the fixed text to be used as
part of the caption text together with the float number if present,
e.g., |Figure|. This information is passed to the caption formatting
template so the actual formatting is defined there.

The \texttt{numbered-boolean} defines whether or not floats of this
type are numbered.

The \texttt{numbered-id} key defines the name of the counter to use
when numbering floats. If absent a counter with the same name as the
type is assumed. By using the same counter with different types it is
possible to use a single numbering scheme---in that case the
\texttt{sequence-class-id} for these types should probably be
identical as well to avoid strange numbering sequences within the
document.

The \texttt{numbered-within-id} key defines the name of the ``within''
counter, i.e., the counter which if stepped resets the numbering. If
the value is empty or not set the float type is numbered in a single
sequence throughout the document.

The \texttt{numbered-action} key defines the representation of the
float number, as used in the caption and by the |\ref|, |\label|
mechanism. The default is |\arabic{|\meta{counter}|}|.

The \texttt{body-decls} key can hold formating instructions that
should apply to the float body. They can assume a normalized
formatting environment already set up by the algorithm.

The declaration of a new float type automatically defines the
necessary user document environments.


\subsection{Float area declarations}

Any float area that is going to be used at some stage by the algorithm
needs to be declared beforehand. This is done through the
|\DeclareFloatArea| command which takes two arguments: the name of the
area (which has to follow the conventions explained in
section~\vref{overlap}) and a list of key/value pairs describing the
characteristics of the area.
\begin{verbatim}
  \DeclareFloatArea{t22}
     {
       class-close-list = {t11,b11},
       all-close-list   = {t12,t32},
       max-float-num    = 2,
     }
\end{verbatim}
 As of today an area is characterized through the maximum number of
floats it is allowed to receive (\texttt{max-float-num}) and through
two lists which tell the algorithm which other areas are affected by
adding a float to the current area. The list \texttt{class-close-list}
enumerates all areas which are not allowed to receive additional
floats of the same sequence class as the float that has been placed 
into the current area, while
the list \texttt{all-close-list} contains the information about all
areas that are to be completely closed the moment a float is received
in the current area.

The \texttt{class-close-list} key is primarily intended to specify a
partial order on the areas to ensure that floats are not getting out
of sequence in the output. For example, the above declaration says: if
a float is placed into area |t22|, i.e., a top area starting at column
two and spanning two columns, then the single column areas |t11| and
|b11| (i.e., those of the first column) are closed for floats of the
same class. However, assuming this example is part of a declaration for
a four column layout which could have areas like |t14| or |t13|, there
is nothing said about closing those areas. Thus in this particular
layout a float spanning three or four columns would still be allowed
to go on top.

On the other hand the \texttt{all-close-list} key is available to
ensure more visual constraints, e.g, ``if |t12| gets filled we don't
want to have |b12| filled as well, we only want |b22| in this
case''. In addition it is needed to implement the restriction about
overlapping float areas as described in section~\vref{overlap}, e.g.,
in the example declaration |t12| and |t32| are closed since they
partly overlap with |t22|.\footnote{As mentioned before, this
restriction might be lifted in a later version of the algorithm; as
long as it is required one could alternatively add those areas behind
the scenes to avoid runtime problems.}





\subsection{Footnote formatting declarations}

The formatting of footnotes is specified by declaring instance(s) of
type |footnotesetup|. At the moment three templates are available
though they should be considered only as prototypes: the template
|std| produces conventional footnotes below each column, the template
|ftnright| collects all footnotes and typesets them in the rightmost
column, and the |margin| template collects and typesets them in the
right outer margin.

The keys of the above templates are providing only a rudimentary
flexibility (to say it positively), in a production version all of
them would need a large number of extensions.
As an example
\begin{verbatim}
  \DeclareInstance{footnotesetup}
    {mainmatter}{std}
    {
     text-sep    = 14pt plus 3pt,
     max-height  = 8in,
    }
\end{verbatim}
would declare the named instance |mainmatter| that provides footnotes
below columns with a separation of |14pt+| and a maximum height for
footnotes per column being |8in|.

Instances like this can then be used in the declaration for a
particular page layout as explained below. Alternatively one could use
unnamed instances there using the |\UseTemplate| method.


\subsection{Page setup declarations}

At the heart of the layout declaration are instances of the type
|pagesetup2.|\footnote{The number 2 has historical reasons and will
vanish again at some point in the future.} 
An example setup showing all currently available keys is given in
%
\ifnum\ProcessControl=0
  \begin{table*}
  \caption{Example declaration for the \texttt{pagesetup2} template
     showing all currently available keys}\label{tab:pagesetup2}
\else
  \def\SPANCNT{2}%
  \begin{table}[t12,b12]
  \caption{Example declaration for the \texttt{pagesetup2} template
     showing all currently available keys}\label{tab:pagesetup2}
  \begin{minipage}{1.3\columnwidth}
\fi
%
\footnotesize
\begin{verbatim}
  \DeclareInstance{pagesetup2}{mainmatter}{std}
    {
% column specification
     column-num                             = 2,
     column-width                           = 220pt,
     column-height                          = 610pt,
     column-sep                             = 20pt,
% float constraint specification
     max-float-num                          = 3,
     float-callout-constraint               = after,
     float-callout-span-constraint          = strict,
     bottom-float-footnote-constraint       = forbidden,
     flush-float-callout-constraint         = page,
     flush-float-callout-span-constraint    = flexible,
     flush-bottom-float-footnote-constraint = none,
% area specification
     area-list                              = {t12,t11,b11,b12,t21,b21},
     defer-class-close-list                 = {t12,t11,b11,b12,t21,b21},
     defer-all-close-list                   = ,
     float-float-sep     = 15pt,
     float-text-sep      = 30pt minus 8pt,
     float-area-sep      = 15pt,
     float-inline-sep    = 6pt minus 2pt,
% grid specification
     grid-point-sep      = 12pt,
% footnote etc specification 
     footnote-setup                         = mainmatter,
    }
\end{verbatim}
\ifnum\ProcessControl=0
  \end{table*}
\else
  \end{minipage}
  \end{table}\def\SPANCNT{}%
\fi
%
table~\vref{tab:pagesetup2}.


\subsubsection{Column specification}

The first four keys (|column-num|, |column-width|, |column-height|,
and  |column-sep|) describe the column structure of the page layout
being defined, i.e., in this case a two-column layout.

\subsubsection{Float constraint specification}

The following four keys define the standard constraints for the
algorithm when placing floats: \texttt{max-\allowbreak
float-\allowbreak num} is the maximum number of floats that can go on
a normal page,
\texttt{float-\allowbreak callout-\allowbreak constraint} defines 
what kind of relations between float and call-out are allowed,
possible values are explained on~\vpageref{fl-callout-relations}.

\texttt{float-\allowbreak callout-\allowbreak span-\allowbreak
constraint} handles the interpretation of spanning floats and is
explained on~\vpageref{fl-span-relation}, and
\texttt{bottom-\allowbreak float-\allowbreak footnote-\allowbreak
constraint} defines whether or not bottom floats are allowed in case
of footnotes. 

The last three constraints are replaced by
\texttt{flush-\allowbreak float-\allowbreak callout-\allowbreak
constraint}, \texttt{flush-\allowbreak float-\allowbreak
 callout-\allowbreak span-\allowbreak constraint}, and
\texttt{flush-\allowbreak bottom-\allowbreak float-\allowbreak
footnote-\allowbreak constraint} in case flushing can't be done
without relaxing the conditions (\texttt{max-\allowbreak
float-\allowbreak num} is disregarded in that case automatically).

\subsubsection{Float area specification}

The key |area-list| defines all float areas that are allowed in this
page layout as well as defining the order in which the areas are tried
when placing floats.  
The keys \texttt{defer-class-close-list} and
\texttt{defer-all-close-list} define the ``closing actions'' for the
special area which receives the floats that could not be placed. E.g.,
if a float of a certain class can't be placed then all areas listed in
\texttt{defer-class-close-list} will be closed for this class of
floats. In other
words the two keys are comparable to the ones available for area
declarations.

Thus these keys together with the keys from the area
declarations are most important to guarantee a sensible order of
floats on the formatted page.

In an earlier implementation of the algorithm a simpler scheme was
used: there was a single area list which was shortened whenever a
float couldn't be placed into it thereby confining the remaining
floats to this restricted selection. This works fine as long as there
are mainly single column floats since in this case the area can be
reasonably ordered into a single sequence. However the moment spanning
floats are supported the situation gets less straightforward. Is it
allowed to place a later float into |t12| if there is already a float
in the area |t11|?
%% bb -- that's easy!  a wider float always takes precedence over a
%%       narrower one in the same column span, so t21 wins.
%% FMi -- so if there is a t11 float t21 is always closed and a later
%%        spanning float will always be deferred -> i doubt that all
%%        designers/editors would like to see this as a forced rule
%%        :-)

It is quite likely that the current controls will turn out to be too
crude. This will be seen once a suitable number of layouts have been
produced under this scheme (or couldn't be produced because they
turned out to be unspecifiable).

There needs to be space between floats in an area and areas need to be
separated from each other, as well as from the column text. For this we
have the following keys:
\texttt{float-float-sep} is the separation between two floats in an area,
\texttt{float-area-sep} is the separation between two vertically
adjacent areas, and \texttt{float-text-sep} finally is the separation
between a float area and the column text.\footnote{A possible
extension would be to allow ornamental material in place of white space.}
The separation between inline floats and surrounding text is given by
\texttt{float-inline-sep}.


\subsubsection{Grid specification}

To produce a grid based design the \texttt{grid-point-sep} needs to be
given a positive dimension. This defines the distance between grid
points on which the algorithm aligns column text, inline floats,
etc.\footnote{Setting this parameter is not sufficient: to make grid
setting possible several other parameters need to be set to suitable
values as well, e.g., the distance between baselines should be
compatible and the column height need to be a multiple of this value.}

To align column text at a grid point the algorithm will extend the
\texttt{float-text-sep} space. Alternatively, if the nearest grid
point can be reached by shrinking that space (assuming its
specification contains a |minus| component) the algorithm will use
that grid point instead. In a similar fashion the space around an
inline float will be determined by the value of
\texttt{float-inline-sep}.



\subsubsection{Footnote, etc.~specification}

Finally the key \texttt{footnote-setup} receives an instance of a
\texttt{footnotesetup} template, therby defining how footnotes are
handled and presented.

What is clearly missing here is handling of other page elements such
as running headers and footers, the folio, etc. This will be added
soon.


\subsection{Float formatting declarations}

For the attachment of captions to floats there exists a prototype
interface using templates of the type |buildfloat|. At the time of
writing, available templates are |centeredbelow|, |centeredabove|, and
|bottomright|, which center the caption below or above the float body
or place it to the right of it, aligned with the bottom of the float
body. All of them would need to be generalized for a production
system to become more flexible.

When trial-formatting a float the algorithm checks for the existence
of a number of |buildfloat| instances and uses the first one that
exists to build the float. More precisely it first checks if an
instance with the name \meta{area}|-|\meta{type} exists, then it looks
for \meta{area}, then for \meta{type}, and finally, if none of them
exists, for an instance with the name |default|. So at least the latter
instance has to be declared by the class.
\begin{verbatim}
  \DeclareInstance{buildfloat}{default}
     {centeredbelow}{}
  \DeclareInstance{buildfloat}{table}
     {centeredabove}{}
  \DeclareInstance{buildfloat}{t31}
     {bottomright}{}
  \DeclareInstance{buildfloat}{t22}
     {bottomright}{}
\end{verbatim}
The example declaration above defines the placement of captions above
tables and below for all other types, with the exception of the areas |t31|
and |t22| where the captions are set to the side.


\section[Performance of the algorithm]
        {Performance of the\\algorithm}

To test the performance of the algorithm we prepared a somewhat
ridiculous test file containing three types of floats (``figures'',
``tables'', and ``algorithms'') with a total number of 47 floats. 
The chosen
layout had 3 columns and 11 potential float areas.  Figure captions
have been placed below the float while with tables and algorithms the
caption was placed on top. The exception was the top areas adjacent to
the outer margin: floats placed there got their captions placed to
the right and partly into the margin.  Footnotes were collected for all
columns and placed in the outer margin.

Floats had to strictly follow  their call-out and a maximum of
ten floats was allowed per page, i.e., roughly three per column.

Since the document contained many floats early on (24 on page one) and
the first of these was especially constructed to be not placeable the
first time around, the algorithm had to work hard to place all the
dangling floats. Table~\vref{tab:stats} 
%
\begin{table}
\footnotesize
\begin{minipage}{\columnwidth}
\begin{verbatim}
 STATS: floats waiting = 24 on page 1
 STATS: trials = 286
 STATS: floats waiting = 19 on page 2 (float page)
 STATS: trials = 159
 STATS: floats waiting = 37 on page 2
 STATS: trials = 397
 STATS: floats waiting = 19 on page 3 (float page)
 STATS: trials = 166
 STATS: floats waiting = 7 on page 4 (float page)
 STATS: trials = 41
 STATS: floats waiting = 20 on page 4
 STATS: trials = 204
 STATS: floats waiting = 5 on page 5 (float page)
 STATS: trials = 27
 STATS: floats waiting = 12 on page 5
 STATS: trials = 108
 STATS: floats waiting = 0 on page 6 (float page)
 STATS: trials = 0
 STATS: floats waiting = 6 on page 6
 STATS: trials = 57
          ...
 STATS: floats waiting = 6 on page 12 (float page)
 STATS: trials = 26
 STATS: floats waiting = 6 on page 12
 STATS: trials = 37
 STATS: floats waiting = 0 on page 13
 STATS: trials = 0
\end{verbatim}
\end{minipage}
\caption{Statistics from the algorithm}\label{tab:stats}
\end{table}
%
shows some statistics as produced by the algorithm on the number of
trials necessary (the highest number was $397$ for $37$ floats; by
comparison, equation~(\ref{eq:binom})~\vpageref{eq:binom} would give
$22595200368$ which would probably take a bit longer to
evaluate). Note that on the third page the algorithm was able to
produce a float page, on all other pages the float page trial was
unsuccessful.

Table~\vref{tab:runs} shows the running times 
%
\begin{table}
\footnotesize
\begin{tabular}{rrr}
     &    \llap{P\textsc{iii}} (650MHz) & 486DX4  (75MHz)\\[6pt]
&\multicolumn2{c}{no tracing} \\[4pt]
real &    0m1.533s & 0m27.633s \\
user &    0m1.460s & 0m26.940s \\
sys  &    0m0.050s & 0m0.690s \\[6pt]
&\multicolumn2{c}{progress information} \\[4pt]
real &    0m3.116s & 0m36.885s \\
user &    0m1.740s & 0m34.470s \\
sys  &    0m0.080s & 0m2.420s \\[6pt]
&\multicolumn2{c}{full tracing} \\[4pt]
real &    0m7.833s &  1m22.480s \\
user &    0m2.720s &  1m7.890s \\
sys  &    0m0.280s &  0m12.360s \\
\end{tabular}
\caption{Running times of the algorithm}\label{tab:runs}
\end{table}
%
needed to produce the final document of 13 pages when the algorithm is
used with different tracing settings. The test machines were a Pentium
\textsc{iii}
650 machine and an older laptop with a 486 processor. In both cases
\TeX{} was run straight from a \TeX{} Live~4 CD.

These times show that the algorithm has an acceptable time performance
since even on a 486 the average time to produce a page is roughly 2
seconds.



\section{Outlook}

While the current algorithm performs well there are several areas in
which its functionality could and probably should be extended. The most
important points are given in the following list.
\begin{itemize}
\item
  Balancing of partial pages, comparable to the way the
  \texttt{multicol} package works, should be implemented to allow for
  layouts where, for example, a heading should span across all
  columns.
\item
  We intend to provide more control over the marginal areas, allowing
  for marginal floats as well as other objects in the margin, properly
  interacting with each other.
\item
  Without much effort the algorithm could be extended to properly
  support double-spreads so this should be added some time soon.
\item
 Once the algorithm has decided which floats to place onto a page one
 could add a post-processing step in which the placement could be
 reconsidered according to different rules. For example, if the
 call-out relation is |page| then floats will tend to be placed in the
 left-hand columns. This is fine as long as there are many floats to
 process but on a page with only a few floats one might want to
 redistribute them differently once it is clear which floats could go
 onto the page.
\item
 Since it is known beforehand how many floats are actively waiting to
 be placed, one could use a different algorithm that tries all
 possible combinations as long as there are only a limited number of
 floats to be placed. The boundary at which the algorithm changes
 behavior could be made customizable so that people with faster
 machines (or more patience) could have the search for optimum running
 for as many floats as they like.
\end{itemize}

%\showfloatlists



\bibliographystyle{plain}
\bibliography{string,book,tub99}

\vfill

\end{document}


% $Log$
% Revision 1.3  2004/10/01 21:46:48  mittelba
% many further updates, still a lot to do
%
% Revision 1.2  2004/09/27 20:06:28  mittelba
% in the middle of normalizing to expl3 syntax
%
% Revision 1.1  2001/07/26 19:55:12  latex3
% original web distrib
%
% Revision 1.19  2000/08/11 07:08:04  latex3
% small updates
%
% Revision 1.18  2000/08/04 15:39:04  latex3
% added documentation on grid layout
%
% Revision 1.17  2000/07/22 06:29:55  latex3
% documentation float class sequence
% documentation float type options
% documentation new page setup options
%
% Revision 1.16  2000/07/02 19:33:15  latex3
% as send off to TUG 2000
%
% Revision 1.15  2000/06/29 17:13:56  latex3
% corrections by David
%
% Revision 1.14  2000/06/22 20:41:58  latex3
% made formatting possibilities interactive
%
% Revision 1.13  2000/06/16 10:10:30  latex3
% added handling of spanning floats / call-out constraints
%
% Revision 1.12  2000/06/15 15:19:03  latex3
% documented new syntax for area names
%
% Revision 1.11  2000/06/15 09:16:51  latex3
% added documentation on fuzzy flushing
%
% Revision 1.10  2000/06/08 15:32:37  latex3
% fixing typo in the math display
%
% Revision 1.9  2000/05/19 08:23:11  latex3
% incorporated Barbara's and Phil's copy-editing suggestions
%
% Revision 1.8  2000/05/09 19:02:42  latex3
% not integrated BB's edits yet
%
% Revision 1.7  2000/05/09 13:35:01  latex3
% next version to BB
%
% Revision 1.6  2000/05/08 20:41:57  latex3
% my additions before folding in Barbara's edits
%
% Revision 1.5  2000/05/08 15:33:25  latex3
% as send to Barbara
%
% Revision 1.4  2000/05/08 11:22:37  latex3
% *** empty log message ***
%
% Revision 1.3  2000/05/04 08:53:37  latex3
% update title
%
% Revision 1.2  2000/05/04 08:51:12  latex3
% first draft as send to David and Chris
%

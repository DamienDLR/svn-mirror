% \iffalse
%
%% File l3rand.dtx Copyright (C) 2011,2012 The LaTeX3 Project
%%
%% It may be distributed and/or modified under the conditions of the
%% LaTeX Project Public License (LPPL), either version 1.3c of this
%% license or (at your option) any later version.  The latest version
%% of this license is in the file
%%
%%    http://www.latex-project.org/lppl.txt
%%
%% This file is part of the "l3trial bundle" (The Work in LPPL)
%% and all files in that bundle must be distributed together.
%%
%% The released version of this bundle is available from CTAN.
%%
%% -----------------------------------------------------------------------
%%
%% The development version of the bundle can be found at
%%
%%    http://www.latex-project.org/svnroot/experimental/trunk/
%%
%% for those people who are interested.
%%
%%%%%%%%%%%
%% NOTE: %%
%%%%%%%%%%%
%%
%%   Snapshots taken from the repository represent work in progress and may
%%   not work or may contain conflicting material!  We therefore ask
%%   people _not_ to put them into distributions, archives, etc. without
%%   prior consultation with the LaTeX Project Team.
%%
%% -----------------------------------------------------------------------
%%
%
%<*driver|package>
\RequirePackage{expl3}
\GetIdInfo$Id$
  {L3 Experimental pseudo-random number generator}
%</driver|package>
%<*driver>
\documentclass[full]{l3doc}
\usepackage{amsmath}
\begin{document}
  \DocInput{\jobname.dtx}
\end{document}
%</driver>
% \fi
%
% \title{^^A
%   The \pkg{l3rand} package\\ Pseudo-random number generator^^A
%   \thanks{This file describes v\ExplFileVersion,
%      last revised \ExplFileDate.}^^A
% }
%
% \author{^^A
%  The \LaTeX3 Project\thanks
%    {^^A
%      E-mail:
%        \href{mailto:latex-team@latex-project.org}
%          {latex-team@latex-project.org}^^A
%    }^^A
% }
%
% \date{Released \ExplFileDate}
%
% \maketitle
%
% \begin{documentation}
%
% \section{\pkg{l3rand} documentation}
%
% \LaTeX3 comes with a pseudo-random number generator,
% which has a period slightly greater than $2^{32}$.
% The functions which produce random numbers can be used
% in an expandable context, but \enquote{cleaning}
% the random number generator with \cs{random_clean:}
% as often as possible improves performance.
%
% For example, to produce a random $10$-letter \enquote{word},
% you could do
% \begin{verbatim}
%   \cs_generate_variant:Nn \tl_item:nn { nx }
%   \prg_replicate:nn { 10 }
%     {
%       \tl_item:nx
%         { ABCDEFGHIJKLMNOPQRSTUVWXYZ }
%         { \rand_range:nn {0} {25} }
%     }
%   \rand_clean:
% \end{verbatim}
%
% \begin{function}{\rand_clean:}
%   \begin{syntax}
%     \cs{rand_clean:}
%   \end{syntax}
%   This non-expandable function improves the performance
%   of the expandable commands by reactualizing some data
%   which they use. Use it liberally, optimally between every
%   call to \cs{rand_get:} or \cs{rand_range:nn}.
% \end{function}
%
% \begin{function}[EXP]{\rand_get:}
%   \begin{syntax}
%     \cs{rand_get:}
%   \end{syntax}
%   Expands to a pseudo-random halfword integer,
%   equidistributed in the interval $[0,65535]$.
% \end{function}
%
% \begin{macro}[EXP]{\rand_range:nn}
%   \begin{syntax}
%     \cs{rand_range:nn} \Arg{min} \Arg{max}
%   \end{syntax}
%   Expands to a pseudo-random integer, equidistributed
%   in the range between \meta{min} and \meta{max} inclusive.
% \end{macro}
%
% \begin{function}{\rand_seed:n}
%   \begin{syntax}
%     \cs{rand_seed:n} \Arg{int}
%   \end{syntax}
%   Sets the seed to \meta{int}.
%   By default, the seed is $1414618349$.
% \end{function}
%
% \begin{function}{\rand_seed_from_time:}
%   \begin{syntax}
%     \cs{rand_seed_from_time:}
%   \end{syntax}
%   Sets the seed to a value determined using the date and time.
% \end{function}
%
% \begin{variable}{\g_rand_a_int, \g_rand_b_int}
%   Power-users might want to change those from their
%   default values $a=4266$ and $b=10090$.
% \end{variable}
%
% Open questions: plenty which I forgot.
% \begin{itemize}
% \item Add a function to pick a random item in a sequence, a tl,
%   a clist, a property list.
% \item Should the \meta{max} be included for \cs{rand_range:nn}?
%   Pro: otherwise, \cs{c_max_int} is unattainable. Contra: not
%   consistent with \cs{seq_item:n} and friends. Pro: consistent
%   with \cs{int_step_inline:nnnn}.
% \item Should expandable errors state which module they come from?
% \end{itemize}
%
% \end{documentation}
%
% \begin{implementation}
%
% \section{\pkg{l3rand} implementation}
%
%    \begin{macrocode}
%<*initex|package>
%    \end{macrocode}
%
%    \begin{macrocode}
%<*package>
\ProvidesExplPackage
  {\ExplFileName}{\ExplFileDate}{\ExplFileVersion}{\ExplFileDescription}
\package_check_loaded_expl:
%</package>
%    \end{macrocode}
%
% Generate pseudo-random numbers with a generalization of linear
% congruential generators (see Knuth, section 3.2.1. and further).
% We work modulo the prime number $p = 65537$, and use the linear
% recursion relation $X_n \equiv 4266 X_{n-1} + 10090 X_{n-2} \mod p$.
% For every $n$, $X_n \in [0, 65536]$, and this relation always
% gives a positive result with no overflow. The |\int_mod:nn| function
% operates safely to produce the next number in $[0,65536]$.
%
% These values were taken so that the polynomial $x^2 - 4266 x - 10090$
% is primitive modulo $p$, which ensures that the recursion has a
% maximal period, namely $p^2 - 1 > 2^{32}$. Pairs of consecutive
% random numbers take all combined values modulo $p$, except for
% the pair $(0,0)$. In order to get a fully uniform distribution
% we skip any $0$ appearing in the sequence, and subtract $1$ to
% each number, thus getting numbers in the range $[0,65535]$.
%
% We denote the last pseudo-random number by $X$ and the previous one
% by $Y$. There will be two channels of computation for those numbers,
% non-expandably, and with a more convoluted expandable technique.
%
% \subsection{Variables}
%
% \begin{variable}{\c_rand_p_int}
%   The modulus $p=65537$ is stored as \cs{c_rand_p_int}.
%   It must be prime, and at least $2^{16}$. Changing that
%   value requires updating \cs{rand_get_end:www}.
%    \begin{macrocode}
\int_const:Nn \c_rand_p_int { 65537 }
%    \end{macrocode}
% \end{variable}
%
% \begin{variable}{\g_rand_a_int, \g_rand_b_int}
%   The coefficients of the linear recursion $a$ and $b$
%   are \cs{g_rand_a_int} and \cs{g_rand_b_int}.
%   The values for $a$ and $b$ must be such that the corresponding unital
%   quadratic polynomial is primitive modulo $p$. Also, details below
%   require them to be positive, and such that $(p-1)(a+b)$ does not
%   overflow.
%    \begin{macrocode}
\int_new:N \g_rand_a_int
\int_new:N \g_rand_b_int
\int_gset:Nn \g_rand_a_int { 4266 }
\int_gset:Nn \g_rand_b_int { 10090 }
%    \end{macrocode}
% \end{variable}
%
% \begin{variable}{\g_rand_X_int, \g_rand_Y_int}
% \begin{macro}{\rand_seed:n}
% \begin{macro}{\rand_seed_from_time:}
%   Starting from an arbitrary \TeX{} integer (in the range
%   $[-2^{31}+1, 2^{31}-1]$), we build two blocks of $16$ bits.
%   One key requirement is that $(0,0)$ cannot be generated,
%   and that the two resulting numbers are at most $65536$.
%    \begin{macrocode}
\int_new:N \g_rand_X_int
\int_new:N \g_rand_Y_int
\cs_new_protected:Npn \rand_seed:n #1
  {
    \int_gset:Nn \g_rand_X_int
      {
        32768 + \int_div_truncate:nn {#1} {65536}
        \int_compare:nNnF {#1} < \c_zero { + \c_one }
      }
    \int_gset:Nn \g_rand_Y_int { \int_mod:nn {#1} {65536} }
    \prg_replicate:nn \c_ten \rand_step:
  }
\cs_new_protected_nopar:Npn \rand_seed_from_time:
  {
    \exp_args:Nf
    \rand_seed:n
      {
        \int_eval:n
          {
            ( ( \tex_year:D + 99*\tex_month:D + \tex_day:D ) * 9999
              + \tex_time:D )
            \int_if_even:nF { \tex_time:D } { *\c_minus_one }
          }
      }
  }
%    \end{macrocode}
% \end{macro}
% \end{macro}
% \end{variable}
%
% \begin{variable}{\l_rand_tmp_int}
%   A temporary variable used to store the value of $X$
%   when computing the next $(X,Y)$ pair.
%    \begin{macrocode}
\int_new:N \l_rand_tmp_int
%    \end{macrocode}
% \end{variable}
%
% \subsection{Halfword random integers}
%
% To avoid overflow, we work in a rather small modulus, and the raw
% random integers which are produced lie in the range $[0,p-1]$,
% which is a little bit more than a halfword ($2^{16}$). In this
% section, we build those halfword random integers, which will be
% put together in the next section to cater for arbitrary ranges.
%
% \begin{macro}[int]{\rand_step:}
%   Generating new terms in the pseudo-random sequence is quite
%   easy when assignments are allowed. Namely, we wish to replace
%   $(X, Y)$ by $( aX + bY \mod p, X)$. Save $X$ in a temporary
%   variable, compute the new value of $X$, then store the old
%   value of $X$ in $Y$. Since $aX+bY$ is always positive (and
%   does not overflow), |\int_mod:nn| produces a non-negative
%   result, at most $p-1$.
%    \begin{macrocode}
\cs_new_protected_nopar:Npn \rand_step:
  {
    \int_set_eq:NN \l_rand_tmp_int \g_rand_X_int
    \int_gset:Nn \g_rand_X_int
      {
        \int_mod:nn
          {
            \g_rand_a_int * \g_rand_X_int
            + \g_rand_b_int * \g_rand_Y_int
          }
          { \c_rand_p_int }
      }
    \int_gset_eq:NN \g_rand_Y_int \l_rand_tmp_int
  }
%    \end{macrocode}
% \end{macro}
%
% We now work towards getting pseudo-random numbers expandably.
% The only way for \TeX{} to ive different results when expanding
% the same command is if its internal state has changed.
% Unfortunately, we cannot change $X$ and $Y$ expandably. The
% only change in the internal state of \TeX{} that can be performed
% expandably is \enquote{raising flags} (described below). We can
% only store in this manner the number $n$ of times the expandable
% pseudo-random number generator was called. To produce a new
% random number, start from $X$ and $Y$ and apply the recursion
% relation $n$ times, and raise one more flag (so as to increase $n$).
%
% \begin{macro}[int]{\rand_aux_flag_off:x}
% \begin{macro}[int,EXP]{\rand_aux_flag_on:x}
% \begin{macro}[int,EXP,TF]{\rand_aux_flag_if_on:x}
%   A flag is \enquote{on} if it is equal to \tn{relax}, and
%   \enquote{off} if it is \texttt{undefined}. In an expansion
%   context, we can test whether a flag is on or off, and turn
%   flags on: indeed, in a \cs{cs:w} \ldots{} \cs{cs_end:}
%   construction (and in |c| arguments), if the corresponding
%   control sequence is undefined, it gets assigned \tn{relax}.
%   Turning a flag back off from \tn{relax} to \texttt{undefined}
%   is not expandable.
%   In the current application, the flag label, |#1| is always
%   a positive integer.
%    \begin{macrocode}
\cs_new_protected_nopar:Npn \rand_aux_flag_off:x #1
  { \cs_set_eq:cN { l_rand_#1_flag } \c_undefined:D }
\cs_new_nopar:Npn \rand_aux_flag_on:x #1
  { \exp_args:Nc \use_none:n { l_rand_#1_flag } }
\prg_new_conditional:Npnn \rand_aux_flag_if_on:x #1 { T, TF }
  {
    \if_cs_exist:w l_rand_#1_flag \cs_end:
      \prg_return_true:
    \else:
      \prg_return_false:
    \fi:
  }
%    \end{macrocode}
% \end{macro}
% \end{macro}
% \end{macro}
%
% \begin{macro}[EXP]{\rand_get:}
% \begin{macro}[EXP,aux]{\rand_get_loop:www}
% \begin{macro}[EXP,aux]{\rand_get_end:www}
%   Count the number of raised flags, and do one step of the
%   recursion for each raised flag. The first argument of
%   \cs{rand_get_loop:www} is the label for the flag,
%   the second and third are the temporary $X$ and $Y$.
%   The new values after one step are $aX+bY \mod p$ and $X$.
%   Once we have gone through all the raised flags, the current
%   value of $X$ is roughly to be returned to the user. However,
%   to get a proper halfword, we throw away any zero, and subtract
%   $1$: the resulting range is $[0,2^{16}-1]$, equidistributed.
%    \begin{macrocode}
\cs_new_nopar:Npn \rand_get:
  {
    \int_eval:n
      {
        \rand_get_loop:www 1 ;
        \g_rand_X_int ;
        \g_rand_Y_int ;
      }
  }
\cs_new_nopar:Npn \rand_get_loop:www #1 ; #2 ; #3 ;
  {
    \rand_aux_flag_if_on:xTF {#1}
      {
        \exp_after:wN \rand_get_loop:www
        \int_value:w \int_eval:w \c_one +
      }
      { \exp_after:wN \rand_get_end:www \int_value:w }
    #1 \exp_after:wN ;
    \int_value:w
      \int_mod:nn
        { \g_rand_a_int * #2 + \g_rand_b_int * #3 }
        { \c_rand_p_int }
    ;
    #2 ;
  }
\cs_new_nopar:Npn \rand_get_end:www #1 ; #2 ; #3 ;
  {
    \rand_aux_flag_on:x {#1}
    \if_int_compare:w #2 = \c_zero
      \rand_aux_flag_on:x { \int_eval:n { #1 + \c_one } }
      \int_mod:nn { \g_rand_b_int * #3 } { \c_rand_p_int }
    \else:
      #2
    \fi:
    - \c_one
  }
%    \end{macrocode}
% \end{macro}
% \end{macro}
% \end{macro}
%
% \begin{macro}{\rand_clean:}
% \begin{macro}[aux]{\rand_clean_loop:w}
%   When using \cs{rand_get:} repeatedly, we need a mechanism
%   to eliminate the flags (\enquote{clean} the mess that was
%   brought by insisting on being expandable). Otherwise,
%   producing the $n$-th pseudo-random number takes a time
%   proportional to $n$. Cleaning involves both putting the
%   flags down and updating the values of $X$ and $Y$
%   by calling \cs{rand_step:}.
%    \begin{macrocode}
\cs_new_protected_nopar:Npn \rand_clean:
  { \rand_clean_loop:w 1 ; }
\cs_new_protected_nopar:Npn \rand_clean_loop:w #1 ;
  {
    \rand_aux_flag_if_on:xT {#1}
      {
        \rand_aux_flag_off:x {#1}
        \rand_step:
        \exp_after:wN \rand_clean_loop:w
        \int_use:N \int_eval:w \c_one + #1 \exp_after:wN ;
      }
  }
%    \end{macrocode}
% \end{macro}
% \end{macro}
%
% \subsection{Random integers in arbitrary ranges}
%
% \begin{macro}[EXP]{\rand_range:nn}
% \begin{macro}[EXP,aux]{\rand_range_aux:ww}
%   First evaluate the arguments of \cs{rand_range:nn}
%   because they would otherwise be computed many times.
%   This could be especially bad in some typical cases
%   such as picking up a random item from a sequence.
%   \begin{verbatim}
%     \seq_item:Nn \l_foo_seq
%       { \rand_range:nn {0} { \seq_count:N \l_foo_seq - 1 } }
%   \end{verbatim}
%   Once this is evaluated, check that the bounds are ordered
%   correctly. If not, return the smallest of the two,
%   and raise an error when relevant. If the bounds are correctly
%   ordered, check whether the difference |#2-#1| might cause an
%   overflow or not, and branch accordingly to
%   \cs{rand_range_large:nn} or \cs{rand_range_small:nn}.
%    \begin{macrocode}
\cs_new:Npn \rand_range:nn #1#2
  {
    \int_eval:n
      {
        \exp_after:wN \rand_range_aux:ww
        \int_value:w \int_eval:w #1 \exp_after:wN ;
        \int_value:w \int_eval:w #2 ;
      }
  }
\cs_new_nopar:Npn \rand_range_aux:ww #1 ; #2 ;
  {
    \if_int_compare:w #1 < #2 \exp_stop_f:
      \if_int_compare:w
          \int_eval:w #2 / 65536 - #1 / 65536 \int_eval_end:
          < 32767 \exp_stop_f:
        \rand_range_small:nn {#1} {#2}
      \else:
        \rand_range_large:nn {#1} {#2}
      \fi:
    \else:
      \if_int_compare:w #1 > #2 \exp_stop_f:
        \msg_expandable_kernel_error:nnnn
          { rand } { bad-range } {#1} {#2}
      \fi:
      #1
    \fi:
  }
%    \end{macrocode}
% \end{macro}
% \end{macro}
%
% \begin{macro}[EXP,aux]{\rand_range_small:nn}
% \begin{macro}[EXP,aux]{\rand_range_small_aux:nnn}
%   |\rand_get: * 32768 + \rand_get: / \c_two| produces an uniformly
%   distributed integer in the interval $[0,2^{31}-1]$. The idea is then
%   to rescale that interval to the desired range. We wish all the
%   integers in the result to have equal probability of appearing, thus,
%   among the $2^{31}$ numbers, that our random number generator can
%   produce, we will only keep
%   $k(\mathtt{\#2} - \mathtt{\#1} + 1) < 2^{31}$ of them for the
%   largest possible integer $k$. This integer is found using
%   \cs{int_div_truncate:nn}. The \enquote{outer}
%   \cs{int_div_truncate:nn} does the range rescaling and produces a
%   result which is typically between $0$ and
%   $\mathtt{\#2} - \mathtt{\#1}$ inclusive. Adding |#1| to that
%   randomly generated number would end. However, we still need to
%   throw away some of the $2^{31}$ numbers, namely those which give
%   a result larger than |#2|; in that case, we start again from
%   \cs{rand_range_small:nn} with new random numbers.
%    \begin{macrocode}
\cs_new_nopar:Npn \rand_range_small:nn #1#2
  {
    \exp_args:Nf \rand_range_small_aux:nnn
      {
        \int_eval:n
          {
            #1 +
            \int_div_truncate:nn
              { \rand_get: * 32768 + \rand_get: / \c_two }
              { \int_div_truncate:nn \c_max_int { #2 - #1 + \c_one } }
          }
      }
      {#1}
      {#2}
  }
\cs_new_nopar:Npn \rand_range_small_aux:nnn #1#2#3
  {
    \if_int_compare:w #1 > #3 \exp_stop_f:
      \rand_range_small:nn {#2} {#3}
    \else:
      #1
    \fi:
  }
%    \end{macrocode}
% \end{macro}
% \end{macro}
%
% \begin{macro}[EXP,aux]{\rand_range_large:nn}
% \begin{macro}[EXP,aux]{\rand_range_large_aux:wwnn}
% \begin{macro}[EXP,aux]{\rand_range_large_aux:nwnn}
%   In this case, |#2| and |#1| are too far apart to be able
%   to compute their difference safely. This implies that the
%   interval covers at least half of the interval
%   $[-(2^{31}-1),2^{31}-1]$. Thus, a viable (but slightly wasteful)
%   strategy is to produce random numbers in that interval,
%   and throw them away if they don't fit in the desired range.
%   The first step is to expand the random numbers, then
%   \cs{rand_range_large_aux:wwnn} feeds an integer in the
%   relevant interval to the next auxiliary, and if it fails,
%   it calls the original \cs{rand_range_large:nn} function,
%   to get new random numbers. The third function is responsible
%   for filtering out the results which fall outside the relevant
%   range. Again, it calls \cs{rand_range_large:nn} in case
%   of failure.
%    \begin{macrocode}
\cs_new_nopar:Npn \rand_range_large:nn
  {
    \exp_after:wN \rand_range_large_aux:wwnn
      \int_value:w \rand_get: \exp_after:wN ;
      \int_value:w \rand_get: ;
  }
\cs_new_nopar:Npn \rand_range_large_aux:wwnn #1 ; #2 ; #3 #4
  {
    \if_int_compare:w #1 = \c_zero
      \if_int_compare:w #2 > \c_zero
        \exp_args:Nf \rand_range_large_aux:nwnn
          { \int_eval:n { - \c_max_int + #2 - \c_one } }
      \fi:
    \else:
      \exp_args:Nf \rand_range_large_aux:nwnn
        { \int_eval:n { ( #1 - 32768 ) * 65536 + #2 } }
    \fi:
    \rand_range_large:nn {#3} {#4}
  }
\cs_new_nopar:Npn \rand_range_large_aux:nwnn
    #1 #2 \rand_range_large:nn #3 #4
  {
    #2
    \if_int_compare:w #1 > #3 \exp_stop_f:
      \if_int_compare:w #1 < #4 \exp_stop_f:
        #1
        \exp_after:wN \exp_after:wN
        \exp_after:wN \use_none:nnn
      \fi:
    \fi:
    \rand_range_large:nn {#3} {#4}
  }
%    \end{macrocode}
% \end{macro}
% \end{macro}
% \end{macro}
%
% \subsection{Messages}
%
%    \begin{macrocode}
\msg_kernel_new:nnn { rand } { bad-range }
  { Invalid~range~[#1,#2]~for~random~numbers. }
%    \end{macrocode}
%
% \subsection{Initialization of the seed}
%
% The random number generator is seeded with a completely arbitrary number.
%    \begin{macrocode}
\rand_seed:n { 1414618349 }
%    \end{macrocode}
%
%    \begin{macrocode}
%</initex|package>
%    \end{macrocode}
%
% \end{implementation}
%
% \PrintIndex
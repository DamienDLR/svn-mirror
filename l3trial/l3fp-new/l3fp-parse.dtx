% \iffalse meta-comment
%
%% File: l3fp-parse.dtx Copyright (C) 2011 The LaTeX3 Project
%%
%% It may be distributed and/or modified under the conditions of the
%% LaTeX Project Public License (LPPL), either version 1.3c of this
%% license or (at your option) any later version.  The latest version
%% of this license is in the file
%%
%%    http://www.latex-project.org/lppl.txt
%%
%% This file is part of the "l3trial bundle" (The Work in LPPL)
%% and all files in that bundle must be distributed together.
%%
%% The released version of this bundle is available from CTAN.
%%
%% -----------------------------------------------------------------------
%%
%% The development version of the bundle can be found at
%%
%%    http://www.latex-project.org/svnroot/experimental/trunk/
%%
%% for those people who are interested.
%%
%%%%%%%%%%%
%% NOTE: %%
%%%%%%%%%%%
%%
%%   Snapshots taken from the repository represent work in progress and may
%%   not work or may contain conflicting material!  We therefore ask
%%   people _not_ to put them into distributions, archives, etc. without
%%   prior consultation with the LaTeX Project Team.
%%
%% -----------------------------------------------------------------------
%%
%
%<*driver|package>
\RequirePackage{l3fp-new}
\GetIdInfo$Id$
  {L3 Experimental floating-point expression parsing}
%</driver|package>
%<*driver>
\documentclass[full]{l3doc}
\usepackage{amsmath}
\usepackage{l3fp-new}
\begin{document}
  \DocInput{\jobname.dtx}
\end{document}
%</driver>
% \fi
%
% \title{The \textsf{l3fp-parse} package\thanks{This file
%         has version number \fileversion, last
%         revised \filedate.}\\
% Floating point expression parsing}
% \author{^^A
%  The \LaTeX3 Project\thanks
%    {^^A
%      E-mail:
%        \href{mailto:latex-team@latex-project.org}
%          {latex-team@latex-project.org}^^A
%    }^^A
% }
% \date{Released \filedate}
%
% \maketitle
% \tableofcontents
%
%^^A To typeset the examples of expansion control, I'm using a hand-made
%^^A environment.
% \newcommand{\fpOperation}[1]
%   {\textcolor[rgb]{.6,.2,.2}{\ttfamily#1}}
% \newcommand{\fpPrecedence}[1]
%   {\textcolor[rgb]{.2,.2,.6}{\ttfamily#1}}
% \newcommand{\fpExpand}[2]
%   {\underline{\textcolor{red}{#1{#2}}}}
% \newenvironment{l3fp-code-example}
%   {\begin{quote}^^A
%       \edef\^{\string^}^^A
%       \let\*\fpExpand
%       \let\o\fpOperation
%       \let\p\fpPrecedence
%       \def\!{\begingroup\def\!{\endgroup\par}\color[gray]{0.5}}^^A
%       \ttfamily\frenchspacing
%   }{\end{quote}}
%
% \begin{documentation}
%
% \section{Evaluating an expression}
%
% \begin{function}[EXP]{\fp_parse:n}
%   \begin{syntax}
%     \cs{fp_parse:n} \Arg{floating point expression}
%   \end{syntax}
%   This f-expands to the internal floating point number obtained
%   as follows. It reads through the \meta{floating point expression},
%   f-expanding tokens from left to right until only non-expandable
%   tokens remain, and parses this expanded result, converting
%   floating point numbers to an internal form, and performing
%   the infix operations on those.
%   \begin{texnote}
%     Registers (integers, toks, etc.) are automatically unpacked,
%     without requiring a function such as \cs{int_use:N}. Invalid
%     tokens remaining after f-expansion will lead to unrecoverable
%     low-level TeX errors.\footnote{Bruno: describe what happens
%       in cases like $2\cs{c_three} = 6$. (In fact, not yet.)}
%   \end{texnote}
% \end{function}
%
% \section{Work plan}\label{subsec:fp-parse-workplan}
%
% The task at hand is non-trivial, and some previous failed attempts have
% shown me that the code ends up giving unreadable logs, so we'd better get
% it (almost) right the first time. Let us thus first discuss precisely
% the design before starting to write the code. To simplify matters,
% we first consider expressions with integers only.
%
% \subsection{Storing results}
%
% The main issue in parsing expressions expandably is: \enquote{where
%   in the input stream should the result be put?}
%
% One option is to place the result at the end of the expression,
% but this has several drawbacks:
% \begin{itemize}
% \item firstly it means that for long expressions we would be reaching
%   all the way to the end of the expression at every step of the
%   calculation, which can be rather expensive;
% \item secondly, when parsing parenthesized sub-expressions, we would
%   naturally place the result after the corresponding closing parenthesis.
%   But since \cs{fp_parse:n} does not assume that its argument is expanded,
%   this closing parenthesis may be hidden in a macro, and not present yet,
%   causing havoc.
% \end{itemize}
%
% The other natural option is to store the result at the start of the
% expression, and carry it as an argument of each macro. This does not
% really work either: in order to expand what follows on the input stream,
% we need to skip at each step over all the tokens in the result using
% \cs{exp_after:wN}. But this requires adding many \cs{exp_after:wN} to
% the result at each step, also an expensive process.
%
% Hence, we need to go for some fine expansion control: the result is
% stored \emph{before} the start\ldots{} A toy model that illustrates this
% idea is to try and add some positive integers which may be hidden
% within macros, or registers. Assume that one number has already been
% found, and that we want to parse the next number. The current status
% of the code may look as follows.
% \begin{quote}\ttfamily
%   \cs{exp_after:wN} \cs{add:ww}
%   \cs{int_value:w} 12345 \cs{exp_after:wN} ; \newline
%   \cs{tex_romannumeral:D} -`0 \cs{clean:w} \meta{stuff}
% \end{quote}
% Hitting this construction by one step of expansion expands
% \cs{exp_after:wN}, which triggers the primitive \cs{int_value:w},
% which reads an integer, \texttt{12345}. This integer is unfinished,
% causing the second \cs{exp_after:wN} to expand, and trigger
% the construction \cs{tex_romannumeral:D} |-`0|, which f-expands
% \cs{clean:w} (see \pkg{l3expan.dtx} for an explanation). Assume
% then that \cs{clean:w} is such that it expands \meta{stuff} to
% \emph{e.g.}, |333444;|. Once \cs{clean:w} is done expanding, we
% will obtain essentially
% \begin{quote}\ttfamily
%   \cs{exp_after:wN} \cs{add:ww} \cs{int_value:w} 12345 ; 333444 ;
% \end{quote}
% where in fact \cs{exp_after:wN} has already been expanded, and
% \cs{int_value:w} has already seen \texttt{12345}. Now,
% \cs{int_value:w} sees the \texttt{;}, and stops expanding, and
% we are left with
% \begin{quote}\ttfamily
%   \cs{add:ww} 12345 ; 333444 ;
% \end{quote}
% which can safely perform the addition by grabbing two arguments
% delimited by \texttt{;}.
%
% On this toy example, we could note that if we were to continue
% parsing the expression, then the following number should also
% be cleaned up before the next use of a binary operation such as
% \cs{add:ww}. Just like \cs{int_value:w} \texttt{12345}
% \cs{exp_after:wN} \texttt{;} expanded what follows once, we need
% \cs{add:ww} to do the calculation, and in the process to expand
% the following once. This is also true in our real application:
% all the functions of the form \cs{fp_...:ww} expand the what
% follows once. This comes at the cost of leaving tokens in the
% input stack, and we will need to be careful to waste as little
% as possible of this precious memory.
%
%
% \subsection{Precedence}
%
% A major point to keep in mind when parsing expressions is that
% different operators have different precedence. The true analog
% of our toy \cs{clean:w} macro must thus take care of that. For
% definiteness, let us assume that the operation which prompted
% \cs{clean:w} was a multiplication. Then \cs{clean:w} (expand
% and) read digits until the number is ended by some operation.
% If this is \texttt{+} or~\texttt{-}, then the multiplication
% should be calculated next, so \cs{clean:w} can simply decide
% that its job is done. However, if the operator we find is |^|,
% then this operation must be performed before returning control
% to the multiplication. This means that we need to \cs{clean:w}
% the number following |^|, and perform the calculation, then just
% end our job.
%
% Hence, each time a number is cleaned, the precedence of the
% following operation must be compared to that of the previous
% operation. The process of course has to happen recursively.
% For instance, |1+2^3*4| would involve the following steps.
% \begin{itemize}
% \item |1| is cleaned up.
% \item |2| is cleaned up.
% \item The precedences of |+| and |^| are compared. Since the
%   latter is higher, the second operand of |^| should be cleaned.
% \item |3| is cleaned up.
% \item The precedences of |^| and |*| are compared. Since the
%   former is higher, the cleaning step stops.
% \item Compute |2^3 = 8|.
% \item We now have |1+8*4|, and the operation |+| is still
%   looking for a second operand. Clean |8|.
% \item The precedences of |+| and |*| are compared. Since the
%   latter is higher, the second operand of |*| should be cleaned.
% \item |4| is cleaned up, and the end of the expression is reached.
% \item Compute |8*4 = 32|.
% \item We now have |1+8*4|, and the operation |+| is still
%   looking for a second operand. Clean |32|, and reach the end
%   of the expression.
% \item Compute |1+32 = 33|.
% \end{itemize}
% Here, there is some (expensive) redundant work: the results of
% computations should not need to be cleaned again. Thus the true definition
% is slightly more elaborate.
%
% The precedence of |(| and |)| are defined to be equal, and smaller than
% the precedence of |+| and |-|, itself smaller than |*| and |/|, smaller,
% finally, then the power operator |**| (or |^|).
%
%
% \subsection{Infix operators}
%
% The implementation that was chosen is slightly wasteful: it causes
% more nesting than necessary.\footnote{Bruno: clarify.}
% However, it is simpler to implement and to explain than a slightly
% optimized variant.\footnote{Bruno: some day, I should
%   implement the optimized version, and compare.}
%
% The cornerstone of that method is a pair of functions,
% \cs{until} and \cs{one}, which both take as their first
% argument the precedence (an integer) of the last operation.
% The f-expansion of
% \begin{quote}
%   \cs{until} \meta{prec} \cs{one} \meta{prec} \meta{stuff}
% \end{quote}
% is the internal floating point obtained by \enquote{cleaning}
% numbers which follow in the input stream, and performing
% computations until reaching an operation with a precedence
% less than or equal to \meta{prec}. This is followed by a control
% sequence of the form \cs{infix_?}, namely,
% \begin{quote}
%   \meta{floating point} \cs{infix_?}
% \end{quote}
% where |?| is the operation following that number in the input
% stream (we thus know that this operation has at most the
% precedence \meta{prec}, otherwise it would have been performed
% already).
%
% How is that expansion achieved? First, \cs{one} \meta{prec}
% reads one \meta{floating point} number, and converts it to an
% internal form, then the following operation, say |*|, is
% packed in the form \cs{infix_*}, which is fed the \meta{prec}.
% This function (one per infix operator) compares \meta{prec}
% with the precedence of the operator we just read (here |*|).
% If \meta{prec} is higher, our job is finished, and \cs{one}
% leaves \cs{c_true_bool} so that \cs{until} knows to stop.
% Otherwise, \cs{infix_*} triggers a new pair
% \cs{until} \meta{prec(*)} \cs{one} \meta{prec(*)},
% which produces the second operand \meta{floating point 2}
% for the multiplication:
% \begin{quote}
%   \cs{until} \meta{prec} \meta{floating point} \newline
%   \texttt{...} \meta{floating point 2} |;| \cs{infix_?}
% \end{quote}
% The dots are \cs{c_false_bool} \cs{fp_mul:ww}. The boolean
% tells \cs{until} that it is not done, and it expands
% (essentially) to
% \begin{quote}
%   \cs{until} \meta{prec}
%   \cs{fp_mul:ww} \meta{floating point} \meta{floating point 2}
%   \cs{tex_romannumeral:D} \texttt{-`0} \cs{infix_?} \meta{prec}
% \end{quote}
% making \TeX{} expand \cs{fp_mul:ww} before \cs{until}.
% As implemented in \pkg{l3fp-basics}, this operation expands
% what follows exactly once. This triggers \cs{tex_romannumeral:D},
% which fully expands \cs{infix_?} \meta{prec}. This compares
% the precedence of the next operation, |?|, and \meta{prec},
% and leaves a boolean (and possibly more things), which is then
% checked by \cs{until} \meta{prec} to know if the result
% of the multiplication is the end of the story, or if |?|
% should be computed as well before \cs{until} \meta{prec} ends.
%
% This should be easier to see on an example. To each infix
% operator, for instance, |*|, is associated the following data:
% \begin{itemize}
% \item a test function, \cs{infix_*}, which conditionally continues
%   the calculation or waits to be hit again by expansion;
% \item a function \fpOperation{*} (notation for \cs{fp_mul:ww})
%   which performs the actual calculation;
% \item an integer, \fpPrecedence{*}, which encodes the precedence of
%   the operator (typically, \cs{c_zero}, \cs{c_one}, or \cs{c_two}).
% \end{itemize}
% The token that is currently being expanded is underlined,
% and in red. Tokens that have not yet been read (and could
% still be hidden in macros) are in gray.
%
% In a first reading, the disinction between the \meta{precedence}
% \fpPrecedence{+}, the operation \fpOperation{+}, and the character
% token |+| should not matter. It is only required to accomodate for
% multi-token infix operators such as |**|: indeed, when controlling
% expansion, we need to skip over those tokens using \cs{exp_after:wN},
% and this only skips one token. Thus |**| needs to be replaced by a
% single token (either its precedence or its calculating function,
% depending on the place).
%
% To end the computation cleanly, we add a trailing right
% parenthesis, and give |(| and |)| the lowest precedence,
% so that \cs{until}\fpPrecedence{(} \cs{one}\fpPrecedence{(}
% reads numbers and performs operations until meeting a right
% parenthesis. This is discussed more precisely in the next section.
%
% \begin{l3fp-code-example}
%   \cs{until}\p(    \*\cs{one}\p( \! 11 + 2**3 * 5 - 9 )\!
%   \cs{until}\p( 1  \*\cs{one}\p( \! 1  + 2**3 * 5 - 9 )\!
%   \cs{until}\p( 11 \*\cs{one}\p( \!    + 2**3 * 5 - 9 )\!
%   \cs{until}\p( 11; \*\cs{infix_+}\p( \! 2**3 * 5 - 9 )\!
%   \cs{until}\p( 11; F \o+ \cs{until}\p+ \*\cs{one}\p+ \! 2**3 * 5 - 9 )\!
%   \cs{until}\p( 11; F \o+ \cs{until}\p+ 2 \*\cs{one}\p+ \!  **3 * 5 - 9 )\!
%   \cs{until}\p( 11; F \o+ \cs{until}\p+ 2; \*\cs{infix_**}\p+ \! 3 * 5 - 9 )\!
%   \cs{until}\p( 11; F \o+ \cs{until}\p+ 2;
%       F \o{**} \cs{until}\p{**} \*\cs{one}\p{**} \! 3 * 5 - 9 )\!
%   \cs{until}\p( 11; F \o+ \cs{until}\p+ 2;
%       F \o{**} \cs{until}\p{**} 3 \*\cs{one}\p{**} \! * 5 - 9 )\!
%   \cs{until}\p( 11; F \o+ \cs{until}\p+ 2;
%       F \o{**} \cs{until}\p{**} 3; \*\cs{infix_*}\p{**} \! 5 - 9 )\!
%   \cs{until}\p( 11; F \o+ \cs{until}\p+ 2;
%       F \o{**} \*\cs{until}\p{**} 3; T \cs{infix_*} \! 5 - 9 )\!
%   \cs{until}\p( 11; F \o+ \*\cs{until}\p+ 2;
%       F \o{**} 3; \cs{infix_*} \! 5 - 9 )\!
%   \cs{until}\p( 11; F \o+ \cs{until}\p+ \*\o{**} 2; 3;
%       \cs{infix_*}\p+ \! 5 - 9 )\!
%   \cs{until}\p( 11; F \o+ \cs{until}\p+ 8; \*\cs{infix_*}\p+ \! 5 - 9 )\!
%   \cs{until}\p( 11; F \o+ \cs{until}\p+ 8;
%       F \o* \cs{until}\p* \*\cs{one}\p* \! 5 - 9 )\!
%   \cs{until}\p( 11; F \o+ \cs{until}\p+ 8;
%       F \o* \cs{until}\p* 5 \*\cs{one}\p* \! - 9 )\!
%   \cs{until}\p( 11; F \o+ \cs{until}\p+ 8;
%       F \o* \cs{until}\p* 5; \*\cs{infix_-}\p* \! 9 )\!
%   \cs{until}\p( 11; F \o+ \cs{until}\p+ 8;
%       F \o* \*\cs{until}\p* 5; T \cs{infix_-} \! 9 )\!
%   \cs{until}\p( 11; F \o+ \*\cs{until}\p+ 8; F \o* 5; \cs{infix_-} \! 9 )\!
%   \cs{until}\p( 11; F \o+ \cs{until}\p+ \*\o{*} 8; 5; \cs{infix_-}\p+ \! 9 )\!
%   \cs{until}\p( 11; F \o+ \cs{until}\p+ 40; \*\cs{infix_-}\p+ \! 9 )\!
%   \cs{until}\p( 11; F \o+ \*\cs{until}\p+ 40; T \cs{infix_-} \! 9 )\!
%   \*\cs{until}\p( 11; F \o+ 40; \cs{infix_-} \! 9 )\!
%   \cs{until}\p( \*\o{+} 11; 40; \cs{infix_-}\p( \! 9 )\!
%   \cs{until}\p( 51; \*\cs{infix_-}\p( \! 9 )\!
%   \cs{until}\p( 51; F \o- \cs{until}\p- \*\cs{one}\p- \! 9 )\!
%   \cs{until}\p( 51; F \o- \cs{until}\p- 9 \*\cs{one}\p- \! )\!
%   \cs{until}\p( 51; F \o- \cs{until}\p- 9; \*\cs{infix_)}\p- \!\!
%   \cs{until}\p( 51; F \o- \*\cs{until}\p- 9; T \cs{infix_)} \!\!
%   \*\cs{until}\p( 51; F \o- 9; \cs{infix_)} \!\!
%   \cs{until}\p( \*\o{-} 51; 9; \cs{infix_)}\p( \!\!
%   \cs{until}\p( 42; \*\cs{infix_)}\p( \!\!
%   \*\cs{until}\p( 42; T \cs{infix_)} \!\!
%   42; \cs{infix_)} \!\!
% \end{l3fp-code-example}
%
% The only missing step is to clean the output by removing \cs{infix_)},
% and possibly checking that nothing else remains.
%
% \subsection{Prefix operators, parentheses, and functions}
%
% Prefix operators (typically the unary |-|) and parentheses are
% taken care of by the same mechanism, and functions (\texttt{sin},
% \texttt{exp}, etc.) as well. Finding the argument of the unary
% |-|, for instance, is very similar to grabbing the second operand
% of a binary infix operator, with a small subtelty on precedence
% explained below. Once that argument is found, its sign can be
% flipped. A left parenthesis is just a prefix operator which
% removes the closing parenthesis (with some extra checks).
%
% Detecting prefix operators is done by \cs{one}. Before looking
% for a number, it tests the first character. If it is a digit, a
% dot, or a register, then we have a number. Otherwise, it is put
% in a function, \cs{prefix_?} (where |?| is roughly that first
% character), which is expanded. For instance, with a left
% parenthesis we would have the following.
% \begin{l3fp-code-example}
%   \*\cs{one}\p* \! ( 2 + 3 ) \!
%   \*\cs{prefix_(}\p* \! 2 + 3 ) \!
%   \o(\p* \cs{until}\p( \*\cs{one}\p( \! 2 + 3 ) \!
%   ... \!\!
%   \o(\p* 5; \cs{infix_)} \! \!
% \end{l3fp-code-example}
% As usual, the \cs{until}--\cs{one} pair reads and compute
% until reaching an operator of precedence at most \fpPrecedence{(}.
% Then \fpOperation{(} removes \cs{infix_)} and looks ahead for
% the next operation, comparing its precedence with the precedence
% \fpPrecedence{*} of the previous operation (in fact, this comparison
% is done by the relevant \cs{infix_?} built from the next operation).
%
% To support multi-character function (and constant) names, we
% may need to put more than one character in the \cs{prefix_?}
% construction. See implementation for details.
%
% Note that contrarily to \cs{infix_?} functions, the \cs{prefix_?}
% functions perform no test on their argument (which is once more
% the previous precedence), since we know that we need a number,
% and must never stop there.
%
% Functions are implemented as prefix operators with infinitely high
% precedence, so that their argument is the first number that can
% possibly be built. For instance, something like the following could
% happen in a computation
% \begin{l3fp-code-example}
%   \*\cs{one}\p* \! sqrt 4 + 3 ) \!
%   \*\cs{prefix_sqrt}\p* \! 4 + 3 ) \!
%   \o{sqrt}\p* \cs{until}\p{$\infty$} \*\cs{one}\p{$\infty$} \! 4 + 3 ) \!
%   ... \!\!
%   \o{sqrt}\p* 4; \cs{infix_+} \! 3 ) \!
%   2; \*\cs{infix_+}\p* \! 3 ) \!
% \end{l3fp-code-example}
%
% Lonely example, to be put somewhere: |2+sin 1 * 3| is $2+(\sin(1)\times 3)$.
%
% A further complication arises in the case of the unary |-| sign:
% |-3**2| should be $-(3^2)=-9$, and not $(-3)^2=9$. Easy, just give
% |-| a lower precedence, equal to that of the infix |+| and |-|.
% Unfortunately, this fails in subtle cases such as |3**-2*4|,
% yielding $3^{-2\times 4}$ instead of the correct $3^{-2}\times 4$.
% In fact, a unary |-| should only perform operations whose precedence
% is greater than that of the last operation, as well as
% |-|.\footnote{Taking into account the precedence of \texttt{-} itself
%   only matters when it follows a left parenthesis:
%   \texttt{(-2*4+3)} should give \texttt{((-8)+3)}, not \texttt{(-(8+3))}.}
% Thus, \cs{prefix_-} \meta{prec} expands to something like
% \begin{l3fp-code-example}
%   \o- \meta{prec} \cs{until}\p? \*\cs{one} \p?
% \end{l3fp-code-example}
% where \fpPrecedence{?} is the maximum of \meta{prec} and the
% precedence of |-|. Once the argument of |-| is found, \fpOperation{-}
% gets its opposite, and leaves it for the previous operation to use.
%
% An example with parentheses.
%
% \begin{l3fp-code-example}
%   \cs{until}\p(    \*\cs{one}\p( \! 11 * ( 2 + 3 ) - 9 )\!
%   \cs{until}\p( 1  \*\cs{one}\p( \! 1  * ( 2 + 3 ) - 9 )\!
%   \cs{until}\p( 11 \*\cs{one}\p( \!    * ( 2 + 3 ) - 9 )\!
%   \cs{until}\p( 11; \*\cs{infix_*}\p( \!   ( 2 + 3 ) - 9 )\!
%   \cs{until}\p( 11; F \o* \cs{until}\p* \*\cs{one}\p* \! ( 2 + 3 ) - 9 )\!
%   \cs{until}\p( 11; F \o* \cs{until}\p* \*\cs{prefix_(}\p* \! 2 + 3 ) - 9 )\!
%   \cs{until}\p( 11; F \o* \cs{until}\p* \o(\p* \cs{until}\p( \*\cs{one}\p( \! 2 + 3 ) - 9 )\!
%   \cs{until}\p( 11; F \o* \cs{until}\p* \o(\p* \cs{until}\p( 2 \*\cs{one}\p( \! + 3 ) - 9 )\!
%   \cs{until}\p( 11; F \o* \cs{until}\p* \o(\p* \cs{until}\p( 2; \*\cs{infix_+}\p( \! 3 ) - 9 )\!
%   \cs{until}\p( 11; F \o* \cs{until}\p* \o(\p* \cs{until}\p( 2; F \o+ \cs{until}\p+ \*\cs{one}\p+ \! 3)-9)\!
%   \cs{until}\p( 11; F \o* \cs{until}\p* \o(\p* \cs{until}\p( 2; F \o+ \cs{until}\p+ 3 \*\cs{one}\p+ \! )-9)\!
%   \cs{until}\p( 11; F \o* \cs{until}\p* \o(\p* \cs{until}\p( 2; F \o+ \cs{until}\p+ 3; \*\cs{infix_)}\p+ \! -9)\!
%   \cs{until}\p( 11; F \o* \cs{until}\p* \o(\p* \cs{until}\p( 2; F \o+ \*\cs{until}\p+ 3; T \cs{infix_)} \! -9)\!
%   \cs{until}\p( 11; F \o* \cs{until}\p* \o(\p* \*\cs{until}\p( 2; F \o+ 3; \cs{infix_)} \! - 9 )\!
%   \cs{until}\p( 11; F \o* \cs{until}\p* \o(\p* \cs{until}\p( \*\o+ 2; 3; \cs{infix_)}\p( \! - 9 )\!
%   \cs{until}\p( 11; F \o* \cs{until}\p* \o(\p* \cs{until}\p( 5; \*\cs{infix_)}\p( \! - 9 )\!
%   \cs{until}\p( 11; F \o* \cs{until}\p* \o(\p* \*\cs{until}\p( 5; T \cs{infix_)} \! - 9 )\!
%   \cs{until}\p( 11; F \o* \cs{until}\p* \*\o(\p* 5; \cs{infix_)} \! - 9 )\!
%   \cs{until}\p( 11; F \o* \cs{until}\p* 5; \*\cs{infix_-}\p* \! 9 )\!
%   \cs{until}\p( 11; F \o* \*\cs{until}\p* 5; T \cs{infix_-} \! 9 )\!
%   \*\cs{until}\p( 11; F \o* 5; \cs{infix_-} \! 9 )\!
%   \cs{until}\p( \*\o* 11; 5; \cs{infix_-}\p( \! 9 )\!
%   \cs{until}\p( 55; \* \cs{infix_-}\p( \! 9 )\!
%   \cs{until}\p( 55; F \o- \cs{until}\p- \*\cs{one}\p- \! 9 )\!
%   \cs{until}\p( 55; F \o- \cs{until}\p- 9 \*\cs{one}\p- \! )\!
%   \cs{until}\p( 55; F \o- \cs{until}\p- 9; \*\cs{infix_)}\p- \!\!
%   \cs{until}\p( 55; F \o- \*\cs{until}\p- 9; T \cs{infix_)} \!\!
%   \*\cs{until}\p( 55; F \o- 9; \cs{infix_)} \!\!
%   \cs{until}\p( \*\o- 55; 9; \cs{infix_)}\p( \!\!
%   \cs{until}\p( 47; \*\cs{infix_)}\p( \!\!
%   \*\cs{until}\p( 47; T \cs{infix_)} \!\!
%   47; \cs{infix_)} \!\!
% \end{l3fp-code-example}
%
% The end of this (sub)section was not revised yet
%
% \begin{itemize}
% \item If it is a sign (|-| or |+|), then any following sign will be
%   combined with this initial sign, forming \cs{prefix_+} or \cs{prefix_-}.
% \item If it is a letter, then any following letter is grabbed, forming
%   for instance \cs{prefix_sin} or \cs{prefix_sinh}.
% \item Otherwise, only one token\footnote{Some support for multi-character
%     prefix operator may be added in the future, but right now, I don't
%     see a use for it. Perhaps, for including comments inside
%     the computation itself??} is grabbed, for instance \cs{prefix_(}.
% \end{itemize}
%
% Functions may take several arguments, possibly an unknown
% number\footnote{Keyword argument support may be added later.},
% for instance \texttt{round(1.23456,2)}.
% \begin{itemize}
% \item \texttt{round} is made into \cs{prefix_round}, which tries to
%   grab one number using \cs{one}.
% \item This builds \cs{prefix_(}, which uses \cs{one} to grab one
%   number, calculating as necessary. The comma is given the same
%   precedence as parentheses, and thus ends the calculation of the
%   argument of \texttt{round}.
% \item \texttt{round} now has its first argument. It can check whether
%   the argument was closed by |,| or |)|, and branch accordingly.
% \item If it was a comma, then the first argument is skipped over,
%   through an expensive set of \cs{exp_after:wN}, and the second
%   argument can be grabbed. Here it is simply an integer, easier
%   to parse by building upon \cs{etex_numexpr:D}.
% \item The closing parenthesis (or another comma) is seen, and the
%   control is given back to \cs{prefix_round}.
% \end{itemize}
%
% \subsection{Type detection}
%
% The type of data should be detected by reading the first few tokens,
% before calling the type-specific function of \file{l3fp-clean}. Or
% should the type be obtained after the semicolon which indicates the
% end of the thing? And placed there?
%
% Also to grab exponent correctly, build \cs{fp_<abc>:w} when seeing
% some non-numeric |abc| while still looking to complete a number (or
% other data). Then, if \cs{fp_postfix_<type>_<abc>:w} exists, use it.^^A...?
%
%
%\end{documentation}
%
%\begin{implementation}
%
%
%\section{Implementation}
%
%
%    We start by ensuring that the required packages are loaded.
%    \begin{macrocode}
%<*package>
\ProvidesExplPackage
  {\filename}{\filedate}{\fileversion}{\filedescription}
\package_check_loaded_expl:
%</package>
%<*initex|package>
%    \end{macrocode}
%
%
% \subsection{Main functions}
%
% \begin{macro}{\fp_parse:n}
%   Thanks to \cs{tex_romannumeral:D}, this expands to an
%   internal floating point number after two expansion steps,
%   just like \cs{int_eval:n} and others.\footnote{Bruno:
%     should the function be renamed \cs{fp_eval:n}?}
%    \begin{macrocode}
\cs_new:Npn \fp_parse:n #1
  { \tex_romannumeral:D \fp_parse:w #1 \fp_parse_end: }
\cs_new:Npn \fp_parse:w
  {
    \exp_after:wN \fp_parse_after:wN
    \tex_romannumeral:D -`0
    \fp_parse_uo:Nw \c_fp_parse_min_prec_int
    \fp_aux_expand:w
  }
\cs_new:Npn \fp_parse_after:wN #1; #2
  {
    %\cs_if_eq:cNF {fp_parse_infix_):N} #2 {\ERROR}
    \c_zero
    #1;
  }
\cs_new:Npn \fp_parse_end: {)}
\int_const:Nn \c_fp_parse_min_prec_int { \c_zero }
%    \end{macrocode}
% \end{macro}
%
% \begin{macro}{\fp_parse_uo:Nw}
%   This is just a shorthand which sets up both \cs{fp_parse_until}
%   and \cs{fp_parse_one} with the same precedence. Note the
%   trailing \cs{tex_romannumeral:D}. This function should be
%   used with much care.
%    \begin{macrocode}
\cs_new:Npn \fp_parse_uo:Nw #1
  {
    \exp_after:wN \fp_parse_until:NwN
    \exp_after:wN #1
    \tex_romannumeral:D -`0
    \exp_after:wN \fp_parse_one:NN
    \exp_after:wN #1
    \tex_romannumeral:D
  }
%    \end{macrocode}
% \end{macro}
%
% \begin{macro}{\fp_parse_until:NwN}
%   \begin{syntax}
%     \cs{fp_parse_until:NwN} \meta{prec} \meta{number} \meta{bool}
%   \end{syntax}
%   If \meta{bool} is true, then \meta{number} is the floating
%   point number that we are looking for, and this expands to
%   \meta{number}. If \meta{bool} is false, then the input
%   stream actually looks like
%   \begin{quote}
%     \cs{fp_parse_until:NwN} \meta{prec} \meta{number 1} \meta{false}
%     \meta{oper} \meta{number 2} \cs{infix_?}
%   \end{quote}
%   and we must feed \meta{prec} to \cs{infix_?}, and perform
%   \meta{oper} on \meta{number} and \meta{number 2}: this
%   triggers the expansion of \cs{infix_?} \meta{prec}, continuing
%   the computation (or stopping). In that case, the function \cs{until}
%   yields
%   \begin{quote}
%     \cs{fp_parse_until:NwN} \meta{prec}
%     \meta{oper} \meta{number 1} \meta{number 2}
%     \cs{tex_romannumeral:D} |-`0| \cs{infix_?} \meta{prec}
%   \end{quote}
%   expanding \meta{oper} next.
%    \begin{macrocode}
\cs_new:Npn \fp_parse_until:NwN #1 #2; #3
  {
    \if_bool:N #3
      \exp_after:wN \use_none:nn
    \fi:
    \fp_parse_until_aux:NwNwN #1 #2;
  }
\cs_new:Npn \fp_parse_until_aux:NwNwN #1 #2; #3 #4; #5
  {
    \exp_after:wN \fp_parse_until:NwN
    \exp_after:wN #1
    \tex_romannumeral:D -`0
    #3 #2; #4;
    \tex_romannumeral:D -`0 #5 #1
  }
%    \end{macrocode}
% \end{macro}
%
% \subsection{Parsing one number}
%
% \begin{macro}{\fp_parse_one:NN}
%   Function called \cs{one} at other places. It grabs one
%   fp number, and packs what follows in an \cs{infix_} function.
%
%   |#1| is the previous \meta{prec}. Since we fully expanded,
%   and only registers and characters are alllowed in expressions,
%   |#2| can be
%   \begin{itemize}
%   \item \cs{tex_relax:D} in some form. That can be an internal floating
%     point, a premature end, or an unitialized register. In an
%     \cs{int_eval:n}, we would get \enquote{missing number, treated as $0$}.
%   \item A register. To be unpacked. But note that \cs{c_minus_one} is
%     not equivalent to $-\cs{c_one}$.\footnote{Bruno: elaborate.}
%   \item A digit, or a dot. That's the start of a number.
%   \item A letter, which starts a \enquote{prefix} word, either
%     constant or function (possibly unknown).
%   \item A sign: in that case we grab all following signs, and turn
%     them into a single \cs{prefix_+} or \cs{prefix_-} for efficiency.
%   \item A left parenthesis.
%   \item Any other character is an error.\footnote{Bruno: rephrase that,
%       since really, it could be extended.}
%   \end{itemize}
%   The two first cases are separated out by an \cs{if_catcode:w} check,
%   digits are also singled out, and then we use the \cs{grab_letters}
%   facility.
%    \begin{macrocode}
\cs_new:Npn \fp_parse_one:NN #1 #2
  {
    \if_catcode:w \tex_relax:D #2
      \if_meaning:w \tex_relax:D #2
        \exp_after:wN \exp_after:wN
        \exp_after:wN \fp_parse_one_relax:NN
      \else:
        \exp_after:wN \exp_after:wN
        \exp_after:wN \fp_parse_one_register:NN
      \fi:
    \else:
      \if_num:w 9<1#2 \exp_stop_f:
        \exp_after:wN \exp_after:wN
        \exp_after:wN \fp_parse_one_digit:NN
      \else:
        \exp_after:wN \exp_after:wN
        \exp_after:wN \fp_parse_one_other:NN
      \fi:
    \fi:
    #1 #2
  }
\cs_new:Npn \fp_parse_one_relax:NN #1 #2
  {
    \exp_after:wN \fp_parse_one_after:NwN
    \exp_after:wN #1
    \tex_romannumeral:D -`0
      \exp_after:wN \fp_cfs_sign_after:Nwn
      \int_use:N \int_eval:w 1 -
        \fp_cfs_lead_relax:N #2
  }
\cs_new:Npn \fp_parse_one_register:NN #1 #2
  {
    \exp_after:wN \fp_parse_one_after:NwN
    \exp_after:wN #1
    \tex_romannumeral:D -`0
    \exp_after:wN \fp_cfs_sign:w \tex_the:D #2
  }
\cs_new:Npn \fp_parse_one_digit:NN #1 #2
  {
    \exp_after:wN \fp_parse_one_after:NwN
    \exp_after:wN #1
    \tex_romannumeral:D -`0
      \exp_after:wN \fp_cfs_sign_after:Nwn
      \exp_after:wN 0
      \exp_after:wN E
        \int_use:N \int_eval:w \c_zero
        \fp_cfs_trim_loop:N #2
  }
\cs_new:Npn \fp_parse_one_after:NwN #1 #2; #3
  {
    \fp_aux_exp_after_fp:wN #2;
    \tex_romannumeral:D -`0 %^^A must test for #3=\relax.
    \use:c {fp_parse_infix_#3:N} #1
  }
%    \end{macrocode}
% \end{macro}
%
%
% \subsection{Prefix operators}
%
% \begin{macro}{\fp_parse_one_other:NN}
%   The interesting bit is \cs{fp_parse_one_other:NN}. It separates
%   letters from non-letters and builds the appropriate \cs{prefix}
%   function. If it is not defined (is \cs{tex_relax:D}), make it
%   a signalling \texttt{nan}. We don't look for an argument, as the
%   unknown \enquote{prefix} can also be a (mistyped) constant such
%   as \texttt{Inf}.
%    \begin{macrocode}
\cs_new:Npn \fp_parse_one_other:NN #1 #2
  {
    \exp_after:wN \fp_parse_one_other_aux:NN
    \exp_after:wN #1
    \cs:w fp_parse_prefix_
      \if_num:w \int_eval:w \tex_uccode:D `#2 / \c_twenty_six = \c_three
        #2
        \exp_after:wN \fp_aux_grab_letters:N
        \tex_romannumeral:D
      \else:
        #2 ;
        \exp_after:wN \cs_end:
        \tex_romannumeral:D
      \fi:
      \fp_aux_expand:w
  }
\cs_new:Npn \fp_parse_one_other_aux:NN #1 #2
  {
    \if_meaning:w \tex_relax:D #2
      \fp_parse_one_other_error:w
    \fi:
    #2 #1
  }
\cs_new:Npn \fp_parse_one_other_error:w \fi: #1 #2 #3
  {
    \fi:
    \exp_after:wN \fp_aux_snan_fp:N
    \exp_after:wN #1
    \tex_romannumeral:D -`0
    \cs:w fp_parse_infix_#3:N \cs_end: #2
  } %^^A must test for #3=\relax
%    \end{macrocode}
% \end{macro}
%
% \begin{macro}{\fp_parse_prefix_inf;}
% \begin{macro}{\fp_parse_prefix_nan;}
%   This ought to give the infinite floating point.
%    \begin{macrocode}
\cs_new:cpn {fp_parse_prefix_inf;} #1 #2
  {
    \exp_after:wN \c_plus_inf_fp
    \tex_romannumeral:D -`0
    \cs:w fp_parse_infix_#2:N \cs_end: #1
  } %^^A must test for #2=\relax.
\cs_new_eq:cc {fp_parse_prefix_infty;} {fp_parse_prefix_inf;}
\cs_new_eq:cc {fp_parse_prefix_infinity;} {fp_parse_prefix_inf;}
\cs_new:cpn {fp_parse_prefix_nan;} #1 #2
  {
    \exp_after:wN \c_empty_qnan_fp
    \tex_romannumeral:D -`0
    \cs:w fp_parse_infix_#2:N \cs_end: #1
  } %^^A must test for #2=\relax.
%    \end{macrocode}
% \end{macro}
% \end{macro}
%
% \begin{macro}[EXP]{\fp_parse_prefix_+;,\fp_parse_prefix_-;}
% \begin{macro}[EXP,aux]{\fp_parse_one_after_unary:NNwN}
%   Above, |#2| is \emph{e.g.}, \cs{fp_neg:w}, and expands once
%   after the calculation.\footnote{Bruno: explain.}
%    \begin{macrocode}
\cs_new:Npn \fp_parse_one_after_unary:NNwN #1 #2 #3; #4
  { #2 #3; \tex_romannumeral:D -`0 #4 #1 }
%    \end{macrocode}
%   A unary |+| is trivial. Unary |-| is harder. As an optimization,
%   if the next character is also a |-|, we do like a unary |+|.
%    \begin{macrocode}
\cs_new_eq:cN {fp_parse_prefix_+;} \fp_parse_one:NN
\cs_new:cpn {fp_parse_prefix_-;} #1 #2
  {
    \if:w - #2
      \fp_parse_prefix_double_minus:Nw #1
    \else:
      \exp_after:wN \fp_parse_one_after_unary:NNwN
      \exp_after:wN #1
      \exp_after:wN \fp_neg:w
      \tex_romannumeral:D -`0
      \if_num:w \c_one < #1
        \fp_parse_uo:Nw #1
      \else:
        \fp_parse_uo:Nw \c_one
      \fi:
    \fi:
    \fp_aux_expand:w
    #2
  }
\cs_new:Npn \fp_parse_prefix_double_minus:Nw
    #1 #2 \fi: \fp_aux_expand:w #3
  {
    \fi:
    \exp_after:wN \fp_parse_one:NN
    \exp_after:wN #1
    \tex_romannumeral:D \fp_aux_expand:w
  }
%    \end{macrocode}
% \end{macro}
% \end{macro}
%
% \begin{macro}{\fp_parse_prefix_(;}
%   This really ought to have a correct argument spec,
%   but it is technically complicated.
%    \begin{macrocode}
\cs_new:cpn {fp_parse_prefix_(;} #1
  {
    \exp_after:wN \fp_parse_lparen_after:NwN
    \exp_after:wN #1
    \tex_romannumeral:D -`0
    \fp_parse_uo:Nw \c_fp_parse_min_prec_int
    \fp_aux_expand:w
  }
\cs_new:Npn \fp_parse_lparen_after:NwN #1 #2; #3
  { \fp_parse_one_after:NwN #1 #2; }
%    \end{macrocode}
% \end{macro}
%
% \begin{macro}{\fp_parse_prefix_.;}
%   This function is called when a number starts with a dot.
%    \begin{macrocode}
\cs_new:cpn {fp_parse_prefix_.;} #1
  {
    \exp_after:wN \fp_parse_one_after:NwN
    \exp_after:wN #1
    \tex_romannumeral:D -`0
      \exp_after:wN \fp_cfs_sign_after:Nwn
      \exp_after:wN 0
      \exp_after:wN E
        \int_use:N \int_eval:w \c_zero
        \fp_cfs_strim_loop:N
  }
%    \end{macrocode}
% \end{macro}
%
% \begin{macro}{\fp_parse_prefix_ln;}
%    \begin{macrocode}
\cs_new:cpn {fp_parse_prefix_ln;} #1
  {
    \exp_after:wN \fp_parse_one_after_unary:NNwN
    \exp_after:wN #1
    \exp_after:wN \fp_ln:w
    \tex_romannumeral:D -`0
    \fp_parse_uo:Nw \c_ten %^^A change precedence to max.
    \fp_aux_expand:w
  }
%    \end{macrocode}
% \end{macro}
%
% \subsection{Infix operators}
%
% \begin{macro}[aux]{\fp_parse_def_infix:nNN,\fp_parse_def_infix_aux:NNN}
%   As described in the \enquote{work plan}, each infix operator
%   has an associated \cs{infix} function, a computing function,
%   and precedence, given as arguments to \cs{fp_parse_def_infix_aux:NNN}.
%   The latter two are only needed when defining the \cs{infix} function.
%    \begin{macrocode}
\cs_new:Npn \fp_parse_def_infix:nNN #1
  {
    \exp_args:Nc \fp_parse_def_infix_aux:NNN
      {fp_parse_infix_#1:N}
  }
\cs_new:Npn \fp_parse_def_infix_aux:NNN #1#2#3
  {
    \cs_new:Npn #1 ##1
      {
        \if_num:w ##1 < #3
          \exp_after:wN \c_false_bool
          \exp_after:wN #2
          \tex_romannumeral:D -`0
          \fp_parse_uo:Nw #3
        \else:
          \exp_after:wN \c_true_bool
          \exp_after:wN #1
          \tex_romannumeral:D %^^A is expansion ok there?
        \fi:
        \fp_aux_expand:w
      }
  }
%    \end{macrocode}
% \end{macro}
%
% \begin{macro}{\fp_parse_infix_+:N,\fp_parse_infix_-:N,
%     \fp_parse_infix_*:N,\fp_parse_infix_/:N}
% Using the general mechanism for arithmetic operations.
%    \begin{macrocode}
\fp_parse_def_infix:nNN {+} \fp_add:ww \c_one
\fp_parse_def_infix:nNN {-} \fp_sub:ww \c_one
\fp_parse_def_infix:nNN {*} \fp_mul:ww \c_two
\fp_parse_def_infix:nNN {/} \fp_div:ww \c_two
%    \end{macrocode}
%    \begin{macrocode}
\cs_new:cpx {fp_parse_infix_):N} #1
  { \c_true_bool \exp_not:c {fp_parse_infix_):N} }
%    \end{macrocode}
% \end{macro}
%
%
%    \begin{macrocode}
%</initex|package>
%    \end{macrocode}
%
%\end{implementation}
%
%\PrintChanges
%
%\PrintIndex
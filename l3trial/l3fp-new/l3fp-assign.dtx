% \iffalse
%% 
%% File: l3fp-assign.dtx Copyright (C) 2011 LaTeX3 project
%%
%% It may be distributed and/or modified under the conditions of the
%% LaTeX Project Public License (LPPL), either version 1.3c of this
%% license or (at your option) any later version.  The latest version
%% of this license is in the file
%%
%%    http://www.latex-project.org/lppl.txt
%%
%% This file is part of the ``l3trial bundle'' (The Work in LPPL)
%% and all files in that bundle must be distributed together.
%%
%% The released version of this bundle is available from CTAN.
%%
%% -----------------------------------------------------------------------
%%
%% The development version of the bundle can be found at
%%
%%    http://www.latex-project.org/svnroot/experimental/trunk/
%%
%% for those people who are interested.
%%
%%%%%%%%%%%
%% NOTE: %%
%%%%%%%%%%%
%%
%%   Snapshots taken from the repository represent work in progress and may
%%   not work or may contain conflicting material!  We therefore ask
%%   people _not_ to put them into distributions, archives, etc. without
%%   prior consultation with the LaTeX Project Team.
%%
%% -----------------------------------------------------------------------
%<*driver|package>
\RequirePackage{expl3,l3fp-aux,l3fp-convert,l3fp-basics}
\GetIdInfo$Id: l3fp-assign.dtx 0000 0000-00-00 00:00:00Z bruno $
  {L3 Experimental floating-point assignments}
%</driver|package>
%<*driver>
\documentclass[full]{l3doc}
\usepackage{amsmath}
\usepackage{l3fp-basics}
\begin{document}
  \tableofcontents
  \DocInput{\jobname.dtx}
\end{document}
%</driver>
% \fi
%
% \title{The \textsf{l3fp-assign} package\thanks{This file
%         has version number \fileversion, last
%         revised \filedate.}\\
% Floating point expressions}
% \author{^^A
%  The \LaTeX3 Project\thanks
%    {^^A
%      E-mail:
%        \href{mailto:latex-team@latex-project.org}
%          {latex-team@latex-project.org}^^A
%    }^^A
% }
% \date{Released \filedate}
% \maketitle
%
% \begin{documentation}
%   
% \end{documentation}
% 
% \begin{implementation}
%
% \section{Implementation}
%
% 
% We start by ensuring that the required packages are loaded.
%    \begin{macrocode}
%<*package>
\ProvidesExplPackage
  {\filename}{\filedate}{\fileversion}{\filedescription}
\package_check_loaded_expl:
%</package>
%<*initex|package>
%    \end{macrocode}
% 
% \subsection{Assigning values}
%
% \begin{macro}[aux]{\fp_assign_raw:Nn,\fp_assign_graw:Nn}
%   These are used internally. Contrarily to
%   \cs{fp_assign_set:Nn}, there is no parsing.
%    \begin{macrocode}
\cs_new_eq:NN \fp_assign_raw:Nx  \cs_set_protected_nopar:Npx
\cs_new_eq:NN \fp_assign_graw:Nx \cs_gset_protected_nopar:Npx
%    \end{macrocode}
% \end{macro}
% 
% \begin{macro}{\fp_assign_new:N}
%    \begin{macrocode}
\cs_new_protected_nopar:Npn \fp_assign_new:N #1
  { \cs_new_eq:NN #1 \c_empty_qnan_fp }
\cs_new_protected_nopar:Npn \fp_assign_const:Nn #1 #2
  {
    \chk_if_free_cs:N #1
    \fp_assign_graw:Nx #1 { \fp_convert_from_str:n {#2} }
  }
\cs_new_protected_nopar:Npn \fp_assign_set:Nn #1 #2
  {
    \fp_assign_raw:Nx #1 { \fp_convert_from_str:n {#2} }
  }
\cs_new_protected_nopar:Npn \fp_assign_gset:Nn #1 #2
  {
    \fp_assign_graw:Nx #1 { \fp_convert_from_str:n {#2} }
  }
%    \end{macrocode}
% \end{macro}
%
% \begin{macro}{\fp_assign_set_from_old:NN}
%    \begin{macrocode}
\cs_new_protected_nopar:Npn \fp_assign_set_from_old:NN #1 #2
  {
    \fp_assign_raw:Nx #1 { \fp_convert_from_old:N #2 }
  }
%    \end{macrocode}
% \end{macro}
%
% \begin{macro}{\fp_assign_gset:Nn}
%   \begin{syntax}
%     \cs{fp_assign_gset:Nn} \meta{new fp var} \Arg{token list}
%   \end{syntax}
%   Sets the \meta{new fp var} locally to the value that we manage to
%   get out of \meta{token list}.
%    \begin{macrocode}
%    \end{macrocode}
% \end{macro} 
%
% \subsection{Operations on new floating point numbers}
%
% \begin{macro}{\fp_assign_gadd:Nn}
%    \begin{macrocode}
\cs_new_protected:Npn \fp_assign_gadd:Nn #1 #2
  {
    \fp_assign_graw:Nx #1
      { \fp_add:nn { #1 } { \fp_convert_from_str:n {#2} } }
  }
\cs_new_protected:Npn \fp_assign_gsub:Nn #1 #2
  {
    \fp_assign_graw:Nx #1
      { \fp_sub:nn { #1 } { \fp_convert_from_str:n {#2} } }
  }
%    \end{macrocode}
% \end{macro}
% 
%
% \subsection{Operations on old floating point numbers}
%
% \begin{macro}{\fp_old_generic:NNn}
%   Just for tests\ldots{}
%    \begin{macrocode}
\cs_new_protected:Npn \fp_old_generic:NNNn #1 #2 #3 #4
  {
    #1 #3
      {
        \fp_convert_to_old:n
          {
            #2
              { \fp_convert_from_old:N #3 }
              { \fp_convert_from_str:n {#4} }
          }
      }
  }
%    \end{macrocode}
% \end{macro}
%
% \begin{macro}{\fp_old_add:Nn,\fp_old_sub:Nn}
% \begin{macro}{\fp_old_gadd:Nn,\fp_old_gsub:Nn}
%   Just for tests\ldots{}
%    \begin{macrocode}
\cs_new:Npn \fp_old_add:Nn  { \fp_old_generic:NNNn \tl_set:Nf  \fp_add:nn }
\cs_new:Npn \fp_old_sub:Nn  { \fp_old_generic:NNNn \tl_set:Nf  \fp_sub:nn }
\cs_new:Npn \fp_old_gadd:Nn { \fp_old_generic:NNNn \tl_gset:Nf \fp_add:nn }
\cs_new:Npn \fp_old_gsub:Nn { \fp_old_generic:NNNn \tl_gset:Nf \fp_sub:nn }
%    \end{macrocode}
% \end{macro}
% \end{macro}
%
%    \begin{macrocode}
%</initex|package>
%    \end{macrocode}
%
%\end{implementation}
%
%\PrintChanges
%
%\PrintIndex
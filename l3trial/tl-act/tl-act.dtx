% \iffalse meta-comment
%
%% File: tl-act.dtx Copyright (C) 2011 The LaTeX3 Project
%%
%% It may be distributed and/or modified under the conditions of the
%% LaTeX Project Public License (LPPL), either version 1.3c of this
%% license or (at your option) any later version.  The latest version
%% of this license is in the file
%%
%%    http://www.latex-project.org/lppl.txt
%%
%% This file is part of the "l3trial bundle" (The Work in LPPL)
%% and all files in that bundle must be distributed together.
%%
%% The released version of this bundle is available from CTAN.
%%
%% -----------------------------------------------------------------------
%%
%% The development version of the bundle can be found at
%%
%%    http://www.latex-project.org/svnroot/experimental/trunk/
%%
%% for those people who are interested.
%%
%%%%%%%%%%%
%% NOTE: %%
%%%%%%%%%%%
%%
%%   Snapshots taken from the repository represent work in progress and may
%%   not work or may contain conflicting material!  We therefore ask
%%   people _not_ to put them into distributions, archives, etc. without
%%   prior consultation with the LaTeX3 Project.
%%
%% -----------------------------------------------------------------------
%
%<*driver|package>
\RequirePackage{expl3}
\GetIdInfo$Id$
  {L3 Experimental Actions on Token Lists}
%</driver|package>
%<*driver>
\documentclass[full]{l3doc}
\usepackage{amsmath}
 %\usepackage{tl-act}
\begin{document}
  \DocInput{\jobname.dtx}
\end{document}
%</driver>
% \fi
%
%
% \title{^^A
%   The \textsf{tl-act} package: actions on token lists^^A
%   \thanks{This file describes v\ExplFileVersion,
%     last revised \ExplFileDate.}^^A
% }
%
% \author{^^A
%  The \LaTeX3 Project\thanks
%    {^^A
%      E-mail:
%        \href{mailto:latex-team@latex-project.org}
%          {latex-team@latex-project.org}^^A
%    }^^A
% }
%
% \date{Released \ExplFileDate}
%
% \maketitle
%
% \begin{documentation}
%
%   This module builds upon the experimental \cs{@@_act:NNNnn},
%   and provides some extra functionalities to manipulate
%   token lists expandably on a token by token basis.
%
%   \begin{function}[EXP]{\tl_act_expand:n}
%     \begin{syntax}
%       \cs{tl_act_expand:n} \Arg{token list}
%     \end{syntax}
%     After two steps of expansion, yields a fully expanded
%     \meta{token list}, as in the case of x-type arguments.
%     Caveats:
%     \begin{itemize}
%     \item catcode 1 and 2 characters must be balanced at each step
%       of the expansion. It requires some care when we do brace tricks
%       elsewhere in \LaTeX3. See for instance a small change I made in
%       \cs{group_align_safe_begin:}, but I think that every expandable
%       function we now have is safe in this respect.
%     \item catcode 1 and 2 characters are automatically converted
%       to |{| and |}| (in charcode) when the corresponding group
%       \enquote{reaches the start} of the token list.
%     \item Furthermore, the expansion of such a group must remain within
%       the \enquote{bounds} of that group. But this is basically the
%       same as the first condition, and all of the \LaTeX3 functions
%       fit the bill.
%     \end{itemize}
%     It doesn't seem like too extreme conditions to
%   \end{function}
%
% \end{documentation}
%
% \begin{implementation}
%
% \section{\pkg{tl-act} implementation}
%
%    \begin{macrocode}
%<*package>
%    \end{macrocode}
%    
%    \begin{macrocode}
%<@@=tl>
%    \end{macrocode}
%
%    \begin{macrocode}
\ProvidesExplPackage
  {\ExplFileName}{\ExplFileDate}{\ExplFileVersion}{\ExplFileDescription}
%    \end{macrocode}
%
% \begin{macro}{\tl_act_expand:n}
%   As usual, feed \cs{@@_act:NNNnn} three functions, an empty
%   parameter text, and the \meta{tokens} to expand.
%    \begin{macrocode}
\cs_new:Npn \tl_act_expand:n
  {
    \tex_romannumeral:D
    \@@_act:NNNnn
      \tl_act_expand_normal:nN
      \tl_act_expand_group:nn
      \tl_act_expand_space:n
      { }
  }
%    \end{macrocode}
%   Spaces are simply output. Groups need to be recursed in.
%    \begin{macrocode}
\cs_new:Npn \tl_act_expand_space:n #1
  { \tl_act_output:n {~} }
\cs_new:Npn \tl_act_expand_group:nn #1
  {
    \tl_act_group_recurse:Nnn
      \tl_act_output:n
      { \tl_act_expand:n }
  }
%    \end{macrocode}
%   The case of a normal token requires more testing. If it does not
%   expand in x-type arguments, then it should not expand here either,
%   and we can simply output it as it is. Otherwise, we need to detect
%   the primitives \cs{unexpanded}, \cs{noexpand} and \cs{the}, which
%   have a somewhat special role in expansion contexts.
%    \begin{macrocode}
\cs_new:Npn \tl_act_expand_normal:nN #1 #2
  {
    \token_if_expands_in_x:NTF #2
      {
        \str_case_x:nnn { \token_to_meaning:N #2 }
          {
            { \token_to_str:N \noexpand } { \tl_act_expand_noexpand:w }
            { \token_to_str:N \unexpanded } { \tl_act_expand_unexpanded:w }
            { \token_to_str:N \the } { \tl_act_expand_the:w }
          }
          { \use_i:nn { \exp_after:wN \tl_act_loop:w #2 } }
      }
      { \tl_act_output:n {#2} }
  }
%    \end{macrocode}
%   In the case of \cs{noexpand}, look ahead. If it is either a space
%   or a group, then we can simply forget \cs{noexpand} as it has no effect.
%   Otherwise, output the next token, and loop.
%    \begin{macrocode}
\cs_new:Npn \tl_act_expand_noexpand:w \tl_act_loop:w #1 \q_tl_act_stop
  {
    \tl_if_head_N_type:nTF {#1}
      { \tl_act_expand_noexpand_head:N }
      { \tl_act_loop:w }
    #1 \q_tl_act_stop
  }
\cs_new:Npn \tl_act_expand_noexpand_head:N #1
  {
    \tl_act_output:n {#1}
    \tl_act_loop:w
  }
%    \end{macrocode}
%   To implement \cs{unexpanded}, we need to be able to
%   detect and ignore what the \TeX{}book calls \meta{filler}:
%   any set of spaces and relax tokens. Then grab a
%   \meta{general text} and output it, then loop. Since one
%   of our assumptions is that the token list is balanced
%   at all times, we don't have to support ^^A|{|
%   \cs{unexpanded} \cs{c_group_begin_token}\ldots|}|.
%    \begin{macrocode}
\cs_new:Npn \tl_act_expand_unexpanded:w \tl_act_loop:w
  {
    \exp_after:wN \tl_act_expand_unexpanded_aux:n
    \tex_romannumeral:D \tl_act_expand_filler:w
  }
\cs_new:Npn \tl_act_expand_unexpanded_aux:n #1
  {
    \tl_act_output:n {#1}
    \tl_act_loop:w
  }
%    \end{macrocode}
%   \meta{Fillers} are not yet expanded properly.
%    \begin{macrocode}
\cs_new:Npn \tl_act_expand_filler:w { -`0 }
% \cs_new:Npn \tl_act_expand_filler:w
%   {
%     \exp_after:wN \tl_act_expand_filler_aux:w
%     \tex_romannumeral:D -`0
%   }
% \cs_new:Npn \tl_act_expand_filler_aux:w #1 \q_tl_act_stop
%   {
%     \tl_if_head_normal:nTF {#1}
%       { \tl_act_expand_filler_aux_ii:N }
%       {
%         \tl_if_head_begin_group:nTF {#1}
%           { \c_zero }
%           {
%             \exp_after:wN \tl_act_expand_filler:w
%             \tex_romannumeral:D -`0
%           }
%       }
%     #1 \q_tl_act_stop
%   }
% \cs_new:Npn \tl_act_expand_filler_aux_ii:N #1
%   {
%     \token_if_eq_meaning:NNTF #1 \tex_relax:D
%       { \exp_after:wN \tl_act_expand_filler:w \use_none:n }
%       {
%         \exp_after:wN \if_meaning:w \exp_not:N #1 #1
%         ...
%         \token_if_eq_catcode:NNTF #1 \c_space_token
%           { \exp_after:wN \tl_act_expand_filler:w \use_none:n }
%           { \c_zero }
%       }
%   }
%    \end{macrocode}
% For \cs{the}, things are harder: it can act on one token or on
% more than one token. However, in most cases, there is no need
% to worry about expansion, and we can simply expand \cs{the}.
% The only cases where preventing further expansion is important
% is for toks registers, and toks parameters. The most troublesome
% case is the \cs{toks} primitive itself, because it needs to
% look ahead for a register number, and that can take very many forms.
%    \begin{macrocode}
\cs_new:Npn \tl_act_expand_the:w #1 \q_tl_act_stop
  {
    \__msg_expandable_error:n {Not~Implemented~Yet!~Sorry.}
    \tl_act_end:w
  }
%    \end{macrocode}
% \end{macro}
%
% \begin{macro}[pTF]{\token_if_expands_in_x:N}
%   This is true for expandable non-protected commands.
%    \begin{macrocode}
\cs_set:Npn \cs_tmp:w #1
  {
    \prg_new_conditional:Npnn \token_if_expands_in_x:N ##1 { TF }
      {
        \exp_after:wN \if_meaning:w \exp_not:N ##1 ##1
          \prg_return_false:
        \else:
          \str_if_eq_x:nnTF
            { \token_to_str:N \X }
            {
              \exp_after:wN \token_if_expands_in_x_aux:w
              \token_to_meaning:N ##1 #1 \q_stop
              X
            }
            \prg_return_false:
            \prg_return_true:
        \fi:
      }
    \cs_new:Npn \token_if_expands_in_x_aux:w ##1 #1 ##2 \q_stop {##1}
  }
\exp_args:No \cs_tmp:w { \tl_to_str:n { protected } }
%    \end{macrocode}
% \end{macro}
%
%    \begin{macrocode}
%</package>
%    \end{macrocode}
%
% \end{implementation}
%
% \PrintIndex

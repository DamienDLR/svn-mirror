% Copyright 2008 The LaTeX Project
\documentclass{ltnews}
 
\publicationmonth{December}
\publicationyear{2008}
\publicationissue{18}
 
\begin{document}
 
\maketitle

\section{Welcome to \LaTeX3}

Momentum is slowly starting to build again behind the \LaTeX3 project. For the last few releases of \TeX~Live, the experimental programming foundation for \LaTeX3 has been available under the name \textsf{expl3}. At first, this was designed as a `marketing exercise', despite large warnings that the code would probably change in the future, we wanted to show that there was progress being made, no matter how slowly, on a successor to \LaTeX. Since then, some people have looked at the code, provided feedback, and~--- most importantly~--- actually tried using it. Although it is yet early days, we believe that the ideas behind the code are sound and there are only `cosmetic improvements' that need to be made before \textsf{expl3} is ready for the \LaTeX~package author masses.

\section{What currently exists}

The current state of the \LaTeX3 code consists of two main branches: the \textsf{expl3} modules that define the underlying programming environment, and the `\textsf{xpackage}s', which are a suite of packages that are written with the \textsf{expl3} programming interface and provide some higher-level functionality for what will one day become \LaTeX3 proper. The majority of the work that has been spent in recent years has been focussed towards developing the \textsf{expl3} programming interface. This programming interface is designed to be used \emph{on top} of \LaTeXe, so new packages can take advantage of the new features while still drawing on the vast number of \LaTeXe\ packages on CTAN.

\section{What's happening now}

In preparation for a refactor of the \textsf{expl3} code, a test suite is being written in order to provide the safety of regression tests for any and all functions that we end up changing. As these tests are being written, our assumptions about what should be called what and the underlying naming conventions for the functions and datatypes defined in \LaTeX3 are being questioned, challenged, and noted for further rumination.

At time of writing, we are approximately half-way through writing the test suite. Once this task is complete, which we hope to take place in the first half of 2009, we will be ready to make changes without worrying about breaking anything. 

\section{What's happening soon}

So what do we want to change? The current \textsf{expl3} codebase has portions that literally date to the pre-\LaTeXe\ days, while other modules have been more recently conceived. It is quite apparent when reading through the sources that some unification and tidying up of the different interfaces would improve the simplicity and consistency of the code~--- both the coding interface and the actual implementation. In many cases, such changes will mean nothing more than a tweak or a rename; one example is from \verb=\token_to_string:N= to \verb=\token_to_str:N=.

Beyond these minor changes, we are also re-thinking the exact notation behind the way functions are defined. There are currently a handful of different types of arguments that functions may be passed (from an untouched single token to a complete expansion of a token list) and we're not entirely happy with how the original choices have evolved now that the system has grown somewhat. We have received good feedback from several people on ways that we could improve the argument syntax, and as part of the refactoring of the \textsf{expl3} packages we hope to address the problems that we currently perceive in the present syntax.

\section{What's happening later}

It's a little early to start talking about what will happen when the refactoring process has finished. We hope to start freezing the interface at this point, and we hope that more package authors will be interested in using the new ideas to write their own code. The scope of features provided by the \textsf{expl3} modules will only continue to grow as it is used more and more in the real world.

Some new and/or experimental packages will be moving to use the \textsf{expl3} programming interface, including \textsf{breqn}, \textsf{siunitx}, \textsf{fontspec}, and \textsf{unicode-math}. (Which is one reason towards the lack of progress in these latter two in recent times.) These will provide improvements to everyday \LaTeX\ users who haven't even heard of the \LaTeX3 Project.

Looking towards the long term, \LaTeX3 as a document preparation system needs to be written almost from scratch, using \textsf{expl3} and the \textsf{xpackages}. A high-level user syntax needs to be designed (based, of course, on \LaTeXe~--- but hopefully improved) and scores of packages will be incorporated into the `out-of-the-box' default document templates. \LaTeXe\ has stood up to the test of time~--- some fifteen years and still going strong in its niches~--- and it is now time to write a successor that will survive another score.

With Lua\TeX\ around the corner, are we being naive by continuing to write code in the weird programming environment that is \TeX? We don't think so; something needs to be done to begin to modernise \LaTeX. \LaTeX3 will certainly run on Lua\TeX, and when the time is right will also take advantage of its new features. Consider what we're doing now the groundwork for a migration to where-ever \TeX\ is going in the next few years.

\section{What can you do?}

The \LaTeX~Project website contains the links for obtaining the most current versions of the code: \url{http://www.latex-project.org/code.html}. More feedback is always necessary, and your comments and suggestions are most welome. If you are a package author, why not browse the repository and the reference documentation \texttt{source3.pdf} to see how we've approached this new way of writing \LaTeX\ packages? Who knows~--- it might end up making your life easier!

\end{document}
 

















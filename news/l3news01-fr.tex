\documentclass[a4paper, 10pt]{article}
\usepackage[utf8]{inputenc}
\usepackage[T1]{fontenc}
\usepackage{textcomp}
\usepackage{lmodern}
\usepackage{fixltx2e}

\usepackage{multicol, geometry, ragged2e, shortvrb, fancyhdr, titlesec, xspace}
\usepackage[babel=true, expansion=false]{microtype}
\usepackage[british, frenchb]{babel}
\usepackage[colorlinks=true]{hyperref}

\frenchbsetup{AutoSpacePunctuation=false, og=«, fg=»}
\newcommand\eng[1]{\foreignlanguage{english}{\emph{#1}}}

\geometry{hmargin=1.2cm, vmargin=2.4cm, includefoot}
\pagestyle{empty}

\setcounter{secnumdepth}{-1}
\titleformat\section{\sffamily\slshape\large}{}{0pt}{}
\titleformat\subsection[runin]{\bfseries}{}{0pt}{}[.]
\titlespacing\section{0pt}{\medskipamount}{\medskipamount}
\titlespacing\subsection{0pt}{\medskipamount}{1em}

\fancypagestyle{firstpage}{%
  \fancyhf{}
  \renewcommand\headrulewidth{0pt}
  \renewcommand\footrulewidth{0pt}
  \cfoot{\parbox\linewidth{\footnotesize
  Les nouvelles \latexx et le logiciel \latex vous sont fournis par l'équipe
  du projet LaTeX3 ; copyright 2009, tous droits réservés.\\
  Traduction française par Manuel \bsc{Pégourié-Gonnard} avec l'aimable
  autorisation de l'équipe du projet \latexx.\\
  La traduction se veut fidèle, mais seul l'original, disponible sur
  \url{http://www.latex-project.org/l3news/}, fait autorité.}}
  }

% #1 = Nième édition, mois AAAA
\newcommand\lnewsbegin[1]{%
  \RaggedRight
  \Huge
  Nouvelles de \latexx
  \\ \normalsize
  #1%
  \par \vspace{1pc}
  \thispagestyle{firstpage}
  \begin{multicols*}{2} \setlength\parindent{1pc}
  }

\newcommand\lnewsend{%
  \end{multicols*}
  }

\newcommand\newcolumn{\vfill\columnbreak}
\newcommand\tsvp{\par\medskip
  {\raggedleft \footnotesize (T. S. V. P.) \par}
  \newcolumn}

\newcommand\tex{\TeX\xspace}
\newcommand\latex{\LaTeX\xspace}
\newcommand\latexe{\LaTeXe\xspace}
\newcommand\latexx{\LaTeX3\xspace}
\newcommand\xetex{Xe\TeX\xspace}
\newcommand\luatex{Lua\TeX\xspace}
\newcommand\texlive{\TeX\,Live\xspace}
\newcommand\miktex{MiK\TeX\xspace}

\newcommand\fnurl[1]{\footnote{\url{#1}}}
\newcommand\pk{\textsf}
\MakeShortVerb\|

\endinput


\begin{document}

\lnewsbegin{Première édition, février 2009}

\section{Bienvenue dans \latexx}

La projet \latexx connaît un regain de dynamisme. Les fondations
expérimentales de la programmation pour \latexx sont disponibles, sous le nom
de \pk{expl3}, depuis quelques versions de \TeX\ Live. Malgré les gros
avertissements disant que le code changerait probablement à l'avenir, nous
voulions montrer que des progrès étaient en cours, même s'ils étaient lents.
Depuis, plusieurs personnes ont regardé le code, fourni des retours, et (c'est
le plus important) essayé de l'utiliser réellement. Bien qu'on en soit aux
commencements, nous pensons que les idées sous-tendant le code sont solides et
que seuls quelques « changements cosmétiques » sont nécessaires avant que
\pk{expl3} ne soit prêt pour l'ensemble des auteurs de modules \latex.

\section{Ce qui existe déjà}

Le code actuel de \latexx consiste en deux branches principales : les modules
\pk{expl3} qui définissent l'environnement de programmation de base, et les
\pk{xpackages}, qui sont une suites de modules écrits avec l'interface de
programmation \pk{expl3} et fournissent des fonctionnalités de plus haut
niveau pour ce qui deviendra un jour \latexx à proprement parler. \pk{expl3}
et \pk{xpackages} sont tous deux conçus pour être utilisés \emph{avec}
\latexe, de sorte que les modules nouvellement écrits puissent profiter des
nouvelles fonctionnalités, tout en étant utilisables en même temps que les
nombreux modules \latex2 du CTAN.

\section{Ce qui se passe actuellement}

Pour préparer une petite révision du code de \pk{expl3}, nous sommes en train
d'écrire une suite de test exhaustive pour chaque module. Ces tests nous
permettent de modifier l'implémentation et de vérifier que le code marche
encore comme avant. Ils mettent aussi en valeur des petits problèmes ou
omissions dans le code. Pendant que nous écrivons les tests, nos suppositions
sur ce qui doit s'appeler comment, les conventions de nommage sous-jacentes
pour les fonctions et les types de données sont remises en question, et
ces questions notées pour être ruminées davantage.

Au moment où ces lignes sont rédigées, nous avons écrit à peu près la moitié
de la suite de tests. Une fois ce travail terminé, ce qui est prévu pour le
premier semestre 2009, nous serons prêts à faire des changements sans nous
inquiéter de casser quoi que ce soit.

\newcolumn

\section{Ce qui se passera bientôt}

Bon, que voulons-nous changer ? Il y a dans le code \pk{expl3} des portions
datant de la période pré-\latexe, et d'autres modules conçus plus récemment.
Il est visible en lisant les sources d'un bout à l'autre qu'un peu
d'unification et de ménage augmenterait la simplicité et la cohérence du code.
Dans plusieurs cas, un tel changement ne serait qu'un simple ajustement ou
renommage.

Au-delà de ces changements mineurs, nous repensons aussi actuellement la
notation exacte qui gouverne les définitions de fonctions. Il y a actuellement
un certain nombre de types d'arguments qu'on peut passer aux fonctions (d'un
simple lexème non modifié au développement complet d'une liste de lexèmes) et
nous ne sommes pas totalement satisfaits de la façon dont ont évolué les
choix initiaux, maintenant que le système a un peu grandi. Nous avons eu des
bons retours de plusieurs personnes sur des façons potentielles d'améliorer la
syntaxe des arguments, et nous espérons attaquer les problèmes que nous voyons
maintenant dans la syntaxe actuelle au cours des prochains changements dans
les modules \pk{expl3}.

\section{Ce qui se passera plus tard}

Une fois finis les changements que nous venons d'évoquer, nous commencerons à
geler le cœur de l'interface des modules \pk{expl3}, et nous espérons que plus
d'auteurs de modules auront envie d'utiliser ces nouvelles idées pour écrire
leur propre code. Alors que les fonctions de base demeureront inchangées, plus
de fonctionnalités et de nouveaux modules seront ajoutés lorsque \latexx
commencera à grandir.

Quelques modules nouveaux et/ou expérimentaux vont être modifiés pour utiliser
l'interface de programmation \pk{expl3}, comme \pk{breqn}, \pk{mathtools},
\pk{empheq}, \pk{fontspec} et \pk{unicode-math}. (C'est une des raisons de
l'absence d'avancées dans les deux derniers récemment.) Il y aura également
une version du module \pk{siunitx} écrite en \pk{expl3}, parallèlement à la
version \latexe. Ces développements apporteront des améliorations à des
utilisateurs quotidiens de \latex qui n'ont même pas entendu parler du projet
\latexx.

Concernant le plus long terme, \latexx en tant que système de préparation de
documents doit être écrit pratiquement en partant de rien. Il faut concevoir
une syntaxe de plus haut niveau, et une multitude de modules sera utilisée
comme source d'inspiration pour les modèles de document par défaut. \latexe a
résisté à l'épreuve du temps (quelque 15 ans et toujours bien vivant) et il
est maintenant temps d'écrire un successeur qui survivra pour les prochaines
décennies.

\lnewsend

\end{document}

% Copyright 2011 The LaTeX Project
\documentclass{ltnews}
\renewcommand{\LaTeXNews}{\LaTeX3~News}

\usepackage{hologo}

\publicationmonth{January}
\publicationyear{2011}
\publicationissue{5}

\begin{document}
\maketitle

\section{Happy new year}

Seasons greetings for 2011!
As the previous news issue was released late, this season's issue will be shorter than usual.

\section{The LPPL is now OSI-approved}

We are happy to report that earlier this year the \LaTeX\ Project Public License (LPPL) has been approved by the OSI as an open source licence.\footnote{\url{http://www.opensource.org/licenses/lppl}} Frank Mittelbach will be publishing a retrospective of the LPPL in an upcoming TUGboat article.

\section{Reflections on 2010}

We are pleased to see the continued development and discussion in the \TeX\ world.
The \LaTeX\ ecosystem continues to see new developments and a selection of notable news from last year include:
\begin{itemize}
\item[June] The TUG~2010 conference was held; videos, slides, and papers from \LaTeX3 Project members are available from our website.\footnote{\url{http://www.latex-project.org/papers/}}
\item[Aug.] 
The \TeX\ Stack Exchange\footnote{\url{http://tex.stackexchange.com}} question'n'answer website has been created and grown quickly. At time of writing, some 2800 people have asked 2600 questions with 5600 answers total, and 2200 users are currently visiting daily.
\item[Sept.] \TeX\ Live 2010 was released: each year the shipping date is earlier; the production process is becoming more streamlined and we congratulate all involved for their hard work.
\item[Oct.] TLContrib\footnote{\url{http://tlcontrib.metatex.org/}} was opened by Taco Hoekwater as a way to update a \TeX~Live installation with material that is not distributable through \verb|tlmgr| such as binary packages, non-free code, or test versions.
\item[Nov.] Philip Lehman released the first stable version of \textsf{biblatex}. One of the most ambitious \LaTeX\ packages in recent memory, \textsf{biblatex} is a highly flexible package for managing citation cross-referencing and bibliography typesetting. In `beta' status for a few years now, reaching a stable release is a great milestone.
\item[Dec.] Lua\TeX\ 0.65. We are happy to see Lua\TeX\ development steadily continuing. \LaTeX\ users may use Lua\TeX\ with the \verb|lualatex| program. Like \verb|xelatex|, this allows \LaTeX\ documents to use multilingual OpenType fonts and Unicode text input.
\end{itemize}

\section{Current progress}

[Will: work on my thesis has prevented any significant work recently in the \LaTeX\ world. But I hope this to change later this year!]

The \textsf{expl3} programming modules continue to see revision and expansion; we have added a Lua\TeX\ module but \textsf{expl3} continues to support all three of pdf\LaTeX, \XeLaTeX, and Lua\LaTeX\ equally.

The \textsf{l3fp} module has been extended and improved.
The \textsf{l3coffin} module has been added based on the original \textsf{xcoffins} package introduced at TUG~2010 as reported in the last news issue; this code is now available from CTAN for testing and feedback.

We have consolidated the \textsf{l3int} and \textsf{l3intexpr} modules (which were separate for historical purposes); all integer/count-related functions are now contained within the `\textsf{int}' code and now have prefix \verb|\int_|. Backwards compatibility is preserved for now but eventually we will drop support for the older \verb|\intexpr_| function names.

\section{Goals for 2011}



\end{document}




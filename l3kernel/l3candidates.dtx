% \iffalse meta-comment
%
%% File: l3candidates.dtx Copyright(C) 2012 The LaTeX3 Project
%%
%% It may be distributed and/or modified under the conditions of the
%% LaTeX Project Public License (LPPL), either version 1.3c of this
%% license or (at your option) any later version.  The latest version
%% of this license is in the file
%%
%%    http://www.latex-project.org/lppl.txt
%%
%% This file is part of the "l3kernel bundle" (The Work in LPPL)
%% and all files in that bundle must be distributed together.
%%
%% The released version of this bundle is available from CTAN.
%%
%% -----------------------------------------------------------------------
%%
%% The development version of the bundle can be found at
%%
%%    http://www.latex-project.org/svnroot/experimental/trunk/
%%
%% for those people who are interested.
%%
%%%%%%%%%%%
%% NOTE: %%
%%%%%%%%%%%
%%
%%   Snapshots taken from the repository represent work in progress and may
%%   not work or may contain conflicting material!  We therefore ask
%%   people _not_ to put them into distributions, archives, etc. without
%%   prior consultation with the LaTeX Project Team.
%%
%% -----------------------------------------------------------------------
%%
%
%<*driver|package>
\RequirePackage{l3names}
\GetIdInfo$Id: l3candidates.dtx 3633 2012-05-12 20:11:32Z joseph $
  {L3 Experimental additions to l3kernel}
%</driver|package>
%<*driver>
\documentclass[full]{l3doc}
\begin{document}
  \DocInput{\jobname.dtx}
\end{document}
%</driver>
% \fi
%
% \title{^^A
%   The \textsf{l3candidates} package\\ Experimental additions to
%   \pkg{l3kernel}^^A
%   \thanks{This file describes v\ExplFileVersion,
%     last revised \ExplFileDate.}^^A
% }
%
% \author{^^A
%  The \LaTeX3 Project\thanks
%    {^^A
%      E-mail:
%        \href{mailto:latex-team@latex-project.org}
%          {latex-team@latex-project.org}^^A
%    }^^A
% }
%
% \date{Released \ExplFileDate}
%
% \maketitle
%
% \begin{documentation}
% 
% This module provides a space in which functions can be added to
% \pkg{l3kernel} (\pkg{expl3}) while still being experimental. As such, the
% functions here may not remain in their current form, or indeed at all,
% in \pkg{l3kernel} in the future. In contrast to the material in
% \pkg{l3experimental}, the functions here are all \emph{small} additions to
% the kernel. We encourage programmers to test them out and report back on
% the \texttt{LaTeX-L} mailing list.
% 
% \section{Additions to \pkg{l3basics}}
%
% \begin{function}[EXP, TF]{\cs_if_exist_use:N, \cs_if_exist_use:c}
%   \begin{syntax}
%     \cs{cs_if_exist_use:NTF} \meta{control sequence} \Arg{true code} \Arg{false code}
%   \end{syntax}
%   If the \meta{control sequence} exists, leave it in the input stream,
%   followed by the \meta{true code} (unbraced). Otherwise, leave the
%   \meta{false} code in the input stream. For example,
%   \begin{verbatim}
%     \cs_set:Npn \mypkg_use_character:N #1
%       { \cs_if_exist_use:cF { mypkg_#1:n } { \mypkg_default:N #1 } }
%   \end{verbatim}
%   calls the function |\mypkg_#1:n| if it exists, and falls back to
%   a default action otherwise. This could also be done (more slowly)
%   using \cs{prg_case_str:xxn}.
%   \begin{texnote}
%     The \texttt{c} variants do not introduce the \meta{control sequence}
%     in the hash table if it is not there.
%   \end{texnote}
% \end{function}
% 
% \section{Additions to \pkg{l3box}}
% 
% \subsection{Affine transformations}
%
% Affine transformations are changes which (informally) preserve straight
% lines. Simple translations are affine transformations, but are better handled
% in \TeX{} by doing the translation first, then inserting an unmodified box.
% On the other hand, rotation and resizing of boxed material can best be
% handled by modifying boxes. These transformations are described here.
%
% \begin{function}{\box_resize:Nnn, \box_resize:cnn}
%   \begin{syntax}
%     \cs{box_resize:Nnn} \meta{box} \Arg{x-size} \Arg{y-size}
%   \end{syntax}
%   Resize the \meta{box} to \meta{x-size} horizontally and \meta{y-size}
%   vertically (both of the sizes are dimension expressions).
%   The \meta{y-size} is the vertical size (height plus depth) of
%   the box. The updated \meta{box} will be an hbox, irrespective of the nature
%   of the \meta{box}  before the resizing is applied. Negative sizes will
%   cause the material in the \meta{box} to be reversed in direction, but the
%   reference point of the \meta{box} will be unchanged. The resizing applies
%   within the current \TeX{} group level.
% \end{function}
%
% \begin{function}
%   {\box_resize_to_ht_plus_dp:Nn, \box_resize_to_ht_plus_dp:cn}
%   \begin{syntax}
%     \cs{box_resize_to_ht_plus_dp:Nn} \meta{box} \Arg{y-size}
%   \end{syntax}
%   Resize the \meta{box} to \meta{y-size} vertically, scaling the horizontal
%   size by the same amount (\meta{y-size} is a dimension expression).
%   The \meta{y-size} is the vertical size (height plus depth) of
%   the box.
%   The updated \meta{box} will be an hbox, irrespective of the nature
%   of the \meta{box}  before the resizing is applied. A negative size will
%   cause the material in the \meta{box} to be reversed in direction, but the
%   reference point of the \meta{box} will be unchanged. The resizing applies
%   within the current \TeX{} group level.
% \end{function}
%
% \begin{function}{\box_resize_to_wd:Nn, \box_resize_to_wd:cn}
%   \begin{syntax}
%     \cs{box_resize_to_wd:Nn} \meta{box} \Arg{x-size}
%   \end{syntax}
%   Resize the \meta{box} to \meta{x-size} horizontally, scaling the vertical
%   size by the same amount (\meta{x-size} is a dimension expression).
%   The updated \meta{box} will be an hbox, irrespective of the nature
%   of the \meta{box}  before the resizing is applied. A negative size will
%   cause the material in the \meta{box} to be reversed in direction, but the
%   reference point of the \meta{box} will be unchanged. The resizing applies
%   within the current \TeX{} group level.
% \end{function}
%
% \begin{function}{\box_rotate:Nn, \box_rotate:cn}
%   \begin{syntax}
%     \cs{box_rotate:Nn} \meta{box} \Arg{angle}
%   \end{syntax}
%   Rotates the \meta{box} by \meta{angle} (in degrees) anti-clockwise about
%   its reference point. The reference point of the updated box will be moved
%   horizontally such that it is at the left side of the smallest rectangle
%   enclosing the rotated material.
%   The updated \meta{box} will be an hbox, irrespective of the nature
%   of the \meta{box} before the rotation is applied. The rotation applies
%   within the current \TeX{} group level.
% \end{function}
%
% \begin{function}{\box_scale:Nnn, \box_scale:cnn}
%   \begin{syntax}
%     \cs{box_scale:Nnn} \meta{box} \Arg{x-scale} \Arg{y-scale}
%   \end{syntax}
%   Scales the \meta{box} by factors \meta{x-scale} and \meta{y-scale} in
%   the horizontal and vertical directions, respectively (both scales are
%   integer expressions). The updated \meta{box} will be an hbox, irrespective
%   of the nature of the \meta{box} before the scaling is applied. Negative
%   scalings will cause the material in the \meta{box} to be reversed in
%   direction, but the reference point of the \meta{box} will be unchanged.
%   The scaling applies within the current \TeX{} group level.
% \end{function}
%
% \subsection{Viewing part of a box}
%
% \begin{function}[added = 2011-11-13]{\box_clip:N, \box_clip:c}
%   \begin{syntax}
%     \cs{box_clip:N} \meta{box}
%   \end{syntax}
%   Clips the \meta{box} in the output so that only material inside the
%   bounding box is displayed in the output. The updated \meta{box} will be an
%   hbox, irrespective of the nature of the \meta{box} before the clipping is
%   applied. The clipping applies within the current \TeX{} group level.
%
%   \textbf{This function is experimental}
%   \begin{texnote}
%     Clipping is implemented by the driver, and as such the full content of
%     the box is places in the output file. Thus clipping does not remove
%     any information from the raw output, and hidden material can therefore
%     be viewed by direct examination of the file.
%   \end{texnote}
% \end{function}
%
% \begin{function}{\box_trim:Nnnnn, \box_trim:cnnnn}
%   \begin{syntax}
%     \cs{box_trim:Nnnnn} \meta{box} \Arg{left} \Arg{bottom} \Arg{right} \Arg{top}
%   \end{syntax}
%   Adjusts the bounding box of the \meta{box} \meta{left} is removed from
%   the left-hand edge of the bounding box, \meta{right} from the right-hand
%   edge and so fourth. All adjustments are \meta{dimension expressions}.
%   Material output of the bounding box will still be displayed in the output
%   unless \cs{box_clip:N} is subsequently applied.
%   The updated \meta{box} will be an
%   hbox, irrespective of the nature of the \meta{box} before the viewport
%   operation is applied. The clipping applies within the current \TeX{}
%   group level.
% \end{function}
%
% \begin{function}{\box_viewport:Nnnnn, \box_viewport:cnnnn}
%   \begin{syntax}
%     \cs{box_viewport:Nnnnn} \meta{box} \Arg{llx} \Arg{lly} \Arg{urx} \Arg{ury}
%   \end{syntax}
%   Adjusts the bounding box of the \meta{box} such that it has lower-left
%   co-ordinates (\meta{llx}, \meta{lly}) and upper-right co-ordinates
%   (\meta{urx}, \meta{ury}). All four co-ordinate positions are
%   \meta{dimension expressions}. Material output of the bounding box will
%   still be displayed in the output unless \cs{box_clip:N} is
%   subsequently applied.
%   The updated \meta{box} will be an
%   hbox, irrespective of the nature of the \meta{box} before the viewport
%   operation is applied. The clipping applies within the current \TeX{}
%   group level.
% \end{function}
%
% \end{documentation}
%
% \begin{implementation}
%
% \section{\pkg{l3candidates} Implementation}
%
%    \begin{macrocode}
%<*initex|package>
%    \end{macrocode}
%
%    \begin{macrocode}
%<*package>
\ProvidesExplPackage
  {\ExplFileName}{\ExplFileDate}{\ExplFileVersion}{\ExplFileDescription}
\package_check_loaded_expl:
%</package>
%    \end{macrocode}
%    
% \subsection{Additions to \pkg{l3baiscs}}
%
% \begin{macro}[EXP,TF]{\cs_if_exist_use:N, \cs_if_exist_use:c}
% \begin{macro}[EXP]{\cs_if_exist_use:N, \cs_if_exist_use:c}
%   The \cs{cs_if_exist_use:\ldots{}} functions cannot be implemented
%   as conditionals because the true branch must leave both the control
%   sequence itself and the true code in the input stream.
%   For the \texttt{c} variants, we are careful not to put the control
%   sequence in the hash table if it does not exist.
%    \begin{macrocode}
\cs_set:Npn \cs_if_exist_use:NTF #1#2
  { \cs_if_exist:NTF #1 { #1 #2 } }
\cs_set:Npn \cs_if_exist_use:NF #1
  { \cs_if_exist:NTF #1 { #1 } }
\cs_set:Npn \cs_if_exist_use:NT #1 #2
  { \cs_if_exist:NTF #1 { #1 #2 } { } }
\cs_set:Npn \cs_if_exist_use:N #1
  { \cs_if_exist:NTF #1 { #1 } { } }
\cs_set:Npn \cs_if_exist_use:cTF #1#2
  { \cs_if_exist:cTF {#1} { \use:c {#1} #2 } }
\cs_set:Npn \cs_if_exist_use:cF #1
  { \cs_if_exist:cTF {#1} { \use:c {#1} } }
\cs_set:Npn \cs_if_exist_use:cT #1#2
  { \cs_if_exist:cTF {#1} { \use:c {#1} #2 } { } }
\cs_set:Npn \cs_if_exist_use:c #1
  { \cs_if_exist:cTF {#1} { \use:c {#1} } { } }
%    \end{macrocode}
% \end{macro}
% \end{macro}
%    
% \subsection{Additions to \pkg{l3box}}
% 
% \subsubsection{Affine transformations}
%
% \begin{variable}{\l_box_angle_fp}
%   When rotating boxes, the angle itself may be needed by the
%   engine-dependent code. This is done using the \pkg{fp} module so
%   that the value is tidied up properly.
%    \begin{macrocode}
\fp_new:N \l_box_angle_fp
%    \end{macrocode}
% \end{variable}
%
% \begin{variable}{\l_box_cos_fp, \l_box_sin_fp}
%   These are used to hold the calculated sine and cosine values while
%   carrying out a rotation.
%    \begin{macrocode}
\fp_new:N \l_box_cos_fp
\fp_new:N \l_box_sin_fp
%    \end{macrocode}
% \end{variable}
%
% \begin{variable}
%   {\l_box_top_dim, \l_box_bottom_dim, \l_box_left_dim, \l_box_right_dim}
%   These are the positions of the four edges of a box before
%   manipulation.
%    \begin{macrocode}
\dim_new:N \l_box_top_dim
\dim_new:N \l_box_bottom_dim
\dim_new:N \l_box_left_dim
\dim_new:N \l_box_right_dim
%    \end{macrocode}
% \end{variable}
%
% \begin{variable}
%  {
%    \l_box_top_new_dim,  \l_box_bottom_new_dim ,
%    \l_box_left_new_dim, \l_box_right_new_dim
%  }
%   These are the positions of the four edges of a box after
%   manipulation.
%    \begin{macrocode}
\dim_new:N \l_box_top_new_dim
\dim_new:N \l_box_bottom_new_dim
\dim_new:N \l_box_left_new_dim
\dim_new:N \l_box_right_new_dim
%    \end{macrocode}
% \end{variable}
%
% \begin{variable}{\l_box_internal_box}
%   Scratch space.
%    \begin{macrocode}
\box_new:N \l_box_internal_box
%    \end{macrocode}
% \end{variable}
%
% \begin{macro}{\box_rotate:Nn}
% \begin{macro}[aux]{\box_rotate_aux:N}
% \begin{macro}[aux]{\box_rotate_x:nnN, \box_rotate_y:nnN}
% \begin{macro}[aux]
%   {
%     \box_rotate_quadrant_one:,   \box_rotate_quadrant_two:,
%     \box_rotate_quadrant_three:, \box_rotate_quadrant_four:
%   }
%   Rotation of a box starts with working out the relevant sine and
%   cosine. The actual rotation is in an auxiliary to keep the flow slightly
%   clearer
%    \begin{macrocode}
\cs_new_protected:Npn \box_rotate:Nn #1#2
  {
    \hbox_set:Nn #1
      {
        \group_begin:
          \fp_set:Nn \l_box_angle_fp {#2}
          \fp_set:Nn \l_box_sin_fp { sin ( \l_box_angle_fp * deg ) }
          \fp_set:Nn \l_box_cos_fp { cos ( \l_box_angle_fp * deg ) }
          \box_rotate_aux:N #1
        \group_end:
    }
  }
%    \end{macrocode}
%   The edges of the box are then recorded: the left edge will
%   always be at zero. Rotation of the four edges then takes place: this is
%   most efficiently done on a quadrant by quadrant basis.
%    \begin{macrocode}
\cs_new_protected:Npn \box_rotate_aux:N #1
  {
    \dim_set:Nn \l_box_top_dim    {  \box_ht:N #1 }
    \dim_set:Nn \l_box_bottom_dim { -\box_dp:N #1 }
    \dim_set:Nn \l_box_right_dim  {  \box_wd:N #1 }
    \dim_zero:N \l_box_left_dim
%    \end{macrocode}
%   The next step is to work out the $x$ and $y$ coordinates of vertices of
%   the rotated box in relation to its original coordinates. The box can be
%   visualized with vertices $B$, $C$, $D$ and $E$ is illustrated
%   (Figure~\ref{fig:rotation}). The vertex $O$ is the reference point on the
%   baseline, and in this implementation is also the centre of rotation.
%   \begin{figure}
%     \centering
%     \setlength{\unitlength}{3pt}^^A
%     \begin{picture}(34,36)(12,44)
%       \thicklines
%       \put(20,52){\dashbox{1}(20,21){}}
%       \put(20,80){\line(0,-1){36}}
%       \put(12,58){\line(1, 0){34}}
%       \put(41,59){A}
%       \put(40,74){B}
%       \put(21,74){C}
%       \put(21,49){D}
%       \put(40,49){E}
%       \put(21,59){O}
%     \end{picture}
%     \caption{Co-ordinates of a box prior to rotation.}
%     \label{fig:rotation}
%   \end{figure}
%   The formulae are, for a point $P$ and angle $\alpha$:
%   \[
%     \begin{array}{l}
%       P'_x = P_x - O_x \\
%       P'_y = P_y - O_y \\
%       P''_x =  ( P'_x \cos(\alpha)) - ( P'_y \sin(\alpha) ) \\
%       P''_y =  ( P'_x \sin(\alpha)) + ( P'_y \cos(\alpha) ) \\
%       P'''_x = P''_x + O_x + L_x \\
%       P'''_y = P''_y + O_y
%    \end{array}
%   \]
%   The \enquote{extra} horizontal translation $L_x$ at the end is calculated
%   so that the leftmost point of the resulting box has $x$-coordinate $0$.
%   This is desirable as \TeX{} boxes must have the reference point at
%   the left edge of the box. (As $O$ is always $(0,0)$, this part of the
%   calculation is omitted here.)
%    \begin{macrocode}
    \fp_compare:nNnTF \l_box_sin_fp > \c_zero_fp
      {
        \fp_compare:nNnTF \l_box_cos_fp > \c_zero_fp
          { \box_rotate_quadrant_one: }
          { \box_rotate_quadrant_two: }
      }
      {
        \fp_compare:nNnTF \l_box_cos_fp < \c_zero_fp
          { \box_rotate_quadrant_three: }
          { \box_rotate_quadrant_four: }
      }
%    \end{macrocode}
%   The position of the box edges are now known, but the box at this
%   stage be misplaced relative to the current \TeX{} reference point. So the
%   content of the box is moved such that the reference point of the
%   rotated box will be in the same place as the original.
%    \begin{macrocode}
    \hbox_set:Nn \l_box_internal_box { \box_use:N #1 }
    \hbox_set:Nn \l_box_internal_box
      {
        \tex_kern:D -\l_box_left_new_dim
        \hbox:n
          {
            \driver_box_rotate_begin:
            \box_use:N \l_box_internal_box
            \driver_box_rotate_end:
          }
      }
%    \end{macrocode}
%   Tidy up the size of the box so that the material is actually inside
%   the bounding box. The result can then be used to reset the original
%   box.
%    \begin{macrocode}
    \box_set_ht:Nn \l_box_internal_box {  \l_box_top_new_dim }
    \box_set_dp:Nn \l_box_internal_box { -\l_box_bottom_new_dim }
    \box_set_wd:Nn \l_box_internal_box
      { \l_box_right_new_dim - \l_box_left_new_dim }
    \box_use:N \l_box_internal_box
  }
%    \end{macrocode}
% \end{macro}
% \end{macro}
%   These functions take a general point $(|#1|, |#2|)$ and rotate its
%   location about the origin, using the previously-set sine and cosine
%   values. Each function gives only one component of the location of the
%   updated point. This is because for rotation of a box each step needs
%   only one value, and so performance is gained by avoiding working
%   out both $x'$ and $y'$ at the same time. Contrast this with
%   the equivalent function in the \pkg{l3coffins} module, where both parts
%   are needed.
%    \begin{macrocode}
\cs_new_protected:Npn \box_rotate_x:nnN #1#2#3
  {
    \dim_set:Nn #3
      {
        \fp_to_dim:n
          {
                \l_box_cos_fp * \dim_to_fp:n {#1}
            - ( \l_box_sin_fp * \dim_to_fp:n {#2} )
          }
      }  
  }
\cs_new_protected:Npn \box_rotate_y:nnN #1#2#3
  {
    \dim_set:Nn #3
      {
        \fp_to_dim:n
          {
              \l_box_sin_fp * \dim_to_fp:n {#1}
            + \l_box_cos_fp * \dim_to_fp:n {#2}
          }
      }  
  }
%    \end{macrocode}
%   Rotation of the edges is done using a different formula for each
%   quadrant. In every case, the top and bottom edges only need the
%   resulting $y$-values, whereas the left and right edges need the
%   $x$-values. Each case is a question of picking out which corner
%   ends up at with the maximum top, bottom, left and right value. Doing
%   this by hand means a lot less calculating and avoids lots of
%   comparisons.
%    \begin{macrocode}
\cs_new_protected:Npn \box_rotate_quadrant_one:
  {
    \box_rotate_y:nnN \l_box_right_dim \l_box_top_dim
      \l_box_top_new_dim
    \box_rotate_y:nnN \l_box_left_dim  \l_box_bottom_dim
      \l_box_bottom_new_dim
    \box_rotate_x:nnN \l_box_left_dim  \l_box_top_dim
      \l_box_left_new_dim
    \box_rotate_x:nnN \l_box_right_dim \l_box_bottom_dim
      \l_box_right_new_dim
  }
\cs_new_protected:Npn \box_rotate_quadrant_two:
  {
    \box_rotate_y:nnN \l_box_right_dim \l_box_bottom_dim
      \l_box_top_new_dim
    \box_rotate_y:nnN \l_box_left_dim  \l_box_top_dim
      \l_box_bottom_new_dim
    \box_rotate_x:nnN \l_box_right_dim  \l_box_top_dim
      \l_box_left_new_dim
    \box_rotate_x:nnN \l_box_left_dim   \l_box_bottom_dim
      \l_box_right_new_dim
  }
\cs_new_protected:Npn \box_rotate_quadrant_three:
  {
    \box_rotate_y:nnN \l_box_left_dim  \l_box_bottom_dim
      \l_box_top_new_dim
    \box_rotate_y:nnN \l_box_right_dim \l_box_top_dim
      \l_box_bottom_new_dim
    \box_rotate_x:nnN \l_box_right_dim \l_box_bottom_dim
      \l_box_left_new_dim
    \box_rotate_x:nnN \l_box_left_dim   \l_box_top_dim
      \l_box_right_new_dim
  }
\cs_new_protected:Npn \box_rotate_quadrant_four:
  {
    \box_rotate_y:nnN \l_box_left_dim  \l_box_top_dim
      \l_box_top_new_dim
    \box_rotate_y:nnN \l_box_right_dim \l_box_bottom_dim
      \l_box_bottom_new_dim
    \box_rotate_x:nnN \l_box_left_dim  \l_box_bottom_dim
      \l_box_left_new_dim
    \box_rotate_x:nnN \l_box_right_dim \l_box_top_dim
      \l_box_right_new_dim
  }
%    \end{macrocode}
% \end{macro}
% \end{macro}
%
% \begin{variable}{\l_box_scale_x_fp, \l_box_scale_y_fp}
%   Scaling is potentially-different in the two axes.
%    \begin{macrocode}
\fp_new:N \l_box_scale_x_fp
\fp_new:N \l_box_scale_y_fp
%    \end{macrocode}
% \end{variable}
%
% \begin{macro}{\box_resize:Nnn, \box_resize:cnn}
% \begin{macro}[aux]{\box_resize_aux:Nnn}
%   Resizing a box starts by working out the various dimensions of the
%   existing box.
%    \begin{macrocode}
\cs_new_protected:Npn \box_resize:Nnn #1#2#3
  {
    \hbox_set:Nn #1
      {
        \group_begin:
          \dim_set:Nn \l_box_top_dim    {  \box_ht:N #1 }
          \dim_set:Nn \l_box_bottom_dim { -\box_dp:N #1 }
          \dim_set:Nn \l_box_right_dim  {  \box_wd:N #1 }
          \dim_zero:N \l_box_left_dim
%    \end{macrocode}
%   The $x$-scaling and resulting box size is easy enough to work
%   out: the dimension is that given as |#2|, and the scale is simply the
%   new width divided by the old one.
%    \begin{macrocode}
          \fp_set:Nn \l_box_scale_x_fp
            { \dim_to_fp:n {#2} / ( \dim_to_fp:n \l_box_right_dim ) }
%    \end{macrocode}
%   The $y$-scaling needs both the height and the depth of the current box.
%    \begin{macrocode}
          \fp_set:Nn \l_box_scale_y_fp
            {
              \dim_to_fp:n {#3} /
                ( \dim_to_fp:n { \l_box_top_dim - \l_box_bottom_dim } )
            }
%    \end{macrocode}
%   Hand off to the auxiliary which does the work.
%    \begin{macrocode}
          \box_resize_aux:Nnn #1 {#2} {#3}
        \group_end:
      }
  }
\cs_generate_variant:Nn \box_resize:Nnn { c }
%    \end{macrocode}
%   With at least one real scaling to do, the next phase is to find the new
%   edge co-ordinates. In the $x$~direction this is relatively easy: just
%   scale the right edge. This is done using the absolute value of the
%   scale so that the new edge is in the correct place. In the $y$~direction,
%   both dimensions have to be scaled, and this again needs the absolute
%   scale value. Once that is all done, the common resize/rescale code can
%   be employed.
%    \begin{macrocode}
\cs_new_protected:Npn \box_resize_aux:Nnn #1#2#3
  {
    \dim_compare:nNnTF {#2} > \c_zero_dim
      { \dim_set:Nn \l_box_right_new_dim {#2} }
      { \dim_set:Nn \l_box_right_new_dim { \c_zero_dim - ( #2 ) } }
    \dim_compare:nNnTF {#3} > \c_zero_dim
      {
        \dim_set:Nn \l_box_top_new_dim
          { \fp_use:N \l_box_scale_y_fp \l_box_top_dim }
        \dim_set:Nn \l_box_bottom_new_dim
          { \fp_use:N \l_box_scale_y_fp \l_box_bottom_dim }
      }
      {
        \dim_set:Nn \l_box_top_new_dim
          { - \fp_use:N \l_box_scale_y_fp \l_box_top_dim }
        \dim_set:Nn \l_box_bottom_new_dim
          { - \fp_use:N \l_box_scale_y_fp \l_box_bottom_dim }
      }
    \box_resize_common:N #1
  }
%    \end{macrocode}
% \end{macro}
% \end{macro}
%
% \begin{macro}{\box_resize_to_ht_plus_dp:Nn, \box_resize_to_ht_plus_dp:cn}
% \begin{macro}{\box_resize_to_wd:Nn, \box_resize_to_wd:cn}
%   Scaling to a total height or to a width is a simplified version of the main
%   resizing operation, with the scale simply copied between the two parts. The
%   internal auxiliary is called using the scaling value twice, as the sign for
%   both parts is needed (as this allows the same internal code to be used as
%   for the general case).
%    \begin{macrocode}
\cs_new_protected:Npn \box_resize_to_ht_plus_dp:Nn #1#2
  {
    \hbox_set:Nn #1
      {
        \group_begin:
          \dim_set:Nn \l_box_top_dim    {  \box_ht:N #1 }
          \dim_set:Nn \l_box_bottom_dim { -\box_dp:N #1 }
          \dim_set:Nn \l_box_right_dim  {  \box_wd:N #1 }
          \dim_zero:N \l_box_left_dim
          \fp_set:Nn \l_box_scale_y_fp
            {
              \dim_to_fp:n {#2} /
                ( \dim_to_fp:n { \l_box_top_dim - \l_box_bottom_dim } )
            }
          \fp_set_eq:NN \l_box_scale_x_fp \l_box_scale_y_fp
          \box_resize_aux:Nnn #1 {#2} {#2}
        \group_end:
      }
  }
\cs_generate_variant:Nn \box_resize_to_ht_plus_dp:Nn { c }
\cs_new_protected:Npn \box_resize_to_wd:Nn #1#2
  {
    \hbox_set:Nn #1
      {
        \group_begin:
          \dim_set:Nn \l_box_top_dim    {  \box_ht:N #1 }
          \dim_set:Nn \l_box_bottom_dim { -\box_dp:N #1 }
          \dim_set:Nn \l_box_right_dim  {  \box_wd:N #1 }
          \dim_zero:N \l_box_left_dim
          \fp_set:Nn \l_box_scale_x_fp
            { \dim_to_fp:n {#2} / ( \dim_to_fp:n \l_box_right_dim ) }
          \fp_set_eq:NN \l_box_scale_y_fp \l_box_scale_x_fp
          \box_resize_aux:Nnn #1 {#2} {#2}
        \group_end:
      }
  }
\cs_generate_variant:Nn \box_resize_to_wd:Nn { c }
%    \end{macrocode}
% \end{macro}
% \end{macro}
%
% \begin{macro}{\box_scale:Nnn, \box_scale:cnn}
% \begin{macro}[aux]{\box_scale_aux:Nnn}
%   When scaling a box, setting the scaling itself is easy enough. The
%   new dimensions are also relatively easy to find, allowing only for
%   the need to keep them positive in all cases. Once that is done then
%   after a check for the trivial scaling a hand-off can be made to the
%   common code. The dimension scaling operations are carried out using
%   the \TeX{} mechanism as it avoids needing to use \texttt{fp}
%   operations.
%    \begin{macrocode}
\cs_new_protected:Npn \box_scale:Nnn #1#2#3
  {
    \hbox_set:Nn #1
      {
        \group_begin:
          \fp_set:Nn \l_box_scale_x_fp {#2}
          \fp_set:Nn \l_box_scale_y_fp {#3}
          \dim_set:Nn \l_box_top_dim    {  \box_ht:N #1 }
          \dim_set:Nn \l_box_bottom_dim { -\box_dp:N #1 }
          \dim_set:Nn \l_box_right_dim  {  \box_wd:N #1 }
          \dim_zero:N \l_box_left_dim
          \box_scale_aux:Nnn #1 {#2} {#3}
        \group_end:
      }
  }
\cs_generate_variant:Nn \box_scale:Nnn { c }
\cs_new_protected:Npn \box_scale_aux:Nnn #1#2#3
  {
    \fp_compare:nNnTF \l_box_scale_y_fp > \c_zero_fp
      {
        \dim_set:Nn \l_box_top_new_dim    { #3 \l_box_top_dim }
        \dim_set:Nn \l_box_bottom_new_dim { #3 \l_box_bottom_dim }
      }
      {
        \dim_set:Nn  \l_box_top_new_dim    { -#3 \l_box_bottom_dim }
        \dim_set:Nn  \l_box_bottom_new_dim { -#3 \l_box_top_dim }
      }
    \fp_compare:nNnTF \l_box_scale_x_fp > \c_zero_fp
      { \l_box_right_new_dim #2 \l_box_right_dim }
      { \l_box_right_new_dim -#2 \l_box_right_dim }
    \box_resize_common:N #1
  }
%    \end{macrocode}
% \end{macro}
% \end{macro}
%
% \begin{macro}[int]{\box_resize_common:N}
%   The main resize function places in input into a box which will start
%   of with zero width, and includes the handles for engine rescaling.
%    \begin{macrocode}
\cs_new_protected:Npn \box_resize_common:N #1
  {
    \hbox_set:Nn \l_box_internal_box
      {
        \driver_box_scale_begin:
        \hbox_overlap_right:n { \box_use:N #1 }
        \driver_box_scale_end:
      }
%    \end{macrocode}
%   The new height and depth can be applied directly.
%    \begin{macrocode}
    \box_set_ht:Nn \l_box_internal_box { \l_box_top_new_dim }
    \box_set_dp:Nn \l_box_internal_box { \l_box_bottom_new_dim }
%    \end{macrocode}
%   Things are not quite as obvious for the width, as the reference point
%   needs to remain unchanged. For positive scaling factors resizing the
%   box is all that is needed. However, for case of a negative scaling
%   the material must be shifted such that the reference point ends up in
%   the right place.
%    \begin{macrocode}
    \fp_compare:nNnTF \l_box_scale_x_fp < \c_zero_fp
      {
        \hbox_to_wd:nn { \l_box_right_new_dim }
          {
            \tex_kern:D \l_box_right_new_dim
            \box_use:N \l_box_internal_box
            \tex_hss:D
          }
      }
      {
        \box_set_wd:Nn \l_box_internal_box { \l_box_right_new_dim }
        \box_use:N \l_box_internal_box
      }
  }
%    \end{macrocode}
% \end{macro}
%
% \subsubsection{Viewing part of a box}
%
% \begin{macro}{\box_clip:N, \box_clip:c}
%   A wrapper around the driver-dependent code.
%    \begin{macrocode}
\cs_new_protected:Npn \box_clip:N #1
  { \hbox_set:Nn #1 { \driver_box_use_clip:N #1 } }
\cs_generate_variant:Nn \box_clip:N { c }
%    \end{macrocode}
% \end{macro}
%
% \begin{macro}{\box_trim:Nnnnn, \box_trim:cnnnn}
%   Trimming from the left- and right-hand edges of the box is easy. The total
%   width is set to remove from the right, and a skip will shift the material
%   to remove from the left.
%    \begin{macrocode}
\cs_new_protected:Npn \box_trim:Nnnnn #1#2#3#4#5
  {
    \box_set_wd:Nn #1 { \box_wd:N #1 - (#4) - (#2) }
    \hbox_set:Nn #1
      {
        \skip_horizontal:n { - \dim_eval:n {#2} }
        \box_use:N #1
      }
%    \end{macrocode}
%   For the height and depth, there is a need to watch the baseline is
%   respected. Material always has to stay on the correct side, so trimming
%   has to check that there is enough material to trim.
%    \begin{macrocode}
    \dim_compare:nNnTF { \box_dp:N #1 } > {#3}
      { \box_set_dp:Nn #1 { \box_dp:N #1 - (#3) } }
      {
        \hbox_set:Nn #1
          { \box_move_down:nn { #3 - \box_dp:N #1 } { \box_use:N #1 } }
        \box_set_dp:Nn #1 \c_zero_dim
      }
    \dim_compare:nNnTF { \box_ht:N #1 } > {#5}
      { \box_set_ht:Nn #1 { \box_ht:N #1 - (#5) } }
      {
        \hbox_set:Nn #1
          { \box_move_up:nn { #5 - \box_ht:N #1 } { \box_use:N #1 } }
        \box_set_ht:Nn #1 \c_zero_dim
      }
  }
\cs_generate_variant:Nn \box_trim:Nnnnn { c }
%    \end{macrocode}
% \end{macro}
%
% \begin{macro}{\box_viewport:Nnnnn, \box_viewport:cnnnn}
%   The same general logic as for clipping, but with absolute dimensions.
%   Thus again width is easy and height is harder.
%    \begin{macrocode}
\cs_new_protected:Npn \box_viewport:Nnnnn #1#2#3#4#5
  {
    \box_set_wd:Nn #1 { (#4) - (#2) }
    \hbox_set:Nn #1
      {
        \skip_horizontal:n { - \dim_eval:n {#2} }
        \box_use:N #1
      }
    \dim_compare:nNnTF {#3} > \c_zero_dim
      {
        \hbox_set:Nn #1 { \box_move_down:nn {#3} { \box_use:N #1 } }
        \box_set_dp:Nn #1 \c_zero_dim
      }
      { \box_set_dp:Nn #1 { - \dim_eval:n {#3} } }
    \dim_compare:nNnTF {#5} > \c_zero_dim
      { \box_set_ht:Nn #1 {#5} }
      {
        \hbox_set:Nn #1
          { \box_move_up:nn { -\dim_eval:n {#5} } { \box_use:N #1 } }
        \box_set_ht:Nn #1 \c_zero_dim
      }
  }
\cs_generate_variant:Nn \box_viewport:Nnnnn { c }
%    \end{macrocode}
% \end{macro}
%
%    \begin{macrocode}
%</initex|package>
%    \end{macrocode}
%
% \end{implementation}
%
% \PrintIndex

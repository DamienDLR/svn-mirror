% \iffalse meta-comment
%
%% File: l3msg.dtx Copyright (C) 2009-2012 The LaTeX3 Project
%%
%% It may be distributed and/or modified under the conditions of the
%% LaTeX Project Public License (LPPL), either version 1.3c of this
%% license or (at your option) any later version.  The latest version
%% of this license is in the file
%%
%%    http://www.latex-project.org/lppl.txt
%%
%% This file is part of the "l3kernel bundle" (The Work in LPPL)
%% and all files in that bundle must be distributed together.
%%
%% The released version of this bundle is available from CTAN.
%%
%% -----------------------------------------------------------------------
%%
%% The development version of the bundle can be found at
%%
%%    http://www.latex-project.org/svnroot/experimental/trunk/
%%
%% for those people who are interested.
%%
%%%%%%%%%%%
%% NOTE: %%
%%%%%%%%%%%
%%
%%   Snapshots taken from the repository represent work in progress and may
%%   not work or may contain conflicting material!  We therefore ask
%%   people _not_ to put them into distributions, archives, etc. without
%%   prior consultation with the LaTeX3 Project.
%%
%% -----------------------------------------------------------------------
%
%<*driver|package>
\RequirePackage{l3names}
\GetIdInfo$Id$
  {L3 Messages}
%</driver|package>
%<*driver>
\documentclass[full]{l3doc}
\begin{document}
  \DocInput{\jobname.dtx}
\end{document}
%</driver>
% \fi
%
% \title{^^A
%   The \pkg{l3msg} package\\ Messages^^A
%   \thanks{This file describes v\ExplFileVersion,
%      last revised \ExplFileDate.}^^A
% }
%
% \author{^^A
%  The \LaTeX3 Project\thanks
%    {^^A
%      E-mail:
%        \href{mailto:latex-team@latex-project.org}
%          {latex-team@latex-project.org}^^A
%    }^^A
% }
%
% \date{Released \ExplFileDate}
%
% \maketitle
%
% \begin{documentation}
%
% Messages need to be passed to the user by modules, either when errors
% occur or to indicate how the code is proceeding. The \pkg{l3msg}
% module provides a consistent method for doing this (as opposed to
% writing directly to the terminal or log).
%
% The system used by \pkg{l3msg} to create messages divides the process
% into two distinct parts. Named messages are created in the first part
% of the process; at this stage, no decision is made about the type of
% output that the message will produce. The second part of the process
% is actually producing a message. At this stage a choice of message
% \emph{class} has to be made, for example \texttt{error},
% \texttt{warning} or \texttt{info}.
%
% By separating out the creation and use of messages, several benefits
% are available. First, the messages can be altered later without
% needing details of where they are used in the code. This makes it
% possible to alter the language used, the detail level and so on.
% Secondly, the output which results from a given message can be
% altered. This can be done on a message class, module or message name
% basis. In this way, message behaviour can be altered and messages can
% be entirely suppressed.
%
% \section{Creating new messages}
%
% ^^A Wrong: "\msg_new:nnnn" is (as it should) global.
% ^^A However, redirections are indeed local.
% All messages have to be created before they can be used. All message
% setting is local, with the general assumption that messages will
% be managed as part of module set up outside of any \TeX{} grouping.
%
% The text of messages will automatically by wrapped to the length
% available in the console. As a result, formatting is only needed
% where it will help to show meaning. In particular, |\\| may be
% used to force a new line and \verb*|\ | forces an explicit space.
% Additionally, |\{|, |\#|, |\}|, |\%| and |\~| can be used to produce
% the corresponding character.
%
% Messages may be subdivided \emph{by one level} using the~|/|
% character.  This is used within the message filtering system to allow
% for example the \LaTeX{} kernel messages to belong to the module
% \texttt{LaTeX} while still being filterable at a more granular level.
% Thus for example
% \begin{verbatim}
%   \msg_new:nnnn { mymodule } { submodule / message } ...
% \end{verbatim}
% will allow only those messages from the \texttt{submodule} to be filtered
% out.
%
% \begin{function}[updated = 2011-08-16]{\msg_new:nnnn, \msg_new:nnn}
%   \begin{syntax}
%     \cs{msg_new:nnnn} \Arg{module} \Arg{message} \Arg{text} \Arg{more text}
%   \end{syntax}
%   Creates a \meta{message} for a given \meta{module}.
%   The message will be defined to first give \meta{text} and then
%   \meta{more text} if the user requests it. If no \meta{more text} is
%   available then a standard text is given instead. Within \meta{text}
%   and \meta{more text} four parameters (|#1| to |#4|) can be used:
%   these will be supplied and expanded at the time the message is
%   used. An error will be raised if the \meta{message} already exists.
% \end{function}
%
% \begin{function}{\msg_set:nnnn, \msg_set:nnn, \msg_gset:nnnn, \msg_gset:nnn}
%   \begin{syntax}
%     \cs{msg_set:nnnn} \Arg{module} \Arg{message} \Arg{text} \Arg{more text}
%   \end{syntax}
%   Sets up the text for a \meta{message} for a given \meta{module}.
%   The message will be defined to first give \meta{text} and then
%   \meta{more text} if the user requests it. If no \meta{more text} is
%   available then a standard text is given instead. Within \meta{text}
%   and \meta{more text} four parameters (|#1| to |#4|) can be used:
%   these will be supplied an expanded at the time the message is used.
% \end{function}
%
% \begin{function}[EXP, pTF, added = 2012-03-03]{\msg_if_exist:nn}
%   \begin{syntax}
%     \cs{msg_if_exist_p:nn} \Arg{module} \Arg{message}
%     \cs{msg_if_exist:nnTF} \Arg{module} \Arg{message} \Arg{true code} \Arg{false code}
%   \end{syntax}
%   Tests whether the \meta{message} for the \meta{module} is currently
%   defined.
% \end{function}
%
% \section{Contextual information for messages}
%
% \begin{function}[rEXP]{\msg_line_context:}
%   \begin{syntax}
%     \cs{msg_line_context:}
%   \end{syntax}
%   Prints the current line number when a message is given, and
%   thus suitable for giving context to messages. The number itself
%   is proceeded by the text |on line|.
% \end{function}
%
% \begin{function}[EXP]{\msg_line_number:}
%   \begin{syntax}
%     \cs{msg_line_number:}
%   \end{syntax}
%   Prints the current line number when a message is given.
% \end{function}
%
% \begin{function}[EXP]{\msg_fatal_text:n}
%   \begin{syntax}
%     \cs{msg_fatal_text:n} \Arg{module}
%   \end{syntax}
%   Produces the standard text
%   \begin{quote}
%     \ttfamily
%     Fatal \meta{module} error
%   \end{quote}
%   This function can be redefined to alter the language in which the
%   message is given, using |#1| as the name of the \meta{module} to
%   be included.
% \end{function}
%
% \begin{function}[EXP]{\msg_critical_text:n}
%   \begin{syntax}
%     \cs{msg_critical_text:n} \Arg{module}
%   \end{syntax}
%   Produces the standard text
%   \begin{quote}
%     \ttfamily
%     Critical \meta{module} error
%   \end{quote}
%   This function can be redefined to alter the language in which the
%   message is given, using |#1| as the name of the \meta{module} to
%   be included.
% \end{function}
%
% \begin{function}[EXP]{\msg_error_text:n}
%   \begin{syntax}
%     \cs{msg_error_text:n} \Arg{module}
%   \end{syntax}
%   Produces the standard text
%   \begin{quote}
%     \ttfamily
%     \meta{module} error
%   \end{quote}
%   This function can be redefined to alter the language in which the
%   message is given, using |#1| as the name of the \meta{module} to
%   be included.
% \end{function}
%
% \begin{function}[EXP]{\msg_warning_text:n}
%   \begin{syntax}
%     \cs{msg_warning_text:n} \Arg{module}
%   \end{syntax}
%   Produces the standard text
%   \begin{quote}
%     \ttfamily
%     \meta{module} warning
%   \end{quote}
%   This function can be redefined to alter the language in which the
%   message is given, using |#1| as the name of the \meta{module} to
%   be included.
% \end{function}
%
% \begin{function}[EXP]{\msg_info_text:n}
%   \begin{syntax}
%     \cs{msg_info_text:n} \Arg{module}
%   \end{syntax}
%   Produces the standard text:
%   \begin{quote}
%     \ttfamily
%     \meta{module} info
%   \end{quote}
%   This function can be redefined to alter the language in which the
%   message is given, using |#1| as the name of the \meta{module} to
%   be included.
% \end{function}
% 
% \begin{function}[EXP]{\msg_see_documentation_text:n}
%   \begin{syntax}
%     \cs{msg_see_documentation_text:n} \Arg{module}
%   \end{syntax}
%   Produces the standard text
%   \begin{quote}
%     \ttfamily
%     
%     See the \meta{module} documentation for further information.
%   \end{quote}
%   This function can be redefined to alter the language in which the
%   message is given, using |#1| as the name of the \meta{module} to
%   be included.
% \end{function}
%
% \section{Issuing messages}
%
% Messages behave differently depending on the message class.  The
% kernel defines several common message classes, but more can be
% created.  The following describes the standard behaviour of each class
% if no redirection of the class or message is active.
%
% When issuing messages, any arguments passed should use
% \cs{tl_to_str:n} or \cs{token_to_str:N} to prevent unwanted expansion
% of the material.  In all cases, the message may be issued
% supplying~$0$ to~$4$ arguments.  If the number of arguments supplied
% here does not match the number in the definition of the message, extra
% arguments will be ignored, or empty arguments added (of course the
% sense of the message may be impaired).
%
% \begin{function}[updated = 2012-04-26]{\msg_class_set:nn}
%   \begin{syntax}
%     \cs{msg_class_set:nn} \Arg{class} \Arg{code}
%   \end{syntax}
%   Sets a \meta{class} to output a message, using \meta{code}
%   to process the message text. The \meta{class} should be a text
%   value, while the \meta{code} may be any arbitrary material.
%   The \meta{code} will receive~$6$ arguments: the module
%   name (|#1|), the message name (|#2|) and the four arguments
%   taken by the message text (|#3| to |#6|).
% \end{function}
%
% \begin{function}[updated = 2012-04-27]
%   {
%     \msg_fatal:nnxxxx ,
%     \msg_fatal:nnxxx  ,
%     \msg_fatal:nnxx   ,
%     \msg_fatal:nnx    ,
%     \msg_fatal:nn
%   }
%   \begin{syntax}
%     \cs{msg_fatal:nnxxxx} \Arg{module} \Arg{message} \Arg{arg one} \Arg{arg two} \Arg{arg three} \Arg{arg four}
%   \end{syntax}
%   Issues \meta{module} error \meta{message}, passing \meta{arg one} to
%   \meta{arg four} to the text-creating functions. After issuing a
%   fatal error the \TeX{} run will halt.
% \end{function}
%
% \begin{function}[updated = 2012-04-27]
%   {
%     \msg_critical:nnxxxx ,
%     \msg_critical:nnxxx  ,
%     \msg_critical:nnxx   ,
%     \msg_critical:nnx    ,
%     \msg_critical:nn
%   }
%   \begin{syntax}
%     \cs{msg_critical:nnxxxx} \Arg{module} \Arg{message} \Arg{arg one} \Arg{arg two} \Arg{arg three} \Arg{arg four}
%   \end{syntax}
%   Issues \meta{module} error \meta{message}, passing \meta{arg one} to
%   \meta{arg four} to the text-creating functions.  After issuing a
%   critical error, \TeX{} will stop reading the current input file.
%   This may halt the \TeX{} run (if the current file is the main file)
%   or may abort reading a sub-file.
%   \begin{texnote}
%     The \TeX{} \tn{endinput} primitive is used to exit the file.  In
%     particular, the rest of the current line remains in the input
%     stream. ^^A Can we fix that?
%   \end{texnote}
% \end{function}
%
% \begin{function}[updated = 2012-04-27]
%   {
%     \msg_error:nnxxxx ,
%     \msg_error:nnxxx  ,
%     \msg_error:nnxx   ,
%     \msg_error:nnx    ,
%     \msg_error:nn
%   }
%   \begin{syntax}
%     \cs{msg_error:nnxxxx} \Arg{module} \Arg{message} \Arg{arg one} \Arg{arg two} \Arg{arg three} \Arg{arg four}
%   \end{syntax}
%   Issues \meta{module} error \meta{message}, passing \meta{arg one} to
%   \meta{arg four} to the text-creating functions.  The error will
%   interrupt processing and issue the text at the terminal.  After user
%   input, the run will continue.
% \end{function}
%
% \begin{function}[updated = 2012-04-27]
%   {
%     \msg_warning:nnxxxx ,
%     \msg_warning:nnxxx  ,
%     \msg_warning:nnxx   ,
%     \msg_warning:nnx    ,
%     \msg_warning:nn
%   }
%   \begin{syntax}
%     \cs{msg_warning:nnxxxx} \Arg{module} \Arg{message} \Arg{arg one} \Arg{arg two} \Arg{arg three} \Arg{arg four}
%   \end{syntax}
%   Issues \meta{module} warning \meta{message}, passing \meta{arg one}
%   to \meta{arg four} to the text-creating functions. The warning text
%   will be added to the log file and the terminal, but the \TeX{} run
%   will not be interrupted.
% \end{function}
%
% \begin{function}[updated = 2012-04-27]
%   {
%     \msg_info:nnxxxx ,
%     \msg_info:nnxxx  ,
%     \msg_info:nnxx   ,
%     \msg_info:nnx    ,
%     \msg_info:nn
%   }
%   \begin{syntax}
%     \cs{msg_info:nnxxxx} \Arg{module} \Arg{message} \Arg{arg one} \Arg{arg two} \Arg{arg three} \Arg{arg four}
%   \end{syntax}
%   Issues \meta{module} information \meta{message}, passing
%   \meta{arg one} to \meta{arg four} to the text-creating functions.
%   The information text will be added to the log file.
% \end{function}
%
% \begin{function}[updated = 2012-04-27]
%   {
%     \msg_log:nnxxxx ,
%     \msg_log:nnxxx  ,
%     \msg_log:nnxx   ,
%     \msg_log:nnx    ,
%     \msg_log:nn
%   }
%   \begin{syntax}
%     \cs{msg_log:nnxxxx} \Arg{module} \Arg{message} \Arg{arg one} \Arg{arg two} \Arg{arg three} \Arg{arg four}
%   \end{syntax}
%   Issues \meta{module} information \meta{message}, passing
%   \meta{arg one} to \meta{arg four} to the text-creating functions.
%   The information text will be added to the log file: the output
%   is briefer than \cs{msg_info:nnxxxx}.
% \end{function}
%
% \begin{function}[updated = 2012-04-27]
%   {
%     \msg_none:nnxxxx ,
%     \msg_none:nnxxx  ,
%     \msg_none:nnxx   ,
%     \msg_none:nnx    ,
%     \msg_none:nn
%   }
%   \begin{syntax}
%     \cs{msg_none:nnxxxx} \Arg{module} \Arg{message} \Arg{arg one} \Arg{arg two} \Arg{arg three} \Arg{arg four}
%   \end{syntax}
%   Does nothing: used as a message class to prevent any output at
%   all (see the discussion of message redirection).
% \end{function}
%
% \section{Redirecting messages}
%
% Each message has a \enquote{name}, which can be used to alter the behaviour
% of the message when it is given. Thus we might have
% \begin{verbatim}
%   \msg_new:nnnn { module } { my-message } { Some~text } { Some~more~text }
% \end{verbatim}
% to define a message, with
% \begin{verbatim}
%   \msg_error:nn { module } { my-message }
% \end{verbatim}
% when it is used. With no filtering, this will raise an error. However, we
% could alter the behaviour with
% \begin{verbatim}
%   \msg_redirect_class:nn { error } { warning }
% \end{verbatim}
% to turn all errors into warnings, or with
% \begin{verbatim}
%   \msg_redirect_module:nnn { module } { error } { warning }
% \end{verbatim}
% to alter only messages from that module, or even
% \begin{verbatim}
%   \msg_redirect_name:nnn { module } { my-message } { warning }
% \end{verbatim}
% to target just one message. Redirection applies first to individual messages,
% then to messages from one module and finally to messages of one class. Thus
% it is possible to select out an individual message for special treatment
% even if the entire class is already redirected.
%
% Multiple redirections are possible.  Redirections can be cancelled by
% providing an empty argument for the target class.  Redirection to a
% missing class will raise errors immediately.  Infinite loops are
% prevented by eliminating the redirection starting from the target of
% the redirection that caused the loop to appear.  Namely, if
% redirections are requested as $A \to B$, $B \to C$ and $C \to A$ in
% this order, then the $A \to B$ redirection is cancelled.
%
% \begin{function}[updated = 2012-04-27]{\msg_redirect_class:nn}
%   \begin{syntax}
%     \cs{msg_redirect_class:nn} \Arg{class one} \Arg{class two}
%   \end{syntax}
%   Changes the behaviour of messages of \meta{class one} so that they
%   are processed using the code for those of \meta{class two}.
% \end{function}
%
% \begin{function}[updated = 2012-04-27]{\msg_redirect_module:nnn}
%   \begin{syntax}
%     \cs{msg_redirect_module:nnn} \Arg{module} \Arg{class one} \Arg{class two}
%   \end{syntax}
%   Redirects message of \meta{class one} for \meta{module} to act as
%   though they were from \meta{class two}. Messages of \meta{class one}
%   from sources other than \meta{module} are not affected by this
%   redirection. This function can be used to make some messages
%   \enquote{silent} by default. For example, all of the
%   \texttt{warning} messages of \meta{module} could be turned off with:
%   \begin{verbatim}
%     \msg_redirect_module:nnn { module } { warning } { none }
%   \end{verbatim}
% \end{function}
%
% \begin{function}[updated = 2012-04-27]{\msg_redirect_name:nnn}
%   \begin{syntax}
%     \cs{msg_redirect_name:nnn} \Arg{module} \Arg{message} \Arg{class}
%   \end{syntax}
%   Redirects a specific \meta{message} from a specific \meta{module} to
%   act as a member of \meta{class} of messages.  No further redirection
%   is performed.  This function can be used to make a selected message
%   \enquote{silent} without changing global parameters:
%   \begin{verbatim}
%     \msg_redirect_name:nnn { module } { annoying-message } { none }
%   \end{verbatim}
% \end{function}
%
% \section{Low-level message functions}
%
% The lower-level message functions should usually be accessed from the
% higher-level system. However, there are occasions where direct
% access to these functions is desirable.
%
% \begin{function}[EXP]{\msg_newline:, \msg_two_newlines:}
%   \begin{syntax}
%     \cs{msg_newline:}
%   \end{syntax}
%   Forces a new line in a message. This is a low-level function, which
%   will not include any additional printing information in the message:
%   contrast with |\\| in messages. The |two| version adds two lines.
% \end{function}
%
% \begin{function}[updated = 2012-04-26]{\msg_interrupt:xxx}
%   \begin{syntax}
%     \cs{msg_interrupt:xxx} \Arg{first line} \Arg{text} \Arg{extra text}
%   \end{syntax}
%   Interrupts the \TeX{} run, issuing a formatted message comprising
%   \meta{first line} and \meta{text} laid out in the format
%   \begin{verbatim}
%     !!!!!!!!!!!!!!!!!!!!!!!!!!!!!!!!!!!!!!!!!!!!!!!!
%     !
%     ! <first line>
%     !
%     ! <text>
%     !...............................................
%   \end{verbatim}
%   where the \meta{text} will be wrapped to fit within the current
%   line length. The user may then request more information, at which
%   stage the \meta{extra text} will be shown in the terminal in the
%   format
%   \begin{verbatim}
%     |'''''''''''''''''''''''''''''''''''''''''''''''
%     |  <extra text>
%     |...............................................
%   \end{verbatim}
%   where the \meta{extra text} will be wrapped to fit within the current
%   line length.
% \end{function}
%
% \begin{function}[updated = 2012-06-05]{\msg_log:x}
%   \begin{syntax}
%     \cs{msg_log:x} \Arg{text}
%   \end{syntax}
%   Writes to the log file with the \meta{text} laid out in the format
%   \begin{verbatim}
%     .................................................
%     . <text>
%     .................................................
%   \end{verbatim}
%   where the \meta{text} will be wrapped to fit within the current
%   line length.
% \end{function}
%
% \begin{function}[updated = 2012-06-05]{\msg_term:x}
%   \begin{syntax}
%     \cs{msg_term:x} \Arg{text}
%   \end{syntax}
%   Writes to the terminal and log file with the \meta{text} laid out in the
%   format
%   \begin{verbatim}
%     *************************************************
%     * <text>
%     *************************************************
%   \end{verbatim}
%   where the \meta{text} will be wrapped to fit within the current
%   line length.
% \end{function}
%
% \section{Kernel-specific functions}
%
% Messages from \LaTeX3 itself are handled by the general message system,
% but have their own functions. This allows some text to be pre-defined,
% and also ensures that serious errors can be handled properly.
%
% \begin{function}[updated = 2011-08-16]
%   {\__msg_kernel_new:nnnn, \__msg_kernel_new:nnn}
%   \begin{syntax}
%     \cs{__msg_kernel_new:nnnn} \Arg{module} \Arg{message} \Arg{text} \Arg{more text}
%   \end{syntax}
%   Creates a kernel \meta{message} for a given \meta{module}.
%   The message will be defined to first give \meta{text} and then
%   \meta{more text} if the user requests it. If no \meta{more text} is
%   available then a standard text is given instead. Within \meta{text}
%   and \meta{more text} four parameters (|#1| to |#4|) can be used:
%   these will be supplied and expanded at the time the message is used.
%   An error will be raised if the \meta{message} already exists.
% \end{function}
%
% \begin{function}{\__msg_kernel_set:nnnn, \__msg_kernel_set:nnn}
%   \begin{syntax}
%     \cs{__msg_kernel_set:nnnn} \Arg{module} \Arg{message} \Arg{text} \Arg{more text}
%   \end{syntax}
%   Sets up the text for a kernel \meta{message} for a given \meta{module}.
%   The message will be defined to first give \meta{text} and then
%   \meta{more text} if the user requests it. If no \meta{more text} is
%   available then a standard text is given instead. Within \meta{text}
%   and \meta{more text} four parameters (|#1| to |#4|) can be used:
%   these will be supplied and expanded at the time the message is used.
% \end{function}
%
% \begin{function}[updated = 2012-04-27]
%   {
%     \__msg_kernel_fatal:nnxxxx ,
%     \__msg_kernel_fatal:nnxxx  ,
%     \__msg_kernel_fatal:nnxx   ,
%     \__msg_kernel_fatal:nnx    ,
%     \__msg_kernel_fatal:nn
%   }
%   \begin{syntax}
%     \cs{__msg_kernel_fatal:nnxxxx} \Arg{module} \Arg{message} \Arg{arg one} \Arg{arg two} \Arg{arg three} \Arg{arg four}
%   \end{syntax}
%   Issues kernel \meta{module} error \meta{message}, passing \meta{arg one}
%   to \meta{arg four} to the text-creating functions. After issuing a
%   fatal error the \TeX{} run will halt. Cannot be redirected.
% \end{function}
%
% \begin{function}[updated = 2012-04-27]
%   {
%     \__msg_kernel_error:nnxxxx ,
%     \__msg_kernel_error:nnxxx  ,
%     \__msg_kernel_error:nnxx   ,
%     \__msg_kernel_error:nnx    ,
%     \__msg_kernel_error:nn
%   }
%   \begin{syntax}
%     \cs{__msg_kernel_error:nnxxxx} \Arg{module} \Arg{message} \Arg{arg one} \Arg{arg two} \Arg{arg three} \Arg{arg four}
%   \end{syntax}
%   Issues kernel \meta{module} error \meta{message}, passing \meta{arg one}
%   to
%   \meta{arg four} to the text-creating functions. The error will
%   stop processing and issue the text at the terminal. After user input,
%   the run will continue. Cannot be redirected.
% \end{function}
%
% \begin{function}[updated = 2012-04-27]
%   {
%     \__msg_kernel_warning:nnxxxx ,
%     \__msg_kernel_warning:nnxxx  ,
%     \__msg_kernel_warning:nnxx   ,
%     \__msg_kernel_warning:nnx    ,
%     \__msg_kernel_warning:nn
%   }
%   \begin{syntax}
%     \cs{__msg_kernel_warning:nnxxxx} \Arg{module} \Arg{message} \Arg{arg one} \Arg{arg two} \Arg{arg three} \Arg{arg four}
%   \end{syntax}
%   Issues kernel \meta{module} warning \meta{message}, passing
%   \meta{arg one} to
%   \meta{arg four} to the text-creating functions. The warning text
%   will be added to the log file, but the \TeX{} run will not be
%   interrupted.
% \end{function}
%
% \begin{function}[updated = 2012-04-27]
%   {
%     \__msg_kernel_info:nnxxxx ,
%     \__msg_kernel_info:nnxxx  ,
%     \__msg_kernel_info:nnxx   ,
%     \__msg_kernel_info:nnx    ,
%     \__msg_kernel_info:nn
%   }
%   \begin{syntax}
%     \cs{__msg_kernel_info:nnxxxx} \Arg{module} \Arg{message} \Arg{arg one} \Arg{arg two} \Arg{arg three} \Arg{arg four}
%   \end{syntax}
%   Issues kernel \meta{module} information \meta{message}, passing
%   \meta{arg one} to \meta{arg four} to the text-creating functions.
%   The information text will be added to the log file.
% \end{function}
%
% \section{Expandable errors}
%
% In a few places, the \LaTeX3 kernel needs to produce errors in an
% expansion only context. This must be handled internally very
% differently from normal error messages, as none of the tools
% to print to the terminal or the log file are expandable.
% However, the interface is similar, with the important caveat that the
% message text and arguments are not expanded, and messages should be
% very short.
%
% \begin{function}[EXP, added = 2011-11-23]
%   {
%     \__msg_kernel_expandable_error:nnnnnn,
%     \__msg_kernel_expandable_error:nnnnn,
%     \__msg_kernel_expandable_error:nnnn,
%     \__msg_kernel_expandable_error:nnn,
%     \__msg_kernel_expandable_error:nn
%   }
%   \begin{syntax}
%     \cs{__msg_kernel_expandable_error:nnnnnn} \Arg{module} \Arg{message} \Arg{arg one} \Arg{arg two} \Arg{arg three} \Arg{arg four}
%   \end{syntax}
%   Issues an error, passing \meta{arg one} to \meta{arg four}
%   to the text-creating functions. The resulting string must
%   be shorter than a line, otherwise it will be cropped.
% \end{function}
%
% \begin{function}[EXP, added = 2011-08-11, updated = 2011-08-13]
%   {\__msg_expandable_error:n}
%   \begin{syntax}
%     \cs{__msg_expandable_error:n} \Arg{error message}
%   \end{syntax}
%   Issues an \enquote{Undefined error} message from \TeX{} itself,
%   and prints the \meta{error message}. The \meta{error message}
%   must be short: it is cropped at the end of one line.
%   \begin{texnote}
%     This function expands to an empty token list after two steps.
%     Tokens inserted in response to \TeX{}'s prompt are read with
%     the current category code setting, and inserted just after
%     the place where the error message was issued.
%   \end{texnote}
% \end{function}
%
% \section{Internal \pkg{l3msg} functions}
%
% The following functions are used in several kernel modules.
%
% \begin{function}{\__msg_aux_use:nn, \__msg_aux_use:nnxxxx}
%   \begin{syntax}
%     \cs{__msg_aux_use:nnxxxx} \Arg{module} \Arg{message} \Arg{arg one} \Arg{arg two} \Arg{arg three} \Arg{arg four}
%   \end{syntax}
%   Prints the \meta{message} from \meta{module} in the terminal,
%   without formatting.
% \end{function}
%
% \begin{function}{\__msg_aux_show:x}
%   \begin{syntax}
%     \cs{__msg_aux_show:x} \Arg{formatted string}
%   \end{syntax}
%   Shows the \meta{formatted string} on the terminal.  After expansion,
%   unless it is empty, the \meta{formatted string} must contain |>|,
%   and the part of \meta{formatted string} before the first |>| is
%   removed. Failure to do so causes low-level \TeX{} errors.
% \end{function}
%
% \begin{function}{\__msg_aux_show:Nnx}
%   \begin{syntax}
%     \cs{__msg_aux_show:Nnx} \meta{variable} \Arg{module} \Arg{token list}
%   \end{syntax}
%   Auxiliary common to \pkg{l3clist}, \pkg{l3prop} and \pkg{seq},
%   which displays an appropriate message and the contents of the variable.
% \end{function}
%
% \end{documentation}
%
% \begin{implementation}
%
% \section{\pkg{l3msg} implementation}
%
%    \begin{macrocode}
%<*initex|package>
%    \end{macrocode}
%    
%    \begin{macrocode}
%<@@=msg>
%    \end{macrocode}
%
%    \begin{macrocode}
%<*package>
\ProvidesExplPackage
  {\ExplFileName}{\ExplFileDate}{\ExplFileVersion}{\ExplFileDescription}
\__expl_package_check:
%</package>
%    \end{macrocode}
%
% \begin{variable}{\l_@@_internal_tl}
%   A general scratch for the module.
%    \begin{macrocode}
\tl_new:N \l_@@_internal_tl
%    \end{macrocode}
% \end{variable}
%
% \subsection{Creating messages}
%
% Messages are created and used separately, so there two parts to
% the code here. First, a mechanism for creating message text.
% This is pretty simple, as there is not actually a lot to do.
%
% \begin{variable}{\c_@@_text_prefix_tl, \c_@@_more_text_prefix_tl}
%   Locations for the text of messages.
%    \begin{macrocode}
\tl_const:Nn \c_@@_text_prefix_tl      { msg~text~>~ }
\tl_const:Nn \c_@@_more_text_prefix_tl { msg~extra~text~>~ }
%    \end{macrocode}
% \end{variable}
%
% \begin{macro}[EXP,pTF]{\msg_if_exist:nn}
%   Test whether the control sequence containing the message text exists
%   or not.
%    \begin{macrocode}
\prg_new_conditional:Npnn \msg_if_exist:nn #1#2 { p , T , F , TF }
  {
    \cs_if_exist:cTF { \c_@@_text_prefix_tl #1 / #2 }
      { \prg_return_true: } { \prg_return_false: }
  }
%    \end{macrocode}
% \end{macro}
%
% \begin{macro}[aux]{\__chk_if_free_msg:nn}
%   This auxiliary is similar to \cs{__chk_if_free_cs:N}, and is used when
%   defining messages with \cs{msg_new:nnnn}. It could be inlined in
%   \cs{msg_new:nnnn}, but the experimental \pkg{l3trace} module needs
%   to disable this check when reloading a package with the extra
%   tracing information.
%    \begin{macrocode}
\cs_new_protected:Npn \__chk_if_free_msg:nn #1#2
  {
    \msg_if_exist:nnT {#1} {#2}
      {
        \@@_kernel_error:nnxx { msg } { message-already-defined }
          {#1} {#2}
      }
  }
%<*package>
\tex_ifodd:D \l@expl@log@functions@bool
  \cs_gset_protected:Npn \__chk_if_free_msg:nn #1#2
    {
      \msg_if_exist:nnT {#1} {#2}
        {
          \@@_kernel_error:nnxx { msg } { message-already-defined }
            {#1} {#2}
        }
      \iow_log:x { Defining~message~#1/#2~ \msg_line_context: }
    }
\fi:
%</package>
%    \end{macrocode}
% \end{macro}
%
% \begin{macro}{\msg_new:nnnn, \msg_new:nnn}
% \begin{macro}{\msg_gset:nnnn,\msg_gset:nnn}
% \begin{macro}{\msg_set:nnnn, \msg_set:nnn}
%   Setting a message simply means saving the appropriate text
%   into two functions. A sanity check first.
%    \begin{macrocode}
\cs_new_protected:Npn \msg_new:nnnn #1#2
  {
    \__chk_if_free_msg:nn {#1} {#2}
    \msg_gset:nnnn {#1} {#2}
  }
\cs_new_protected:Npn \msg_new:nnn #1#2#3
  { \msg_new:nnnn {#1} {#2} {#3} { } }
\cs_new_protected:Npn \msg_set:nnnn #1#2#3#4
  {
    \cs_set:cpn { \c_@@_text_prefix_tl #1 / #2 }
      ##1##2##3##4 {#3}
    \cs_set:cpn { \c_@@_more_text_prefix_tl #1 / #2 }
      ##1##2##3##4 {#4}
  }
\cs_new_protected:Npn \msg_set:nnn #1#2#3
  { \msg_set:nnnn {#1} {#2} {#3} { } }
\cs_new_protected:Npn \msg_gset:nnnn #1#2#3#4
  {
    \cs_gset:cpn { \c_@@_text_prefix_tl #1 / #2 }
      ##1##2##3##4 {#3}
    \cs_gset:cpn { \c_@@_more_text_prefix_tl #1 / #2 }
      ##1##2##3##4 {#4}
  }
\cs_new_protected:Npn \msg_gset:nnn #1#2#3
  { \msg_gset:nnnn {#1} {#2} {#3} { } }
%    \end{macrocode}
% \end{macro}
% \end{macro}
% \end{macro}
%
% \subsection{Messages: support functions and text}
%
% \begin{variable}
%   {
%     \c_msg_coding_error_text_tl ,
%     \c_msg_continue_text_tl     ,
%     \c_msg_critical_text_tl     ,
%     \c_msg_fatal_text_tl        ,
%     \c_msg_help_text_tl         ,
%     \c_msg_no_info_text_tl      ,
%     \c_msg_on_line_tl           ,
%     \c_msg_return_text_tl       ,
%     \c_msg_trouble_text_tl
%   }
% Simple pieces of text for messages.
%    \begin{macrocode}
\tl_const:Nn \c_msg_coding_error_text_tl
  {
    This~is~a~coding~error.
    \\ \\
  }
\tl_const:Nn \c_msg_continue_text_tl
  { Type~<return>~to~continue }
\tl_const:Nn \c_msg_critical_text_tl
  { Reading~the~current~file~will~stop }
\tl_const:Nn \c_msg_fatal_text_tl
  { This~is~a~fatal~error:~LaTeX~will~abort }
\tl_const:Nn \c_msg_help_text_tl
  { For~immediate~help~type~H~<return> }
\tl_const:Nn \c_msg_no_info_text_tl
  {
    LaTeX~does~not~know~anything~more~about~this~error,~sorry.
    \c_msg_return_text_tl
  }
\tl_const:Nn \c_msg_on_line_text_tl { on~line }
\tl_const:Nn \c_msg_return_text_tl
  {
    \\ \\
    Try~typing~<return>~to~proceed.
    \\
    If~that~doesn't~work,~type~X~<return>~to~quit.
  }
\tl_const:Nn \c_msg_trouble_text_tl
  {
    \\ \\
    More~errors~will~almost~certainly~follow: \\
    the~LaTeX~run~should~be~aborted.
  }
%    \end{macrocode}
% \end{variable}
%
% \begin{macro}{\msg_newline:, \msg_two_newlines:}
%   New lines are printed in the same way as for low-level file writing.
%    \begin{macrocode}
\cs_new_nopar:Npn \msg_newline: { ^^J }
\cs_new_nopar:Npn \msg_two_newlines: { ^^J ^^J }
%    \end{macrocode}
% \end{macro}
%
% \begin{macro}{\msg_line_number:, \msg_line_context:}
%   For writing the line number nicely. \cs{msg_line_context:} was set up
%   earlier, so this is not \texttt{new}.
%    \begin{macrocode}
\cs_new_nopar:Npn \msg_line_number: { \int_use:N \tex_inputlineno:D }
\cs_gset_nopar:Npn \msg_line_context:
  {
    \c_msg_on_line_text_tl
    \c_space_tl
    \msg_line_number:
  }
%    \end{macrocode}
% \end{macro}
%
% \subsection{Showing messages: low level mechanism}
%
% \begin{macro}{\msg_interrupt:xxx}
%   The low-level interruption macro is rather opaque, unfortunately.
%   Depending on the availability of more information there is a choice
%   of how to set up the further help.  We feed the extra help text and
%   the message itself to a wrapping auxiliary, in this order because we
%   must first setup \TeX{}'s \tn{errhelp} register before issuing an
%   \tn{errmessage}.
%    \begin{macrocode}
\cs_new_protected:Npn \msg_interrupt:xxx #1#2#3
  {
    \tl_if_empty:nTF {#3}
      {
        \@@_interrupt_wrap:xx { \\ \c_msg_no_info_text_tl }
          {#1 \\\\ #2 \\\\ \c_msg_continue_text_tl }
      }
      {
        \@@_interrupt_wrap:xx { \\ #3 }
          {#1 \\\\ #2 \\\\ \c_msg_help_text_tl }
      }
  }
%    \end{macrocode}
% \end{macro}
%
% \begin{macro}[aux]{\@@_interrupt_wrap:xx}
% \begin{macro}[aux]{\@@_interrupt_more_text:n}
%   First setup \TeX{}'s \tn{errhelp} register with the extra help |#1|,
%   then build a nice-looking error message with |#2|.  Everything is
%   done using \texttt{x}-type expansion as the new line markers are
%   different for the two type of text and need to be correctly set up.
%   The auxiliary \cs{@@_interrupt_more_text:n} receives its argument
%   as a line-wrapped string, which is thus unaffected by expansion.
%    \begin{macrocode}
\cs_new_protected:Npn \@@_interrupt_wrap:xx #1#2
  {
    \iow_wrap:nnnN {#1} { | ~ } { } \@@_interrupt_more_text:n
    \iow_wrap:nnnN {#2} { ! ~ } { } \@@_interrupt_text:n
  }
\cs_new_protected:Npn \@@_interrupt_more_text:n #1
  {
    \exp_args:Nx \tex_errhelp:D
      {
        |'''''''''''''''''''''''''''''''''''''''''''''''
        #1 \iow_newline:
        |...............................................
      }
  }
%    \end{macrocode}
% \end{macro}
% \end{macro}
%
% \begin{macro}[aux]{\@@_interrupt_text:n}
%   The business end of the process starts by producing some visual
%   separation of the message from the main part of the log.  The error
%   message needs to be printed with everything made
%   \enquote{invisible}: \TeX{}'s own information involves the macro in
%   which \tn{errmessage} is called, and the end of the argument of the
%   \tn{errmessage}, including the closing brace.  We use an active |!|
%   to call the \tn{errmessage} primitive, and end its argument with
%   \cs{use_none:n} \Arg{dots} which fills the output with dots.  Two
%   trailing closing braces are turned into spaces to hide them as well.
%   The group in which we alter the definition of the active |!| is
%   closed before producing the message: this ensures that tokens
%   inserted by typing |I| in the command-line will bee inserted after
%   the message is entirely cleaned up.
%    \begin{macrocode}
\group_begin:
  \char_set_lccode:nn {`\{} {`\ }
  \char_set_lccode:nn {`\}} {`\ }
  \char_set_lccode:nn {`\&} {`\!}
  \char_set_catcode_active:N \&
\tl_to_lowercase:n
  {
    \group_end:
    \cs_new_protected:Npn \@@_interrupt_text:n #1
      {
        \iow_term:x
          {
            \iow_newline:
            !!!!!!!!!!!!!!!!!!!!!!!!!!!!!!!!!!!!!!!!!!!!!!!!
            \iow_newline:
            !
          }
        \group_begin:
          \cs_set_protected_nopar:Npn &
            {
              \tex_errmessage:D
                {
                  #1
                  \use_none:n
                    { ............................................ }
                }
            }
          \exp_after:wN
        \group_end:
        &
      }
  }
%    \end{macrocode}
% \end{macro}
%
% \begin{macro}{\msg_log:x}
% \begin{macro}{\msg_term:x}
%   Printing to the log or terminal without a stop is rather easier.
%   A bit of simple visual work sets things off nicely.
%    \begin{macrocode}
\cs_new_protected:Npn \msg_log:x #1
  {
    \iow_log:n { ................................................. }
    \iow_wrap:nnnN { . ~ #1} { . ~ } { } \iow_log:n
    \iow_log:n { ................................................. }
  }
\cs_new_protected:Npn \msg_term:x #1
  {
    \iow_term:n { ************************************************* }
    \iow_wrap:nnnN { * ~ #1} { * ~ } { } \iow_term:n
    \iow_term:n { ************************************************* }
  }
%    \end{macrocode}
% \end{macro}
% \end{macro}
%
% \subsection{Displaying messages}
%
% \LaTeX{} is handling error messages and so the \TeX{} ones are disabled.
% This is already done by the \LaTeXe{} kernel, so to avoid messing up any
% deliberate change by a user this is only set in format mode.
%    \begin{macrocode}
%<*initex>
\int_gset_eq:NN \tex_errorcontextlines:D \c_minus_one
%</initex>
%    \end{macrocode}
%
% \begin{macro}
%   {
%     \msg_fatal_text:n    ,
%     \msg_critical_text:n ,
%     \msg_error_text:n    ,
%     \msg_warning_text:n  ,
%     \msg_info_text:n
%   }
%   A function for issuing messages: both the text and order could
%   in principle vary.
%    \begin{macrocode}
\cs_new:Npn \msg_fatal_text:n #1 { Fatal~#1~error }
\cs_new:Npn \msg_critical_text:n #1 { Critical~#1~error }
\cs_new:Npn \msg_error_text:n #1 { #1~error }
\cs_new:Npn \msg_warning_text:n #1 { #1~warning }
\cs_new:Npn \msg_info_text:n #1 { #1~info }
%    \end{macrocode}
% \end{macro}
%
% \begin{macro}{\msg_see_documentation_text:n}
%   Contextual footer information.  The \LaTeX{} module only comprises
%   \LaTeX3 code, so we refer to the \LaTeX3 documentation rather than
%   simply \enquote{\LaTeX{}}.
%    \begin{macrocode}
\cs_new:Npn \msg_see_documentation_text:n #1
  {
    \\ \\ See~the~
    \str_if_eq:nnTF {#1} { LaTeX } { LaTeX3 } {#1} ~
    documentation~for~further~information.
  }
%    \end{macrocode}
% \end{macro}
%
% \begin{macro}{\msg_class_set:nn}
%   Setting up a message class does two tasks. Any existing redirection
%   is cleared, and the various message functions are created
%   to simply use the code stored for the message.
%    \begin{macrocode}
\cs_new_protected:Npn \msg_class_set:nn #1#2
  {
    \cs_if_exist:cTF { msg_ #1 _code:nnxxxx }
      \cs_set_protected:cpn
      \cs_new_protected:cpn
        { msg_ #1 _code:nnxxxx } ##1##2##3##4##5##6 {#2}
    \prop_clear_new:c { l_msg_redirect_ #1 _prop }
    \cs_set_protected_nopar:cpn { msg_ #1 :nnxxxx }
      { \@@_use:nnnxxxx {#1} }
    \cs_set_protected:cpx { msg_ #1 :nnxxx } ##1##2##3##4##5
      { \exp_not:c { msg_ #1 :nnxxxx } {##1} {##2} {##3} {##4} {##5} { } }
    \cs_set_protected:cpx { msg_ #1 :nnxx } ##1##2##3##4
      { \exp_not:c { msg_ #1 :nnxxxx } {##1} {##2} {##3} {##4} { } { } }
    \cs_set_protected:cpx { msg_ #1 :nnx } ##1##2##3
      { \exp_not:c { msg_ #1 :nnxxxx } {##1} {##2} {##3} { } { } { } }
    \cs_set_protected:cpx { msg_ #1 :nn } ##1##2
      { \exp_not:c { msg_ #1 :nnxxxx } {##1} {##2} { } { } { } { } }
  }
%    \end{macrocode}
% \end{macro}
%
% \begin{macro}[aux]{\@@_class_chk_exist:nT}
%   Checking that a message class exists.  We build this from
%   \cs{cs_if_free:cTF} rather than \cs{cs_if_exist:cTF} because that
%   avoids reading the second argument earlier than necessary.
%    \begin{macrocode}
\cs_new:Npn \@@_class_chk_exist:nT #1
  {
    \cs_if_free:cTF { msg_ #1 _code:nnxxxx }
      { \@@_kernel_error:nnx { msg } { message-class-unknown } {#1} }
  }
%    \end{macrocode}
% \end{macro}
%
% \begin{macro}
%   {
%     \msg_fatal:nnxxxx ,
%     \msg_fatal:nnxxx  ,
%     \msg_fatal:nnxx   ,
%     \msg_fatal:nnx    ,
%     \msg_fatal:nn
%   }
%   For fatal errors, after the error message \TeX{} bails out.
%    \begin{macrocode}
\msg_class_set:nn { fatal }
  {
    \msg_interrupt:xxx
      { \msg_fatal_text:n {#1} : ~ "#2" }
      {
        \use:c { \c_@@_text_prefix_tl #1 / #2 } {#3} {#4} {#5} {#6}
        \msg_see_documentation_text:n {#1}
      }
      { \c_msg_fatal_text_tl }
    \tex_end:D
  }
%    \end{macrocode}
% \end{macro}
%
% \begin{macro}
%   {
%     \msg_critical:nnxxxx ,
%     \msg_critical:nnxxx  ,
%     \msg_critical:nnxx   ,
%     \msg_critical:nnx    ,
%     \msg_critical:nn
%   }
%   Not quite so bad: just end the current file.
%    \begin{macrocode}
\msg_class_set:nn { critical }
  {
    \msg_interrupt:xxx
      { \msg_critical_text:n {#1} : ~ "#2" }
      {
        \use:c { \c_@@_text_prefix_tl #1 / #2 } {#3} {#4} {#5} {#6}
        \msg_see_documentation_text:n {#1}
      }
      { \c_msg_critical_text_tl }
    \tex_endinput:D
  }
%    \end{macrocode}
% \end{macro}
%
% \begin{macro}
%   {
%     \msg_error:nnxxxx ,
%     \msg_error:nnxxx  ,
%     \msg_error:nnxx   ,
%     \msg_error:nnx    ,
%     \msg_error:nn
%   }
% \begin{macro}[aux]{\msg_error_aux:cxxxxn, \msg_no_more_text:xxxx}
%   For an error, the interrupt routine is called.  We check if there is
%   a \enquote{more text} by comparing that control sequence with a
%   permanently empty text.
%    \begin{macrocode}
\msg_class_set:nn { error }
  {
    \msg_error_aux:cxxxxn
      { \c_@@_more_text_prefix_tl #1 / #2 }
      {#3} {#4} {#5} {#6}
      {
        \msg_interrupt:xxx
          { \msg_error_text:n {#1} : ~ "#2" }
          {
            \use:c { \c_@@_text_prefix_tl #1 / #2 } {#3} {#4} {#5} {#6}
            \msg_see_documentation_text:n {#1}
          }
     }
  }
\cs_new_protected:Npn \msg_error_aux:cxxxxn #1#2#3#4#5#6
  {
    \cs_if_eq:cNTF {#1} \msg_no_more_text:xxxx
      { #6 { } }
      { #6 { \use:c {#1} {#2} {#3} {#4} {#5} } }
  }
\cs_new:Npn \msg_no_more_text:xxxx #1#2#3#4 { }
%    \end{macrocode}
% \end{macro}
% \end{macro}
%
% \begin{macro}
%   {
%     \msg_warning:nnxxxx ,
%     \msg_warning:nnxxx  ,
%     \msg_warning:nnxx   ,
%     \msg_warning:nnx    ,
%     \msg_warning:nn
%   }
%   Warnings are printed to the terminal.
%    \begin{macrocode}
\msg_class_set:nn { warning }
  {
    \msg_term:x
      {
        \msg_warning_text:n {#1} : ~ "#2" \\ \\
        \use:c { \c_@@_text_prefix_tl #1 / #2 } {#3} {#4} {#5} {#6}
      }
  }
%    \end{macrocode}
% \end{macro}
%
% \begin{macro}
%   {
%     \msg_info:nnxxxx ,
%     \msg_info:nnxxx  ,
%     \msg_info:nnxx   ,
%     \msg_info:nnx    ,
%     \msg_info:nn
%   }
%   Information only goes into the log.
%    \begin{macrocode}
\msg_class_set:nn { info }
  {
    \msg_log:x
      {
        \msg_info_text:n {#1} : ~ "#2" \\ \\
        \use:c { \c_@@_text_prefix_tl #1 / #2 } {#3} {#4} {#5} {#6}
      }
  }
%    \end{macrocode}
% \end{macro}
%
% \begin{macro}
%   {
%     \msg_log:nnxxxx ,
%     \msg_log:nnxxx  ,
%     \msg_log:nnxx   ,
%     \msg_log:nnx    ,
%     \msg_log:nn
%   }
%   \enquote{Log} data is very similar to information, but with no extras
%   added.
%    \begin{macrocode}
\msg_class_set:nn { log }
  {
    \msg_log:x
      { \use:c { \c_@@_text_prefix_tl #1 / #2 } {#3} {#4} {#5} {#6} }
  }
%    \end{macrocode}
% \end{macro}
%
% \begin{macro}
%   {
%     \msg_none:nnxxxx ,
%     \msg_none:nnxxx  ,
%     \msg_none:nnxx   ,
%     \msg_none:nnx    ,
%     \msg_none:nn
%   }
%   The \texttt{none} message type is needed so that input can be gobbled.
%    \begin{macrocode}
\msg_class_set:nn { none } { }
%    \end{macrocode}
% \end{macro}
%
% \begin{variable}{\l_@@_class_tl, \l_@@_current_class_tl}
%   Support variables needed for the redirection system.
%    \begin{macrocode}
\tl_new:N \l_@@_class_tl
\tl_new:N \l_@@_current_class_tl
%    \end{macrocode}
% \end{variable}
%
% \begin{variable}{\l_@@_redirect_prop}
%   For redirection of individually-named messages
%    \begin{macrocode}
\prop_new:N \l_@@_redirect_prop
%    \end{macrocode}
% \end{variable}
%
% \begin{variable}{\l_@@_hierarchy_seq}
%   During redirection, split the message name into a sequence with
%   items |{/module/submodule}|, |{/module}|, and |{}|.
%    \begin{macrocode}
\seq_new:N \l_@@_hierarchy_seq
%    \end{macrocode}
% \end{variable}
%
% \begin{variable}{\l_@@_class_loop_seq}
%   Classes encountered when following redirections to check for loops.
%    \begin{macrocode}
\seq_new:N \l_@@_class_loop_seq
%    \end{macrocode}
% \end{variable}
%
% \begin{macro}[int]{\@@_use:nnnxxxx}
% \begin{macro}[aux]
%   {
%     \@@_use_redirect_name:n  , \@@_use_hierarchy:nwwN ,
%     \@@_use_redirect_module:n, \@@_use_code:
%   }
%   Actually using a message is a multi-step process.  First, some
%   safety checks on the message and class requested.  The code and
%   arguments are then stored to avoid passing them around.  The
%   assignment to \cs{@@_use_code:} is similar to \cs{tl_set:Nn}.
%   The message is eventually produced with whatever \cs{l_@@_class_tl}
%   is when \cs{@@_use_code:} is called.
%    \begin{macrocode}
\cs_new_protected:Npn \@@_use:nnnxxxx #1#2#3#4#5#6#7
  {
    \msg_if_exist:nnTF {#2} {#3}
      {
        \@@_class_chk_exist:nT {#1}
          {
            \tl_set:Nn \l_@@_current_class_tl {#1}
            \cs_set_protected_nopar:Npx \@@_use_code:
              {
                \exp_not:n
                  {
                    \use:c { msg_ \l_@@_class_tl _code:nnxxxx }
                      {#2} {#3} {#4} {#5} {#6} {#7}
                  }
              }
            \@@_use_redirect_name:n { #2 / #3 }
          }
      }
      { \@@_kernel_error:nnxx { msg } { message-unknown } {#2} {#3} }
  }
\cs_new_protected_nopar:Npn \@@_use_code: { }
%    \end{macrocode}
%   The first check is for a individual message redirection. If this
%   applies then no further redirection is attempted.  Otherwise, split
%   the message name into \texttt{module/submodule/message} (with an
%   arbitrary number of slashes), and store |{/module/submodule}|,
%   |{/module}| and |{}| into \cs{l_@@_hierarchy_seq}.  We will then
%   map through this sequence, applying the most specific redirection.
%    \begin{macrocode}
\cs_new_protected:Npn \@@_use_redirect_name:n #1
  {
    \prop_get:NnNTF \l_@@_redirect_prop { / #1 } \l_@@_class_tl
      { \@@_use_code: }
      {
        \seq_clear:N \l_@@_hierarchy_seq
        \@@_use_hierarchy:nwwN { }
          #1 \q_mark \@@_use_hierarchy:nwwN
          /  \q_mark \use_none_delimit_by_q_stop:w
          \q_stop
        \@@_use_redirect_module:n { }
      }
  }
\cs_new_protected:Npn \@@_use_hierarchy:nwwN #1#2 / #3 \q_mark #4
  {
    \seq_put_left:Nn \l_@@_hierarchy_seq {#1}
    #4 { #1 / #2 } #3 \q_mark #4
  }
%    \end{macrocode}
%   At this point, the items of \cs{l_@@_hierarchy_seq} are the
%   various levels at which we should look for a redirection.
%   Redirections which are less specific than the argument of
%   \cs{@@_use_redirect_module:n} are not attempted.  This argument is
%   empty for a class redirection, \texttt{/module} for a module
%   redirection, \emph{etc.}  Loop through the sequence to find the most
%   specific redirection, with module |##1|.  The loop is interrupted
%   after testing for a redirection for |##1| equal to the argument |#1|
%   (least specific redirection allowed).  When a redirection is found,
%   break the mapping, then if the redirection targets the same class,
%   output the code with that class, and otherwise set the target as the
%   new current class, and search for further redirections.  Those
%   redirections should be at least as specific as |##1|.
%    \begin{macrocode}
\cs_new_protected:Npn \@@_use_redirect_module:n #1
  {
    \seq_map_inline:Nn \l_@@_hierarchy_seq
      {
        \prop_get:cnNTF { l_msg_redirect_ \l_@@_current_class_tl _prop }
          {##1} \l_@@_class_tl
          {
            \seq_map_break:n
              {
                \tl_if_eq:NNTF \l_@@_current_class_tl \l_@@_class_tl
                  { \@@_use_code: }
                  {
                    \tl_set_eq:NN \l_@@_current_class_tl \l_@@_class_tl
                    \@@_use_redirect_module:n {##1}
                  }
              }
          }
          {
            \str_if_eq:nnT {##1} {#1}
              {
                \tl_set_eq:NN \l_@@_class_tl \l_@@_current_class_tl
                \seq_map_break:n { \@@_use_code: }
              }
          }
      }
  }
%    \end{macrocode}
% \end{macro}
% \end{macro}
%
% \begin{macro}{\msg_redirect_name:nnn}
%   Named message will always use the given class even if that class is
%   redirected further.  An empty target class cancels any existing
%   redirection for that message.
%    \begin{macrocode}
\cs_new_protected:Npn \msg_redirect_name:nnn #1#2#3
  {
    \tl_if_empty:nTF {#3}
      { \prop_remove:Nn \l_@@_redirect_prop { / #1 / #2 } }
      {
        \@@_class_chk_exist:nT {#3}
          { \prop_put:Nnn \l_@@_redirect_prop { / #1 / #2 } {#3} }
      }
  }
%    \end{macrocode}
% \end{macro}
%
% \begin{macro}{\msg_redirect_class:nn, \msg_redirect_module:nnn}
% \begin{macro}[aux]{\@@_redirect:nnn, \@@_redirect_loop_chk:nnn}
%   If the target class is empty, eliminate the corresponding
%   redirection.  Otherwise, add the redirection.  We must then check
%   for a loop: as an initialization, we start by storing the initial
%   class in \cs{l_@@_current_class_tl}.
%    \begin{macrocode}
\cs_new_protected_nopar:Npn \msg_redirect_class:nn
  { \@@_redirect:nnn { } }
\cs_new_protected:Npn \msg_redirect_module:nnn #1
  { \@@_redirect:nnn { / #1 } }
\cs_new_protected:Npn \@@_redirect:nnn #1#2#3
  {
    \@@_class_chk_exist:nT {#2}
      {
        \tl_if_empty:nTF {#3}
          { \prop_remove:cn { l_msg_redirect_ #2 _prop } {#1} }
          {
            \@@_class_chk_exist:nT {#3}
              {
                \prop_put:cnn { l_msg_redirect_ #2 _prop } {#1} {#3}
                \tl_set:Nn \l_@@_current_class_tl {#2}
                \seq_clear:N \l_@@_class_loop_seq
                \@@_redirect_loop_chk:nnn {#2} {#3} {#1}
              }
          }
      }
  }
%    \end{macrocode}
%   Since multiple redirections can only happen with increasing
%   specificity, a loop requires that all steps are of the same
%   specificity.  The new redirection can thus only create a loop with
%   other redirections for the exact same module, |#1|, and not
%   submodules.  After some initialization above, follow redirections
%   with \cs{l_@@_class_tl}, and keep track in
%   \cs{l_@@_class_loop_seq} of the various classes encountered.  A
%   redirection from a class to itself, or the absence of redirection
%   both mean that there is no loop.  A redirection to the initial class
%   marks a loop.  To break it, we must decide which redirection to
%   cancel.  The user most likely wants the newly added redirection to
%   hold with no further redirection.  We thus remove the redirection
%   starting from |#2|, target of the new redirection.  Note that no
%   message is emitted by any of the underlying functions: otherwise we
%   may get an infinite loop because of a message from the message
%   system itself.
%    \begin{macrocode}
\cs_new_protected:Npn \@@_redirect_loop_chk:nnn #1#2#3
  {
    \seq_put_right:Nn \l_@@_class_loop_seq {#1}
    \prop_get:cnNT { l_msg_redirect_ #1 _prop } {#3} \l_@@_class_tl
      {
        \str_if_eq_x:nnF { \l_@@_class_tl } {#1}
          {
            \tl_if_eq:NNTF \l_@@_class_tl \l_@@_current_class_tl
              {
                \prop_put:cnn { l_msg_redirect_ #2 _prop } {#3} {#2}
                \@@_kernel_warning:nnxx { msg } { redirect-loop }
                  { \l_@@_class_loop_seq } {#3}
              }
              { \@@_redirect_loop_chk:onn \l_@@_class_tl {#2} {#3} }
          }
      }
  }
\cs_generate_variant:Nn \@@_redirect_loop_chk:nnn { o }
%    \end{macrocode}
% \end{macro}
% \end{macro}
%
% \subsection{Kernel-specific functions}
%
% \begin{macro}{\@@_kernel_new:nnnn, \@@_kernel_new:nnn}
% \begin{macro}{\@@_kernel_set:nnnn, \@@_kernel_set:nnn}
%   The kernel needs some messages of its own. These are created using
%   pre-built functions. Two functions are provided: one more general
%   and one which only has the short text part.
%    \begin{macrocode}
\cs_new_protected:Npn \@@_kernel_new:nnnn #1#2
  { \msg_new:nnnn { LaTeX } { #1 / #2 } }
\cs_new_protected:Npn \@@_kernel_new:nnn #1#2
  { \msg_new:nnn { LaTeX } { #1 / #2 } }
\cs_new_protected:Npn \@@_kernel_set:nnnn #1#2
  { \msg_set:nnnn { LaTeX } { #1 / #2 } }
\cs_new_protected:Npn \@@_kernel_set:nnn #1#2
  { \msg_set:nnn { LaTeX } { #1 / #2 } }
%    \end{macrocode}
% \end{macro}
% \end{macro}
%
% \begin{macro}[aux]{\@@_kernel_new_class:npn}
%   All the functions for kernel messages come in variants ranging
%   from~$0$ to~$4$ arguments.  Those with less than~$4$ arguments are
%   defined in terms of the $4$-argument variant, in a way very similar
%   to \cs{msg_class_set:nn}.  This auxiliary is destroyed at the end of
%   the group.
%    \begin{macrocode}
\group_begin:
  \cs_set_protected:Npn \@@_kernel_new_class:npn #1
    {
      \cs_new_protected:cpx { @@_ #1 :nnxxx } ##1##2##3##4##5
        { \exp_not:c { @@_ #1 :nnxxxx } {##1} {##2} {##3} {##4} {##5} { } }
      \cs_new_protected:cpx { @@_ #1 :nnxx } ##1##2##3##4
        { \exp_not:c { @@_ #1 :nnxxxx } {##1} {##2} {##3} {##4} { } { } }
      \cs_new_protected:cpx { @@_ #1 :nnx } ##1##2##3
        { \exp_not:c { @@_ #1 :nnxxxx } {##1} {##2} {##3} { } { } { } }
      \cs_new_protected:cpx { @@_ #1 :nn } ##1##2
        { \exp_not:c { @@_ #1 :nnxxxx } {##1} {##2} { } { } { } { } }
      \cs_new_protected:cpn { @@_ #1 :nnxxxx }
    }
%    \end{macrocode}
% \end{macro}
%
% \begin{macro}
%   {
%     \@@_kernel_fatal:nnxxxx ,
%     \@@_kernel_fatal:nnxxx  ,
%     \@@_kernel_fatal:nnxx   ,
%     \@@_kernel_fatal:nnx    ,
%     \@@_kernel_fatal:nn     ,
%     \@@_kernel_error:nnxxxx ,
%     \@@_kernel_error:nnxxx  ,
%     \@@_kernel_error:nnxx   ,
%     \@@_kernel_error:nnx    ,
%     \@@_kernel_error:nn     ,
%   }
%   Neither fatal kernel errors nor kernel errors can be redirected.  We
%   directly use the code for (non-kernel) fatal errors and errors,
%   adding the \enquote{\LaTeX{}} module name.  Three functions are
%   already defined by \pkg{l3basics}; we need to undefine them to avoid
%   errors.
%    \begin{macrocode}
  \@@_kernel_new_class:npn { kernel_fatal } #1#2
    { \msg_fatal_code:nnxxxx { LaTeX } { #1 / #2 } }
  \cs_undefine:N \@@_kernel_error:nnxx
  \cs_undefine:N \@@_kernel_error:nnx
  \cs_undefine:N \@@_kernel_error:nn
  \@@_kernel_new_class:npn { kernel_error } #1#2
    { \msg_error_code:nnxxxx { LaTeX } { #1 / #2 } }
%    \end{macrocode}
% \end{macro}
%
% \begin{macro}
%   {
%     \@@_kernel_warning:nnxxxx ,
%     \@@_kernel_warning:nnxxx  ,
%     \@@_kernel_warning:nnxx   ,
%     \@@_kernel_warning:nnx    ,
%     \@@_kernel_warning:nn     ,
%     \@@_kernel_info:nnxxxx    ,
%     \@@_kernel_info:nnxxx     ,
%     \@@_kernel_info:nnxx      ,
%     \@@_kernel_info:nnx       ,
%     \@@_kernel_info:nn        ,
%   }
%   Kernel messages which can be redirected simply use the machinery for
%   normal messages, with the module name \enquote{\LaTeX{}}.
%    \begin{macrocode}
  \@@_kernel_new_class:npn { kernel_warning } #1#2
    { \msg_warning:nnxxxx { LaTeX } { #1 / #2 } }
  \@@_kernel_new_class:npn { kernel_info } #1#2
    { \msg_info:nnxxxx { LaTeX } { #1 / #2 } }
%    \end{macrocode}
% \end{macro}
%
% End the group to eliminate \cs{@@_kernel_new_class:npn}.
%    \begin{macrocode}
\group_end:
%    \end{macrocode}
%
% Error messages needed to actually implement the message system
% itself.
%    \begin{macrocode}
\@@_kernel_new:nnnn { msg } { message-already-defined }
  { Message~'#2'~for~module~'#1'~already~defined. }
  {
    \c_msg_coding_error_text_tl
    LaTeX~was~asked~to~define~a~new~message~called~'#2'\
    by~the~module~'#1':~this~message~already~exists.
    \c_msg_return_text_tl
  }
\@@_kernel_new:nnnn { msg } { message-unknown }
  { Unknown~message~'#2'~for~module~'#1'. }
  {
    \c_msg_coding_error_text_tl
    LaTeX~was~asked~to~display~a~message~called~'#2'\\
    by~the~module~'#1':~this~message~does~not~exist.
    \c_msg_return_text_tl
  }
\@@_kernel_new:nnnn { msg } { message-class-unknown }
  { Unknown~message~class~'#1'. }
  {
    LaTeX~has~been~asked~to~redirect~messages~to~a~class~'#1':\\
    this~was~never~defined.
    \c_msg_return_text_tl
  }
%    \end{macrocode}
%
% \begin{macro}[aux]{\@@_loop:n}
%   For message redirection loops, the first argument is a sequence with
%   all classes involved in the loop.
%    \begin{macrocode}
\cs_new:Npn \@@_loop:n #1 { {#1} ~ => ~ }
\@@_kernel_new:nnnn { msg } { redirect-loop }
  {
    Message~redirection~loop~caused~by~
    { \seq_item:Nn #1 { 1 } } ~ => ~ { \seq_item:Nn #1 { 2 } }
    \tl_if_empty:nF {#2} { ~for~module~'\use_none:n #2' } .~
    Cancelling~
    { \seq_item:Nn #1 { 2 } } ~ => ~
    { \seq_item:Nn #1 { \int_mod:nn { 2 } { \seq_count:N #1 } + 1 } } .
  }
  {
    Adding~the~message~redirection~
    { \seq_item:Nn #1 { 1 } } ~ => ~ { \seq_item:Nn #1 { 2 } }
    \tl_if_empty:nF {#2} { ~for~the~module~'\use_none:n #2' } ~
    created~the~loop\\\\
      \iow_indent:n
        {
          \seq_map_function:NN #1 \@@_loop:n
          { \seq_item:Nn #1 { 1 } } \\\\
        }
    The~redirection~ { \seq_item:Nn #1 { 2 } } ~ => ~
    { \seq_item:Nn #1 { \int_mod:nn { 2 } { \seq_count:N #1 } + 1 } } ~
    was~overridden.
  }
%    \end{macrocode}
% \end{macro}
%
% Messages for earlier kernel modules.
%    \begin{macrocode}
\@@_kernel_new:nnnn { kernel } { bad-number-of-arguments }
  { Function~'#1'~cannot~be~defined~with~#2~arguments. }
  {
    \c_msg_coding_error_text_tl
    LaTeX~has~been~asked~to~define~a~function~'#1'~with~
    #2~arguments.~
    TeX~allows~between~0~and~9~arguments~for~a~single~function.
  }
\@@_kernel_new:nnnn { kernel } { command-already-defined }
  { Control~sequence~#1~already~defined. }
  {
    \c_msg_coding_error_text_tl
    LaTeX~has~been~asked~to~create~a~new~control~sequence~'#1'~
    but~this~name~has~already~been~used~elsewhere. \\ \\
    The~current~meaning~is:\\
    \ \ #2
  }
\@@_kernel_new:nnnn { kernel } { command-not-defined }
  { Control~sequence~#1~undefined. }
  {
    \c_msg_coding_error_text_tl
    LaTeX~has~been~asked~to~use~a~command~#1,~but~this~has~not~
    been~defined~yet.
  }
\@@_kernel_new:nnnn { kernel } { empty-search-pattern }
  { Empty~search~pattern. }
  {
    \c_msg_coding_error_text_tl
    LaTeX~has~been~asked~to~replace~an~empty~pattern~by~'#1':~that~
    would~lead~to~an~infinite~loop!
  }
\@@_kernel_new:nnnn { kernel } { out-of-registers }
  { No~room~for~a~new~#1. }
  {
    TeX~only~supports~\int_use:N \c_max_register_int \
    of~each~type.~All~the~#1~registers~have~been~used.~
    This~run~will~be~aborted~now.
  }
\@@_kernel_new:nnnn { kernel } { missing-colon }
  { Function~'#1'~contains~no~':'. }
  {
    \c_msg_coding_error_text_tl
    Code-level~functions~must~contain~':'~to~separate~the~
    argument~specification~from~the~function~name.~This~is~
    needed~when~defining~conditionals~or~when~building~a~
    parameter~text~from~the~number~of~arguments~of~the~function.
  }
\@@_kernel_new:nnnn { kernel } { protected-predicate }
  { Predicate~'#1'~must~be~expandable. }
  {
    \c_msg_coding_error_text_tl
    LaTeX~has~been~asked~to~define~'#1'~as~a~protected~predicate.~
    Only~expandable~functions~can~appear~in~the~definition~of~predicates.
  }
\@@_kernel_new:nnnn { kernel } { scanmark-already-defined }
  { Scan~mark~#1~already~defined. }
  {
    \c_msg_coding_error_text_tl
    LaTeX~has~been~asked~to~create~a~new~scan~mark~'#1'~
    but~this~name~has~already~been~used~for~a~scan~mark.
  }
\@@_kernel_new:nnnn { kernel } { variable-not-defined }
  { Variable~#1~undefined. }
  {
    \c_msg_coding_error_text_tl
    LaTeX~has~been~asked~to~show~a~variable~#1,~but~this~has~not~
    been~defined~yet.
  }
%    \end{macrocode}
%
% Some errors only appear in expandable settings,
% hence don't need a \enquote{more-text} argument.
%    \begin{macrocode}
\@@_kernel_new:nnn { kernel } { bad-variable }
  { Erroneous~variable~#1 used! }
\@@_kernel_new:nnn { kernel } { misused-sequence }
  { A~sequence~was~misused. }
\@@_kernel_new:nnn { kernel } { negative-replication }
  { Negative~argument~for~\prg_replicate:nn. }
\@@_kernel_new:nnn { kernel } { unknown-comparison }
  { Relation~symbol~'#1'~unknown:~use~=,~<,~>,~==,~!=,~<=,~>=. }
\@@_kernel_new:nnn { kernel } { zero-step }
  { Zero~step~size~for~step~function~#1. }
%    \end{macrocode}
%
% Messages used by the \enquote{\texttt{show}} functions.
%    \begin{macrocode}
\@@_kernel_new:nnn { kernel } { show-clist }
  {
    The~comma~list~
    \str_if_eq:nnF {#1} { \l__clist_internal_clist } { \token_to_str:N #1~}
    \clist_if_empty:NTF #1
      { is~empty }
      { contains~the~items~(without~outer~braces): }
  }
\@@_kernel_new:nnn { kernel } { show-prop }
  {
    The~property~list~\token_to_str:N #1~
    \prop_if_empty:NTF #1
      { is~empty }
      { contains~the~pairs~(without~outer~braces): }
  }
\@@_kernel_new:nnn { kernel } { show-seq }
  {
    The~sequence~\token_to_str:N #1~
    \seq_if_empty:NTF #1
      { is~empty }
      { contains~the~items~(without~outer~braces): }
  }
\@@_kernel_new:nnn { kernel } { show-no-stream }
  { No~ #1 ~streams~are~open }
\@@_kernel_new:nnn { kernel } { show-open-streams }
  { The~following~ #1 ~streams~are~in~use: }
%    \end{macrocode}
%
% \subsection{Expandable errors}
%
% \begin{macro}[int]{\@@_expandable_error:n}
% \begin{macro}[aux]{\@@_expandable_error:w}
%   In expansion only context, we cannot use the normal means of
%   reporting errors. Instead, we feed \TeX{} an undefined control
%   sequence, \cs{LaTeX3 error:}. It is thus interrupted, and shows
%   the context, which thanks to the odd-looking \cs{use:n} is
%   \begin{verbatim}
%     <argument> \LaTeX3 error:
%                               The error message.
%   \end{verbatim}
%   In other words, \TeX{} is processing the argument of \cs{use:n},
%   which is \cs{LaTeX3 error:} \meta{error message}.
%   Then \cs{@@_expandable_error:w} cleans up. In fact, there
%   is an extra subtlety: if the user inserts tokens for error recovery,
%   they should be kept. Thus we also use an odd space character
%   (with category code $7$) and keep tokens until that space character,
%   dropping everything else until \cs{q_stop}. The \cs{c_zero} prevents
%   losing braces around the user-inserted text if any, and stops the
%   expansion of \tn{romannumeral}.
%    \begin{macrocode}
\group_begin:
\char_set_catcode_math_superscript:N \^
\char_set_lccode:nn { `^ } { `\  }
\char_set_lccode:nn { `L } { `L }
\char_set_lccode:nn { `T } { `T }
\char_set_lccode:nn { `X } { `X }
\tl_to_lowercase:n
  {
    \cs_new:Npx \@@_expandable_error:n #1
      {
        \exp_not:n
          {
            \tex_romannumeral:D
            \exp_after:wN \exp_after:wN
            \exp_after:wN \@@_expandable_error:w
            \exp_after:wN \exp_after:wN
            \exp_after:wN \c_zero
          }
        \exp_not:N \use:n { \exp_not:c { LaTeX3~error: } ^ #1 } ^
      }
    \cs_new:Npn \@@_expandable_error:w #1 ^ #2 ^ { #1 }
  }
\group_end:
%    \end{macrocode}
% \end{macro}
% \end{macro}
%
% \begin{macro}
%   {
%     \@@_kernel_expandable_error:nnnnnn,
%     \@@_kernel_expandable_error:nnnnn,
%     \@@_kernel_expandable_error:nnnn,
%     \@@_kernel_expandable_error:nnn,
%     \@@_kernel_expandable_error:nn
%   }
%   The command built from the csname
%   |\c_@@_text_prefix_tl LaTeX / #1 / #2|
%   takes four arguments and builds the error text, which is fed to
%   \cs{@@_expandable_error:n}.
%    \begin{macrocode}
\cs_new:Npn \@@_kernel_expandable_error:nnnnnn #1#2#3#4#5#6
  {
    \exp_args:Nf \@@_expandable_error:n
      {
        \exp_args:NNc \exp_after:wN \exp_stop_f:
          { \c_@@_text_prefix_tl LaTeX / #1 / #2 }
          {#3} {#4} {#5} {#6}
      }
  }
\cs_new:Npn \@@_kernel_expandable_error:nnnnn #1#2#3#4#5
  {
    \@@_kernel_expandable_error:nnnnnn
      {#1} {#2} {#3} {#4} {#5} { }
  }
\cs_new:Npn \@@_kernel_expandable_error:nnnn #1#2#3#4
  {
    \@@_kernel_expandable_error:nnnnnn
      {#1} {#2} {#3} {#4} { } { }
  }
\cs_new:Npn \@@_kernel_expandable_error:nnn #1#2#3
  {
    \@@_kernel_expandable_error:nnnnnn
      {#1} {#2} {#3} { } { } { }
  }
\cs_new:Npn \@@_kernel_expandable_error:nn #1#2
  {
    \@@_kernel_expandable_error:nnnnnn
      {#1} {#2} { } { } { } { }
  }
%    \end{macrocode}
% \end{macro}
%
% \subsection{Showing variables}
%
% Functions defined in this section are used for diagnostic functions
% in \pkg{l3clist}, \pkg{l3file}, \pkg{l3prop}, \pkg{l3seq}, \pkg{xtemplate}
%
% \begin{macro}[int]{\@@_aux_use:nn}
% \begin{macro}[int]{\@@_aux_use:nnxxxx}
%   Print the text of a message to the terminal, without formatting.
%    \begin{macrocode}
\cs_new_protected:Npn \@@_aux_use:nn #1#2
  { \@@_aux_use:nnxxxx {#1} {#2} { } { } { } { } }
\cs_new_protected:Npn \@@_aux_use:nnxxxx #1#2#3#4#5#6
  {
    \iow_wrap:nnnN
      {
        \use:c { \c_@@_text_prefix_tl #1 / #2 }
          {#3} {#4} {#5} {#6}
      }
      { } { } \iow_term:x
  }
%    \end{macrocode}
% \end{macro}
% \end{macro}
%
% \begin{macro}[int]{\@@_aux_show:Nnx}
% \begin{macro}[int]{\@@_aux_show:x}
% \begin{macro}[aux,EXP]{\@@_aux_show:w}
%   The arguments of \cs{@@_aux_show:Nnx} are
%   \begin{itemize}
%   \item The \meta{variable} to be shown.
%   \item The \texttt{TF} emptiness conditional for that type of variables.
%   \item The type of the variable.
%   \item A mapping of the form \cs{seq_map_function:NN} \meta{variable}
%     \cs{@@_aux_show:n}, which produces the formatted string.
%   \end{itemize}
%   We remove a new line and \verb*|> | from the first item using
%   a \texttt{w}-type auxiliary, and the fact that \texttt{f}-expansion
%   removes a space. To avoid a low-level \TeX{} error if there is
%   an empty argument, a simple test is used to keep the output
%   \enquote{clean}. The odd \cs{exp_after:wN} which expands the closing
%   brace improves the output slightly.
%    \begin{macrocode}
\cs_new_protected:Npn \@@_aux_show:Nnx #1#2#3
  {
    \cs_if_exist:NTF #1
      {
        \@@_aux_use:nnxxxx { LaTeX / kernel } { show- #2 } {#1} { } { } { }
        \@@_aux_show:x {#3}
      }
      {
        \@@_kernel_error:nnx { kernel } { variable-not-defined }
          { \token_to_str:N #1 }
      }
  }
\cs_new_protected:Npn \@@_aux_show:x #1
  {
    \tl_set:Nx \l_@@_internal_tl {#1}
    \tl_if_empty:NTF \l_@@_internal_tl
      { \etex_showtokens:D \exp_after:wN { } }
      {
        \exp_args:Nf \etex_showtokens:D
          {
            \exp_after:wN \exp_after:wN
            \exp_after:wN \@@_aux_show:w
            \exp_after:wN \l_@@_internal_tl
          }
      }
  }
\cs_new:Npn \@@_aux_show:w #1 > { }
%    \end{macrocode}
% \end{macro}
% \end{macro}
% \end{macro}
%
% \begin{macro}[aux,EXP]{\@@_aux_show:n}
% \begin{macro}[aux,EXP]{\@@_aux_show:nn}
% \begin{macro}[aux,EXP]{\@@_aux_show_unbraced:nn}
%   Each item in the variable is formatted using one of
%   the following functions.
%    \begin{macrocode}
\cs_new:Npn \@@_aux_show:n #1
  {
    \iow_newline: > \c_space_tl \c_space_tl { \exp_not:n {#1} }
  }
\cs_new:Npn \@@_aux_show:nn #1#2
  {
    \iow_newline: > \c_space_tl \c_space_tl { \exp_not:n {#1} }
    \c_space_tl \c_space_tl => \c_space_tl \c_space_tl { \exp_not:n {#2} }
  }
\cs_new:Npn \@@_aux_show_unbraced:nn #1#2
  {
    \iow_newline: > \c_space_tl \c_space_tl \exp_not:n {#1}
    \c_space_tl \c_space_tl => \c_space_tl \c_space_tl \exp_not:n {#2}
  }
%    \end{macrocode}
% \end{macro}
% \end{macro}
% \end{macro}
%
% \subsection{Deprecated functions}
%
% Deprecated on 2011-05-27, for removal by 2011-08-31.
%
% \begin{macro}{\msg_class_new:nn}
% This is only ever used in a |set| fashion.
%    \begin{macrocode}
%<*deprecated>
\cs_new_eq:NN \msg_class_new:nn \msg_class_set:nn
%</deprecated>
%    \end{macrocode}
% \end{macro}
%
% \begin{macro}
%   {
%     \msg_trace:nnxxxx, \msg_trace:nnxxx, \msg_trace:nnxx,
%     \msg_trace:nnx,    \msg_trace:nn
%   }
%   The performance here is never going to be good enough for tracing
%   code, so let's be realistic.
%    \begin{macrocode}
%<*deprecated>
\cs_new_eq:NN \msg_trace:nnxxxx \msg_log:nnxxxx
\cs_new_eq:NN \msg_trace:nnxxx  \msg_log:nnxxx
\cs_new_eq:NN \msg_trace:nnxx   \msg_log:nnxx
\cs_new_eq:NN \msg_trace:nnx    \msg_log:nnx
\cs_new_eq:NN \msg_trace:nn     \msg_log:nn
%</deprecated>
%    \end{macrocode}
%\end{macro}
%
% \begin{macro}{\msg_generic_new:nnn}
% \begin{macro}{\msg_generic_new:nn}
% \begin{macro}{\msg_generic_set:nnn}
% \begin{macro}{\msg_generic_set:nn}
% \begin{macro}{\msg_direct_interrupt:xxxxx}
% \begin{macro}{\msg_direct_log:xx}
% \begin{macro}{\msg_direct_term:xx}
%  These were all too low-level.
%    \begin{macrocode}
%<*deprecated>
\cs_new_protected:Npn \msg_generic_new:nnn #1#2#3 { \deprecated }
\cs_new_protected:Npn \msg_generic_new:nn  #1#2   { \deprecated }
\cs_new_protected:Npn \msg_generic_set:nnn #1#2#3 { \deprecated }
\cs_new_protected:Npn \msg_generic_set:nn  #1#2   { \deprecated }
\cs_new_protected:Npn \msg_direct_interrupt:xxxxx #1#2#3#4#5 { \deprecated }
\cs_new_protected:Npn \msg_direct_log:xx #1#2  { \deprecated }
\cs_new_protected:Npn \msg_direct_term:xx #1#2 { \deprecated }
%</deprecated>
%    \end{macrocode}
% \end{macro}
% \end{macro}
% \end{macro}
% \end{macro}
% \end{macro}
% \end{macro}
% \end{macro}
%
% \begin{macro}{\@@_kernel_bug:x}
% \begin{variable}{\c_@@_kernel_bug_text_tl, \c_@@_kernel_bug_more_text_tl}
%    \begin{macrocode}
%<*deprecated>
\cs_set_protected:Npn \@@_kernel_bug:x #1
  {
    \msg_interrupt:xxx { \c_@@_kernel_bug_text_tl }
      {
        #1
       \msg_see_documentation_text:n { LaTeX3 }
      }
      { \c_@@_kernel_bug_more_text_tl }
  }
\tl_const:Nn \c_@@_kernel_bug_text_tl
  { This~is~a~LaTeX~bug:~check~coding! }
\tl_const:Nn \c_@@_kernel_bug_more_text_tl
  {
    There~is~a~coding~bug~somewhere~around~here. \\
    This~probably~needs~examining~by~an~expert.
    \c_msg_return_text_tl
  }
%</deprecated>
%    \end{macrocode}
% \end{variable}
% \end{macro}
%
%    \begin{macrocode}
%</initex|package>
%    \end{macrocode}
%
% \end{implementation}
%
% \PrintIndex

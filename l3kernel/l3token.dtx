% \iffalse meta-comment
%
%% File: l3token.dtx Copyright (C) 2005-2012 The LaTeX3 Project
%%
%% It may be distributed and/or modified under the conditions of the
%% LaTeX Project Public License (LPPL), either version 1.3c of this
%% license or (at your option) any later version.  The latest version
%% of this license is in the file
%%
%%    http://www.latex-project.org/lppl.txt
%%
%% This file is part of the "l3kernel bundle" (The Work in LPPL)
%% and all files in that bundle must be distributed together.
%%
%% The released version of this bundle is available from CTAN.
%%
%% -----------------------------------------------------------------------
%%
%% The development version of the bundle can be found at
%%
%%    http://www.latex-project.org/svnroot/experimental/trunk/
%%
%% for those people who are interested.
%%
%%%%%%%%%%%
%% NOTE: %%
%%%%%%%%%%%
%%
%%   Snapshots taken from the repository represent work in progress and may
%%   not work or may contain conflicting material!  We therefore ask
%%   people _not_ to put them into distributions, archives, etc. without
%%   prior consultation with the LaTeX3 Project.
%%
%% -----------------------------------------------------------------------
%
%<*driver|package>
\RequirePackage{l3names}
\GetIdInfo$Id$
  {L3 Experimental token manipulation}
%</driver|package>
%<*driver>
\documentclass[full]{l3doc}
\begin{document}
  \DocInput{\jobname.dtx}
\end{document}
%</driver>
% \fi
%
% \title{^^A
%   The \pkg{l3token} package\\ Token manipulation^^A
%   \thanks{This file describes v\ExplFileVersion,
%      last revised \ExplFileDate.}^^A
% }
%
% \author{^^A
%  The \LaTeX3 Project\thanks
%    {^^A
%      E-mail:
%        \href{mailto:latex-team@latex-project.org}
%          {latex-team@latex-project.org}^^A
%    }^^A
% }
%
% \date{Released \ExplFileDate}
%
% \maketitle
%
% \begin{documentation}
%
% This module deals with tokens. Now this is perhaps not the most
% precise description so let's try with a better description: When
% programming in \TeX{}, it is often desirable to know just what a
% certain token is: is it a control sequence or something
% else. Similarly one often needs to know if a control sequence is
% expandable or not, a macro or a primitive, how many arguments it
% takes etc. Another thing of great importance (especially when it
% comes to document commands) is looking ahead in the token stream to
% see if a certain character is present and maybe even remove it or
% disregard other tokens while scanning. This module provides
% functions for both and as such will have two primary function
% categories: |\token| for anything that deals with tokens and
% |\peek| for looking ahead in the token stream.
%
% Most of the time we will be using the term \enquote{token} but most of the
% time the function we're describing can equally well by used on a
% control sequence as such one is one token as well.
%
% We shall refer to list of tokens as |tlist|s and such lists
% represented by a single control sequence is a \enquote{token list variable}
% |tl var|. Functions for these two types are found in the \textsf{l3tl}
% module.
%
% \section{All possible tokens}
%
% Let us start by reviewing every case that a given token can fall into.
% It is very important to distinguish two aspects of a token: its meaning,
% and what it looks like.
%
% For instance, \cs{if:w}, \cs{if_charcode:w}, and \cs{tex_if:D} are
% three for the same internal operation of \TeX{}, namely the primitive
% testing the next two characters for equality of their character code.
% They behave identically in many situations. However, \TeX{}
% distinguishes them when searching for a delimited argument. Namely, the
% example function \cs{show_until_if:w} defined below will take everything
% until \cs{if:w} as an argument, despite the presence of other copies of
% \cs{if:w} under different names.
% \begin{verbatim}
% \cs_new:Npn \show_until_if:w #1 \if:w { \tl_show:n {#1} }
% \show_until_if:w \tex_if:D \if_charcode:w \if:w
% \end{verbatim}
%
% \section{Character tokens}
%
% \begin{function}
%   {
%     \char_set_catcode_escape:N           ,
%     \char_set_catcode_group_begin:N      ,
%     \char_set_catcode_group_end:N        ,
%     \char_set_catcode_math_toggle:N      ,
%     \char_set_catcode_alignment:N        ,
%     \char_set_catcode_end_line:N         ,
%     \char_set_catcode_parameter:N        ,
%     \char_set_catcode_math_superscript:N ,
%     \char_set_catcode_math_subscript:N   ,
%     \char_set_catcode_ignore:N           ,
%     \char_set_catcode_space:N            ,
%     \char_set_catcode_letter:N           ,
%     \char_set_catcode_other:N            ,
%     \char_set_catcode_active:N           ,
%     \char_set_catcode_comment:N          ,
%     \char_set_catcode_invalid:N
%   }
%   \begin{syntax}
%     \cs{char_set_catcode_letter:N} \meta{character}
%   \end{syntax}
%   Sets the category code of the \meta{character} to that indicated in
%   the function name. Depending on the current category code of the
%   \meta{token} the escape token may also be needed:
%   \begin{verbatim}
%     \char_set_catcode_other:N \%
%   \end{verbatim}
%   The assignment is local.
% \end{function}
%
% \begin{function}
%   {
%     \char_set_catcode_escape:n           ,
%     \char_set_catcode_group_begin:n      ,
%     \char_set_catcode_group_end:n        ,
%     \char_set_catcode_math_toggle:n       ,
%     \char_set_catcode_alignment:n        ,
%     \char_set_catcode_end_line:n         ,
%     \char_set_catcode_parameter:n        ,
%     \char_set_catcode_math_superscript:n ,
%     \char_set_catcode_math_subscript:n   ,
%     \char_set_catcode_ignore:n           ,
%     \char_set_catcode_space:n            ,
%     \char_set_catcode_letter:n           ,
%     \char_set_catcode_other:n            ,
%     \char_set_catcode_active:n           ,
%     \char_set_catcode_comment:n          ,
%     \char_set_catcode_invalid:n
%   }
%   \begin{syntax}
%     \cs{char_set_catcode_letter:n} \Arg{integer expression}
%   \end{syntax}
%   Sets the category code of the \meta{character} which has character
%   code as given by the \meta{integer expression}. This version can be
%   used to set up characters which cannot otherwise be given
%   (\emph{cf.}~the \texttt{N}-type variants). The assignment is local.
% \end{function}
%
% \begin{function}{\char_set_catcode:nn}
%   \begin{syntax}
%     \cs{char_set_catcode:nn} \Arg{intexpr_1} \Arg{intexpr_2}
%   \end{syntax}
%   These functions set the category code of the \meta{character} which
%   has character code as given by the \meta{integer expression}.
%   The first \meta{integer expression}
%   is the character code and the second is the category code to apply.
%   The setting applies within the current \TeX{} group. In general, the
%   symbolic functions \cs{char_set_catcode_\meta{type}} should be preferred,
%   but there are cases where these lower-level functions may be useful.
% \end{function}
%
% \begin{function}[EXP]{\char_value_catcode:n}
%   \begin{syntax}
%     \cs{char_value_catcode:n} \Arg{integer expression}
%   \end{syntax}
%   Expands to the current category code of the \meta{character} with
%   character code given by the
%   \meta{integer expression}.
% \end{function}
%
% \begin{function}{\char_show_value_catcode:n}
%   \begin{syntax}
%     \cs{char_show_value_catcode:n} \Arg{integer expression}
%   \end{syntax}
%   Displays the current category code of the \meta{character} with
%   character code given by the  \meta{integer expression} on the
%   terminal.
% \end{function}
%
% \begin{function}{\char_set_lccode:nn}
%   \begin{syntax}
%     \cs{char_set_lcode:nn} \Arg{intexpr_1} \Arg{intexpr_2}
%   \end{syntax}
%   This function set up the behaviour of \meta{character} when
%   found inside \cs{tl_to_lowercase:n}, such that \meta{character1}
%   will be converted into \meta{character2}. The two \meta{characters}
%   may be specified using an \meta{integer expression} for the character code
%   concerned. This may include the \TeX{} |`|\meta{character}
%   method for converting a single character into its character
%   code:
%   \begin{verbatim}
%     \char_set_lccode:nn { `\A } { `\a } % Standard behaviour
%     \char_set_lccode:nn { `\A } { `\A + 32 }
%     \char_set_lccode:nn { 50 } { 60 }
%   \end{verbatim}
%   The setting applies within the current \TeX{} group.
% \end{function}
%
% \begin{function}[EXP]{\char_value_lccode:n}
%   \begin{syntax}
%     \cs{char_value_lccode:n} \Arg{integer expression}
%   \end{syntax}
%   Expands to the current lower case code of the \meta{character} with
%   character code given by the
%   \meta{integer expression}.
% \end{function}
%
% \begin{function}{\char_show_value_lccode:n}
%   \begin{syntax}
%     \cs{char_show_value_lccode:n} \Arg{integer expression}
%   \end{syntax}
%   Displays the current lower case code of the \meta{character} with
%   character code given by the  \meta{integer expression} on the
%   terminal.
% \end{function}
%
% \begin{function}{\char_set_uccode:nn}
%   \begin{syntax}
%     \cs{char_set_uccode:nn} \Arg{intexpr_1} \Arg{intexpr_2}
%   \end{syntax}
%   This function set up the behaviour of \meta{character} when
%   found inside \cs{tl_to_uppercase:n}, such that \meta{character1}
%   will be converted into \meta{character2}. The two \meta{characters}
%   may be specified using an \meta{integer expression} for the character code
%   concerned. This may include the \TeX{} |`|\meta{character}
%   method for converting a single character into its character
%   code:
%   \begin{verbatim}
%     \char_set_uccode:nn { `\a } { `\A } % Standard behaviour
%     \char_set_uccode:nn { `\A } { `\A - 32 }
%     \char_set_uccode:nn { 60 } { 50 }
%   \end{verbatim}
%   The setting applies within the current \TeX{} group.
% \end{function}
%
% \begin{function}[EXP]{\char_value_uccode:n}
%   \begin{syntax}
%     \cs{char_value_uccode:n} \Arg{integer expression}
%   \end{syntax}
%   Expands to the current upper case code of the \meta{character} with
%   character code given by the
%   \meta{integer expression}.
% \end{function}
%
% \begin{function}{\char_show_value_uccode:n}
%   \begin{syntax}
%     \cs{char_show_value_uccode:n} \Arg{integer expression}
%   \end{syntax}
%   Displays the current upper case code of the \meta{character} with
%   character code given by the  \meta{integer expression} on the
%   terminal.
% \end{function}
%
% \begin{function}{\char_set_mathcode:nn}
%   \begin{syntax}
%     \cs{char_set_mathcode:nn} \Arg{intexpr_1} \Arg{intexpr_2}
%   \end{syntax}
%   This function sets up the math code of \meta{character}.
%   The \meta{character} is specified as
%   an \meta{integer expression} which will be used as the character
%   code of the relevant character. The setting applies within the
%   current \TeX{} group.
% \end{function}
%
% \begin{function}[EXP]{\char_value_mathcode:n}
%   \begin{syntax}
%     \cs{char_value_mathcode:n} \Arg{integer expression}
%   \end{syntax}
%   Expands to the current math code of the \meta{character} with
%   character code given by the
%   \meta{integer expression}.
% \end{function}
%
% \begin{function}{\char_show_value_mathcode:n}
%   \begin{syntax}
%     \cs{char_show_value_mathcode:n} \Arg{integer expression}
%   \end{syntax}
%   Displays the current math code of the \meta{character} with
%   character code given by the  \meta{integer expression} on the
%   terminal.
% \end{function}
%
% \begin{function}{\char_set_sfcode:nn}
%   \begin{syntax}
%     \cs{char_set_sfcode:nn} \Arg{intexpr_1} \Arg{intexpr_2}
%   \end{syntax}
%   This function sets up the space factor for the \meta{character}.
%   The \meta{character} is specified as
%   an \meta{integer expression} which will be used as the character
%   code of the relevant character. The setting applies within the
%   current \TeX{} group.
% \end{function}
%
% \begin{function}[EXP]{\char_value_sfcode:n}
%   \begin{syntax}
%     \cs{char_value_sfcode:n} \Arg{integer expression}
%   \end{syntax}
%   Expands to the current space factor for the \meta{character} with
%   character code given by the
%   \meta{integer expression}.
% \end{function}
%
% \begin{function}{\char_show_value_sfcode:n}
%   \begin{syntax}
%     \cs{char_show_value_sfcode:n} \Arg{integer expression}
%   \end{syntax}
%   Displays the current space factor for the \meta{character} with
%   character code given by the  \meta{integer expression} on the
%   terminal.
% \end{function}
%
% \begin{variable}[added = 2012-01-23]{\l_char_active_seq}
%   Used to track which tokens will require special handling at the document
%   level as they are of category \meta{active} (catcode~$13$). Each entry in
%   the sequence consists of a single active character. Active tokens should be
%   added to the sequence when they are defined for general document use.
% \end{variable}
%
% \begin{variable}[added = 2012-01-23]{\l_char_special_seq}
%   Used to track which tokens will require special handling when working with
%   verbatim-like material at the document level as they are not of categories
%   \meta{letter} (catcode~$11$) or \meta{other} (catcode~$12$). Each entry in
%   the sequence consists of a single escaped token, for example |\\| for the
%   backslash or |\{| for an opening brace.^^A \}
%   Escaped tokens should be added to the sequence when they are defined for
%   general document use.
% \end{variable}
%
% \section{Generic tokens}
%
% \begin{function}{\token_new:Nn}
%   \begin{syntax}
%     \cs{token_new:Nn} \meta{token1} \Arg{token_2}
%   \end{syntax}
%   Defines \meta{token1} to globally be a snapshot of \meta{token2}.
%   This will be an implicit representation of \meta{token2}.
% \end{function}
%
% ^^A Because it's late I can't figure out a better way to handle this:
% \ExplSyntaxOn
%   \cs_set_eq:NN \c_alignment_token @
%   \cs_set_eq:NN \c_parameter_token @
% \ExplSyntaxOff
% \begin{variable}
%   {
%     \c_group_begin_token,
%     \c_group_end_token,
%     \c_math_toggle_token,
%     \c_alignment_token,
%     \c_parameter_token,
%     \c_math_superscript_token,
%     \c_math_subscript_token,
%     \c_space_token
%   }
%   These are implicit tokens which have the category code described
%   by their name. They are used internally for test purposes but
%   are also available to the programmer for other uses.
% \end{variable}
% \ExplSyntaxOn
%   \cs_set_eq:NN \c_alignment_token &
%   \cs_set_eq:NN \c_parameter_token #
% \ExplSyntaxOff
%
% \begin{variable}
%   {
%     \c_catcode_letter_token,
%     \c_catcode_other_token
%   }
%   These are implicit tokens which have the category code described
%   by their name. They are used internally for test purposes and should
%   not be used other than for category code tests.
% \end{variable}
%
% \begin{variable}{\c_catcode_active_tl}
%   A token list containing an active token. This is used internally
%   for test purposes and should not be used other than in
%   appropriately-constructed category code tests.
% \end{variable}
%
% \section{Converting tokens}
%
% \begin{function}[EXP]{\token_to_meaning:N}
%   \begin{syntax}
%     \cs{token_to_meaning:N} \meta{token}
%   \end{syntax}
%   Inserts the current meaning of the \meta{token} into the input
%   stream as a series of characters of category code $12$ (other).
%   This will be the primitive \TeX{} description of the \meta{token},
%   thus for example both functions defined by \cs{cs_set_nopar:Npn}
%   and token list variables defined using \cs{tl_new:N} will be described
%   as |macro|s.
%   \begin{texnote}
%     This is the \TeX{} primitive \tn{meaning}.
%   \end{texnote}
% \end{function}
%
% \begin{function}[EXP]{\token_to_str:N, \token_to_str:c}
%   \begin{syntax}
%     \cs{token_to_str:N} \meta{token}
%   \end{syntax}
%   Converts the given \meta{token} into a series of characters with
%   category code $12$ (other). The current escape character will be
%   the first character in the sequence, although this will also have
%   category code $12$ (the escape character is part of the
%   \meta{token}). This function requires only a single expansion.
%   \begin{texnote}
%     \cs{token_to_str:N} is the \TeX{} primitive \tn{string} renamed.
%   \end{texnote}
% \end{function}
%
% \section{Token conditionals}
%
% \begin{function}[EXP,pTF]{\token_if_group_begin:N}
%   \begin{syntax}
%     \cs{token_if_group_begin_p:N} \meta{token} \\
%     \cs{token_if_group_begin:NTF} \meta{token} \Arg{true code} \Arg{false code}
%   \end{syntax}
%   Tests if \meta{token} has the category code of a begin group token
%   (|{| when normal \TeX{} category codes are in ^^A }
%   force).
%   Note that an explicit begin group token cannot be tested in this way,
%   as it is not a valid \texttt{N}-type argument.
% \end{function}
%
% \begin{function}[EXP,pTF]{\token_if_group_end:N}
%   \begin{syntax}
%     \cs{token_if_group_end_p:N} \meta{token} \\
%     \cs{token_if_group_end:NTF} \meta{token} \Arg{true code} \Arg{false code}
%    \end{syntax}
%   Tests if \meta{token} has the category code of an end group token
%   (^^A {
%   |}| when normal \TeX{} category codes are in force).
%   Note that an explicit end group token cannot be tested in this way,
%   as it is not a valid \texttt{N}-type argument.
% \end{function}
%
% \begin{function}[EXP,pTF]{\token_if_math_toggle:N}
%   \begin{syntax}
%     \cs{token_if_math_toggle_p:N} \meta{token} \\
%     \cs{token_if_math_toggle:NTF} \meta{token} \Arg{true code} \Arg{false code}
%   \end{syntax}
%   Tests if \meta{token} has the category code of a math shift token
%   (|$| when normal \TeX{} category codes are in force).
% \end{function}
%
% \begin{function}[EXP,pTF]{\token_if_alignment:N}
%   \begin{syntax}
%     \cs{token_if_alignment_p:N} \meta{token} \\
%     \cs{token_if_alignment:NTF} \meta{token} \Arg{true code} \Arg{false code}
%   \end{syntax}
%   Tests if \meta{token} has the category code of an alignment token
%   (|&| when normal \TeX{} category codes are in force).
% \end{function}
%
% \begin{function}[EXP,pTF]{\token_if_parameter:N}
%   \begin{syntax}
%     \cs{token_if_parameter_p:N} \meta{token} \\
%     \cs{token_if_alignment:NTF} \meta{token} \Arg{true code} \Arg{false code}
%   \end{syntax}
%   Tests if \meta{token} has the category code of a macro parameter token
%   (|#| when normal \TeX{} category codes are in force).
% \end{function}
%
% \begin{function}[EXP,pTF]{\token_if_math_superscript:N}
%   \begin{syntax}
%     \cs{token_if_math_superscript_p:N} \meta{token} \\
%     \cs{token_if_math_superscript:NTF} \meta{token} \Arg{true code} \Arg{false code}
%   \end{syntax}
%   Tests if \meta{token} has the category code of a superscript token
%   (|^| when normal \TeX{} category codes are in force).
% \end{function}
%
% \begin{function}[EXP,pTF]{\token_if_math_subscript:N}
%   \begin{syntax}
%     \cs{token_if_math_subscript_p:N} \meta{token} \\
%     \cs{token_if_math_subscript:NTF} \meta{token} \Arg{true code} \Arg{false code}
%   \end{syntax}
%   Tests if \meta{token} has the category code of a subscript token
%   (|_| when normal \TeX{} category codes are in force).
% \end{function}
%
% \begin{function}[EXP,pTF]{\token_if_space:N}
%   \begin{syntax}
%     \cs{token_if_space_p:N} \meta{token} \\
%     \cs{token_if_space:NTF} \meta{token} \Arg{true code} \Arg{false code}
%   \end{syntax}
%   Tests if \meta{token} has the category code of a space token.
%   Note that an explicit space token with character code $32$ cannot
%   be tested in this way, as it is not a valid \texttt{N}-type argument.
% \end{function}
%
% \begin{function}[EXP,pTF]{\token_if_letter:N}
%   \begin{syntax}
%     \cs{token_if_letter_p:N} \meta{token} \\
%     \cs{token_if_letter:NTF} \meta{token} \Arg{true code} \Arg{false code}
%   \end{syntax}
%   Tests if \meta{token} has the category code of a letter token.
% \end{function}
%
% \begin{function}[EXP,pTF]{\token_if_other:N}
%   \begin{syntax}
%     \cs{token_if_other_p:N} \meta{token} \\
%     \cs{token_if_other:NTF} \meta{token} \Arg{true code} \Arg{false code}
%   \end{syntax}
%   Tests if \meta{token} has the category code of an \enquote{other}
%   token.
% \end{function}
%
% \begin{function}[EXP,pTF]{\token_if_active:N}
%   \begin{syntax}
%     \cs{token_if_active_p:N} \meta{token} \\
%     \cs{token_if_active:NTF} \meta{token} \Arg{true code} \Arg{false code}
%   \end{syntax}
%   Tests if \meta{token} has the category code of an active character.
% \end{function}
%
% \begin{function}[EXP,pTF]{\token_if_eq_catcode:NN}
%   \begin{syntax}
%     \cs{token_if_eq_catcode_p:NN} \meta{token1} \meta{token2} \\
%     \cs{token_if_eq_catcode:NNTF} \meta{token1} \meta{token2} \Arg{true code} \Arg{false code}
%   \end{syntax}
%   Tests if the two \meta{tokens} have the same category code.
% \end{function}
%
% \begin{function}[EXP,pTF]{\token_if_eq_charcode:NN}
%   \begin{syntax}
%     \cs{token_if_eq_charcode_p:NN} \meta{token1} \meta{token2} \\
%     \cs{token_if_eq_charcode:NNTF} \meta{token1} \meta{token2} \Arg{true code} \Arg{false code}
%   \end{syntax}
%   Tests if the two \meta{tokens} have the same character code.
% \end{function}
%
% \begin{function}[EXP,pTF]{\token_if_eq_meaning:NN}
%   \begin{syntax}
%     \cs{token_if_eq_meaning_p:NN} \meta{token1} \meta{token2} \\
%     \cs{token_if_eq_meaning:NNTF} \meta{token1} \meta{token2} \Arg{true code} \Arg{false code}
%   \end{syntax}
%   Tests if the two \meta{tokens} have the same meaning when expanded.
% \end{function}
%
% \begin{function}[updated = 2011-05-23, EXP,pTF]{\token_if_macro:N}
%   \begin{syntax}
%     \cs{token_if_macro_p:N} \meta{token} \\
%     \cs{token_if_macro:NTF} \meta{token} \Arg{true code} \Arg{false code}
%   \end{syntax}
%   Tests if the \meta{token} is a \TeX{} macro.
% \end{function}
%
% \begin{function}[EXP,pTF]{\token_if_cs:N}
%   \begin{syntax}
%     \cs{token_if_cs_p:N} \meta{token} \\
%     \cs{token_if_cs:NTF} \meta{token} \Arg{true code} \Arg{false code}
%   \end{syntax}
%   Tests if the \meta{token} is a control sequence.
% \end{function}
%
% \begin{function}[EXP,pTF]{\token_if_expandable:N}
%   \begin{syntax}
%     \cs{token_if_expandable_p:N} \meta{token} \\
%     \cs{token_if_expandable:NTF} \meta{token} \Arg{true code} \Arg{false code}
%   \end{syntax}
%   Tests if the \meta{token} is expandable. This test returns \meta{false}
%   for an undefined token.
% \end{function}
%
% \begin{function}[EXP,pTF, updated=2012-01-20]{\token_if_long_macro:N}
%   \begin{syntax}
%     \cs{token_if_long_macro_p:N} \meta{token} \\
%     \cs{token_if_long_macro:NTF} \meta{token} \Arg{true code} \Arg{false code}
%   \end{syntax}
%   Tests if the \meta{token} is a long macro.
% \end{function}
%
% \begin{function}[EXP,pTF, updated=2012-01-20]{\token_if_protected_macro:N}
%   \begin{syntax}
%     \cs{token_if_protected_macro_p:N} \meta{token} \\
%     \cs{token_if_protected_macro:NTF} \meta{token} \Arg{true code} \Arg{false code}
%   \end{syntax}
%   Tests if the \meta{token} is a protected macro: a macro which
%   is both protected and long will return logical \texttt{false}.
% \end{function}
%
% \begin{function}[EXP,pTF, updated=2012-01-20]{\token_if_protected_long_macro:N}
%   \begin{syntax}
%     \cs{token_if_protected_long_macro_p:N} \meta{token} \\
%     \cs{token_if_protected_long_macro:NTF} \meta{token} \Arg{true code} \Arg{false code}
%   \end{syntax}
%   Tests if the \meta{token} is a protected long macro.
% \end{function}
%
% \begin{function}[EXP,pTF, updated=2012-01-20]{\token_if_chardef:N}
%   \begin{syntax}
%     \cs{token_if_chardef_p:N} \meta{token} \\
%     \cs{token_if_chardef:NTF} \meta{token} \Arg{true code} \Arg{false code}
%   \end{syntax}
%   Tests if the \meta{token} is defined to be a chardef.
%   \begin{texnote}
%     Booleans, boxes and small integer constants are implemented as
%     chardefs.
%   \end{texnote}
% \end{function}
%
% \begin{function}[EXP,pTF, updated=2012-01-20]{\token_if_mathchardef:N}
%   \begin{syntax}
%     \cs{token_if_mathchardef_p:N} \meta{token} \\
%     \cs{token_if_mathchardef:NTF} \meta{token} \Arg{true code} \Arg{false code}
%   \end{syntax}
%   Tests if the \meta{token} is defined to be a mathchardef.
% \end{function}
%
% \begin{function}[EXP,pTF, updated=2012-01-20]{\token_if_dim_register:N}
%   \begin{syntax}
%     \cs{token_if_dim_register_p:N} \meta{token} \\
%     \cs{token_if_dim_register:NTF} \meta{token} \Arg{true code} \Arg{false code}
%   \end{syntax}
%   Tests if the \meta{token} is defined to be a dimension register.
% \end{function}
%
% \begin{function}[EXP,pTF, updated=2012-01-20]{\token_if_int_register:N}
%   \begin{syntax}
%     \cs{token_if_int_register_p:N} \meta{token} \\
%     \cs{token_if_int_register:NTF} \meta{token} \Arg{true code} \Arg{false code}
%   \end{syntax}
%   Tests if the \meta{token} is defined to be a integer register.
%   \begin{texnote}
%     Constant integers may be implemented as integer registers,
%     chardefs, or mathchardefs depending on their value.
%   \end{texnote}
% \end{function}
%
% \begin{function}[EXP,pTF, added=2012-02-15]{\token_if_muskip_register:N}
%   \begin{syntax}
%     \cs{token_if_muskip_register_p:N} \meta{token} \\
%     \cs{token_if_muskip_register:NTF} \meta{token} \Arg{true code} \Arg{false code}
%   \end{syntax}
%   Tests if the \meta{token} is defined to be a muskip register.
% \end{function}
%
% \begin{function}[EXP,pTF, updated=2012-01-20]{\token_if_skip_register:N}
%   \begin{syntax}
%     \cs{token_if_skip_register_p:N} \meta{token} \\
%     \cs{token_if_skip_register:NTF} \meta{token} \Arg{true code} \Arg{false code}
%   \end{syntax}
%   Tests if the \meta{token} is defined to be a skip register.
% \end{function}
%
% \begin{function}[EXP,pTF, updated=2012-01-20]{\token_if_toks_register:N}
%   \begin{syntax}
%     \cs{token_if_toks_register_p:N} \meta{token} \\
%     \cs{token_if_toks_register:NTF} \meta{token} \Arg{true code} \Arg{false code}
%   \end{syntax}
%   Tests if the \meta{token} is defined to be a toks register
%   (not used by\LaTeX3).
% \end{function}
%
% \begin{function}[updated = 2011-05-23, EXP,pTF]{\token_if_primitive:N}
%   \begin{syntax}
%     \cs{token_if_primitive_p:N} \meta{token} \\
%     \cs{token_if_primitive:NTF} \meta{token} \Arg{true code} \Arg{false code}
%   \end{syntax}
%   Tests if the \meta{token} is an engine primitive.
% \end{function}
%
% \section{Peeking ahead at the next token}
%
% There is often a need to look ahead at the next token in the input
% stream while leaving it in place. This is handled using the
% \enquote{peek} functions. The generic \cs{peek_after:Nw} is
% provided along with a family of predefined tests for common cases.
% As peeking ahead does \emph{not} skip spaces the predefined tests
% include both a space-respecting and space-skipping version.
%
% \begin{function}{\peek_after:Nw}
%   \begin{syntax}
%     \cs{peek_after:Nw} \meta{function} \meta{token}
%   \end{syntax}
%   Locally sets the test variable \cs{l_peek_token} equal to \meta{token}
%   (as an implicit token, \emph{not} as a token list), and then
%   expands the \meta{function}. The \meta{token} will remain in
%   the input stream as the next item after the \meta{function}.
%   The \meta{token} here may be \verb*| |, |{| or |}| (assuming
%   normal \TeX{} category codes), \emph{i.e.}~it is not necessarily the
%   next argument which would be grabbed by a normal function.
% \end{function}
%
% \begin{function}{\peek_gafter:Nw}
%   \begin{syntax}
%     \cs{peek_gafter:Nw} \meta{function} \meta{token}
%   \end{syntax}
%   Globally sets the test variable \cs{g_peek_token} equal to \meta{token}
%   (as an implicit token, \emph{not} as a token list), and then
%   expands the \meta{function}. The \meta{token} will remain in
%   the input stream as the next item after the \meta{function}.
%   The \meta{token} here may be \verb*| |, |{| or |}| (assuming
%   normal \TeX{} category codes), \emph{i.e.}~it is not necessarily the
%   next argument which would be grabbed by a normal function.
% \end{function}
%
% \begin{variable}{\l_peek_token}
%  Token set by \cs{peek_after:Nw} and available for testing
%  as described above.
% \end{variable}
%
% \begin{variable}{\g_peek_token}
%  Token set by \cs{peek_gafter:Nw} and available for testing
%  as described above.
% \end{variable}
%
% \begin{function}[updated = 2011-07-02, TF]{\peek_catcode:N}
%   \begin{syntax}
%     \cs{peek_catcode:NTF} \meta{test token} \Arg{true code} \Arg{false code}
%   \end{syntax}
%   Tests if the next \meta{token} in the input stream has the same
%   category code as the \meta{test token} (as defined by the test
%   \cs{token_if_eq_catcode:NNTF}). Spaces are respected by the test
%   and the \meta{token} will be left in the input stream after
%   the \meta{true code} or \meta{false code} (as appropriate to the
%   result of the test).
% \end{function}
%
% \begin{function}[updated = 2011-07-02, TF]{\peek_catcode_ignore_spaces:N}
%   \begin{syntax}
%     \cs{peek_catcode_ignore_spaces:NTF} \meta{test token} \Arg{true code} \Arg{false code}
%   \end{syntax}
%   Tests if the next \meta{token} in the input stream has the same
%   category code as the \meta{test token} (as defined by the test
%   \cs{token_if_eq_catcode:NNTF}). Spaces are ignored by the test
%   and the \meta{token} will be left in the input stream after
%   the \meta{true code} or \meta{false code} (as appropriate to the
%   result of the test).
% \end{function}
%
% \begin{function}[updated = 2011-07-02, TF]{\peek_catcode_remove:N}
%   \begin{syntax}
%     \cs{peek_catcode_remove:NTF} \meta{test token} \Arg{true code} \Arg{false code}
%   \end{syntax}
%   Tests if the next \meta{token} in the input stream has the same
%   category code as the \meta{test token} (as defined by the test
%   \cs{token_if_eq_catcode:NNTF}). Spaces are respected by the test
%   and the \meta{token} will be removed from the input stream if the
%   test is true. The function will then place either the
%   \meta{true code} or \meta{false code} in the input stream (as
%   appropriate to the result of the test).
% \end{function}
%
% \begin{function}[updated = 2011-07-02, TF]
%   {\peek_catcode_remove_ignore_spaces:N}
%   \begin{syntax}
%     \cs{peek_catcode_remove_ignore_spaces:NTF} \meta{test token} \Arg{true code} \Arg{false code}
%   \end{syntax}
%   Tests if the next \meta{token} in the input stream has the same
%   category code as the \meta{test token} (as defined by the test
%   \cs{token_if_eq_catcode:NNTF}). Spaces are ignored by the test
%   and the \meta{token} will be removed from the input stream if the
%   test is true. The function will then place either the
%   \meta{true code} or \meta{false code} in the input stream (as
%   appropriate to the result of the test).
% \end{function}
%
% \begin{function}[updated = 2011-07-02, TF]{\peek_charcode:N}
%   \begin{syntax}
%     \cs{peek_charcode:NTF} \meta{test token} \Arg{true code} \Arg{false code}
%   \end{syntax}
%   Tests if the next \meta{token} in the input stream has the same
%   character code as the \meta{test token} (as defined by the test
%   \cs{token_if_eq_charcode:NNTF}). Spaces are respected by the test
%   and the \meta{token} will be left in the input stream after
%   the \meta{true code} or \meta{false code} (as appropriate to the
%   result of the test).
% \end{function}
%
% \begin{function}[updated = 2011-07-02, TF]{\peek_charcode_ignore_spaces:N}
%   \begin{syntax}
%     \cs{peek_charcode_ignore_spaces:NTF} \meta{test token} \Arg{true code} \Arg{false code}
%   \end{syntax}
%   Tests if the next \meta{token} in the input stream has the same
%   character code as the \meta{test token} (as defined by the test
%   \cs{token_if_eq_charcode:NNTF}). Spaces are ignored by the test
%   and the \meta{token} will be left in the input stream after
%   the \meta{true code} or \meta{false code} (as appropriate to the
%   result of the test).
% \end{function}
%
% \begin{function}[updated = 2011-07-02, TF]{\peek_charcode_remove:N}
%   \begin{syntax}
%     \cs{peek_charcode_remove:NTF} \meta{test token} \Arg{true code} \Arg{false code}
%   \end{syntax}
%   Tests if the next \meta{token} in the input stream has the same
%   character code as the \meta{test token} (as defined by the test
%   \cs{token_if_eq_charcode:NNTF}). Spaces are respected by the test
%   and the \meta{token} will be removed from the input stream if the
%   test is true. The function will then place either the
%   \meta{true code} or \meta{false code} in the input stream (as
%   appropriate to the result of the test).
% \end{function}
%
% \begin{function}[updated = 2011-07-02, TF]
%   {\peek_charcode_remove_ignore_spaces:N}
%   \begin{syntax}
%     \cs{peek_charcode_remove_ignore_spaces:NTF} \meta{test token}
%     ~~\Arg{true code} \Arg{false code}
%   \end{syntax}
%    Tests if the next \meta{token} in the input stream has the same
%   character code as the \meta{test token} (as defined by the test
%   \cs{token_if_eq_charcode:NNTF}). Spaces are ignored by the test
%   and the \meta{token} will be removed from the input stream if the
%   test is true. The function will then place either the
%   \meta{true code} or \meta{false code} in the input stream (as
%   appropriate to the result of the test).
% \end{function}
%
% \begin{function}[updated = 2011-07-02, TF]{\peek_meaning:N}
%   \begin{syntax}
%     \cs{peek_meaning:NTF} \meta{test token} \Arg{true code} \Arg{false code}
%   \end{syntax}
%   Tests if the next \meta{token} in the input stream has the same
%   meaning as the \meta{test token} (as defined by the test
%   \cs{token_if_eq_meaning:NNTF}). Spaces are respected by the test
%   and the \meta{token} will be left in the input stream after
%   the \meta{true code} or \meta{false code} (as appropriate to the
%   result of the test).
% \end{function}
%
% \begin{function}[updated = 2011-07-02, TF]{\peek_meaning_ignore_spaces:N}
%   \begin{syntax}
%     \cs{peek_meaning_ignore_spaces:NTF} \meta{test token} \Arg{true code} \Arg{false code}
%   \end{syntax}
%   Tests if the next \meta{token} in the input stream has the same
%   meaning as the \meta{test token} (as defined by the test
%   \cs{token_if_eq_meaning:NNTF}). Spaces are ignored by the test
%   and the \meta{token} will be left in the input stream after
%   the \meta{true code} or \meta{false code} (as appropriate to the
%   result of the test).
% \end{function}
%
% \begin{function}[updated = 2011-07-02, TF]{\peek_meaning_remove:N}
%   \begin{syntax}
%     \cs{peek_meaning_remove:NTF} \meta{test token} \Arg{true code} \Arg{false code}
%   \end{syntax}
%   Tests if the next \meta{token} in the input stream has the same
%   meaning as the \meta{test token} (as defined by the test
%   \cs{token_if_eq_meaning:NNTF}). Spaces are respected by the test
%   and the \meta{token} will be removed from the input stream if the
%   test is true. The function will then place either the
%   \meta{true code} or \meta{false code} in the input stream (as
%   appropriate to the result of the test).
% \end{function}
%
% \begin{function}[updated = 2011-07-02, TF]
%   {\peek_meaning_remove_ignore_spaces:N}
%   \begin{syntax}
%     \cs{peek_meaning_remove_ignore_spaces:NTF} \meta{test token}
%     ~~\Arg{true code} \Arg{false code}
%   \end{syntax}
%   Tests if the next \meta{token} in the input stream has the same
%   meaning as the \meta{test token} (as defined by the test
%   \cs{token_if_eq_meaning:NNTF}). Spaces are ignored by the test
%   and the \meta{token} will be removed from the input stream if the
%   test is true. The function will then place either the
%   \meta{true code} or \meta{false code} in the input stream (as
%   appropriate to the result of the test).
% \end{function}
%
% \section{Decomposing a macro definition}
%
% These functions decompose \TeX{} macros into their constituent
% parts: if the \meta{token} passed is not a macro then no decomposition
% can occur. In the later case, all three functions leave \cs{scan_stop:}
% in the input stream.
%
% \begin{function}[EXP]{\token_get_arg_spec:N}
%   \begin{syntax}
%     \cs{token_get_arg_spec:N} \meta{token}
%   \end{syntax}
%   If the \meta{token} is a macro, this function will leave
%   the primitive \TeX{} argument specification in input stream as
%   a string of tokens of category code $12$ (with spaces having category
%   code $10$). Thus for example for a token \cs{next} defined by
%   \begin{verbatim}
%     \cs_set:Npn \next #1#2 { x #1 y #2 }
%   \end{verbatim}
%   will leave |#1#2| in the input stream. If the \meta{token} is
%   not a macro then \cs{scan_stop:} will be left in the input stream
%   \begin{texnote}
%     If the arg~spec. contains the string |->|, then the |spec| function
%     will produce incorrect results.
%   \end{texnote}
% \end{function}
%
% \begin{function}[EXP]{\token_get_replacement_spec:N}
%   \begin{syntax}
%     \cs{token_get_replacement_spec:N} \meta{token}
%   \end{syntax}
%   If the \meta{token} is a macro, this function will leave
%   the replacement text in input stream as
%   a string of tokens of category code $12$ (with spaces having category
%   code $10$). Thus for example for a token \cs{next} defined by
%   \begin{verbatim}
%     \cs_set:Npn \next #1#2 { x #1~y #2 }
%   \end{verbatim}
%   will leave \verb|x#1 y#2| in the input stream. If the \meta{token} is
%   not a macro then \cs{scan_stop:} will be left in the input stream
% \end{function}
%
% \begin{function}[EXP]{\token_get_prefix_spec:N}
%   \begin{syntax}
%     \cs{token_get_prefix_spec:N} \meta{token}
%   \end{syntax}
%   If the \meta{token} is a macro, this function will leave
%   the \TeX{} prefixes applicable in input stream as
%   a string of tokens of category code $12$ (with spaces having category
%   code $10$). Thus for example for a token \cs{next} defined by
%   \begin{verbatim}
%     \cs_set:Npn \next #1#2 { x #1~y #2 }
%   \end{verbatim}
%   will leave |\long| in the input stream. If the \meta{token} is
%   not a macro then \cs{scan_stop:} will be left in the input stream
% \end{function}
%
% \end{documentation}
%
% \begin{implementation}
%
% \section{\pkg{l3token} implementation}
%
%    \begin{macrocode}
%<*initex|package>
%    \end{macrocode}
%
%    \begin{macrocode}
%<*package>
\ProvidesExplPackage
  {\ExplFileName}{\ExplFileDate}{\ExplFileVersion}{\ExplFileDescription}
\__expl_package_check:
%</package>
%    \end{macrocode}
%
% \subsection{Character tokens}
%
% \begin{macro}{\char_set_catcode:nn}
% \begin{macro}{\char_value_catcode:n}
% \begin{macro}{\char_show_value_catcode:n}
%   Category code changes.
%    \begin{macrocode}
\cs_new_protected:Npn \char_set_catcode:nn #1#2
  { \tex_catcode:D #1 = \__int_eval:w #2 \__int_eval_end: }
\cs_new:Npn \char_value_catcode:n #1
  { \tex_the:D \tex_catcode:D \__int_eval:w #1\__int_eval_end: }
\cs_new_protected:Npn \char_show_value_catcode:n #1
  { \tex_showthe:D \tex_catcode:D \__int_eval:w #1 \__int_eval_end: }
%    \end{macrocode}
% \end{macro}
% \end{macro}
% \end{macro}
%
% \begin{macro}
%   {
%     \char_set_catcode_escape:N           ,
%     \char_set_catcode_group_begin:N      ,
%     \char_set_catcode_group_end:N        ,
%     \char_set_catcode_math_toggle:N      ,
%     \char_set_catcode_alignment:N        ,
%     \char_set_catcode_end_line:N         ,
%     \char_set_catcode_parameter:N        ,
%     \char_set_catcode_math_superscript:N ,
%     \char_set_catcode_math_subscript:N   ,
%     \char_set_catcode_ignore:N           ,
%     \char_set_catcode_space:N            ,
%     \char_set_catcode_letter:N           ,
%     \char_set_catcode_other:N            ,
%     \char_set_catcode_active:N           ,
%     \char_set_catcode_comment:N          ,
%     \char_set_catcode_invalid:N
%   }
%    \begin{macrocode}
\cs_new_protected:Npn \char_set_catcode_escape:N #1
  { \char_set_catcode:nn { `#1 } \c_zero }
\cs_new_protected:Npn \char_set_catcode_group_begin:N #1
  { \char_set_catcode:nn { `#1 } \c_one }
\cs_new_protected:Npn \char_set_catcode_group_end:N #1
  { \char_set_catcode:nn { `#1 } \c_two }
\cs_new_protected:Npn \char_set_catcode_math_toggle:N #1
  { \char_set_catcode:nn { `#1 } \c_three }
\cs_new_protected:Npn \char_set_catcode_alignment:N #1
  { \char_set_catcode:nn { `#1 } \c_four }
\cs_new_protected:Npn \char_set_catcode_end_line:N #1
  { \char_set_catcode:nn { `#1 } \c_five }
\cs_new_protected:Npn \char_set_catcode_parameter:N #1
  { \char_set_catcode:nn { `#1 } \c_six }
\cs_new_protected:Npn \char_set_catcode_math_superscript:N #1
  { \char_set_catcode:nn { `#1 } \c_seven }
\cs_new_protected:Npn \char_set_catcode_math_subscript:N #1
  { \char_set_catcode:nn { `#1 } \c_eight }
\cs_new_protected:Npn \char_set_catcode_ignore:N #1
  { \char_set_catcode:nn { `#1 } \c_nine }
\cs_new_protected:Npn \char_set_catcode_space:N #1
  { \char_set_catcode:nn { `#1 } \c_ten }
\cs_new_protected:Npn \char_set_catcode_letter:N #1
  { \char_set_catcode:nn { `#1 } \c_eleven }
\cs_new_protected:Npn \char_set_catcode_other:N #1
  { \char_set_catcode:nn { `#1 } \c_twelve }
\cs_new_protected:Npn \char_set_catcode_active:N #1
  { \char_set_catcode:nn { `#1 } \c_thirteen }
\cs_new_protected:Npn \char_set_catcode_comment:N #1
  { \char_set_catcode:nn { `#1 } \c_fourteen }
\cs_new_protected:Npn \char_set_catcode_invalid:N #1
  { \char_set_catcode:nn { `#1 } \c_fifteen }
%    \end{macrocode}
% \end{macro}
%
% \begin{macro}
%   {
%     \char_set_catcode_escape:n           ,
%     \char_set_catcode_group_begin:n      ,
%     \char_set_catcode_group_end:n        ,
%     \char_set_catcode_math_toggle:n      ,
%     \char_set_catcode_alignment:n        ,
%     \char_set_catcode_end_line:n         ,
%     \char_set_catcode_parameter:n        ,
%     \char_set_catcode_math_superscript:n ,
%     \char_set_catcode_math_subscript:n   ,
%     \char_set_catcode_ignore:n           ,
%     \char_set_catcode_space:n            ,
%     \char_set_catcode_letter:n           ,
%     \char_set_catcode_other:n            ,
%     \char_set_catcode_active:n           ,
%     \char_set_catcode_comment:n          ,
%     \char_set_catcode_invalid:n
%   }
%    \begin{macrocode}
\cs_new_protected:Npn \char_set_catcode_escape:n #1
  { \char_set_catcode:nn {#1} \c_zero }
\cs_new_protected:Npn \char_set_catcode_group_begin:n #1
  { \char_set_catcode:nn {#1} \c_one }
\cs_new_protected:Npn \char_set_catcode_group_end:n #1
  { \char_set_catcode:nn {#1} \c_two }
\cs_new_protected:Npn \char_set_catcode_math_toggle:n #1
  { \char_set_catcode:nn {#1} \c_three }
\cs_new_protected:Npn \char_set_catcode_alignment:n #1
  { \char_set_catcode:nn {#1} \c_four }
\cs_new_protected:Npn \char_set_catcode_end_line:n #1
  { \char_set_catcode:nn {#1} \c_five }
\cs_new_protected:Npn \char_set_catcode_parameter:n #1
  { \char_set_catcode:nn {#1} \c_six }
\cs_new_protected:Npn \char_set_catcode_math_superscript:n #1
  { \char_set_catcode:nn {#1} \c_seven }
\cs_new_protected:Npn \char_set_catcode_math_subscript:n #1
  { \char_set_catcode:nn {#1} \c_eight }
\cs_new_protected:Npn \char_set_catcode_ignore:n #1
  { \char_set_catcode:nn {#1} \c_nine }
\cs_new_protected:Npn \char_set_catcode_space:n #1
  { \char_set_catcode:nn {#1} \c_ten }
\cs_new_protected:Npn \char_set_catcode_letter:n #1
  { \char_set_catcode:nn {#1} \c_eleven }
\cs_new_protected:Npn \char_set_catcode_other:n #1
  { \char_set_catcode:nn {#1} \c_twelve }
\cs_new_protected:Npn \char_set_catcode_active:n #1
  { \char_set_catcode:nn {#1} \c_thirteen }
\cs_new_protected:Npn \char_set_catcode_comment:n #1
  { \char_set_catcode:nn {#1} \c_fourteen }
\cs_new_protected:Npn \char_set_catcode_invalid:n #1
  { \char_set_catcode:nn {#1} \c_fifteen }
%    \end{macrocode}
% \end{macro}
%
% \begin{macro}{\char_set_mathcode:nn}
% \begin{macro}{\char_value_mathcode:n}
% \begin{macro}{\char_show_value_mathcode:n}
% \begin{macro}{\char_set_lccode:nn}
% \begin{macro}{\char_value_lccode:n}
% \begin{macro}{\char_show_value_lccode:n}
% \begin{macro}{\char_set_uccode:nn}
% \begin{macro}{\char_value_uccode:n}
% \begin{macro}{\char_show_value_uccode:n}
% \begin{macro}{\char_set_sfcode:nn}
% \begin{macro}{\char_value_sfcode:n}
% \begin{macro}{\char_show_value_sfcode:n}
%   Pretty repetitive, but necessary!
%    \begin{macrocode}
\cs_new_protected:Npn \char_set_mathcode:nn #1#2
  { \tex_mathcode:D #1 = \__int_eval:w #2 \__int_eval_end: }
\cs_new:Npn \char_value_mathcode:n #1
  { \tex_the:D \tex_mathcode:D \__int_eval:w #1\__int_eval_end: }
\cs_new_protected:Npn \char_show_value_mathcode:n #1
  { \tex_showthe:D \tex_mathcode:D \__int_eval:w #1 \__int_eval_end: }
\cs_new_protected:Npn \char_set_lccode:nn #1#2
  { \tex_lccode:D #1 = \__int_eval:w #2 \__int_eval_end: }
\cs_new:Npn \char_value_lccode:n #1
  { \tex_the:D \tex_lccode:D \__int_eval:w #1\__int_eval_end: }
\cs_new_protected:Npn \char_show_value_lccode:n #1
  { \tex_showthe:D \tex_lccode:D \__int_eval:w #1 \__int_eval_end: }
\cs_new_protected:Npn \char_set_uccode:nn #1#2
  { \tex_uccode:D #1 = \__int_eval:w #2 \__int_eval_end: }
\cs_new:Npn \char_value_uccode:n #1
  { \tex_the:D \tex_uccode:D \__int_eval:w #1\__int_eval_end: }
\cs_new_protected:Npn \char_show_value_uccode:n #1
  { \tex_showthe:D \tex_uccode:D \__int_eval:w #1 \__int_eval_end: }
\cs_new_protected:Npn \char_set_sfcode:nn #1#2
  { \tex_sfcode:D #1 = \__int_eval:w #2 \__int_eval_end: }
\cs_new:Npn \char_value_sfcode:n #1
  { \tex_the:D \tex_sfcode:D \__int_eval:w #1\__int_eval_end: }
\cs_new_protected:Npn \char_show_value_sfcode:n #1
  { \tex_showthe:D \tex_sfcode:D \__int_eval:w #1 \__int_eval_end: }
%    \end{macrocode}
% \end{macro}
% \end{macro}
% \end{macro}
% \end{macro}
% \end{macro}
% \end{macro}
% \end{macro}
% \end{macro}
% \end{macro}
% \end{macro}
% \end{macro}
% \end{macro}
%
% \subsection{Generic tokens}
%
% \begin{macro}{\token_new:Nn}
%  Creates a new token.
%    \begin{macrocode}
\cs_new_protected:Npn \token_new:Nn #1#2 { \cs_new_eq:NN #1 #2 }
%    \end{macrocode}
% \end{macro}
%
% \begin{macro}
%   {
%     \c_group_begin_token,
%     \c_group_end_token,
%     \c_math_toggle_token,
%     \c_alignment_token,
%     \c_parameter_token,
%     \c_math_superscript_token,
%     \c_math_subscript_token,
%     \c_space_token,
%     \c_catcode_letter_token,
%     \c_catcode_other_token
%   }
%    We define these useful tokens. We have to do it by hand with the
%    brace tokens for obvious reasons.
%    \begin{macrocode}
\cs_new_eq:NN \c_group_begin_token {
\cs_new_eq:NN \c_group_end_token }
\group_begin:
  \char_set_catcode_math_toggle:N \*
  \token_new:Nn \c_math_toggle_token { * }
  \char_set_catcode_alignment:N \*
  \token_new:Nn \c_alignment_token { * }
  \token_new:Nn \c_parameter_token { # }
  \token_new:Nn \c_math_superscript_token { ^ }
  \char_set_catcode_math_subscript:N \*
  \token_new:Nn \c_math_subscript_token { * }
  \token_new:Nn \c_space_token { ~ }
  \token_new:Nn \c_catcode_letter_token { a }
  \token_new:Nn \c_catcode_other_token { 1 }
\group_end:
%    \end{macrocode}
% \end{macro}
%
% \begin{variable}{\c_catcode_active_tl}
%   Not an implicit token!
%    \begin{macrocode}
\group_begin:
  \char_set_catcode_active:N \*
  \tl_const:Nn \c_catcode_active_tl { \exp_not:N * }
\group_end:
%    \end{macrocode}
% \end{variable}
%
% \begin{variable}{\l_char_active_seq, \l_char_special_seq}
%   Two sequences for dealing with special characters. The first is characters
%   which may be active, and contains the active characters themselves  to
%   allow easy redefinition. The second longer list is for \enquote{special}
%   characters more generally, and these are escaped so that for example
%   bulk code assignments can be carried out. In both cases, the order is
%   by \textsc{ascii} character code (as is done in for example
%   \cs{ExplSyntaxOn}). The only complication is dealing with |_|, which
%   requires the use of \cs{use:n} \emph{and} \cs{use:nn}.
%    \begin{macrocode}
\seq_new:N \l_char_active_seq
\use:n
  {
    \group_begin:
    \char_set_catcode_active:N \"
    \char_set_catcode_active:N \$
    \char_set_catcode_active:N \&
    \char_set_catcode_active:N \^
    \char_set_catcode_active:N \_
    \char_set_catcode_active:N \~
    \use:nn
      {
        \group_end:
        \seq_set_from_clist:Nn \l_char_active_seq
      }
  }
    { { " , $ , & , ^ , _ , ~ } } %$
\seq_new:N \l_char_special_seq
\seq_set_from_clist:Nn \l_char_special_seq
  { \  , \" , \# , \$ , \% , \& , \\ , \^ , \_ , \{ , \} , \~ }
%    \end{macrocode}
% \end{variable}
%
% \subsection{Token conditionals}
%
% \begin{macro}[pTF]{\token_if_group_begin:N}
%   Check if token is a begin group token. We use the constant
%   |\c_group_begin_token| for this.
%    \begin{macrocode}
\prg_new_conditional:Npnn \token_if_group_begin:N #1 { p , T ,  F , TF }
  {
    \if_catcode:w \exp_not:N #1 \c_group_begin_token
      \prg_return_true: \else: \prg_return_false: \fi:
  }
%    \end{macrocode}
% \end{macro}
%
% \begin{macro}[pTF]{\token_if_group_end:N}
%   Check if token is a end group token. We use the constant
%   |\c_group_end_token| for this.
%    \begin{macrocode}
\prg_new_conditional:Npnn \token_if_group_end:N #1 { p , T ,  F , TF }
  {
    \if_catcode:w \exp_not:N #1 \c_group_end_token
      \prg_return_true: \else: \prg_return_false: \fi:
  }
%    \end{macrocode}
% \end{macro}
%
% \begin{macro}[pTF]{\token_if_math_toggle:N}
%   Check if token is a math shift token. We use the constant
%   |\c_math_toggle_token| for this.
%    \begin{macrocode}
\prg_new_conditional:Npnn \token_if_math_toggle:N #1 { p , T ,  F , TF }
  {
    \if_catcode:w \exp_not:N #1 \c_math_toggle_token
      \prg_return_true: \else: \prg_return_false: \fi:
  }
%    \end{macrocode}
% \end{macro}
%
% \begin{macro}[pTF]{\token_if_alignment:N}
%   Check if token is an alignment tab token. We use the constant
%   |\c_alignment_tab_token| for this.
%    \begin{macrocode}
\prg_new_conditional:Npnn \token_if_alignment:N #1 { p , T ,  F , TF }
  {
    \if_catcode:w \exp_not:N #1 \c_alignment_token
      \prg_return_true: \else: \prg_return_false: \fi:
  }
%    \end{macrocode}
% \end{macro}
%
% \begin{macro}[pTF]{\token_if_parameter:N}
%   Check if token is a parameter token. We use the constant
%   |\c_parameter_token| for this. We have to trick \TeX{} a bit to
%   avoid an error message: within a group we prevent
%   \cs{c_parameter_token} from behaving like a macro parameter character.
%   The definitions of \cs{prg_new_conditional:Npnn} are global, so they
%   will remain after the group.
%    \begin{macrocode}
\group_begin:
\cs_set_eq:NN \c_parameter_token \scan_stop:
\prg_new_conditional:Npnn \token_if_parameter:N #1 { p , T ,  F , TF }
  {
    \if_catcode:w \exp_not:N #1 \c_parameter_token
      \prg_return_true: \else: \prg_return_false: \fi:
  }
\group_end:
%    \end{macrocode}
% \end{macro}
%
% \begin{macro}[pTF]{\token_if_math_superscript:N}
%   Check if token is a math superscript token. We use the constant
%   |\c_superscript_token| for this.
%    \begin{macrocode}
\prg_new_conditional:Npnn \token_if_math_superscript:N #1 { p , T ,  F , TF }
  {
    \if_catcode:w \exp_not:N #1 \c_math_superscript_token
      \prg_return_true: \else: \prg_return_false: \fi:
  }
%    \end{macrocode}
% \end{macro}
%
% \begin{macro}[pTF]{\token_if_math_subscript:N}
%   Check if token is a math subscript token. We use the constant
%   |\c_subscript_token| for this.
%    \begin{macrocode}
\prg_new_conditional:Npnn \token_if_math_subscript:N #1 { p , T ,  F , TF }
  {
    \if_catcode:w \exp_not:N #1 \c_math_subscript_token
      \prg_return_true: \else: \prg_return_false: \fi:
  }
%    \end{macrocode}
% \end{macro}
%
% \begin{macro}[pTF]{\token_if_space:N}
%   Check if token is a space token. We use the constant
%   |\c_space_token| for this.
%    \begin{macrocode}
\prg_new_conditional:Npnn \token_if_space:N #1 { p , T ,  F , TF }
  {
    \if_catcode:w \exp_not:N #1 \c_space_token
      \prg_return_true: \else: \prg_return_false: \fi:
  }
%    \end{macrocode}
% \end{macro}
%
% \begin{macro}[pTF]{\token_if_letter:N}
%   Check if token is a letter token. We use the constant
%   |\c_letter_token| for this.
%    \begin{macrocode}
\prg_new_conditional:Npnn \token_if_letter:N #1 { p , T ,  F , TF }
  {
    \if_catcode:w \exp_not:N #1 \c_catcode_letter_token
      \prg_return_true: \else: \prg_return_false: \fi:
  }
%    \end{macrocode}
% \end{macro}
%
% \begin{macro}[pTF]{\token_if_other:N}
%   Check if token is an other char token. We use the constant
%   |\c_other_char_token| for this.
%    \begin{macrocode}
\prg_new_conditional:Npnn \token_if_other:N #1 { p , T ,  F , TF }
  {
    \if_catcode:w \exp_not:N #1 \c_catcode_other_token
      \prg_return_true: \else: \prg_return_false: \fi:
  }
%    \end{macrocode}
% \end{macro}
%
% \begin{macro}[pTF]{\token_if_active:N}
%   Check if token is an active char token. We use the constant
%   |\c_active_char_tl| for this. A technical point is that
%   \cs{c_active_char_tl} is in fact a macro expanding to
%   |\exp_not:N *|, where |*| is active.
%    \begin{macrocode}
\prg_new_conditional:Npnn \token_if_active:N #1 { p , T ,  F , TF }
  {
    \if_catcode:w \exp_not:N #1 \c_catcode_active_tl
      \prg_return_true: \else: \prg_return_false: \fi:
  }
%    \end{macrocode}
% \end{macro}
%
% \begin{macro}[pTF]{\token_if_eq_meaning:NN}
%   Check if the tokens |#1| and |#2| have same meaning.
%    \begin{macrocode}
\prg_new_conditional:Npnn \token_if_eq_meaning:NN #1#2 { p , T ,  F , TF }
  {
    \if_meaning:w  #1  #2
      \prg_return_true: \else: \prg_return_false: \fi:
  }
%    \end{macrocode}
% \end{macro}
%
% \begin{macro}[pTF]{\token_if_eq_catcode:NN}
%  Check if the tokens |#1| and |#2| have same category code.
%    \begin{macrocode}
\prg_new_conditional:Npnn \token_if_eq_catcode:NN #1#2 { p , T ,  F , TF }
  {
    \if_catcode:w \exp_not:N #1 \exp_not:N #2
      \prg_return_true: \else: \prg_return_false: \fi:
  }
%    \end{macrocode}
% \end{macro}
%
% \begin{macro}[pTF]{\token_if_eq_charcode:NN}
%  Check if the tokens |#1| and |#2| have same character code.
%    \begin{macrocode}
\prg_new_conditional:Npnn \token_if_eq_charcode:NN #1#2 { p , T ,  F , TF }
  {
    \if_charcode:w \exp_not:N #1 \exp_not:N #2
      \prg_return_true: \else: \prg_return_false: \fi:
  }
%    \end{macrocode}
% \end{macro}
%
% \begin{macro}[pTF]{\token_if_macro:N}
% \begin{macro}[aux]{\token_if_macro_p_aux:w}
%   When a token is a macro, |\token_to_meaning:N| will always output
%   something like |\long macro:#1->#1| so we could naively check to
%   see if the meaning contains |->|. However, this can fail the five
%   \tn{...mark} primitives, whose meaning has the form
%   |...mark:|\meta{user material}. The problem is that the
%   \meta{user material} can contain |->|.
%
%   However, only characters, macros, and marks can contain the colon
%   character. The idea is thus to grab until the first |:|, and analyse
%   what is left. However, macros can have any combination of |\long|,
%   |\protected| or |\outer| (not used in \LaTeX3) before the string
%   |macro:|. We thus only select the part of the meaning between
%   the first |ma| and the first following |:|. If this string is
%   |cro|, then we have a macro. If the string is |rk|, then we have
%   a mark. The string can also be |cro parameter character | for a
%   colon with a weird category code (namely the usual category code
%   of |#|). Otherwise, it is empty.
%
%   This relies on the fact that |\long|, |\protected|, |\outer|
%   cannot contain |ma|, regardless of the escape character, even if
%   the escape character is |m|\ldots{}
%
%   Both |ma| and |:| must be of category code $12$ (other), and we
%   achieve using the standard lowercasing technique.
%
%    \begin{macrocode}
\group_begin:
\char_set_catcode_other:N \M
\char_set_catcode_other:N \A
\char_set_lccode:nn { `\; } { `\: }
\char_set_lccode:nn { `\T } { `\T }
\char_set_lccode:nn { `\F } { `\F }
\tl_to_lowercase:n
  {
    \group_end:
    \prg_new_conditional:Npnn \token_if_macro:N #1 { p , T ,  F , TF }
      {
        \exp_after:wN \token_if_macro_p_aux:w
        \token_to_meaning:N #1 MA; \q_stop
      }
    \cs_new:Npn \token_if_macro_p_aux:w #1 MA #2 ; #3 \q_stop
      {
        \if_int_compare:w \pdftex_strcmp:D { #2 } { cro } = \c_zero
            \prg_return_true:
        \else:
            \prg_return_false:
        \fi:
      }
  }
%    \end{macrocode}
%  \end{macro}
%  \end{macro}
%
% \begin{macro}[pTF]{\token_if_cs:N}
%   Check if token has same catcode as a control sequence. This
%   follows the same pattern as for \cs{token_if_letter:N} \emph{etc.}
%   We use |\scan_stop:| for this.
%    \begin{macrocode}
\prg_new_conditional:Npnn \token_if_cs:N #1 { p , T ,  F , TF }
  {
    \if_catcode:w \exp_not:N #1 \scan_stop:
      \prg_return_true: \else: \prg_return_false: \fi:
  }
%    \end{macrocode}
% \end{macro}
%
% \begin{macro}[pTF]{\token_if_expandable:N}
%   Check if token is expandable. We use the fact that \TeX{} will
%   temporarily convert |\exp_not:N| \meta{token} into |\scan_stop:| if
%   \meta{token} is expandable.
%    \begin{macrocode}
\prg_new_conditional:Npnn \token_if_expandable:N #1 { p , T ,  F , TF }
  {
    \cs_if_exist:NTF #1
      {
        \exp_after:wN \if_meaning:w \exp_not:N #1 #1
          \prg_return_false: \else: \prg_return_true: \fi:
      }
      { \prg_return_false: }
  }
%    \end{macrocode}
% \end{macro}
%
% \begin{macro}[pTF]
%   {
%     \token_if_chardef:N,               \token_if_mathchardef:N,
%     \token_if_dim_register:N,          \token_if_int_register:N,
%     \token_if_muskip_register:N,
%     \token_if_skip_register:N,         \token_if_toks_register:N,
%                                        \token_if_long_macro:N,
%     \token_if_protected_macro:N,       \token_if_protected_long_macro:N,
%   }
% \begin{macro}[aux]
%   {
%     \token_if_chardef_aux:w,
%     \token_if_dim_register_aux:w,
%     \token_if_int_register_aux:w,
%     \token_if_muskip_register_aux:w,
%     \token_if_skip_register_aux:w,
%     \token_if_toks_register_aux:w,
%     \token_if_protected_macro_aux:w,
%     \token_if_long_macro_aux:w,
%   }
%   Most of these functions have to check the meaning of the token in
%   question so we need to do some checkups on which characters are
%   output by |\token_to_meaning:N|. As usual, these characters have
%   catcode 12 so we must do some serious substitutions in the code
%   below\dots
%    \begin{macrocode}
\group_begin:
  \char_set_lccode:nn { `T } { `T }
  \char_set_lccode:nn { `F } { `F }
  \char_set_lccode:nn { `X } { `n }
  \char_set_lccode:nn { `Y } { `t }
  \char_set_lccode:nn { `Z } { `d }
  \tl_map_inline:nn { A C E G H I K L M O P R S U X Y Z R " }
    { \char_set_catcode:nn { `#1 } \c_twelve }
%    \end{macrocode}
%   We convert the token list to lower case and restore the catcode and
%   lowercase code changes.
%    \begin{macrocode}
\tl_to_lowercase:n
  {
    \group_end:
%    \end{macrocode}
%   First up is checking if something has been defined with
%   \tn{chardef} or \tn{mathchardef}. This is easy since \TeX{}
%   thinks of such tokens as hexadecimal so it stores them as
%   |\char"|\meta{hex~number} or |\mathchar"|\meta{hex~number}.
%   Grab until the first occurrence of |char"|, and compare what
%   preceeds with |\| or |\math|. In fact, the escape character
%   may not be a backslash, so we compare with the result of
%   converting some other control sequence to a string, namely
%   |\char| or |\mathchar| (the auxiliary adds the |char| back).
%    \begin{macrocode}
    \prg_new_conditional:Npnn \token_if_chardef:N #1 { p , T ,  F , TF }
      {
        \__str_if_eq_x_return:nn
          {
            \exp_after:wN \token_if_chardef_aux:w
              \token_to_meaning:N #1 CHAR" \q_stop
          }
          { \token_to_str:N \char }
      }
    \prg_new_conditional:Npnn \token_if_mathchardef:N #1 { p , T ,  F , TF }
      {
        \__str_if_eq_x_return:nn
          {
            \exp_after:wN \token_if_chardef_aux:w
              \token_to_meaning:N #1 CHAR" \q_stop
          }
          { \token_to_str:N \mathchar }
      }
    \cs_new:Npn \token_if_chardef_aux:w #1 CHAR" #2 \q_stop { #1 CHAR }
%    \end{macrocode}
%
%   Dim registers are a little more difficult since their \tn{meaning}
%   has the form |\dimen|\meta{number}, and we must take care of the
%   two primitives \tn{dimen} and \tn{dimendef}.
%    \begin{macrocode}
    \prg_new_conditional:Npnn \token_if_dim_register:N #1 { p , T ,  F , TF }
      {
        \if_meaning:w \tex_dimen:D #1
          \prg_return_false:
        \else:
          \if_meaning:w \tex_dimendef:D #1
            \prg_return_false:
          \else:
            \__str_if_eq_x_return:nn
              {
                \exp_after:wN \token_if_dim_register_aux:w
                  \token_to_meaning:N #1 ZIMEX \q_stop
              }
              { \token_to_str:N \  }
          \fi:
        \fi:
      }
    \cs_new:Npn \token_if_dim_register_aux:w #1 ZIMEX #2 \q_stop { #1 ~ }
%    \end{macrocode}
% Integer registers are one step harder since constants are implemented
% differently from variables, and we also have to take care of the
% primitives \tn{count} and \tn{countdef}.
%    \begin{macrocode}
    \prg_new_conditional:Npnn \token_if_int_register:N #1 { p , T ,  F , TF }
      {
        % \token_if_chardef:NTF #1 { \prg_return_true: }
        %   {
        %     \token_if_mathchardef:NTF #1 { \prg_return_true: }
        %       {
        \if_meaning:w \tex_count:D #1
          \prg_return_false:
        \else:
          \if_meaning:w \tex_countdef:D #1
            \prg_return_false:
          \else:
            \__str_if_eq_x_return:nn
              {
                \exp_after:wN \token_if_int_register_aux:w
                  \token_to_meaning:N #1 COUXY \q_stop
              }
              { \token_to_str:N \  }
          \fi:
        \fi:
        %       }
        %   }
      }
    \cs_new:Npn \token_if_int_register_aux:w #1 COUXY #2 \q_stop { #1 ~ }
%    \end{macrocode}
%   Muskip registers are done the same way as the dimension registers.
%    \begin{macrocode}
    \prg_new_conditional:Npnn \token_if_muskip_register:N #1 { p , T ,  F , TF }
      {
        \if_meaning:w \tex_muskip:D #1
          \prg_return_false:
        \else:
          \if_meaning:w \tex_muskipdef:D #1
            \prg_return_false:
          \else:
            \__str_if_eq_x_return:nn
              {
                \exp_after:wN \token_if_muskip_register_aux:w
                  \token_to_meaning:N #1 MUSKIP \q_stop
              }
              { \token_to_str:N \  }
          \fi:
        \fi:
      }
    \cs_new:Npn \token_if_muskip_register_aux:w #1 MUSKIP #2 \q_stop { #1 ~ }
%    \end{macrocode}
%   Skip registers.
%    \begin{macrocode}
    \prg_new_conditional:Npnn \token_if_skip_register:N #1 { p , T ,  F , TF }
      {
        \if_meaning:w \tex_skip:D #1
          \prg_return_false:
        \else:
          \if_meaning:w \tex_skipdef:D #1
            \prg_return_false:
          \else:
            \__str_if_eq_x_return:nn
              {
                \exp_after:wN \token_if_skip_register_aux:w
                  \token_to_meaning:N #1 SKIP \q_stop
              }
              { \token_to_str:N \  }
          \fi:
        \fi:
      }
    \cs_new:Npn \token_if_skip_register_aux:w #1 SKIP #2 \q_stop { #1 ~ }
%    \end{macrocode}
%   Toks registers.
%    \begin{macrocode}
    \prg_new_conditional:Npnn \token_if_toks_register:N #1 { p , T ,  F , TF }
      {
        \if_meaning:w \tex_toks:D #1
          \prg_return_false:
        \else:
          \if_meaning:w \tex_toksdef:D #1
            \prg_return_false:
          \else:
            \__str_if_eq_x_return:nn
              {
                \exp_after:wN \token_if_toks_register_aux:w
                  \token_to_meaning:N #1 YOKS \q_stop
              }
              { \token_to_str:N \  }
          \fi:
        \fi:
      }
     \cs_new:Npn \token_if_toks_register_aux:w #1 YOKS #2 \q_stop { #1 ~ }
%    \end{macrocode}
%   Protected macros.
%    \begin{macrocode}
    \prg_new_conditional:Npnn \token_if_protected_macro:N #1
      { p , T ,  F , TF }
      {
        \__str_if_eq_x_return:nn
          {
            \exp_after:wN \token_if_protected_macro_aux:w
              \token_to_meaning:N #1 PROYECYEZ~MACRO \q_stop
          }
          { \token_to_str:N \  }
      }
    \cs_new:Npn \token_if_protected_macro_aux:w
      #1 PROYECYEZ~MACRO #2 \q_stop { #1 ~ }
%    \end{macrocode}
%   Long macros and protected long macros share an auxiliary.
%    \begin{macrocode}
    \prg_new_conditional:Npnn \token_if_long_macro:N #1 { p , T ,  F , TF }
      {
        \__str_if_eq_x_return:nn
          {
            \exp_after:wN \token_if_long_macro_aux:w
              \token_to_meaning:N #1 LOXG~MACRO \q_stop
          }
          { \token_to_str:N \  }
      }
    \prg_new_conditional:Npnn \token_if_protected_long_macro:N #1
      { p , T ,  F , TF }
      {
        \__str_if_eq_x_return:nn
          {
            \exp_after:wN \token_if_long_macro_aux:w
              \token_to_meaning:N #1 LOXG~MACRO \q_stop
          }
          { \token_to_str:N \protected \token_to_str:N \  }
      }
    \cs_new:Npn \token_if_long_macro_aux:w #1 LOXG~MACRO #2 \q_stop { #1 ~ }
%    \end{macrocode}
% Finally the |\tl_to_lowercase:n| ends!
%    \begin{macrocode}
  }
%    \end{macrocode}
% \end{macro}
% \end{macro}
%
% \begin{macro}[pTF]{\token_if_primitive:N}
% \begin{macro}[aux]{\token_if_primitive_aux:NNw,
%     \token_if_primitive_aux_space:w,
%     \token_if_primitive_aux_nullfont:N,
%     \token_if_primitive_aux_loop:N,
%     \token_if_primitive_auxii:Nw,
%     \token_if_primitive_aux_undefined:N}
%^^A See http://groups.google.com/group/comp.text.tex/browse_thread/thread/0a72666873f8753d#
%
%   We filter out macros first, because they cause endless trouble later
%   otherwise.
%
%   Primitives are almost distinguished by the fact that the result
%   of \cs{token_to_meaning:N} is formed from letters only. Every other
%   token has either a space (e.g., |the letter A|), a digit
%   (e.g., |\count123|) or a double quote (e.g., |\char"A|).
%
%   Ten exceptions: on the one hand, \cs{c_undefined:D} is not a
%   primitive, but its meaning is |undefined|, only letters;
%   on the other hand, \tn{space}, \tn{italiccorr},
%   \tn{hyphen}, \tn{firstmark}, \tn{topmark},
%   \tn{botmark}, \tn{splitfirstmark}, \tn{splitbotmark},
%   and \tn{nullfont} are primitives, but have non-letters
%   in their meaning.
%
%   We start by removing the two first (non-space) characters from
%   the meaning. This removes the escape character (which may be
%   inexistent depending on \tn{endlinechar}), and takes care
%   of three of the exceptions: \tn{space}, \tn{italiccorr}
%   and \tn{hyphen}, whose meaning is at most two characters.
%   This leaves a string terminated by some |:|, and \cs{q_stop}.
%
%   The meaning of each one of the five \tn{...mark} primitives
%   has the form \meta{letters}|:|\meta{user material}. In other words,
%   the first non-letter is a colon. We remove everything after the first
%   colon.
%
%   We are now left with a string, which we must analyze. For primitives,
%   it contains only letters. For non-primitives, it contains either
%   |"|, or a space, or a digit. Two exceptions remain: \cs{c_undefined:D},
%   which is not a primitive, and \tn{nullfont}, which is a primitive.
%
%   Spaces cannot be grabbed in an undelimited way, so we check them
%   separately. If there is a space, we test for \tn{nullfont}.
%   Otherwise, we go through characters one by one, and stop at the
%   first character less than |`A| (this is not quite a test for
%   \enquote{only letters}, but is close enough to work in this context).
%   If this first character is |:| then we have a primitive, or
%   \cs{c_undefined:D}, and if it is |"| or a digit, then the token
%   is not a primitive.
%
%    \begin{macrocode}
\tex_chardef:D \c_token_A_int = `A ~ %
\group_begin:
\char_set_catcode_other:N \;
\char_set_lccode:nn { `\; } { `\: }
\char_set_lccode:nn { `\T } { `\T }
\char_set_lccode:nn { `\F } { `\F }
\tl_to_lowercase:n {
  \group_end:
  \prg_new_conditional:Npnn \token_if_primitive:N #1 { p , T , F , TF }
    {
      \token_if_macro:NTF #1
        \prg_return_false:
        {
          \exp_after:wN \token_if_primitive_aux:NNw
          \token_to_meaning:N #1 ; ; ; \q_stop #1
        }
    }
  \cs_new:Npn \token_if_primitive_aux:NNw #1#2 #3 ; #4 \q_stop
    {
      \tl_if_empty:oTF { \token_if_primitive_aux_space:w #3 ~ }
        { \token_if_primitive_aux_loop:N #3 ; \q_stop }
        { \token_if_primitive_aux_nullfont:N }
    }
}
\cs_new:Npn \token_if_primitive_aux_space:w #1 ~ { }
\cs_new:Npn \token_if_primitive_aux_nullfont:N #1
  {
    \if_meaning:w \tex_nullfont:D #1
      \prg_return_true:
    \else:
      \prg_return_false:
    \fi:
  }
\cs_new:Npn \token_if_primitive_aux_loop:N #1
  {
    \if_int_compare:w `#1 < \c_token_A_int %
      \exp_after:wN \token_if_primitive_auxii:Nw
      \exp_after:wN #1
    \else:
      \exp_after:wN \token_if_primitive_aux_loop:N
    \fi:
  }
\cs_new:Npn \token_if_primitive_auxii:Nw #1 #2 \q_stop
  {
    \if:w : #1
      \exp_after:wN \token_if_primitive_aux_undefined:N
    \else:
      \prg_return_false:
      \exp_after:wN \use_none:n
    \fi:
  }
\cs_new:Npn \token_if_primitive_aux_undefined:N #1
  {
    \if_cs_exist:N #1
      \prg_return_true:
    \else:
      \prg_return_false:
    \fi:
  }
%    \end{macrocode}
% \end{macro}
% \end{macro}
%
% \subsection{Peeking ahead at the next token}
%
% Peeking ahead is implemented using a two part mechanism. The
% outer level provides a defined interface to the lower level material.
% This allows a large amount of code to be shared. There are four
% cases:
% \begin{enumerate}
%   \item peek at the next token;
%   \item peek at the next non-space token;
%   \item peek at the next token and remove it;
%   \item peek at the next non-space token and remove it.
% \end{enumerate}
%
% \begin{variable}{\l_peek_token}
% \begin{variable}{\g_peek_token}
%   Storage tokens which are publicly documented: the token peeked.
%    \begin{macrocode}
\cs_new_eq:NN \l_peek_token ?
\cs_new_eq:NN \g_peek_token ?
%    \end{macrocode}
% \end{variable}
% \end{variable}
%
% \begin{variable}{\l_peek_search_token}
%   The token to search for as an implicit token:
%   \emph{cf.}~\cs{l_peek_search_tl}.
%    \begin{macrocode}
\cs_new_eq:NN \l_peek_search_token ?
%    \end{macrocode}
% \end{variable}
%
% \begin{variable}{\l_peek_search_tl}
%   The token to search for as an explicit token:
%   \emph{cf.}~\cs{l_peek_search_token}.
%    \begin{macrocode}
\tl_new:N \l_peek_search_tl
%    \end{macrocode}
% \end{variable}
%
% \begin{macro}[aux]
%   {\peek_true:w, \peek_true_aux:w, \peek_false:w, \peek_tmp:w}
%   Functions used by the branching and space-stripping code.
%    \begin{macrocode}
\cs_new_nopar:Npn \peek_true:w  { }
\cs_new_nopar:Npn \peek_true_aux:w  { }
\cs_new_nopar:Npn \peek_false:w { }
\cs_new:Npn \peek_tmp:w { }
%    \end{macrocode}
% \end{macro}
%
% \begin{macro}{\peek_after:Nw}
% \begin{macro}{\peek_after:Nw}
%   Simple wrappers for \tn{futurelet}: no arguments absorbed
%   here.
%    \begin{macrocode}
\cs_new_protected_nopar:Npn \peek_after:Nw
  { \tex_futurelet:D \l_peek_token }
\cs_new_protected_nopar:Npn \peek_gafter:Nw
  { \tex_global:D \tex_futurelet:D \g_peek_token }
%    \end{macrocode}
% \end{macro}
% \end{macro}
%
% \begin{macro}[aux]{\peek_true_remove:w}
%   A function to remove the next token and then regain control.
%    \begin{macrocode}
\cs_new_protected:Npn \peek_true_remove:w
  {
    \group_align_safe_end:
    \tex_afterassignment:D \peek_true_aux:w
    \cs_set_eq:NN \peek_tmp:w
  }
%    \end{macrocode}
% \end{macro}
%
% \begin{macro}[TF]{\peek_token_generic:NN}
%   The generic function stores the test token in both implicit and
%   explicit modes, and the \texttt{true} and \texttt{false} code as
%   token lists, more or less. The two branches have to be absorbed here
%   as the input stream needs to be cleared for the peek function itself.
%    \begin{macrocode}
\cs_new_protected:Npn \peek_token_generic:NNTF #1#2#3#4
  {
    \cs_set_eq:NN \l_peek_search_token #2
    \tl_set:Nn \l_peek_search_tl {#2}
    \cs_set_nopar:Npx \peek_true:w
      {
        \exp_not:N \group_align_safe_end:
        \exp_not:n {#3}
      }
    \cs_set_nopar:Npx \peek_false:w
      {
        \exp_not:N \group_align_safe_end:
        \exp_not:n {#4}
      }
    \group_align_safe_begin:
      \peek_after:Nw #1
  }
\cs_new_protected:Npn \peek_token_generic:NNT #1#2#3
  { \peek_token_generic:NNTF #1 #2 {#3} { } }
\cs_new_protected:Npn \peek_token_generic:NNF #1#2#3
  { \peek_token_generic:NNTF #1 #2 { } {#3} }
%    \end{macrocode}
% \end{macro}
%
% \begin{macro}[TF]{\peek_token_remove_generic:NN}
%   For token removal there needs to be a call to the auxiliary
%   function which does the work.
%    \begin{macrocode}
\cs_new_protected:Npn \peek_token_remove_generic:NNTF #1#2#3#4
  {
    \cs_set_eq:NN \l_peek_search_token #2
    \tl_set:Nn \l_peek_search_tl {#2}
    \cs_set_eq:NN \peek_true:w \peek_true_remove:w
    \cs_set_nopar:Npx \peek_true_aux:w { \exp_not:n {#3} }
    \cs_set_nopar:Npx \peek_false:w
      {
        \exp_not:N \group_align_safe_end:
        \exp_not:n {#4}
      }
    \group_align_safe_begin:
      \peek_after:Nw #1
  }
\cs_new_protected:Npn \peek_token_remove_generic:NNT #1#2#3
  { \peek_token_remove_generic:NNTF #1 #2 {#3} { } }
\cs_new_protected:Npn \peek_token_remove_generic:NNF #1#2#3
  { \peek_token_remove_generic:NNTF #1 #2 { } {#3} }
%    \end{macrocode}
% \end{macro}
%
% \begin{macro}
%   {\peek_execute_branches_catcode:, \peek_execute_branches_meaning:}
%   The category code and meaning tests are straight forward.
%    \begin{macrocode}
\cs_new_nopar:Npn \peek_execute_branches_catcode:
  {
    \if_catcode:w
      \exp_not:N \l_peek_token \exp_not:N \l_peek_search_token
      \exp_after:wN \peek_true:w
    \else:
      \exp_after:wN \peek_false:w
    \fi:
  }
\cs_new_nopar:Npn \peek_execute_branches_meaning:
  {
    \if_meaning:w \l_peek_token \l_peek_search_token
      \exp_after:wN \peek_true:w
    \else:
      \exp_after:wN \peek_false:w
    \fi:
  }
%    \end{macrocode}
% \end{macro}
%
% \begin{macro}{\peek_execute_branches_charcode:}
% \begin{macro}[aux]{\peek_execute_branches_charcode:NN}
%   First the character code test there is a need to worry about \TeX{}
%   grabbing brace group or skipping spaces. These are all tested for
%   using a category code check before grabbing what must be a real
%   single token and doing the comparison.
%    \begin{macrocode}
\cs_new_nopar:Npn \peek_execute_branches_charcode:
  {
    \bool_if:nTF
      {
           \token_if_eq_catcode_p:NN \l_peek_token \c_group_begin_token
        || \token_if_eq_catcode_p:NN \l_peek_token \c_group_end_token
        || \token_if_eq_meaning_p:NN \l_peek_token \c_space_token
      }
      { \peek_false:w }
      {
        \exp_after:wN \peek_execute_branches_charcode_aux:NN
          \l_peek_search_tl
      }
  }
\cs_new:Npn \peek_execute_branches_charcode_aux:NN #1#2
  {
    \if:w \exp_not:N #1 \exp_not:N #2
      \exp_after:wN \peek_true:w
    \else:
      \exp_after:wN \peek_false:w
    \fi:
    #2
  }
%    \end{macrocode}
% \end{macro}
% \end{macro}
%
% \begin{macro}{\peek_ignore_spaces_execute_branches:}
% \begin{macro}[aux]{\peek_ignore_spaces_execute_branches_aux:}
%   This function removes one token at a time with a mechanism that can
%   be applied to things other than spaces.
%    \begin{macrocode}
\cs_new_protected_nopar:Npn \peek_ignore_spaces_execute_branches:
  {
    \token_if_eq_meaning:NNTF \l_peek_token \c_space_token
      {
        \tex_afterassignment:D \peek_ignore_spaces_execute_branches_aux:
        \cs_set_eq:NN \peek_tmp:w
      }
      { \peek_execute_branches: }
  }
\cs_new_protected_nopar:Npn \peek_ignore_spaces_execute_branches_aux:
  { \peek_after:Nw \peek_ignore_spaces_execute_branches: }
%    \end{macrocode}
% \end{macro}
% \end{macro}
%
% \begin{macro}[aux]{\peek_def:nnnn}
% \begin{macro}[aux]{\peek_def_aux:nnnnn}
%   The public functions themselves cannot be defined using
%   \cs{prg_new_conditional:Npnn} and so a couple of auxiliary functions
%   are used. As a result, everything is done inside a group. As a result
%   things are a bit complicated.
%    \begin{macrocode}
\group_begin:
  \cs_set:Npn \peek_def:nnnn #1#2#3#4
    {
      \peek_def_aux:nnnnn {#1} {#2} {#3} {#4} { TF }
      \peek_def_aux:nnnnn {#1} {#2} {#3} {#4} { T }
      \peek_def_aux:nnnnn {#1} {#2} {#3} {#4} { F }
    }
  \cs_set:Npn \peek_def_aux:nnnnn #1#2#3#4#5
    {
      \cs_new_nopar:cpx { #1 #5 }
        {
          \tl_if_empty:nF {#2}
            { \exp_not:n { \cs_set_eq:NN \peek_execute_branches: #2 } }
          \exp_not:c { #3 #5 }
          \exp_not:n {#4}
        }
    }
%    \end{macrocode}
% \end{macro}
% \end{macro}
% \begin{macro}[TF]
%   {
%     \peek_catcode:N, \peek_catcode_ignore_spaces:N,
%     \peek_catcode_remove:N, \peek_catcode_remove_ignore_spaces:N
%   }
%   With everything in place the definitions can take place. First for
%   category codes.
%    \begin{macrocode}
  \peek_def:nnnn { peek_catcode:N }
    { }
    { peek_token_generic:NN }
    { \peek_execute_branches_catcode: }
  \peek_def:nnnn { peek_catcode_ignore_spaces:N }
    { \peek_execute_branches_catcode: }
    { peek_token_generic:NN }
    { \peek_ignore_spaces_execute_branches: }
  \peek_def:nnnn { peek_catcode_remove:N }
    { }
    { peek_token_remove_generic:NN }
    { \peek_execute_branches_catcode: }
  \peek_def:nnnn { peek_catcode_remove_ignore_spaces:N }
    { \peek_execute_branches_catcode: }
    { peek_token_remove_generic:NN }
    { \peek_ignore_spaces_execute_branches: }
%    \end{macrocode}
% \end{macro}
% \begin{macro}[TF]
%   {
%     \peek_charcode:N, \peek_charcode_ignore_spaces:N,
%     \peek_charcode_remove:N, \peek_charcode_remove_ignore_spaces:N
%   }
%   Then for character codes.
%    \begin{macrocode}
  \peek_def:nnnn { peek_charcode:N }
    { }
    { peek_token_generic:NN }
    { \peek_execute_branches_charcode: }
  \peek_def:nnnn { peek_charcode_ignore_spaces:N }
    { \peek_execute_branches_charcode: }
    { peek_token_generic:NN }
    { \peek_ignore_spaces_execute_branches: }
  \peek_def:nnnn { peek_charcode_remove:N }
    { }
    { peek_token_remove_generic:NN }
    { \peek_execute_branches_charcode: }
  \peek_def:nnnn { peek_charcode_remove_ignore_spaces:N }
    { \peek_execute_branches_charcode: }
    { peek_token_remove_generic:NN }
    { \peek_ignore_spaces_execute_branches: }
%    \end{macrocode}
% \end{macro}
% \begin{macro}[TF]
%   {
%     \peek_meaning:N, \peek_meaning_ignore_spaces:N,
%     \peek_meaning_remove:N, \peek_meaning_remove_ignore_spaces:N
%   }
%   Finally for meaning, with the group closed to remove the temporary
%   definition functions.
%    \begin{macrocode}
  \peek_def:nnnn { peek_meaning:N }
    { }
    { peek_token_generic:NN }
    { \peek_execute_branches_meaning: }
  \peek_def:nnnn { peek_meaning_ignore_spaces:N }
    { \peek_execute_branches_meaning: }
    { peek_token_generic:NN }
    { \peek_ignore_spaces_execute_branches: }
  \peek_def:nnnn { peek_meaning_remove:N }
    { }
    { peek_token_remove_generic:NN }
    { \peek_execute_branches_meaning: }
  \peek_def:nnnn { peek_meaning_remove_ignore_spaces:N }
    { \peek_execute_branches_meaning: }
    { peek_token_remove_generic:NN }
    { \peek_ignore_spaces_execute_branches: }
\group_end:
%    \end{macrocode}
% \end{macro}
%
% \subsection{Decomposing a macro definition}
%
% \begin{macro}{\token_get_prefix_spec:N}
% \begin{macro}{\token_get_arg_spec:N}
% \begin{macro}{\token_get_replacement_spec:N}
% \begin{macro}[aux]{\token_get_prefix_arg_replacement_aux:wN}
%   We sometimes want to test if a
%   control sequence can be expanded to reveal a hidden
%   value. However, we cannot just expand the macro blindly as it may
%   have arguments and none might be present. Therefore we define
%   these functions to pick either the prefix(es), the argument
%   specification, or the replacement text from a macro. All of this
%   information is returned as characters with catcode~$12$. If the
%   token in question isn't a macro, the token |\scan_stop:| is
%   returned instead.
%    \begin{macrocode}
\exp_args:Nno \use:nn
  { \cs_new:Npn \token_get_prefix_arg_replacement_aux:wN #1 }
  { \tl_to_str:n { macro : } #2 -> #3 \q_stop #4 }
  { #4 {#1} {#2} {#3} }
\cs_new:Npn \token_get_prefix_spec:N #1
  {
    \token_if_macro:NTF #1
      {
        \exp_after:wN \token_get_prefix_arg_replacement_aux:wN
          \token_to_meaning:N #1 \q_stop \use_i:nnn
      }
      { \scan_stop: }
  }
\cs_new:Npn \token_get_arg_spec:N #1
  {
    \token_if_macro:NTF #1
      {
        \exp_after:wN \token_get_prefix_arg_replacement_aux:wN
          \token_to_meaning:N #1 \q_stop \use_ii:nnn
      }
      { \scan_stop: }
  }
\cs_new:Npn \token_get_replacement_spec:N #1
  {
    \token_if_macro:NTF #1
      {
        \exp_after:wN \token_get_prefix_arg_replacement_aux:wN
          \token_to_meaning:N #1 \q_stop \use_iii:nnn
      }
      { \scan_stop: }
  }
%    \end{macrocode}
% \end{macro}
% \end{macro}
% \end{macro}
% \end{macro}
%
% \subsection{Deprecated functions}
%
% Deprecated on 2011-05-27, for removal by 2011-08-31.
%
% \begin{macro}{\char_set_catcode:w}
% \begin{macro}{\char_set_mathcode:w}
% \begin{macro}{\char_set_lccode:w}
% \begin{macro}{\char_set_uccode:w}
% \begin{macro}{\char_set_sfcode:w}
%   Primitives renamed.
%    \begin{macrocode}
%<*deprecated>
\cs_new_eq:NN \char_set_catcode:w  \tex_catcode:D
\cs_new_eq:NN \char_set_mathcode:w \tex_mathcode:D
\cs_new_eq:NN \char_set_lccode:w   \tex_lccode:D
\cs_new_eq:NN \char_set_uccode:w   \tex_uccode:D
\cs_new_eq:NN \char_set_sfcode:w   \tex_sfcode:D
%</deprecated>
%    \end{macrocode}
% \end{macro}
% \end{macro}
% \end{macro}
% \end{macro}
% \end{macro}
%
% \begin{macro}{\char_value_catcode:w}
% \begin{macro}{\char_show_value_catcode:w}
% \begin{macro}{\char_value_mathcode:w}
% \begin{macro}{\char_show_value_mathcode:w}
% \begin{macro}{\char_value_lccode:w}
% \begin{macro}{\char_show_value_lccode:w}
% \begin{macro}{\char_value_uccode:w}
% \begin{macro}{\char_show_value_uccode:w}
% \begin{macro}{\char_value_sfcode:w}
% \begin{macro}{\char_show_value_sfcode:w}
%   More |w| functions we should not have.
%    \begin{macrocode}
%<*deprecated>
\cs_new_nopar:Npn \char_value_catcode:w { \tex_the:D \char_set_catcode:w }
\cs_new_nopar:Npn \char_show_value_catcode:w
  { \tex_showthe:D \char_set_catcode:w }
\cs_new_nopar:Npn \char_value_mathcode:w { \tex_the:D \char_set_mathcode:w }
\cs_new_nopar:Npn \char_show_value_mathcode:w
  { \tex_showthe:D \char_set_mathcode:w }
\cs_new_nopar:Npn \char_value_lccode:w { \tex_the:D \char_set_lccode:w }
\cs_new_nopar:Npn \char_show_value_lccode:w
  { \tex_showthe:D \char_set_lccode:w }
\cs_new_nopar:Npn \char_value_uccode:w { \tex_the:D \char_set_uccode:w }
\cs_new_nopar:Npn \char_show_value_uccode:w
  { \tex_showthe:D \char_set_uccode:w }
\cs_new_nopar:Npn \char_value_sfcode:w { \tex_the:D \char_set_sfcode:w }
\cs_new_nopar:Npn \char_show_value_sfcode:w
  { \tex_showthe:D \char_set_sfcode:w }
%</deprecated>
%    \end{macrocode}
% \end{macro}
% \end{macro}
% \end{macro}
% \end{macro}
% \end{macro}
% \end{macro}
% \end{macro}
% \end{macro}
% \end{macro}
% \end{macro}
%
% \begin{macro}{\peek_after:NN}
% \begin{macro}{\peek_gafter:NN}
%   The second argument here must be |w|.
%    \begin{macrocode}
%<*deprecated>
\cs_new_eq:NN \peek_after:NN  \peek_after:Nw
\cs_new_eq:NN \peek_gafter:NN \peek_gafter:Nw
%</deprecated>
%    \end{macrocode}
% \end{macro}
% \end{macro}
%
% Functions deprecated 2011-05-28 for removal by 2011-08-31.
%
% \begin{macro}{\c_alignment_tab_token}
% \begin{macro}{\c_math_shift_token}
% \begin{macro}{\c_letter_token}
% \begin{macro}{\c_other_char_token}
%    \begin{macrocode}
%<*deprecated>
\cs_new_eq:NN \c_alignment_tab_token \c_alignment_token
\cs_new_eq:NN \c_math_shift_token    \c_math_toggle_token
\cs_new_eq:NN \c_letter_token        \c_catcode_letter_token
\cs_new_eq:NN \c_other_char_token    \c_catcode_other_token
%</deprecated>
%    \end{macrocode}
% \end{macro}
% \end{macro}
% \end{macro}
% \end{macro}
%
% \begin{macro}{\c_active_char_token}
%   An odd one: this was never a |token|!
%    \begin{macrocode}
%<*deprecated>
\cs_new_eq:NN \c_active_char_token \c_catcode_active_tl
%</deprecated>
%    \end{macrocode}
% \end{macro}
%
% \begin{macro}
%   {
%     \char_make_escape:N           ,
%     \char_make_group_begin:N      ,
%     \char_make_group_end:N        ,
%     \char_make_math_toggle:N      ,
%     \char_make_alignment:N        ,
%     \char_make_end_line:N         ,
%     \char_make_parameter:N        ,
%     \char_make_math_superscript:N ,
%     \char_make_math_subscript:N   ,
%     \char_make_ignore:N           ,
%     \char_make_space:N            ,
%     \char_make_letter:N           ,
%     \char_make_other:N            ,
%     \char_make_active:N           ,
%     \char_make_comment:N          ,
%     \char_make_invalid:N
%   }
% \begin{macro}
%   {
%     \char_make_escape:n           ,
%     \char_make_group_begin:n      ,
%     \char_make_group_end:n        ,
%     \char_make_math_toggle:n      ,
%     \char_make_alignment:n        ,
%     \char_make_end_line:n         ,
%     \char_make_parameter:n        ,
%     \char_make_math_superscript:n ,
%     \char_make_math_subscript:n   ,
%     \char_make_ignore:n           ,
%     \char_make_space:n            ,
%     \char_make_letter:n           ,
%     \char_make_other:n            ,
%     \char_make_active:n           ,
%     \char_make_comment:n          ,
%     \char_make_invalid:n
%   }
%  Two renames in one block!
%    \begin{macrocode}
%<*deprecated>
\cs_new_eq:NN \char_make_escape:N           \char_set_catcode_escape:N
\cs_new_eq:NN \char_make_begin_group:N      \char_set_catcode_group_begin:N
\cs_new_eq:NN \char_make_end_group:N        \char_set_catcode_group_end:N
\cs_new_eq:NN \char_make_math_shift:N       \char_set_catcode_math_toggle:N
\cs_new_eq:NN \char_make_alignment_tab:N    \char_set_catcode_alignment:N
\cs_new_eq:NN \char_make_end_line:N         \char_set_catcode_end_line:N
\cs_new_eq:NN \char_make_parameter:N        \char_set_catcode_parameter:N
\cs_new_eq:NN \char_make_math_superscript:N
  \char_set_catcode_math_superscript:N
\cs_new_eq:NN \char_make_math_subscript:N
  \char_set_catcode_math_subscript:N
\cs_new_eq:NN \char_make_ignore:N           \char_set_catcode_ignore:N
\cs_new_eq:NN \char_make_space:N            \char_set_catcode_space:N
\cs_new_eq:NN \char_make_letter:N           \char_set_catcode_letter:N
\cs_new_eq:NN \char_make_other:N            \char_set_catcode_other:N
\cs_new_eq:NN \char_make_active:N           \char_set_catcode_active:N
\cs_new_eq:NN \char_make_comment:N          \char_set_catcode_comment:N
\cs_new_eq:NN \char_make_invalid:N          \char_set_catcode_invalid:N
\cs_new_eq:NN \char_make_escape:n           \char_set_catcode_escape:n
\cs_new_eq:NN \char_make_begin_group:n      \char_set_catcode_group_begin:n
\cs_new_eq:NN \char_make_end_group:n        \char_set_catcode_group_end:n
\cs_new_eq:NN \char_make_math_shift:n       \char_set_catcode_math_toggle:n
\cs_new_eq:NN \char_make_alignment_tab:n    \char_set_catcode_alignment:n
\cs_new_eq:NN \char_make_end_line:n         \char_set_catcode_end_line:n
\cs_new_eq:NN \char_make_parameter:n        \char_set_catcode_parameter:n
\cs_new_eq:NN \char_make_math_superscript:n
  \char_set_catcode_math_superscript:n
\cs_new_eq:NN \char_make_math_subscript:n
  \char_set_catcode_math_subscript:n
\cs_new_eq:NN \char_make_ignore:n           \char_set_catcode_ignore:n
\cs_new_eq:NN \char_make_space:n            \char_set_catcode_space:n
\cs_new_eq:NN \char_make_letter:n           \char_set_catcode_letter:n
\cs_new_eq:NN \char_make_other:n            \char_set_catcode_other:n
\cs_new_eq:NN \char_make_active:n           \char_set_catcode_active:n
\cs_new_eq:NN \char_make_comment:n          \char_set_catcode_comment:n
\cs_new_eq:NN \char_make_invalid:n          \char_set_catcode_invalid:n
%</deprecated>
%    \end{macrocode}
% \end{macro}
% \end{macro}
%
% \begin{macro}[pTF]{\token_if_alignment_tab:N}
% \begin{macro}[pTF]{\token_if_math_shift:N}
% \begin{macro}[pTF]{\token_if_other_char:N}
% \begin{macro}[pTF]{\token_if_active_char:N}
%    \begin{macrocode}
%<*deprecated>
\cs_new_eq:NN \token_if_alignment_tab_p:N \token_if_alignment_p:N
\cs_new_eq:NN \token_if_alignment_tab:NT  \token_if_alignment:NT
\cs_new_eq:NN \token_if_alignment_tab:NF  \token_if_alignment:NF
\cs_new_eq:NN \token_if_alignment_tab:NTF \token_if_alignment:NTF
\cs_new_eq:NN \token_if_math_shift_p:N \token_if_math_toggle_p:N
\cs_new_eq:NN \token_if_math_shift:NT  \token_if_math_toggle:NT
\cs_new_eq:NN \token_if_math_shift:NF  \token_if_math_toggle:NF
\cs_new_eq:NN \token_if_math_shift:NTF \token_if_math_toggle:NTF
\cs_new_eq:NN \token_if_other_char_p:N \token_if_other_p:N
\cs_new_eq:NN \token_if_other_char:NT  \token_if_other:NT
\cs_new_eq:NN \token_if_other_char:NF  \token_if_other:NF
\cs_new_eq:NN \token_if_other_char:NTF \token_if_other:NTF
\cs_new_eq:NN \token_if_active_char_p:N \token_if_active_p:N
\cs_new_eq:NN \token_if_active_char:NT  \token_if_active:NT
\cs_new_eq:NN \token_if_active_char:NF  \token_if_active:NF
\cs_new_eq:NN \token_if_active_char:NTF \token_if_active:NTF
%</deprecated>
%    \end{macrocode}
% \end{macro}
% \end{macro}
% \end{macro}
% \end{macro}
%
%    \begin{macrocode}
%</initex|package>
%    \end{macrocode}
%
% \end{implementation}
%
% \PrintIndex

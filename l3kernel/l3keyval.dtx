% \iffalse meta-comment
%
%% File: l3keyval.dtx Copyright (C) 2006-2011 The LaTeX3 Project
%%
%% It may be distributed and/or modified under the conditions of the
%% LaTeX Project Public License (LPPL), either version 1.3c of this
%% license or (at your option) any later version.  The latest version
%% of this license is in the file
%%
%%    http://www.latex-project.org/lppl.txt
%%
%% This file is part of the "expl3 bundle" (The Work in LPPL)
%% and all files in that bundle must be distributed together.
%%
%% The released version of this bundle is available from CTAN.
%%
%% -----------------------------------------------------------------------
%%
%% The development version of the bundle can be found at
%%
%%    http://www.latex-project.org/svnroot/experimental/trunk/
%%
%% for those people who are interested.
%%
%%%%%%%%%%%
%% NOTE: %%
%%%%%%%%%%%
%%
%%   Snapshots taken from the repository represent work in progress and may
%%   not work or may contain conflicting material!  We therefore ask
%%   people _not_ to put them into distributions, archives, etc. without
%%   prior consultation with the LaTeX3 Project.
%%
%% -----------------------------------------------------------------------
%
%<*driver|package>
\RequirePackage{l3names}
\GetIdInfo$Id$
  {L3 Experimental key-value parsing}
%</driver|package>
%<*driver>
\documentclass[full]{l3doc}
\begin{document}
  \DocInput{\jobname.dtx}
\end{document}
%</driver>
% \fi
%
% \title{^^A
%   The \pkg{l3keyval} package\\ Key--value parsing^^A
%   \thanks{This file describes v\fileversion, last revised \filedate.}^^A
% }
%
% \author{^^A
%  The \LaTeX3 Project\thanks
%    {^^A
%      E-mail:
%        \href{mailto:latex-team@latex-project.org}
%          {latex-team@latex-project.org}^^A
%    }^^A
% }
%
% \date{Released \filedate}
%
% \maketitle
%
% \begin{documentation}
% 
% A key--value list is input of the form
% \begin{verbatim}
%   KeyOne = ValueOne ,
%   KeyTwo = ValueTwo ,
%   KeyThree   
% \end{verbatim} 
% where each key--value pair is separated by a comma from the rest of
% the list, and each key--value pair does not necessarily contain an
% equals sign or a value! Processing this type of input correctly 
% requires a number of careful steps, to correctly account for
% braces, spaces and the category codes of separators.
% 
% This module provides the low-level machinery for processing arbitrary
% key--value lists. The \pkg{l3keys} module provides a higher-level
% interface for managing run-time settings using key--value input,
% while other parts of \LaTeX3 also use key--value input based on
% \pkg{l3keyval} (for example the \pkg{xtemplate} module).
% 
 %\section{Parsing key--value lists}
%
% The low-level parsing system converts a \meta{key--value list}
% into \meta{keys} and associated \meta{values}. After the parsing phase
% is completed, the resulting keys and values (or keys alone) are
% available for further processing. This is not carried out by the
% low-level parser, and so the parser requires the names of
% two functions along with the key--value list. One function is
% needed to process key--value pairs (\emph{i.e}~two arguments), 
% and a second function if required for keys given without arguments
% (\emph{i.e.}~a single argument).
% 
% The parser does not double |#| tokens or expand any input. The tokens
% |=| and |,| are corrected so that the parser does not \enquote{miss}
% any due to category code changes. Spaces are removed from the ends
% of the keys and values. Values which are given in braces will have
% exactly one set removed, thus
% \begin{verbatim}
%    key = {value here},
% \end{verbatim}
% and
% \begin{verbatim}
%   key = value here,
% \end{verbatim}
% are treated identically.
% 
 %\begin{function}{\keyval_parse:NNn}
%   \begin{syntax}
%     \cs{keyval_parse:NNn} \meta{function1} \meta{function2}
%     ~~\Arg{key--value list}
%   \end{syntax}
%   Parses the \meta{key--value list} into a series of \meta{keys} and
%   associated \meta{values}, or keys alone (if no \meta{value} was
%   given).  \meta{function1} should take one argument, while
%   \meta{function2} should absorb two arguments. After
%   \cs{keyval_parse:NNn} has parsed the \meta{key--value list}, 
%   \meta{function1} will be used to process keys given with no value
%   and \meta{function2} will be used to process keys given with a
%   value. The order of the \meta{keys} in the \meta{key--value list}
%   will be preserved. Thus
%   \begin{verbatim}
%     \keyval_parse:NNn \function:n \function:nn
%       { key1 = value1 , key2 = value2, key3 = , key4 }
%   \end{verbatim}
%   will be converted into an input stream
%   \begin{verbatim}
%     \function:nn { key1 } { value1 }
%     \function:nn { key2 } { value2 }
%     \function:nn { key3 } { }
%     \function:n  { key4 }
%   \end{verbatim}
%   Note that there is a difference between an empty value (an equals
%   sign followed by nothing) and a missing value (no equals sign at
%   all). 
% \end{function}
%
% \end{documentation}
%
% \begin{implementation}
%
% \section{\pkg{l3keyval} implementation}
%
%    \begin{macrocode}
%<*initex|package>
%    \end{macrocode}
%    
%    \begin{macrocode}
%<*package>
\ProvidesExplPackage
  {\filename}{\filedate}{\fileversion}{\filedescription}
\package_check_loaded_expl:
%</package>
%    \end{macrocode}
%    
% \begin{variable}{\g_keyval_level_int}
%   For nesting purposes an integer is needed for the current level.
%    \begin{macrocode}
\int_new:N \g_keyval_level_int
%    \end{macrocode}
% \end{variable}
%
% \begin{variable}{\l_keyval_key_tl, \l_keyval_value_tl}
%   The current key name and value.
%    \begin{macrocode}
\tl_new:N \l_keyval_key_tl
\tl_new:N \l_keyval_value_tl
%    \end{macrocode}
% \end{variable}
%
% \begin{variable}{\l_keyval_sanitise_tl}
% \begin{variable}{\l_keyval_parse_tl}
%   Token list variables for dealing with awkward category codes in the
%   input.
%    \begin{macrocode}
\tl_new:N \l_keyval_sanitise_tl
\tl_new:N \l_keyval_parse_tl
%    \end{macrocode}
%\end{variable}
%\end{variable}
%
% \begin{macro}{\keyval_parse:n}
%   The parsing function first deals with the category codes for
%   |=| and |,|, so that there are no odd events. The input is then
%   handed off to the element by element system. 
%    \begin{macrocode}
\group_begin:
  \char_make_active:n { `\= }
  \char_make_active:n { `\, }
  \char_set_lccode:nn { `\8 } { `\= }
  \char_set_lccode:nn { `\9 } { `\, }
\tl_to_lowercase:n
  {
    \group_end:
    \cs_new_protected:Npn \keyval_parse:n #1
      {
        \group_begin:
          \tl_clear:N \l_keyval_sanitise_tl
          \tl_set:Nn \l_keyval_sanitise_tl {#1}
          \tl_replace_all_in:Nnn \l_keyval_sanitise_tl { = } { 8 }
          \tl_replace_all_in:Nnn \l_keyval_sanitise_tl { , } { 9 }
          \tl_clear:N \l_keyval_parse_tl
          \exp_after:wN \keyval_parse_elt:w \exp_after:wN
            \q_no_value \l_keyval_sanitise_tl 9 \q_nil 9
        \exp_after:wN \group_end:
        \l_keyval_parse_tl
      }
  }
%    \end{macrocode}
% \end{macro}
%
% \begin{macro}{\keyval_parse_elt:w}
%   Each item to be parsed will have \cs{q_no_value} added to the front.
%   Hence the blank test here can always be used to find a totally
%   empty argument. If this is the case, the system loops round. If there
%   is something to parse, there is a check for the \cs{q_nil} marker
%   and if not a hand-off.
%    \begin{macrocode}
\cs_new_protected:Npn \keyval_parse_elt:w #1 ,
  {
    \tl_if_blank:oTF { \use_none:n #1 }
      { \keyval_parse_elt:w \q_no_value }
      {
        \quark_if_nil:oF { \use_ii:nn #1 }
          {
            \keyval_split_key_value:w #1 = = \q_stop
            \keyval_parse_elt:w \q_no_value
          }
      }
  }
%    \end{macrocode}
% \end{macro}
%
% \begin{macro}{\keyval_split_key_value:w}
% \begin{macro}[aux]{\keyval_split_key_value_aux:wTF}
%   The key and value are handled separately. First the key is grabbed and
%   saved as \cs{l_keyval_key_tl}. Then a check is need to see if there is
%   a value at all: if not then the key name is simply added to the output.
%   If there is a value then there is a check to ensure that there was
%   only one |=| in the input (remembering some extra ones are around at
%   the moment to prevent errors). All being well, there is an
%   hand-off to find the value: the \cs{q_nil} is there to prevent loss
%   of braces.
%    \begin{macrocode}
\cs_new_protected:Npn \keyval_split_key_value:w #1 = #2 \q_stop
  {
    \keyval_split_key:w #1 \q_stop
    \str_if_eq:nnTF {#2} { = }
      {
        \tl_put_right:Nx \l_keyval_parse_tl
          {
            \exp_not:c { keyval_key_no_value_elt_ \int_use:N \g_keyval_level_int :n }
              { \exp_not:o \l_keyval_key_tl }
          }
      }
      {
        \keyval_split_key_value_aux:wTF #2 \q_no_value \q_stop
          { \keyval_split_value:w \q_nil #2 }
          { \msg_kernel_error:nn { keyval } { misplaced-equals-sign } }
      }
  }
\cs_new:Npn \keyval_split_key_value_aux:wTF #1 = #2#3 \q_stop
  { \tl_if_head_eq_meaning:nNTF {#3} \q_no_value }
%    \end{macrocode}
% \end{macro}
% \end{macro}
%
% \begin{macro}{\keyval_split_key:w}
% \begin{macro}{\keyval_remove_spaces:w}
% \begin{macro}[aux]{\keyval_split_key_aux:w}
% \begin{macro}[aux]{\keyval_remove_spaces_aux:w}
%   The aim here is to remove spaces and also exactly one set of braces.
%   The spaces are trimmed off from each end using a \enquote{funny}
%   |Q|, which will never turn up in normal use. The idea is that
%   the \texttt{f}-type expansion will stop if it finds an unexpandable
%   token or a space, and will gobble the space. To avoid expanding
%   anything else, the \cs{exp_not:N} works by ensuring that the first
%   non-space token in the setting will stop the \texttt{f}-type
%   expansion. The \cs{use_none:n} is needed to remove the leading
%   quark, while the second setting of \cs{l_keyval_key_tl}
%   removes exactly one set of braces.
%    \begin{macrocode}
\group_begin:
  \char_set_catcode:nn { `\Q } { 3 }
  \cs_new_protected:Npn \keyval_split_key:w #1 \q_stop
    {
      \exp_args:NNf \tl_set:Nn \l_keyval_key_tl
        {
          \exp_after:wN \keyval_remove_spaces:w \exp_after:wN
            \exp_not:N \use_none:n #1 Q ~ Q
        }
      \tl_set:Nx \l_keyval_key_tl
        { \exp_after:wN \keyval_split_key_aux:w \l_keyval_key_tl \q_stop }
    }
  \cs_gset:Npn \keyval_split_key_aux:w #1 \q_stop { \exp_not:n {#1} }
  \cs_gset:Npn \keyval_remove_spaces:w #1 ~ Q { \keyval_remove_spaces_aux:w #1 Q }
  \cs_gset:Npn \keyval_remove_spaces_aux:w #1 Q #2 {#1}
\group_end:
%    \end{macrocode}
% \end{macro}
% \end{macro}
% \end{macro}
% \end{macro}
%
% \begin{macro}{\keyval_split_value:w}
%   Here the value has to be separated from the equals signs and the
%   leading \cs{q_nil} added in to keep the brace levels. Fist the
%   processing function can be added to the output list. If there is no
%   value, setting \cs{l_keyval_value_tl} with three groups removed will
%   leave nothing at all, and so an empty group can be added to the
%   parsed list. On the other hand, if the value is entirely contained
%   within a set of braces then \cs{l_keyval_value_tl} will contain
%   \cs{q_nil} only. In that case, strip off the leading quark using
%   \cs{use_ii:nnn}, which also deals with any spaces.
%    \begin{macrocode}
\cs_new_protected:Npn \keyval_split_value:w #1 = =
  {
    \tl_put_right:Nx \l_keyval_parse_tl
      {
        \exp_not:c { keyval_key_value_elt_ \int_use:N \g_keyval_level_int :nn }
          { \exp_not:o \l_keyval_key_tl }
      }
    \tl_set:Nx \l_keyval_value_tl { \exp_not:o { \use_none:nnn #1 \q_nil \q_nil } }
    \tl_if_empty:NTF \l_keyval_value_tl
      { \tl_put_right:Nn \l_keyval_parse_tl { { } } }
      {
        \quark_if_nil:NTF \l_keyval_value_tl
          {
            \tl_put_right:Nx \l_keyval_parse_tl
              { { \exp_not:o { \use_ii:nnn #1 \q_nil } } }
          }
          { \keyval_split_value_aux:w #1 \q_stop }
      }
  }
%    \end{macrocode}
%   A similar idea to the key code: remove the spaces from each end and
%   deal with one set of braces.
%    \begin{macrocode}
\group_begin:
  \char_set_catcode:nn { `\Q } { 3 }
  \cs_new_protected:Npn \keyval_split_value_aux:w \q_nil #1 \q_stop
    {
      \exp_args:NNf \tl_set:Nn \l_keyval_value_tl
        { \keyval_remove_spaces:w \exp_not:N #1 Q ~ Q }
      \tl_put_right:Nx \l_keyval_parse_tl { { \exp_not:o \l_keyval_value_tl } }
    }
\group_end:
%    \end{macrocode}
% \end{macro}
%
% \begin{macro}{\keyval_parse:NNn}
%   The outer parsing routine just sets up the processing functions and
%   hands off.
%    \begin{macrocode}
\cs_new_protected:Npn \keyval_parse:NNn #1#2#3
  {
    \int_gincr:N \g_keyval_level_int
    \cs_gset_eq:cN { keyval_key_no_value_elt_ \int_use:N \g_keyval_level_int :n } #1
    \cs_gset_eq:cN { keyval_key_value_elt_ \int_use:N \g_keyval_level_int :nn }   #2
    \keyval_parse:n {#3}
    \int_gdecr:N \g_keyval_level_int
  }
%    \end{macrocode}
% \end{macro}
%
% One message for the low level parsing system.
%    \begin{macrocode}
\msg_kernel_new:nnnn { keyval } { misplaced-equals-sign }
  { Misplaced~equals~sign~in~key-value~input~\msg_line_number: }
  {
    LaTeX~is~attempting~to~parse~some~key-value~input~but~found~
    two~equals~signs~not~separated~by~a~comma.
  }
%    \end{macrocode}
%    
% \subsection{Depreciated functions}
% 
% Depreciated on 0000-00-00 for removal by 0000-00-00.
%
% \begin{macro}{\KV_process_space_removal_sanitize:NNn}
% \begin{macro}{\KV_process_space_removal_no_sanitize:NNn}
% \begin{macro}{\KV_process_no_space_removal_no_sanitize:NNn}
% There is just one function for this now.
%    \begin{macrocode}
\cs_new_eq:NN \KV_process_space_removal_sanitize:NNn       \keyval_parse:NNn
\cs_new_eq:NN \KV_process_space_removal_no_sanitize:NNn    \keyval_parse:NNn
\cs_new_eq:NN \KV_process_no_space_removal_no_sanitize:NNn \keyval_parse:NNn
%    \end{macrocode}
% \end{macro}
% \end{macro}
% \end{macro}
%
%    \begin{macrocode}
%</initex|package>
%    \end{macrocode}
% 
% \end{implementation}
% 
% \PrintIndex

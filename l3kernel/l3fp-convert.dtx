% \iffalse meta-comment
%
%% File: l3fp-convert.dtx Copyright(C) 2011-2012 The LaTeX3 Project
%%
%% It may be distributed and/or modified under the conditions of the
%% LaTeX Project Public License (LPPL), either version 1.3c of this
%% license or (at your option) any later version.  The latest version
%% of this license is in the file
%%
%%    http://www.latex-project.org/lppl.txt
%%
%% This file is part of the "l3kernel bundle" (The Work in LPPL)
%% and all files in that bundle must be distributed together.
%%
%% The released version of this bundle is available from CTAN.
%%
%% -----------------------------------------------------------------------
%%
%% The development version of the bundle can be found at
%%
%%    http://www.latex-project.org/svnroot/experimental/trunk/
%%
%% for those people who are interested.
%%
%%%%%%%%%%%
%% NOTE: %%
%%%%%%%%%%%
%%
%%   Snapshots taken from the repository represent work in progress and may
%%   not work or may contain conflicting material!  We therefore ask
%%   people _not_ to put them into distributions, archives, etc. without
%%   prior consultation with the LaTeX Project Team.
%%
%% -----------------------------------------------------------------------
%%
%
%<*driver>
\RequirePackage{l3names}
\GetIdInfo$Id$
  {L3 Floating-point conversion}
\documentclass[full]{l3doc}
\begin{document}
  \DocInput{\jobname.dtx}
\end{document}
%</driver>
% \fi
%
% \title{^^A
%   The \textsf{l3fp-convert} package\\ Floating point conversion^^A
%   \thanks{This file describes v\ExplFileVersion,
%     last revised \ExplFileDate.}^^A
% }
%
% \author{^^A
%  The \LaTeX3 Project\thanks
%    {^^A
%      E-mail:
%        \href{mailto:latex-team@latex-project.org}
%          {latex-team@latex-project.org}^^A
%    }^^A
% }
%
% \date{Released \ExplFileDate}
%
% \maketitle
%
% \begin{documentation}
%
% ^^A todo: tests
% ^^A todo: doc functions
%
% \end{documentation}
%
% \begin{implementation}
%
% \section{\texttt{l3fp-convert} implementation}
%
%    \begin{macrocode}
%<*initex|package>
%    \end{macrocode}
%
% \subsection{Trimming trailing zeros}
%
% \begin{macro}[int, EXP]{\fp_trim_zeros:w}
% \begin{macro}[aux, EXP]
%   {\fp_trim_zeros_loop:w, \fp_trim_zeros_dot:w, \fp_trim_zeros_end:w}
%   If |#1| ends with a $0$, the \texttt{loop} auxiliary takes that zero
%   as an end-delimiter for its first argument, and the second argument
%   is the same \texttt{loop} auxiliary.  Once the last trailing zero is
%   reached, the second argument will be the \texttt{dot} auxiliary,
%   which removes a trailing dot if any.  We then cleanup with the
%   \texttt{end} auxiliary, keeping only the number.
%    \begin{macrocode}
\cs_new:Npn \fp_trim_zeros:w #1 ;
  {
    \fp_trim_zeros_loop:w #1
      ; \fp_trim_zeros_loop:w 0; \fp_trim_zeros_dot:w .; \s__stop
  }
\cs_new:Npn \fp_trim_zeros_loop:w #1 0; #2 { #2 #1 ; #2 }
\cs_new:Npn \fp_trim_zeros_dot:w #1 .; { \fp_trim_zeros_end:w #1 ; }
\cs_new:Npn \fp_trim_zeros_end:w #1 ; #2 \s__stop { #1 }
%    \end{macrocode}
% \end{macro}
% \end{macro}
%
% \subsection{Scientific notation}
%
% Expressing an internal floating point number in scientific notation is
% quite easy: no rounding, and the format is very well defined. We
% simply have to filter out the various special cases.
%
% \begin{macro}
%   {\fp_to_scientific:N, \fp_to_scientific:c, \fp_to_scientific:n}
%   The three functions evaluate their argument, then pass it to
%   \cs{fp_to_scientific:w}.
%    \begin{macrocode}
\cs_new:Npn \fp_to_scientific:N #1 { \exp_after:wN \fp_to_scientific:w #1 }
\cs_generate_variant:Nn \fp_to_scientific:N { c }
\cs_new:Npn \fp_to_scientific:n #1
  {
    \exp_after:wN \fp_to_scientific:w
    \tex_romannumeral:D -`0 \fp_parse:n {#1}
  }
%    \end{macrocode}
% \end{macro}
%
% \begin{macro}[aux,EXP]
%   {
%     \fp_to_scientific:w,
%     \fp_to_scientific_normal:nnnnn,
%     \fp_to_scientific_normal:wNw
%   }
%   First cater for the sign: we need to insert |-| if |#2=2|.  Then
%   filter the special cases.  In the normal case, decrement the
%   exponent and unbrace the $4$ brace groups, then in a second step
%   grab the first digit (previously hidden in braces) to order the
%   various parts correctly.
%    \begin{macrocode}
\cs_new:Npn \fp_to_scientific:w \s_fp \fp_use:w #1 #2
  {
    \if_meaning:w 2 #2 \exp_after:wN - \tex_romannumeral:D -`0 \fi:
    \if_case:w #1 \exp_stop_f:
         \fp_aux_case_return:nw { 0 }
    \or: \exp_after:wN \fp_to_scientific_normal:nnnnn
    \or: \fp_aux_case_return:nw { inf }
    \or: \fp_aux_case_return:nw { nan }
    \fi:
  }
\cs_new:Npn \fp_to_scientific_normal:nnnnn #1 #2#3#4#5 ;
  {
    \if_int_compare:w #1 = \c_one
      \exp_after:wN \fp_to_scientific_normal:wNw
    \else:
      \exp_after:wN \fp_to_scientific_normal:wNw
      \exp_after:wN e
      \int_use:N \__int_eval:w #1 - \c_one
    \fi:
    ; #2 #3 #4 #5 ;
  }
\cs_new:Npn \fp_to_scientific_normal:wNw #1 ; #2#3;
  { \fp_trim_zeros:w #2.#3 ; #1 }
%    \end{macrocode}
% \end{macro}
%
% \subsection{Decimal representation}
%
% \begin{macro}[EXP]
%   {\fp_to_decimal:N, \fp_to_decimal:c, \fp_to_decimal:n}
%   All three variants are based on the same \cs{fp_to_decimal:w} after
%   evaluating their argument to an internal floating point.
%    \begin{macrocode}
\cs_new:Npn \fp_to_decimal:N #1 { \exp_after:wN \fp_to_decimal:w #1 }
\cs_generate_variant:Nn \fp_to_decimal:N { c }
\cs_new_nopar:Npn \fp_to_decimal:n
  {
    \exp_after:wN \fp_to_decimal:w
    \tex_romannumeral:D -`0 \fp_parse:n
  }
%    \end{macrocode}
% \end{macro}
%
% \begin{macro}[EXP,aux]
%   {
%     \fp_to_decimal:w,
%     \fp_to_decimal_normal:nnnnn,
%     \fp_to_decimal_large:Nnnw,
%     \fp_to_decimal_huge:wnnnn,
%   }
%   First take care of the sign, inserting |-| when appropriate.  Then
%   zero and \texttt{nan} give \texttt{0}, after an error message for
%   \texttt{nan}, and infinities return a number larger than the maximum
%   floating point number (this ensures that feeding this back to
%   \cs{fp_parse:n} will correctly give infinity again.  Normal numbers
%   with an exponent in the range $[1,15]$ have that number of digits
%   before the decimal separator: \enquote{decimate} them, and remove
%   leading zeros with \cs{__int_value:w}, then trim trailing zeros and
%   dot.  Normal numbers with an exponent $16$ or larger have no decimal
%   separator, we only need to add trailing zeros.  When the exponent is
%   non-positive, the result should be $0.\meta{zeros}\meta{digits}$,
%   trimmed.
%    \begin{macrocode}
\cs_new:Npn \fp_to_decimal:w \s_fp \fp_use:w #1#2
  {
    \if_meaning:w 2 #2 \exp_after:wN - \tex_romannumeral:D -`0 \fi:
    \if_case:w #1 \exp_stop_f:
         \fp_aux_case_return:nw { 0 }
    \or: \exp_after:wN \fp_to_decimal_normal:nnnnn
    \or: \fp_aux_case_return:nw { \prg_replicate:nn \c_fp_max_exponent_int 9 }
    \else:
      \fp_aux_case_return:nw
        { \fp_error:n { Cannot~convert~'nan'~to~decimal. } 0 }
    \fi:
  }
\cs_new:Npn \fp_to_decimal_normal:nnnnn #1 #2#3#4#5;
  {
    \int_compare:nNnTF {#1} > \c_zero
      {
        \int_compare:nNnTF {#1} < \c_sixteen
          {
            \fp_aux_decimate:nNnnnn { \c_sixteen - #1 }
              \fp_to_decimal_large:Nnnw
          }
          {
            \exp_after:wN \exp_after:wN
            \exp_after:wN \fp_to_decimal_huge:wnnnn
            \prg_replicate:nn { #1 - \c_sixteen } { 0 } ;
          }
        {#2} {#3} {#4} {#5}
      }
      {
        \exp_after:wN \fp_trim_zeros:w
        \exp_after:wN 0
        \exp_after:wN .
        \tex_romannumeral:D -`0 \prg_replicate:nn { - #1 } { 0 }
        #2#3#4#5 ;
      }
  }
\cs_new:Npn \fp_to_decimal_large:Nnnw #1#2#3#4;
  {
    \exp_after:wN \fp_trim_zeros:w \__int_value:w
      \if_int_compare:w #2 > \c_zero
        #2
      \fi:
      \exp_stop_f:
      #3.#4 ;
  }
\cs_new:Npn \fp_to_decimal_huge:wnnnn #1; #2#3#4#5 { #2#3#4#5 #1 }
%    \end{macrocode}
% \end{macro}
%
% \subsection{Token list representation}
%
% \begin{macro}[EXP]{\fp_to_tl:N, \fp_to_tl:c, \fp_to_tl:n}
%   The three functions evaluate their argument, then pass it to
%   \cs{fp_to_tl:w}.
%    \begin{macrocode}
\cs_new:Npn \fp_to_tl:N #1 { \exp_after:wN \fp_to_tl:w #1 }
\cs_generate_variant:Nn \fp_to_tl:N { c }
\cs_new_nopar:Npn \fp_to_tl:n
  {
    \exp_after:wN \fp_to_tl:w
    \tex_romannumeral:D -`0 \fp_parse:n
  }
%    \end{macrocode}
% \end{macro}
%
% \begin{macro}[EXP, aux]
%   {\fp_to_tl:w, \fp_to_tl_aux:nw, \fp_to_tl_normal:nnnnn}
%   First filter special cases.  We express normal numbers in decimal
%   notation if the exponent is in the range $[-2,16]$, and otherwise
%   use scientific notation.
%    \begin{macrocode}
\cs_new:Npn \fp_to_tl:w \s_fp \fp_use:w #1#2
  {
    \if_meaning:w 2 #2 \exp_after:wN - \tex_romannumeral:D -`0 \fi:
    \if_case:w #1 \exp_stop_f:
           \fp_aux_case_return:nw { 0 }
    \or:   \exp_after:wN \fp_to_tl_normal:nnnnn
    \or:   \fp_aux_case_return:nw { inf }
    \else: \fp_aux_case_return:nw { nan }
    \fi:
  }
\cs_new:Npn \fp_to_tl_normal:nnnnn #1
  {
    \if_int_compare:w #1 > \c_sixteen
      \exp_after:wN \fp_to_scientific_normal:nnnnn
    \else:
      \if_int_compare:w #1 < - \c_two
        \exp_after:wN \exp_after:wN
        \exp_after:wN \fp_to_scientific_normal:nnnnn
      \else:
        \exp_after:wN \exp_after:wN
        \exp_after:wN \fp_to_decimal_normal:nnnnn
      \fi:
    \fi:
    {#1}
  }
%    \end{macrocode}
% \end{macro}
%
% \subsection{Formatting}
%
% ^^A todo
%
% \subsection{Convert to dimension or integer}
%
% \begin{macro}[EXP]{\fp_to_dim:N, \fp_to_dim:c, \fp_to_dim:n}
%   The three functions rely on \cs{fp_to_decimal:n} internally.
%    \begin{macrocode}
\cs_new:Npn \fp_to_dim:N #1 { \fp_to_decimal:N #1 pt }
\cs_generate_variant:Nn \fp_to_dim:N { c }
\cs_new:Npn \fp_to_dim:n #1 { \fp_to_decimal:n {#1} pt }
%    \end{macrocode}
% \end{macro}
%
% \begin{macro}[EXP]{\fp_to_int:N, \fp_to_int:c, \fp_to_int:n}
%   The three functions evaluate their argument, then pass it to
%   \cs{fp_to_int:w}.
%    \begin{macrocode}
\cs_new:Npn \fp_to_int:N #1 { \exp_after:wN \fp_to_int:w #1 }
\cs_generate_variant:Nn \fp_to_int:N { c }
\cs_new_nopar:Npn \fp_to_int:n
  {
    \exp_after:wN \fp_to_int:w
    \tex_romannumeral:D -`0 \fp_parse:n
  }
%    \end{macrocode}
% \end{macro}
%
% \begin{macro}[aux, EXP]{\fp_to_int:w}
%   To convert to an integer, first round to $0$ places (to the nearest
%   integer), then express the result as a decimal number: the
%   definition of \cs{fp_to_decimal:w} is such that there will be no
%   trailing dot nor zero.
%    \begin{macrocode}
\cs_new:Npn \fp_to_int:w #1;
  {
    \exp_after:wN \fp_to_decimal:w \tex_romannumeral:D -`0
    \fp_round:Nwn \fp_aux_round_to_nearest:NNN #1; { 0 } \prg_do_nothing:
  }
%    \end{macrocode}
% \end{macro}
%
% \subsection{Convert from a dimension}
%
% \begin{macro}[EXP]{\dim_to_fp:n}
% \begin{macro}[aux, EXP]
%   {
%     \dim_to_fp_test:N,
%     \dim_to_fp_aux:Nw,
%     \dim_to_fp_aux_ii:wNNnnnnnn,
%     \dim_to_fp_aux_iii:wnnnnwN,
%   }
%   The dimension expression (which can in fact be a glue expression)
%   is evaluated, converted to a number (\emph{i.e.}, expressed in
%   scaled points), then multiplied by $2^{-16} = 0.0000152587890625$
%   to give a value expressed in points. We use the auxiliary function
%   \begin{quote}
%     \cs{fp_mul_npos:Nnwnw} \meta{sign} \newline
%     ~~\Arg{exp_1}  \meta{body1} |;| \Arg{exp_2} \meta{body2} |;|
%   \end{quote}
%    \begin{macrocode}
\cs_new:Npn \dim_to_fp:n #1
  {
    \exp_after:wN \dim_to_fp_test:N
    \__int_value:w \etex_glueexpr:D #1 ;
  }
\cs_new:Npn \dim_to_fp_test:N #1
  {
    \if_meaning:w 0 #1
      \exp_after:wN \fp_aux_use_i_until_s:nw
      \exp_after:wN \c_zero_fp
    \else:
      \if_meaning:w - #1
        \exp_after:wN \dim_to_fp_aux:Nw
        \exp_after:wN 2
        \__int_value:w
      \else:
        \exp_after:wN \dim_to_fp_aux:Nw
        \exp_after:wN 0
        \__int_value:w #1
      \fi:
    \fi:
  }
\cs_new:Npn \dim_to_fp_aux:Nw #1 #2;
  {
    \fp_aux_pack_twice_four:wNNNNNNNN \dim_to_fp_aux_ii:wNNnnnnnn ;
    #2 000 0000 00 {10}987654321; #1
  }
\cs_new:Npn \dim_to_fp_aux_ii:wNNnnnnnn #1; #2#3#4#5#6#7#8#9
  { \dim_to_fp_aux_iii:wnnnnwN #1 {#2#300} {0000} ; }
\cs_new:Npn \dim_to_fp_aux_iii:wnnnnwN #1; #2#3#4#5#6; #7
  {
    \fp_mul_npos:Nnwnw #7 {#5} #1 ;
    {-4} {1525} {8789} {0625} {0000} ;
  }
%    \end{macrocode}
% \end{macro}
% \end{macro}
%
% \subsection{Use and eval}
%
% \begin{macro}[EXP]{\fp_use:N, \fp_use:c, \fp_eval:n}
%   The meaning of those functions might vary.  For now, they convert to
%   a real number appropriate as a factor in front of dimensions.
%    \begin{macrocode}
\cs_new_eq:NN \fp_use:N \fp_to_decimal:N
\cs_generate_variant:Nn \fp_use:N { c }
\cs_new_eq:NN \fp_eval:n \fp_to_decimal:n
%    \end{macrocode}
% \end{macro}
% 
% \begin{macro}[EXP]{\fp_abs:n}
%   Trivial but useful
%    \begin{macrocode}
\cs_new:Npn \fp_abs:n #1 { \fp_to_decimal:n { abs ( #1 ) } }
%    \end{macrocode}
% \end{macro}
%
%    \begin{macrocode}
%</initex|package>
%    \end{macrocode}
%
% \end{implementation}
%
% \PrintChanges
%
% \PrintIndex
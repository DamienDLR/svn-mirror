% \iffalse
%% File: l3font.dtx Copyright (C) 2011 LaTeX3 project
%%
%% It may be distributed and/or modified under the conditions of the
%% LaTeX Project Public License (LPPL), either version 1.3c of this
%% license or (at your option) any later version.  The latest version
%% of this license is in the file
%%
%%    http://www.latex-project.org/lppl.txt
%%
%% This file is part of the "l3experimental bundle" (The Work in LPPL)
%% and all files in that bundle must be distributed together.
%%
%% The released version of this bundle is available from CTAN.
%%
%% -----------------------------------------------------------------------
%%
%% The development version of the bundle can be found at
%%
%%    http://www.latex-project.org/svnroot/experimental/trunk/
%%
%% for those people who are interested.
%%
%%%%%%%%%%%
%% NOTE: %%
%%%%%%%%%%%
%%
%%   Snapshots taken from the repository represent work in progress and may
%%   not work or may contain conflicting material!  We therefore ask
%%   people _not_ to put them into distributions, archives, etc. without
%%   prior consultation with the LaTeX Project Team.
%%
%% -----------------------------------------------------------------------
%
%<*driver|package>
\RequirePackage{l3names}
\GetIdInfo$Id$
          {L3 Experimental Font Loading}
%</driver|package>
%<*driver>
%\fi
\ProvidesFile{\ExplFileName.\ExplFileNameExt}
  [\ExplFileDate\space v\ExplFileVersion\space\ExplFileDescription]
%\iffalse
\documentclass[full,checktest]{l3doc}
\begin{document}
\DocInput{l3font.dtx}
\end{document}
%</driver>
% \fi
%
%
% \title{^^A
%   The \pkg{l3font} package\\ Basic font support^^A
%   \thanks{This file describes v\ExplFileVersion,
%      last revised \ExplFileDate.}^^A
% }
%
% \author{^^A
%  The \LaTeX3 Project\thanks
%    {^^A
%      E-mail:
%        \href{mailto:latex-team@latex-project.org}
%          {latex-team@latex-project.org}^^A
%    }^^A
% }
%
% \date{Released \ExplFileDate}
%
% This module covers basic font loading commands.
% Functions are provided to load font faces and extract various
% properties from them.
%
% Some features within are specific to \XeTeX{} and \LuaTeX{}; such functions
% will be explicitly noted.
%
% This module is currently a work in progress as we incorporate \LaTeXe{}'s
% font loading into \pkg{expl3}. The successor to the NFSS
% (tentatively denoted \texttt{xfss}) will provide
% (backwards compatible) user-level functions for font selection.
%
% \section{Naming and scope}
%
% The \texttt{l3font} module largely provides commands for selecting and working
% with typefaces selected at specific sizes.
% (Many fonts have different shapes intended for use at different physical
% sizes.)
% We call a font at some size a \texttt{fontface} which is the function prefix
% used for these commands and also should be the variable suffix for
% denoting variables of this type.
%
% We reserve for future use the module name and variable type \texttt{font} for
% higher-level font selection commands, such as for font families which have
% linked bold, italic, etc., shapes with automatically-chosen optical sizes.
% (I.e., what is currently provided in \LaTeXe{} by the NFSS.)
%
%
% \section{Functions}
%
%
% \begin{function}{\fontface_set:Nn,\fontface_gset:Nn,
%     \fontface_set:cn,\fontface_gset:cn}
%   \begin{syntax}
%     \cs{fontface_set:Nn} <font cs> \Arg{font name}
%     \cs{fontface_set:Nn} \cs{l_tenrm_fontface} |{cmr10}|
%   \end{syntax}
%   Defines <font cs> as a command to select the font defined by <font name>
%   at its design size, which is usually specified by the font designer.
%   For fonts without a typical design size, this will usually be 10\,pt.
% \end{function}
%
% \begin{function}{\fontface_set_at:Nnn,\fontface_gset_at:Nnn,
%     \fontface_set_at:cnn,\fontface_gset_at:cnn}
%   \begin{syntax}
%     \cs{fontface_set_at:Nnn} <font cs> \Arg{font name} \Arg{font size}
%     \cs{fontface_set_at:Nnn} \cs{l_tenrm_fontface} |{cmr10}| |{10pt}|
%   \end{syntax}
%   Defines <cs> as a command to select the font defined by <font name>
%   at the <font size>.
% \end{function}
%
% \begin{function}{\fontface_set_eq:NN,\fontface_gset_eq:NN}
%   \begin{syntax}
%     \cs{fontface_set_eq:NN} <font cs1> <font cs2>
%   \end{syntax}
%   Copies <font cs2> into <font cs1>.
% \end{function}
%
% \begin{function}{\fontface_set_to_current:N,\fontface_gset_to_current:N}
%   \begin{syntax}
%     \cs{fontface_set_to_current:N} <font cs>
%   \end{syntax}
%   Sets <font cs> to the font that is currently selected.
% \end{function}
%
% \begin{function}[pTF]{ \fontface_if_null:N }
%   \begin{syntax}
%     \cs{fontface_if_null:NTF} <font cs> \Arg{true code} \Arg{false code}
%   \end{syntax}
%   Conditional to switch whether the control sequence is the
%   \enquote{null font}.
% \end{function}
%
% \begin{function}{\font_suppress_not_found_error:,
%     \font_enable_not_found_error:}
%   \begin{syntax}
%     \cs{font_suppress_not_found_error:}
%     \cs{font_enable_not_found_error:}
%   \end{syntax}
%   \emph{Not available in pdf\TeX{}.}
%   In \LuaTeX{} or \XeTeX{}, the error when a font is selected but does not
%   exist can be toggled with these two commands. The non-existence of a font
%   can then be tested with the \cs{fontface_if_null_p:N} conditional.
% \end{function}
%
% \end{documentation}
%
% \begin{implementation}
%
% \section{\pkg{l3font} implementation}
%
%    \begin{macrocode}
%<*package>
\ProvidesExplPackage
  {\ExplFileName}{\ExplFileDate}{\ExplFileVersion}{\ExplFileDescription}
%</package>
%<*initex|package>
%    \end{macrocode}
%
% \begin{macro}{\fontface_set:Nn,\fontface_gset:Nn,
%               \fontface_set:cn,\fontface_gset:cn}
%   \begin{arguments}
%   \item csname \item fontname
%   \end{arguments}
%   Note that the fontname needs to be escaped appropriately
%   in \texttt{xetex} or \texttt{luatex}.
%    \begin{macrocode}
\cs_new_protected:Npn \fontface_set:Nn #1#2
  { \tex_font:D #1 = #2 \scan_stop: }
\cs_new_protected_nopar:Npn \fontface_gset:Nn
  { \tex_global:D \fontface_set:Nn }
\cs_generate_variant:Nn \fontface_set:Nn  {c}
\cs_generate_variant:Nn \fontface_gset:Nn {c}
%    \end{macrocode}
% \end{macro}
%
% \begin{macro}{\fontface_set_at:Nnn,\fontface_gset_at:Nnn,
%               \fontface_set_at:cnn,\fontface_gset_at:cnn}
%   \begin{arguments}
%   \item csname \item fontname \item size (dimension)
%   \end{arguments}
%   Note that the fontname needs to be escaped appropriately
%   in \texttt{xetex} or \texttt{luatex}.
%    \begin{macrocode}
\cs_new_protected:Npn \fontface_set_at:Nnn #1#2#3
  { \tex_font:D #1 = #2 ~at~ #3 \scan_stop: }
\cs_new_protected_nopar:Npn \fontface_gset_at:Nnn
  { \tex_global:D \fontface_set_at:Nnn }
\cs_generate_variant:Nn \fontface_set_at:Nnn  {c}
\cs_generate_variant:Nn \fontface_gset_at:Nnn {c}
%    \end{macrocode}
% \end{macro}
%
% \begin{macro}{\fontface_set_eq:NN}
%    \begin{macrocode}
\cs_set_eq:NN \fontface_set_eq:NN \cs_set_eq:NN
\cs_set_protected_nopar:Npn \fontface_gset_eq:NN
  { \tex_global:D \fontface_set_eq:NN }
%    \end{macrocode}
% \end{macro}
%
% \begin{macro}{\fontface_set_to_current:N,\fontface_gset_to_current:N}
%    \begin{macrocode}
\cs_set_protected:Npn \fontface_set_to_current:N #1
  {
    \exp_after:wN \fontface_set_eq:NN
    \exp_after:wN #1
    \tex_the:D \tex_font:D
  }
\cs_set_protected_nopar:Npn \fontface_gset_to_current:N
  { \tex_global:D \fontface_set_eq:NN }
%    \end{macrocode}
% \end{macro}
%
% \begin{macro}{\font_suppress_not_found_error:,\font_enable_not_found_error:}
%   The name of the integer parameter to control differs between \XeTeX{}
%   and \LuaTeX{}. In \pdfTeX{}, the functions produce an error message.
%    \begin{macrocode}
\luatex_if_engine:TF
  {
    \cs_new_protected:Npn \font_suppress_not_found_error:
      { \luatexsuppressfontnotfounderror = \c_one }
    \cs_new_protected:Npn \font_enable_not_found_error:
      { \luatexsuppressfontnotfounderror = \c_zero }
  }
  {
    \xetex_if_engine:TF
      {
        \cs_new_protected:Npn \font_suppress_not_found_error:
          { \suppressfontnotfounderror = \c_one }
        \cs_new_protected:Npn \font_enable_not_found_error:
          { \suppressfontnotfounderror = \c_zero }
      }
      {
        \cs_new_protected:Npn \font_suppress_not_found_error:
          {
            \msg_kernel_warning:nnx {l3font} {cmd-pdftex-unavail}
              { \font_suppress_not_found_error: }
          }
      }
  }
\msg_kernel_new:nnn {l3font} {cmd-pdftex-unavail}
  { The~command~`\exp_not:n{#1}'~is~not~available~for~the~pdfTeX~engine. }
%    \end{macrocode}
% \end{macro}
%
% \begin{macro}[pTF]{\fontface_if_null:N}
%    \begin{macrocode}
\prg_new_conditional:Npnn \fontface_if_null:N #1 { p , TF , T , F }
  {
    \if_meaning:w #1 \tex_nullfont:D
      \prg_return_true:
    \else:
      \prg_return_false:
    \fi:
  }
%    \end{macrocode}
% \end{macro}
%
%    \begin{macrocode}
%</initex|package>
%    \end{macrocode}
%
%
% \end{implementation}
% \PrintIndex
%
% \endinput

% \iffalse meta-comment
%
%% File: l3tl-convert.dtx Copyright (C) 2011 The LaTeX3 Project
%%
%% It may be distributed and/or modified under the conditions of the
%% LaTeX Project Public License (LPPL), either version 1.3c of this
%% license or (at your option) any later version.  The latest version
%% of this license is in the file
%%
%%    http://www.latex-project.org/lppl.txt
%%
%% This file is part of the "l3experimental bundle" (The Work in LPPL)
%% and all files in that bundle must be distributed together.
%%
%% The released version of this bundle is available from CTAN.
%%
%% -----------------------------------------------------------------------
%%
%% The development version of the bundle can be found at
%%
%%    http://www.latex-project.org/svnroot/experimental/trunk/
%%
%% for those people who are interested.
%%
%%%%%%%%%%%
%% NOTE: %%
%%%%%%%%%%%
%%
%%   Snapshots taken from the repository represent work in progress and may
%%   not work or may contain conflicting material!  We therefore ask
%%   people _not_ to put them into distributions, archives, etc. without
%%   prior consultation with the LaTeX3 Project.
%%
%% -----------------------------------------------------------------------
%
%<*driver|package>
\RequirePackage{expl3}
\GetIdInfo$Id: l3tl-convert.dtx 3039 2011-12-08 09:22:35Z bruno $
  {L3 Experimental Token-List Conversion}
%</driver|package>
%<*driver>
\documentclass[full]{l3doc}
\usepackage{amsmath}
\begin{document}
  \DocInput{\jobname.dtx}
\end{document}
%</driver>
% \fi
%
%
% \title{^^A
%   The \textsf{l3tl-convert} package: analysing and building token lists^^A
%   \thanks{This file describes v\ExplFileVersion,
%     last revised \ExplFileDate.}^^A
% }
%
% \author{^^A
%  The \LaTeX3 Project\thanks
%    {^^A
%      E-mail:
%        \href{mailto:latex-team@latex-project.org}
%          {latex-team@latex-project.org}^^A
%    }^^A
% }
%
% \date{Released \ExplFileDate}
%
% \maketitle
%
% \begin{documentation}
%
% \section{\pkg{l3tl-convert} documentation}
%
% This module moslty provides internal functions for use
% in the \pkg{l3regex} module. However, it provides as a
% side-effect a user debugging function, very similar to
% the \cs{ShowTokens} macro from the \pkg{ted} package.
%
% \begin{function}{\tl_display:N, \tl_display:n}
%   \begin{syntax}
%     \cs{tl_display:n} \Arg{token list}
%   \end{syntax}
%   Displays to the terminal the detailed decomposition of
%   the \meta{token list} into tokens,
%   showing the category code of each character token,
%   and the meaning of control sequences and active characters.
% \end{function}
%
% \subsection{Internal}
%
% \begin{function}{\tl_convert:n}
%   \begin{syntax}
%     \cs{tl_convert:n} \Arg{token list}
%   \end{syntax}
%   This function analyses each token in the \meta{token list},
%   converting each \meta{token} to
%   \begin{quote}
%     \Arg{tokens} \cs{s_tl_convert} \meta{catcode} \meta{char code} |,|
%   \end{quote}
%   where \meta{tokens} expands in one step to the \meta{token},
%   and \meta{catcode} and \meta{char code} are the category code
%   and character code of the token if it is a character token,
%   and otherwise are $0$ and $-1$ respectively.
%   The result of the conversion is then stored in
%   \cs{g_tl_convert_result_tl}.
% \end{function}
%
% \end{documentation}
%
% \begin{implementation}
%
% \section{\pkg{l3tl-convert} implementation}
%
%    \begin{macrocode}
%<*initex|package>
%    \end{macrocode}
%
%    \begin{macrocode}
\ProvidesExplPackage
  {\ExplFileName}{\ExplFileDate}{\ExplFileVersion}{\ExplFileDescription}
\RequirePackage{l3str}
%    \end{macrocode}
%
% \subsection{Variables and helper functions}
%
% \begin{variable}{\l_tl_convert_normal_int}
%   The number of normal (\texttt{N}-type argument) tokens
%   since the last special token.
%    \begin{macrocode}
\int_new:N \l_tl_convert_normal_int
%    \end{macrocode}
% \end{variable}
%
% \begin{variable}{\l_tl_convert_index_int}
%   During the first pass, this is the index in the array being built.
%   During the second pass, it is equal to the maximum index in the
%   array from the first pass.
%    \begin{macrocode}
\int_new:N \l_tl_convert_index_int
%    \end{macrocode}
% \end{variable}
%
% \begin{variable}{\l_tl_convert_nesting_int}
%   Nesting depth of explicit begin-group and end-group characters
%   during the first pass. This lets us detect the end of the token
%   list without a reserved end-marker.
%    \begin{macrocode}
\int_new:N \l_tl_convert_nesting_int
%    \end{macrocode}
% \end{variable}
%
% \begin{variable}{\l_tl_convert_char_int}
%   Holds the character code of the token currently considered,
%   in cases where it is a character.
%    \begin{macrocode}
\int_new:N \l_tl_convert_char_int
%    \end{macrocode}
% \end{variable}
%
% \begin{variable}{\l_tl_convert_type_int}
%   When encountering special characters, we record their \enquote{type}
%   in this integer.
%    \begin{macrocode}
\int_new:N \l_tl_convert_type_int
%    \end{macrocode}
% \end{variable}
%
% \begin{variable}{\s_tl_convert}
%   A non-expandable marker.
%    \begin{macrocode}
\cs_new_eq:NN \s_tl_convert \scan_stop:
%    \end{macrocode}
% \end{variable}
%
% \begin{variable}{\g_tl_convert_result_tl}
%   The result of the conversion is stored in this token list,
%   with a succession of items of the form
%   \begin{quote}
%     \meta{tokens} \cs{s_tl_convert} \meta{catcode} \meta{char code} |,|
%   \end{quote}
%    \begin{macrocode}
\tl_new:N \g_tl_convert_result_tl
%    \end{macrocode}
% \end{variable}
%
% \begin{macro}[int]{\tl_convert_extract_charcode:}
% \begin{macro}[aux]{\tl_convert_extract_charcode_aux:w}
%   Extracting the character code from the meaning of \cs{l_tl_convert_token}.
%   This has no error checking, and should only be assumed to work
%   for begin-group and end-group tokens.
%    \begin{macrocode}
\cs_new_protected_nopar:Npn \tl_convert_extract_charcode:
  {
    \exp_after:wN \tl_convert_extract_charcode_aux:w
      \token_to_meaning:N \l_tl_convert_token
  }
\cs_new_protected_nopar:Npn \tl_convert_extract_charcode_aux:w #1 ~ #2 ~ { ` }
%    \end{macrocode}
% \end{macro}
% \end{macro}
%
% \begin{macro}[int]{\tl_convert_cs_space_count:NN}
% \begin{macro}[aux]{\tl_convert_cs_space_count:w}
% \begin{macro}[aux]{\tl_convert_cs_space_count_end:w}
%   Counts the number of spaces in the string representation of its
%   second argument, as well as the number of characters following
%   the last space in that representation, and feeds the two numbers
%   as semicolon-delimited arguments to the first argument.
%    \begin{macrocode}
\cs_new:Npn \tl_convert_cs_space_count:NN #1 #2
  {
    \exp_after:wN #1
    \int_value:w \int_eval:w \c_zero
      \exp_after:wN \tl_convert_cs_space_count:w
        \token_to_str:N #2
        \fi: \tl_convert_cs_space_count_end:w ; ~ !
  }
\cs_new:Npn \tl_convert_cs_space_count:w #1 ~
  {
    \if_false: #1 #1 \fi:
    + \c_one
    \tl_convert_cs_space_count:w
  }
\cs_new:Npn \tl_convert_cs_space_count_end:w ; #1 \fi: #2 !
  { \exp_after:wN ; \int_value:w \str_length_ignore_spaces:n {#1} ; }
%    \end{macrocode}
% \end{macro}
% \end{macro}
% \end{macro}
%
% \subsection{Plan of attack}
%
% Our goal is to produce a token list of the form roughly
% \begin{quote}
%   \Arg{token 1} \cs{s_tl_convert} \meta{catcode 1} \meta{char code 1} |,| \\
%   \Arg{token 2} \cs{s_tl_convert} \meta{catcode 2} \meta{char code 2} |,| \\
%   \ldots{}
%   \Arg{token N} \cs{s_tl_convert} \meta{catcode N} \meta{char code N} |,|
% \end{quote}
% Most but not all tokens can be grabbed as an undelimited
% (\texttt{N}-type) argument by \TeX{}.
% The plan is to have a two pass system. In the first pass,
% locate special tokens, and store them in various \tn{toks} registers.
% In the second pass, which is done within an \texttt{x}-expanding
% assignment, normal tokens are taken in as \texttt{N}-type
% arguments, and special tokens are retrieved from the \tn{toks}
% registers, and removed from the input stream by some means.
%
% To ease the difficult first pass, we first do some setup
% with \cs{tl_convert_i_setup:n}.
% Active characters set equal to non-active characters
% cause trouble, so we disable all active characters by
% setting them equal to \texttt{undefined}.
% We also set there the escape character to be printable
% (backslash, but this later changes to a forward slash):
% this makes it possible to distinguish characters from
% control sequences.
%
% A token has two characteristics: its \tn{meaning},
% and what it looks like for \TeX{} when it is in
% scanning mode (\emph{e.g.}, when capturing parameters
% for a macro).
% For our purposes, the token's meaning can be
% \begin{itemize}
% \item a begin-group token (category code $1$),
%   either space (character code $32$), or non-space;
% \item an end-group token (category code $2$),
%   either space (character code $32$), or non-space;
% \item a space token (category code $10$, character code $32$);
%^^A \item a character with category code $3$, $4$, $6$, $7$, $8$, $11$,
%^^A   or $12$, or with category code $10$ but character code $\neq 32$;
%^^A \item a non-character (primitive, macro, register, font,
%^^A   undefined, \emph{etc.}).
% \item anything else (then the token is always an \texttt{N}-type argument).
% \end{itemize}
% The token itself can \enquote{look like} one of the following
% \begin{itemize}
% \item a non-active character, in which case its meaning
%   is automatically that associated to its character code
%   and category code, we call it \enquote{true} character;
% \item an active character;
% \item a control sequence.
% \end{itemize}
% The only tokens which are not valid \texttt{N}-type arguments are
% true begin-group characters, true end-group characters, and true spaces.
% We will detect those characters by scanning ahead with \tn{futurelet},
% then distinguishing true characters from tokens set equal to them
% using the \tn{string} representation.
%
% The second pass is a simple exercise in expandable loops.
%
% \begin{macro}[int]{\tl_convert:n}
%   Everything is done within a group, and all definitions will be local.
%   We use \cs{group_align_safe_begin/end:} to avoid problems in case
%   \cs{tl_convert:n} is used within an alignment and its argument
%   contains alignment tab tokens.
%    \begin{macrocode}
\cs_new_protected:Npn \tl_convert:n #1
  {
    \group_begin:
      \group_align_safe_begin:
        \tl_convert_setup:n {#1}
        \tl_convert_i:n {#1}
        \tl_convert_ii:n {#1}
      \group_align_safe_end:
    \group_end:
  }
%    \end{macrocode}
% \end{macro}
%
% \subsection{Setup}
%
% \begin{macro}[int]{\tl_convert_setup:n}
% \begin{macro}[aux]{\tl_convert_disable_loop:N}
%   Active characters can cause problems later on in the processing,
%   so the first step is to disable them, by setting them to
%   \texttt{undefined}. Since Unicode contains too many characters
%   to loop over all of them, we instead loop over the input token
%   list as a string: any active character in the token list
%   must appear in its string representation.
%   ^^A todo: ask LuaTeX list for an \ifx\undefined <active char>
%   ^^A which does not add the <active char> in memory.
%    \begin{macrocode}
\cs_new_protected:Npn \tl_convert_setup:n #1
  {
    \int_set:Nn \tex_escapechar:D { 92 }
    \exp_after:wN \tl_convert_disable_loop:N
      \tl_to_str:n {#1} { ~ } { ? ~ \prg_map_break: }
    \prg_break_point:n { }
  }
\group_begin:
  \char_set_catcode_active:N \^^@
  \cs_new_protected_nopar:Npn \tl_convert_disable_loop:N #1
    {
      \tex_lccode:D \c_zero `#1 ~
      \tl_to_lowercase:n { \tex_let:D ^^@ } \c_undefined:D
      \tl_convert_disable_loop:N
    }
\group_end:
%    \end{macrocode}
% \end{macro}
% \end{macro}
%
% \subsection{First pass}
%
% The goal of this pass is to detect special (non-\texttt{N}-type) tokens,
% and count how many \texttt{N}-type tokens lie between special tokens.
% Also, we wish to store some representation of each special token
% in a \tn{toks} register.
%
% After the setup step, we have $11$ types of tokens:
% \begin{itemize}
% \item[1] a true non-space begin-group character;
% \item[2] a true space begin-group character;
% \item[3] a true non-space end-group character;
% \item[4] a true space end-group character;
% \item[5] a true space blank space character;
% \item[6] an undefined active character;
% \item[7] any other true character;
% \item[8] a control sequence equal to a begin-group token (category code $1$);
% \item[9] a control sequence equal to an end-group token (category code $2$);
% \item[10] a control sequence equal to a space token
%   (character code $32$, category code $10$);
% \item[11] any other control sequence.
% \end{itemize}
% Our first tool is \tn{futurelet}. This cannot distinguish
% cases $1$ and $2$ from $8$, nor cases $3$ and $4$ from $9$,
% nor case $5$ from case $10$. Those cases will be distinguished
% by applying the \tn{string} primitive to the following token,
% after possibly changing the escape character to ensure that
% a control sequence's string representation cannot be mistaken
% for the true character.
%
% In cases $6$, $7$, and $11$, the following token is a valid
% \texttt{N}-type argument, so we grab it and distinguish the case
% of a character from a control sequence: in the latter case,
% \cs{str_tail:n} \Arg{token} is non-empty, because the
% escape character is printable.
%
% \begin{macro}{\tl_convert_i:n}
%   We read tokens one by one using \tn{futurelet}.
%   While performing the loop, we keep track of the number of
%   true begin-group characters minus the number of
%   true end-group characters in \cs{l_tl_convert_nesting_int}.
%   This reaches $-1$ when we read the closing brace.
%    \begin{macrocode}
\cs_new_protected:Npn \tl_convert_i:n #1
  {
    \int_zero:N \l_tl_convert_normal_int
    \int_zero:N \l_tl_convert_index_int
    \int_zero:N \l_tl_convert_nesting_int
    \if_false: { \fi: \tl_convert_i_loop:w #1 }
    \int_decr:N \l_tl_convert_index_int
  }
%    \end{macrocode}
% \end{macro}
%
% \begin{macro}[int]{\tl_convert_i_loop:w}
%   Read one character and check its type.
%    \begin{macrocode}
\cs_new_protected_nopar:Npn \tl_convert_i_loop:w
  { \tex_futurelet:D \l_tl_convert_token \tl_convert_i_type:w }
%    \end{macrocode}
% \end{macro}
%
% \begin{macro}[int]{\tl_convert_i_type:w}
%   At this point, \cs{l_tl_convert_token} holds the meaning of the following
%   token. We store in \cs{l_tl_convert_type_int} the meaning of the token ahead:
%   \begin{itemize}
%   \item[0] space token;
%   \item[1] begin-group token;
%   \item[-1] end-group token;
%   \item[2] other.
%   \end{itemize}
%   The values $0$, $1$, $-1$ correspond to how much a true such character
%   changes the nesting level ($2$ is used only here, and is irrelevant later).
%   Then call the auxiliary for each case.
%    \begin{macrocode}
\cs_new_protected_nopar:Npn \tl_convert_i_type:w
  {
    \l_tl_convert_type_int =
      \if_meaning:w \l_tl_convert_token \c_space_token
        \c_zero
      \else:
        \if_catcode:w \exp_not:N \l_tl_convert_token \c_group_begin_token
          \c_one
        \else:
          \if_catcode:w \exp_not:N \l_tl_convert_token \c_group_end_token
            \c_minus_one
          \else:
            \c_two
          \fi:
        \fi:
      \fi:
    \if_case:w \l_tl_convert_type_int
         \exp_after:wN \tl_convert_i_space:w
    \or: \exp_after:wN \tl_convert_i_bgroup:w
    \or: \exp_after:wN \tl_convert_i_safe:N
    \else: \exp_after:wN \tl_convert_i_egroup:w
    \fi:
  }
%    \end{macrocode}
% \end{macro}
%
% \begin{macro}[int]{\tl_convert_i_safe:N}
% \begin{macro}[aux]{\tl_convert_i_cs:ww}
%   The upcoming token is safe: we can simply grab it in a second pass.
%   Since other branches of the code must pass their tokens through
%   \tn{string}, we do it here as well. It's been optimized in an evil way,
%   sorry.
%^^A todo: document
%    \begin{macrocode}
\cs_new_protected:Npn \tl_convert_i_safe:N #1
  {
    \if_charcode:w
        \scan_stop:
        \exp_after:wN \use_none:n \token_to_str:N #1 \prg_do_nothing:
        \scan_stop:
      \int_incr:N \l_tl_convert_normal_int
    \else:
      \tl_convert_cs_space_count:NN \tl_convert_i_cs:ww #1
    \fi:
    \tl_convert_i_loop:w
  }
\cs_new_protected:Npn \tl_convert_i_cs:ww #1; #2;
  {
    \if_num:w #1 > \c_zero
      \tex_skip:D \l_tl_convert_index_int
        = \int_eval:w \l_tl_convert_normal_int + \c_one sp \scan_stop:
      \tex_advance:D \l_tl_convert_index_int #1 \exp_stop_f:
      \l_tl_convert_normal_int #2 \exp_stop_f:
    \else:
      \tex_advance:D \l_tl_convert_normal_int #2 \exp_stop_f:
    \fi:
  }
%    \end{macrocode}
% \end{macro}
% \end{macro}
%
% \begin{macro}[aux]{\tl_convert_i_space:w}
% \begin{macro}[aux]{\tl_convert_i_space_test:w}
%   In this branch, the following token's meaning is a blank space.
%   Apply \tn{string} to that token: if it is a control sequence
%   the result starts with the escape character; otherwise it is
%   a true blank space, whose string representation is also a blank space.
%   We test for that in \cs{tl_convert_i_space_test:w},
%   after grabbing the first character of the string representation
%   as \cs{l_tl_convert_token}.
%   Also, since \cs{tl_convert_i_store:} expects the special token to be
%   stored in the relevant \tn{toks} register, we do that. The extra
%   \cs{exp_not:n} is unnecessary of course, but it makes the treatment
%   of all tokens more homogeneous.
%    \begin{macrocode}
\cs_new_protected_nopar:Npn \tl_convert_i_space:w
  {
    \tex_afterassignment:D \tl_convert_i_space_test:w
    \exp_after:wN \cs_set_eq:NN
    \exp_after:wN \l_tl_convert_token
    \token_to_str:N
  }
\cs_new_protected_nopar:Npn \tl_convert_i_space_test:w
  {
    \if_meaning:w \l_tl_convert_token \c_space_token
      \tex_toks:D \l_tl_convert_index_int { \exp_not:n { ~ } }
      \tl_convert_i_store:
    \else:
      \int_incr:N \l_tl_convert_normal_int
    \fi:
    \tl_convert_i_loop:w
  }
%    \end{macrocode}
% \end{macro}
% \end{macro}
%
% \begin{macro}[int]{\tl_convert_i_bgroup:w}
% \begin{macro}[int]{\tl_convert_i_egroup:w}
% \begin{macro}[aux]{\tl_convert_i_group:nw}
% \begin{macro}[aux]{\tl_convert_i_group_test:w}
%   The token might be either a true character token with
%   catcode $1$ or $2$, or it could be a control sequence.
%   The only tricky case is if the character code happens
%   to be equal to the escape character: then we change
%   the escape character, so that the string representation
%   of the true character and of a control sequence set equal
%   to it start differently.
%    \begin{macrocode}
\group_begin:
  \char_set_catcode_group_begin:N \^^@
  \char_set_catcode_group_end:N \^^E
  \cs_new_protected_nopar:Npn \tl_convert_i_bgroup:w
    { \tl_convert_i_group:nw { \exp_after:wN ^^@ \if_false: ^^E \fi: } }
  \char_set_catcode_group_begin:N \^^B
  \char_set_catcode_group_end:N \^^@
  \cs_new_protected_nopar:Npn \tl_convert_i_egroup:w
    { \tl_convert_i_group:nw { \if_false: ^^B \fi: ^^@ } }
\group_end:
\cs_new_protected_nopar:Npn \tl_convert_i_group:nw #1
  {
    \l_tl_convert_char_int = \tl_convert_extract_charcode: \scan_stop:
    \tex_lccode:D \c_zero \l_tl_convert_char_int
    \tl_to_lowercase:n { \tex_toks:D \l_tl_convert_index_int {#1} }
    \if_num:w \l_tl_convert_char_int = \tex_escapechar:D
      \int_set:Nn \tex_escapechar:D { 139 - \tex_escapechar:D }
    \fi:
    \tex_afterassignment:D \tl_convert_i_group_test:w
    \exp_after:wN \cs_set_eq:NN
    \exp_after:wN \l_tl_convert_token
    \token_to_str:N
  }
\cs_new_protected_nopar:Npn \tl_convert_i_group_test:w
  {
    \if_num:w \tl_convert_extract_charcode: = \l_tl_convert_char_int
      \tl_convert_i_store:
    \else:
      \int_incr:N \l_tl_convert_normal_int
    \fi:
    \tl_convert_i_loop:w
  }
%    \end{macrocode}
% \end{macro}
% \end{macro}
% \end{macro}
% \end{macro}
%
% \begin{macro}[int]{\tl_convert_i_store:}
%   This function is called each time we meet a special token,
%   and the value of \cs{l_tl_convert_type_int} indicates which case
%   we are in:
%   \begin{itemize}
%   \item[-1] end-group character;
%   \item[0] space character;
%   \item[1] begin-group character.
%   \end{itemize}
%   We need to distinguish further the case of a space character
%   (code $32$) from other character codes, because those will
%   behave differently in the second pass. Namely, we change the
%   cases above to
%   \begin{itemize}
%   \item[-2] space end-group character;
%   \item[-1] non-space end-group character;
%   \item[0] space blank space character;
%   \item[1] non-space begin-group character;
%   \item[2] space begin-group character.
%   \end{itemize}
%   This has the property that non-space characters correspond to odd
%   values of \cs{l_tl_convert_type_int}.
%   Also, the \tn{toks} register number \cs{l_tl_convert_index_int}
%   holds a token list which expands to the given special token.
%    \begin{macrocode}
\cs_new_protected_nopar:Npn \tl_convert_i_store:
  {
    \tex_advance:D \l_tl_convert_nesting_int \l_tl_convert_type_int
    \if_num:w \l_tl_convert_char_int = \c_thirty_two
      \tex_multiply:D \l_tl_convert_type_int \c_two
    \fi:
    \tex_skip:D \l_tl_convert_index_int
      = \l_tl_convert_normal_int sp plus \l_tl_convert_type_int sp \scan_stop:
    \int_incr:N \l_tl_convert_index_int
    \int_zero:N \l_tl_convert_normal_int
    \if_num:w \l_tl_convert_nesting_int = \c_minus_one
      \cs_set_eq:NN \tl_convert_i_loop:w \scan_stop:
    \fi:
  }
%    \end{macrocode}
% \end{macro}
%
% \subsection{Second pass}
%
% The second pass is an exercise in expandable loops.
%
% \begin{macro}[int]{\tl_convert_ii:n}
% \begin{macro}[aux]{\tl_convert_ii_loop:w}
%   Start the loop with the index $0$. No need for an end-marker:
%   the loop will stop by itself when the last index is read.
%   We will repeatedly oscillate between reading long stretches
%   of normal tokens, and reading special tokens.
%    \begin{macrocode}
\cs_new_protected:Npn \tl_convert_ii:n #1
  {
    \tl_gset:Nx \g_tl_convert_result_tl
      {
        \tl_convert_ii_loop:w 0; #1
        \prg_break_point:n { }
      }
  }
\cs_new:Npn \tl_convert_ii_loop:w #1;
  {
    \exp_after:wN \tl_convert_ii_normals:ww
      \int_value:w \tex_skip:D #1 ; #1 ;
  }
%    \end{macrocode}
% \end{macro}
% \end{macro}
%
% \begin{macro}[aux]{\tl_convert_ii_normals:ww}
% \begin{macro}[aux]{\tl_convert_ii_normal:wwN}
%   The first argument is the number of normal tokens which remain
%   to be read, and the second argument is the index in the array
%   produced in the first step.
%   A character's string representation is always one character long,
%   while a control sequence is always longer (we have set the escape
%   character to a printable value). In both cases, we leave
%   \cs{exp_not:n} \Arg{token} \cs{s_tl_convert} in the input stream
%   (after \texttt{x}-expansion).
%    \begin{macrocode}
\cs_new:Npn \tl_convert_ii_normals:ww #1;
  {
    \if_num:w #1 = \c_zero
      \tl_convert_ii_special:w
    \fi:
    \tl_convert_ii_normal:wwN #1;
  }
\cs_new:Npn \tl_convert_ii_normal:wwN #1; #2; #3
  {
    \exp_not:n { \exp_not:n { #3 } } \s_tl_convert
    \if_charcode:w
        \scan_stop:
        \exp_after:wN \use_none:n \token_to_str:N #3 \prg_do_nothing:
        \scan_stop:
      \exp_after:wN \tl_convert_ii_char:Nww
    \else:
      \exp_after:wN \tl_convert_ii_cs:Nww
    \fi:
    #3 #1; #2;
  }
%    \end{macrocode}
% \end{macro}
% \end{macro}
%
% \begin{macro}[aux]{\tl_convert_ii_char:Nww}
%   If the normal token we grab is a character, leave
%   \meta{catcode} \meta{charcode} followed by a comma
%   in the input stream, and call \cs{tl_convert_ii_normals:ww}
%   with its first argument decremented.
%    \begin{macrocode}
\cs_new:Npn \tl_convert_ii_char:Nww #1
  {
    \if_meaning:w #1 \c_undefined:D            D \else:
    \if_catcode:w #1 \c_catcode_other_token    C \else:
    \if_catcode:w #1 \c_catcode_letter_token   B \else:
    \if_catcode:w #1 \c_math_toggle_token      3 \else:
    \if_catcode:w #1 \c_alignment_token        4 \else:
    \if_catcode:w #1 \c_math_superscript_token 7 \else:
    \if_catcode:w #1 \c_math_subscript_token   8 \else:
      6
    \fi: \fi: \fi: \fi: \fi: \fi: \fi:
    \int_value:w `#1 ,
    \exp_after:wN \tl_convert_ii_normals:ww
      \int_use:N \int_eval:w \c_minus_one +
  }
%    \end{macrocode}
% \end{macro}
%
% \begin{macro}[aux]{\tl_convert_ii_cs:Nww}
%   If the token we grab is a control sequence, leave
%   |0-1,| (as category code and character code) in the input stream,
%   and call \cs{tl_convert_ii_normals:ww} with updated arguments.
%    \begin{macrocode}
\cs_new:Npn \tl_convert_ii_cs:Nww #1
  {
    0 -1 ,
    \tl_convert_cs_space_count:NN \tl_convert_ii_cs_test:ww #1
  }
\cs_new:Npn \tl_convert_ii_cs_test:ww #1 ; #2 ; #3; #4;
  {
    \exp_after:wN \tl_convert_ii_normals:ww
    \int_use:N \int_eval:w
    \if_num:w #1 = \c_zero
      #3
    \else:
      \tex_skip:D \int_eval:w #4 + #1 \int_eval_end:
    \fi:
    - #2
    \exp_after:wN ;
    \int_use:N \int_eval:w #4 + #1;
  }
%    \end{macrocode}
% \end{macro}
%
% \begin{macro}[aux]{\tl_convert_ii_special:w}
% \begin{macro}[aux]{\tl_convert_ii_special_char:wN}
% \begin{macro}[aux]{\tl_convert_ii_special_space:w}
%   Here, |#1| is the current index in the array built in the first pass.
%   Check whether we reached the end (we shouldn't insert the trailing
%   end-group character that marked the end of the token list in the
%   first pass).
%   Unpack the \tn{toks} register: when \texttt{x}-expanding again,
%   we will get the special token.
%   Then leave the category code in the input stream, followed by
%   the character code, and call \cs{tl_convert_ii_loop:w} with the next index.
%    \begin{macrocode}
\cs_new:Npn \tl_convert_ii_special:w \fi: \tl_convert_ii_normal:wwN 0; #1;
  {
    \fi:
    \if_num:w #1 = \l_tl_convert_index_int
      \exp_after:wN \prg_map_break:
    \fi:
    \tex_the:D \tex_toks:D #1 \s_tl_convert
    \if_case:w \etex_gluestretch:D \tex_skip:D #1 \exp_stop_f:
           A
    \or:   1
    \or:   1
    \else: 2
    \fi:
    \if_int_odd:w \etex_gluestretch:D \tex_skip:D #1 \exp_stop_f:
      \exp_after:wN \tl_convert_ii_special_char:wN \int_use:N
    \else:
      \exp_after:wN \tl_convert_ii_special_space:w \int_use:N
    \fi:
    \int_eval:w \c_one + #1 \exp_after:wN ;
    \token_to_str:N
  }
\cs_new:Npn \tl_convert_ii_special_char:wN #1 ; #2
  {
    \int_value:w `#2 ,
    \tl_convert_ii_loop:w #1 ;
  }
\cs_new:Npn \tl_convert_ii_special_space:w #1 ; ~
  {
    32 ,
    \tl_convert_ii_loop:w #1 ;
  }
%    \end{macrocode}
% \end{macro}
% \end{macro}
% \end{macro}
%
% \subsection{Showing the results}
%
% \begin{macro}{\tl_display:N}
% \begin{macro}{\tl_display:n}
%   Add to \cs{tl_convert:n} a third pass to display tokens to the terminal.
%    \begin{macrocode}
\cs_new_protected:Npn \tl_display:N #1
  {
    \exp_args:No \tl_convert:n {#1}
    \msg_aux_show:Nnx #1
      { tl-display }
      {
        \exp_after:wN \tl_display_loop:wNw \g_tl_convert_result_tl
          \s_tl_convert { ? \prg_map_break: } ,
        \prg_break_point:n { }
      }
  }
\cs_new_protected:Npn \tl_display:n #1
  {
    \tl_set:Nn \l_tl_tmpa_tl {#1}
    \tl_display:N \l_tl_tmpa_tl
  }
%    \end{macrocode}
% \end{macro}
% \end{macro}
%
% \begin{macro}[aux, EXP]{\tl_display_loop:wNw}
%   Here, |#1| \texttt{o}-expands to the token;
%   |#2| is the category code (one hexadecimal digit),
%   $0$ for control sequences;
%   |#3| is the character code, which we ignore.
%   In the cases of control sequences and active characters,
%   the meaning may overflow one line, and we want to truncate
%   it. Those cases are thus separated out.
%    \begin{macrocode}
\cs_new:Npn \tl_display_loop:wNw #1 \s_tl_convert #2 #3,
  {
    \use_none:n #2
    \iow_newline: > \c_space_tl \c_space_tl
    \if_num:w "#2 = \c_zero
      \exp_after:wN \tl_display_cs:n
    \else:
      \if_num:w "#2 = \c_thirteen
        \exp_after:wN \exp_after:wN
        \exp_after:wN \tl_display_active:n
      \else:
        \exp_after:wN \exp_after:wN
        \exp_after:wN \tl_display_normal:n
      \fi:
    \fi:
    {#1}
    \tl_display_loop:wNw
  }
%    \end{macrocode}
% \end{macro}
%
% \begin{macro}[aux]{\tl_display_normal:n}
%   Non-active characters are a simple matter of printing
%   the character, and its meaning. One can check that
%   begin-group and end-group characters do not mess up
%   \TeX{}'s alignment status.
%    \begin{macrocode}
\cs_new:Npn \tl_display_normal:n #1
  {
    \exp_after:wN \token_to_str:N #1 ~
    ( \exp_after:wN \token_to_meaning:N #1 )
  }
%    \end{macrocode}
% \end{macro}
%
% \begin{macro}[aux]{\tl_display_cs:n}
% \begin{macro}[aux]{\tl_display_active:n}
% \begin{macro}[aux]{\tl_display_long:nn}
% \begin{macro}[aux]{\tl_display_long_aux:nnn}
%   Control sequences and active characters are printed in the same way,
%   making sure not to go beyond the \cs{l_iow_line_length_int}. In case
%   of an overflow, we replace the last characters by
%   \cs{c_tl_display_etc_str}.
%    \begin{macrocode}
\cs_new:Npn \tl_display_cs:n #1
  { \exp_args:No \tl_display_long:nn {#1} { control~sequence= } }
\cs_new:Npn \tl_display_active:n #1
  { \exp_args:No \tl_display_long:nn {#1} { active~character= } }
\cs_new:Npn \tl_display_long:nn #1
  {
    \exp_args:Noo \tl_display_long_aux:nnn
      { \token_to_str:N #1 }
      { \token_to_meaning:N #1 }
  }
\cs_new_nopar:Npn \tl_display_long_aux:nnn #1#2#3
  {
    \int_compare:nNnTF
      { \str_length:n { >>> #1 ~ ( #3 #2 ) } }
      > { \l_iow_line_length_int }
      {
        \str_substr:nnn { >>> #1 ~ ( #3 #2 ) } \c_three
          { \l_iow_line_length_int - \str_length:N \c_tl_display_etc_str }
        \c_tl_display_etc_str
      }
      {#1~(#3#2)}
  }
%    \end{macrocode}
% \end{macro}
% \end{macro}
% \end{macro}
% \end{macro}
%
% \begin{variable}{\c_tl_display_etc_str}
%   When a control sequence (or active character)
%   and its meaning are too long to fit in one line
%   of the terminal, the end is replaced by this token list.
%    \begin{macrocode}
\str_const:Nx \c_tl_display_etc_str % (
  { \iow_char:N \\ETC. ) }
%    \end{macrocode}
% \end{variable}
%
%    \begin{macrocode}
\msg_kernel_new:nnn { tl-display } { show }
  {
    Token~list~
    \str_if_eq:nnF {#1} { \l_tl_tmpa_tl } { \token_to_str:N #1~}
    \tl_if_empty:NTF #1
      { is~empty }
      { contains~the~tokens: }
  }
%    \end{macrocode}
%
%    \begin{macrocode}
%</initex|package>
%    \end{macrocode}
%
% \end{implementation}
%
% \PrintIndex

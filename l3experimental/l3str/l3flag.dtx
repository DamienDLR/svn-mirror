% \iffalse meta-comment
%
%% File: l3flag.dtx Copyright (C) 2011 The LaTeX3 Project
%%
%% It may be distributed and/or modified under the conditions of the
%% LaTeX Project Public License (LPPL), either version 1.3c of this
%% license or (at your option) any later version.  The latest version
%% of this license is in the file
%%
%%    http://www.latex-project.org/lppl.txt
%%
%% This file is part of the "l3experimental bundle" (The Work in LPPL)
%% and all files in that bundle must be distributed together.
%%
%% The released version of this bundle is available from CTAN.
%%
%% -----------------------------------------------------------------------
%%
%% The development version of the bundle can be found at
%%
%%    http://www.latex-project.org/svnroot/experimental/trunk/
%%
%% for those people who are interested.
%%
%%%%%%%%%%%
%% NOTE: %%
%%%%%%%%%%%
%%
%%   Snapshots taken from the repository represent work in progress and may
%%   not work or may contain conflicting material!  We therefore ask
%%   people _not_ to put them into distributions, archives, etc. without
%%   prior consultation with the LaTeX3 Project.
%%
%% -----------------------------------------------------------------------
%
%<*driver|package>
\RequirePackage{expl3}
\GetIdInfo$Id: l3flag.dtx 3039 2011-12-08 09:22:35Z bruno $
  {L3 Experimental Flags}
%</driver|package>
%<*driver>
\documentclass[full]{l3doc}
\usepackage{amsmath}
\begin{document}
  \DocInput{\jobname.dtx}
\end{document}
%</driver>
% \fi
%
%
% \title{^^A
%   The \textsf{l3flag} package: expandable flags^^A
%   \thanks{This file describes v\ExplFileVersion,
%     last revised \ExplFileDate.}^^A
% }
%
% \author{^^A
%  The \LaTeX3 Project\thanks
%    {^^A
%      E-mail:
%        \href{mailto:latex-team@latex-project.org}
%          {latex-team@latex-project.org}^^A
%    }^^A
% }
%
% \date{Released \ExplFileDate}
%
% \maketitle
%
% \begin{documentation}
%
% Flags are the only data structure on which \TeX{} can perform
% assignments in expansion only contexts. This module is meant
% mostly for kernel use: in almost all cases, booleans should
% be preferred to flags, because they are faster.
%
% A flag can be in two states: \enquote{lowered} or \enquote{raised}.
% The status of a flag can be tested expandably (just like booleans).
% A flag can also be \emph{raised} expandably. However, a flag cannot
% be lowered expandably.
%
% Flag variables are always local.
%
% Testing for the existence of flags with \cs{cs_if_exist:NTF} will
% always result in a negative answer: from the point of view of
% \TeX{}, flags are either undefined control sequences, or let to
% \cs{scan_stop:}. This has a second consequence: flags don't need
% to be declared with \cs{flag_new:N}, although this has error-checking
% benefits. The ability of using flags without declaring them is used
% in \pkg{l3rand}.
%
% \begin{function}{\flag_new:N, \flag_new:c}
%   \begin{syntax}
%     \cs{flag_new:N} \meta{flag}
%   \end{syntax}
%   Creates a new \meta{flag} or raises an error if the
%   name is already taken. The declaration is global,
%   but \meta{flags} are always local variables.
%   The \meta{flag} will initially be lowered.
% \end{function}
%
% \begin{function}[EXP,pTF]{\flag_test:N, \flag_test:c}
%   \begin{syntax}
%     \cs{flag_test:N} \meta{flag}
%   \end{syntax}
%   This function returns \texttt{true} if the \meta{flag} is raised,
%   and \texttt{false} if the \meta{flag} is lowered.
% \end{function}
%
% \begin{function}[EXP]{\flag_raise:N}
%   \begin{syntax}
%     \cs{flag_raise:N} \meta{flag}
%   \end{syntax}
%   The \meta{flag} is raised. The assignment is local.
%   This function is expandable, despite being an assignment.
% \end{function}
%
% \begin{function}{\flag_lower:N}
%   \begin{syntax}
%     \cs{flag_lower:N} \meta{flag}
%   \end{syntax}
%   The \meta{flag} is lowered. The assignment is local.
% \end{function}
%
% \end{documentation}
%
% \begin{implementation}
%
% \section{\pkg{l3str} implementation}
%
%    \begin{macrocode}
%<*initex|package>
%    \end{macrocode}
%
%    \begin{macrocode}
\ProvidesExplPackage
  {\ExplFileName}{\ExplFileDate}{\ExplFileVersion}{\ExplFileDescription}
%    \end{macrocode}
%
% \begin{variable}{\g_flag_list_tl}
%   We keep track of the list of all defined flags in a token list.
%   This is only used by \cs{flag_new:N} to check that the flag
%   has not yet been defined.
%    \begin{macrocode}
\tl_new:N \g_flag_list_tl
%    \end{macrocode}
% \end{variable}
%
% \begin{macro}{\flag_new:N, \flag_new:c}
%   If the control sequence is already defined, then we raise an error.
%   Otherwise we check if it is already a flag, raising an appropriate
%   error, finally lower the flag globally (this is the only global
%   assignment to a flag\ldots{} perhaps an error?).
%   The \texttt{c} variant can be created without worrying for this function.
%   \begin{macrocode}
\cs_new_protected:Npn \flag_new:N #1
  {
    \tl_if_in:NnTF \g_flag_list_tl {#1}
      {
        \msg_kernel_error:nnx
          { flag } { already-defined }
          { \token_to_str:N #1 }
      }
      {
        \tl_gput_right:Nn \g_flag_list_tl {#1}
        \cs_new_eq:NN #1 \c_undefined:D
      }
  }
\cs_generate_variant:Nn \flag_new:N { c }
%    \end{macrocode}
% \end{macro}
%
% \begin{macro}[EXP, pTF]{\flag_test:N, \flag_test:c}
%   Test if the control sequence is \texttt{undefined} or not.
%    \begin{macrocode}
\prg_new_conditional:Npnn \flag_test:N #1 { p , T , F , TF }
  {
    \if_cs_exist:N #1
      \prg_return_true:
    \else:
      \prg_return_false:
    \fi:
  }
\prg_new_conditional:Npnn \flag_test:c #1 { p , T , F , TF }
  {
    \if_cs_exist:w #1 \cs_end:
      \prg_return_true:
    \else:
      \prg_return_false:
    \fi:
  }
%    \end{macrocode}
% \end{macro}
%
% \begin{macro}[EXP]{\flag_raise:N, \flag_raise:c}
%   Raising a flag expandably relies on the fact that \TeX{}
%   automatically lets undefined control sequences to \tn{relax}
%   when building an unknown csname. We build such a csname, then
%   remove it with \cs{use_none:n}, essentially doing
%   \cs{use_none:c}, but optimized slightly. When a flag is given
%   as an \texttt{N}-type argument, a little more work is needed,
%   converting to a csname before raising that.
%    \begin{macrocode}
\cs_new:Npn \flag_raise:N #1
  { \exp_after:wN \use_none:n \cs:w \cs_to_str:N #1 \cs_end: }
\cs_new:Npn \flag_raise:c #1
  { \exp_after:wN \use_none:n \cs:w #1 \cs_end: }
%    \end{macrocode}
% \end{macro}
%
% \begin{macro}{\flag_lower:N, \flag_lower:c}
%   Simply undefine the control sequence, locally.
%    \begin{macrocode}
\cs_new_protected:Npn \flag_lower:N #1
  { \cs_set_eq:NN #1 \c_undefined:D }
\cs_generate_variant:Nn \flag_lower:N { c }
%    \end{macrocode}
% \end{macro}
%
%    \begin{macrocode}
\msg_kernel_new:nnnn { flag } { already-defined }
  { The~control~sequence~#1~is~already~declared~as~a~flag. }
  {
    LaTeX~was~asked~to~define~the~flag~#1,~but~it~has~already~
    been~defined~as~a~flag.~The~flag~module~is~mostly~meant~
    for~kernel~use,~and~booleans~should~be~preferred.
  }
%    \end{macrocode}
%
%    \begin{macrocode}
%</initex|package>
%    \end{macrocode}
%
% \end{implementation}
%
% \PrintIndex

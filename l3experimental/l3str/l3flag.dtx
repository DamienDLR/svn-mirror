% \iffalse meta-comment
%
%% File: l3flag.dtx Copyright (C) 2011-2012 The LaTeX3 Project
%%
%% It may be distributed and/or modified under the conditions of the
%% LaTeX Project Public License (LPPL), either version 1.3c of this
%% license or (at your option) any later version.  The latest version
%% of this license is in the file
%%
%%    http://www.latex-project.org/lppl.txt
%%
%% This file is part of the "l3experimental bundle" (The Work in LPPL)
%% and all files in that bundle must be distributed together.
%%
%% The released version of this bundle is available from CTAN.
%%
%% -----------------------------------------------------------------------
%%
%% The development version of the bundle can be found at
%%
%%    http://www.latex-project.org/svnroot/experimental/trunk/
%%
%% for those people who are interested.
%%
%%%%%%%%%%%
%% NOTE: %%
%%%%%%%%%%%
%%
%%   Snapshots taken from the repository represent work in progress and may
%%   not work or may contain conflicting material!  We therefore ask
%%   people _not_ to put them into distributions, archives, etc. without
%%   prior consultation with the LaTeX3 Project.
%%
%% -----------------------------------------------------------------------
%
%<*driver|package>
\RequirePackage{expl3}
\GetIdInfo$Id: l3flag.dtx 3039 2011-12-08 09:22:35Z bruno $
  {L3 Experimental Flags}
%</driver|package>
%<*driver>
\documentclass[full]{l3doc}
\usepackage{amsmath}
\begin{document}
  \DocInput{\jobname.dtx}
\end{document}
%</driver>
% \fi
%
%
% \title{^^A
%   The \textsf{l3flag} package: expandable flags^^A
%   \thanks{This file describes v\ExplFileVersion,
%     last revised \ExplFileDate.}^^A
% }
%
% \author{^^A
%  The \LaTeX3 Project\thanks
%    {^^A
%      E-mail:
%        \href{mailto:latex-team@latex-project.org}
%          {latex-team@latex-project.org}^^A
%    }^^A
% }
%
% \date{Released \ExplFileDate}
%
% \maketitle
%
% \begin{documentation}
%
% Flags are the only data structure on which \TeX{} can perform
% assignments in expansion-only contexts. This module is meant mostly
% for kernel use: in almost all cases, booleans or integers should be
% preferred to flags, because they are faster.
%
% A flag can hold any non-negative value. In expansion-only contexts,
% this value can only be increased. The value can also be queried
% expandably. However, it can only be decreased or set to zero
% non-expandably.
%
% Flag variables are always local. They are referenced by a \meta{name}
% of the form \meta{package}\texttt{_}\meta{flag name}, for instance,
% \texttt{str_missing}.
%
% \section{Non-expandable flag commands}
%
% \begin{function}{\flag_new:n}
%   \begin{syntax}
%     \cs{flag_new:n} \Arg{flag name}
%   \end{syntax}
%   Creates a new \meta{flag} with a name given by converting the
%   \meta{flag name} to a string (through \cs{tl_to_str:n}), or raises
%   an error if the name is already taken. The declaration is global,
%   but flags are always local variables.  The \meta{flag} will
%   initially hold the value $0$.
% \end{function}
%
% \begin{function}{\flag_clear:n}
%   \begin{syntax}
%     \cs{flag_clear:n} \Arg{flag name}
%   \end{syntax}
%   The \meta{flag} is set to zero. The assignment is local.
% \end{function}
%
% \begin{function}{\flag_clear_new:n}
%   \begin{syntax}
%     \cs{flag_clear_new:n} \Arg{flag name}
%   \end{syntax}
%   Ensures that the \meta{flag} exists globally by applying
%   \cs{flag_new:n} if necessary, then applies \cs{flag_zero:n}, setting
%   the \meta{flag} locally to zero.
% \end{function}
%
% \section{Expandable flag commands}
%
% \begin{function}[EXP,pTF]{\flag_if_exist:n}
%   \begin{syntax}
%     \cs{flag_if_exist:n} \Arg{flag name}
%   \end{syntax}
%   This function returns \texttt{true} if the \meta{flag name}
%   references a flag that has been defined previously, and
%   \texttt{false} otherwise.
% \end{function}
%
% \begin{function}[EXP,pTF]{\flag_test:n}
%   \begin{syntax}
%     \cs{flag_test:n} \Arg{flag name}
%   \end{syntax}
%   This function returns \texttt{true} if the \meta{flag} is non-zero,
%   and \texttt{false} if the \meta{flag} is zero.
% \end{function}
%
% \begin{function}[EXP]{\flag_height:n}
%   \begin{syntax}
%     \cs{flag_height:n} \Arg{flag name}
%   \end{syntax}
%   Expands to the value of the \meta{flag} as an integer denotation.
% \end{function}
%
% \begin{function}[EXP]{\flag_raise:n}
%   \begin{syntax}
%     \cs{flag_raise:n} \Arg{flag name}
%   \end{syntax}
%   The \meta{flag} is increased by $1$. The assignment is local.
%   This function is expandable, despite being an assignment.
% \end{function}
%
% \end{documentation}
%
% \begin{implementation}
%
% \section{\pkg{l3flag} implementation}
%
%    \begin{macrocode}
%<*initex|package>
%    \end{macrocode}
%
%    \begin{macrocode}
\ProvidesExplPackage
  {\ExplFileName}{\ExplFileDate}{\ExplFileVersion}{\ExplFileDescription}
%    \end{macrocode}
%
% \subsection{Variables}
%
% \begin{variable}{\l_flag_internal_int}
%   Integer used for various scratch purposes.
%    \begin{macrocode}
\int_new:N \l_flag_internal_int
%    \end{macrocode}
% \end{variable}
%
% \subsection{Non-expandable flag commands}
%
% \begin{macro}{\flag_new:n}
% \begin{macro}[aux]{\flag_new_aux:w}
%   For each flag, we define a \enquote{trap} function, which currently
%   simply increases the flag by $1$.
%   \begin{macrocode}
\cs_new_protected:Npn \flag_new:n #1
  { \exp_after:wN \flag_new_aux:w \tl_to_str:n {#1} \q_stop }
\cs_new_protected:Npn \flag_new_aux:w #1 \q_stop
  {
    \cs_new:cpn { flag_trap_#1:n } ##1
      { \exp_after:wN \use_none:n \cs:w l_#1_##1_flag \cs_end: }
    \cs_new_eq:cN { l_#1_0_flag } \c_undefined:D
  }
%    \end{macrocode}
% \end{macro}
% \end{macro}
%
% \begin{macro}{\flag_clear:n}
% \begin{macro}[aux]{\flag_clear_aux:ww}
%   Undefine control sequences, starting from the |_0| flag, upwards,
%   until reaching an undefined control sequence.
%    \begin{macrocode}
\cs_new_protected:Npn \flag_clear:n #1
  {
    \int_zero:N \l_flag_internal_int
    \exp_after:wN \flag_clear_aux:ww
    \exp_after:wN 0
    \exp_after:wN ;
    \tl_to_str:n {#1} \q_stop
  }
\cs_new_protected:Npn \flag_clear_aux:ww #1 ; #2 \q_stop
  {
    \if_cs_exist:w l_#2_#1_flag \cs_end:
    \else:
      \exp_after:wN \use_none_delimit_by_q_stop:w
    \fi:
    \cs_set_eq:cN { l_#2_#1_flag } \c_undefined:D
    \exp_after:wN \flag_clear_aux:ww
    \int_use:N \int_eval:w \c_one + #1 ;
    #2 \q_stop
  }
%    \end{macrocode}
% \end{macro}
% \end{macro}
%
% \begin{macro}{\flag_clear_new:n}
%   A flag exist if \cs{flag_trap_\meta{flag name}:n} exists.
%    \begin{macrocode}
\cs_new_protected:Npn \flag_clear_new:n #1
  { \flag_if_exist:nTF {#1} { \flag_clear:n } { \flag_new:n } {#1} }
%    \end{macrocode}
% \end{macro}
%
% \subsection{Expandable flag commands}
%
% \begin{macro}[EXP, pTF]{\flag_if_exist:n}
%   A \meta{flag} is defined if the corresponding \enquote{trap} is
%   defined.
%    \begin{macrocode}
\prg_new_conditional:Npnn \flag_if_exist:n #1 { p , T , F , TF }
  {
    \cs_if_exist:cTF { flag_trap_ \tl_to_str:n {#1} :n }
      { \prg_return_true: } { \prg_return_false: }
  }
%    \end{macrocode}
% \end{macro}
%
% \begin{macro}[EXP, pTF]{\flag_test:n}
%   Test if the flag is non-zero, by checking the |_0| control sequence.
%    \begin{macrocode}
\prg_new_conditional:Npnn \flag_test:n #1 { p , T , F , TF }
  {
    \if_cs_exist:w l_#1_0_flag \cs_end:
      \prg_return_true:
    \else:
      \prg_return_false:
    \fi:
  }
%    \end{macrocode}
% \end{macro}
%
% \begin{macro}[EXP]{\flag_height:n}
% \begin{macro}[EXP, aux]{\flag_height_loop:ww, \flag_height_end:ww}
%   Extract the value of the flag by going through all of the
%   |_|\meta{integer} control sequences starting from $0$.
%    \begin{macrocode}
\cs_new:Npn \flag_height:n #1
  {
    \exp_after:wN \flag_height_loop:ww
    \exp_after:wN 0
    \exp_after:wN ;
    \tl_to_str:n {#1} \q_stop
  }
\cs_new:Npn \flag_height_loop:ww #1 ; #2 \q_stop
  {
    \if_cs_exist:w l_#2_#1_flag \cs_end:
      \exp_after:wN \flag_height_loop:ww \int_use:N \int_eval:w \c_one +
    \else:
      \exp_after:wN \flag_height_end:ww
    \fi:
    #1 ; #2 \q_stop
  }
\cs_new:Npn \flag_height_end:ww #1 ; #2 \q_stop { #1 }
%    \end{macrocode}
% \end{macro}
% \end{macro}
%
% \begin{macro}[EXP]{\flag_raise:n}
% \begin{macro}[EXP, aux]{\flag_raise_loop:ww, \flag_raise_end:ww}
%   Raising a flag expandably relies on the fact that \TeX{}
%   automatically lets undefined control sequences to \tn{relax}
%   when building an unknown csname. We build such a csname, then
%   remove it with \cs{use_none:n}, essentially doing
%   \cs{use_none:c}, but optimized slightly.
%^^A todo: update comments, since that's hidden inside the "trap".
%    \begin{macrocode}
\cs_new:Npn \flag_raise:n #1
  {
    \exp_after:wN \flag_raise_loop:ww
    \exp_after:wN 0
    \exp_after:wN ;
    \tl_to_str:n {#1} \q_stop
  }
\cs_new:Npn \flag_raise_loop:ww #1 ; #2 \q_stop
  {
    \if_cs_exist:w l_#2_#1_flag \cs_end:
      \exp_after:wN \flag_raise_loop:ww \int_use:N \int_eval:w \c_one +
    \else:
      \exp_after:wN \flag_raise_end:ww
    \fi:
    #1 ; #2 \q_stop
  }
\cs_new:Npn \flag_raise_end:ww #1 ; #2 \q_stop
  { \cs:w flag_trap_#2:n \cs_end: {#1} }
%    \end{macrocode}
% \end{macro}
% \end{macro}
%
%    \begin{macrocode}
%</initex|package>
%    \end{macrocode}
%
% \end{implementation}
%
% \PrintIndex

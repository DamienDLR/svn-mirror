% \iffalse
%% File: l3tlp.dtx Copyright (C) 1990-2005 LaTeX3 project
%%
%% It may be distributed and/or modified under the conditions of the
%% LaTeX Project Public License (LPPL), either version 1.3a of this
%% license or (at your option) any later version.  The latest version
%% of this license is in the file
%%
%%    http://www.latex-project.org/lppl.txt
%%
%% This file is part of the ``expl3 bundle'' (The Work in LPPL)
%% and all files in that bundle must be distributed together.
%%
%% The released version of this bundle is available from CTAN.
%%
%% -----------------------------------------------------------------------
%%
%% The development version of the bundle can be found at
%%
%%    http://www.latex-project.org/cgi-bin/cvsweb.cgi/
%%
%% for those people who are interested.
%%
%%%%%%%%%%%
%% NOTE: %%
%%%%%%%%%%%
%%
%%   Snapshots taken from the repository represent work in progress and may
%%   not work or may contain conflicting material!  We therefore ask
%%   people _not_ to put them into distributions, archives, etc. without
%%   prior consultation with the LaTeX Project Team.
%%
%% -----------------------------------------------------------------------
%
%<*!initex>
%\fi
\def\next#1: #2.dtx,v #3 #4 #5 #6 #7$#8{%^^A}$
  \def\fileversion{#3}%
  \def\filedate{#4}%
%\iffalse
%<*dtx>
%\fi
     \ProvidesFile{#2.dtx}[#4 v#3 #8]%
%\iffalse
%</dtx>
%<package> \ProvidesPackage{#2}[#4 #3 #5 #6]%
%<driver>  \ProvidesFile{#2.drv}[#4 v#3 #8]%
%\fi
}
%\iffalse
%</!initex>
%\fi
\next$Id$
          {L3 Experimental Token List Pointers}
%
% \iffalse
%<*driver>
\documentclass{l3doc}

\begin{document}
\DocInput{l3tlp.dtx}
\end{document}
%</driver>
% \fi
%
%
% \title{The \textsf{l3tlp} package\thanks{This file
%         has version number \fileversion, last
%         revised \filedate.}\\
% Token List Pointers}
% \author{\Team}
% \date{\filedate}
% \maketitle
%
%
%
% \section {Token list pointers}
%
% \LaTeX3 stores token lists in so called `token list pointers'.
% Variables of this type get the suffix "tlp" and functions of this type
% have the prefix "tlp". To use a  token list pointer you simply call
% the corresponding variable.
%
% \subsection{Functions}
%
% \begin{function}{\tlp_new:Nn |
%                  \tlp_new:cn}
% \begin{syntax}
%    "\tlp_new:Nn" <tlp> "{" <initial token list> "}"
% \end{syntax}
% Defines <variable> to be a new variable (or constant) of type token
% list pointer. <initial token list> is the initial value of
% <tlp>. This makes it possible to assign values to a constant
% token list pointer.
% \end{function}
%
% \begin{function}{%
%                  \tlp_use:N |
%                  \tlp_use:c
% }
% \begin{syntax}
%   "\tlp_use:N" <tlp>
% \end{syntax}
% Function that inserts the <tlp> into the processing stream.
% \end{function}
%
% \begin{function}{%
%                  \tlp_set:Nn |
%                  \tlp_set:Nc |
%                  \tlp_set:No |
%                  \tlp_set:Nf |
%                  \tlp_set:Nx |
%                  \tlp_gset:Nn |
%                  \tlp_gset:Nc |
%                  \tlp_gset:No |
%                  \tlp_gset:Nx |
%                  \tlp_gset:cn |
%                  \tlp_gset:cx }
% \begin{syntax}
%   "\tlp_set:Nn" <tlp> "{" <token list> "}"
% \end{syntax}
% Defines <tlp> to hold the token list <token list>. Global
% variants of this command assign the value globally the other variants
% expand the <token list> up to a certain level before the assignment
% or interpret the <tokenlist> as a character list and form a control
% sequence out of it.
% \end{function}
%
% \begin{function}{%
%                  \tlp_clear:N |
%                  \tlp_clear:c |
%                  \tlp_gclear:N |
%                  \tlp_gclear:c
% }
% \begin{syntax}
%   "\tlp_clear:N" <tlp>
% \end{syntax}
% The <tlp> is locally or globally cleared. The "c" variants will
% generate a control sequence name which is then interpreted as
% <tlp> before clearing.
% \end{function}
%
% \begin{function}{%
%                  \tlp_clear_new:N |
%                  \tlp_clear_new:c |
% }
% \begin{syntax}
%   "\tlp_clear_new:N" <tlp>
% \end{syntax}
% These functions check if <tlp> exists. If it does it will be cleared;
% if it doesn't it will be allocated.
% \end{function}
%
% \begin{function}{%
%                  \tlp_put_left:Nn |
%                  \tlp_put_left:No |
%                  \tlp_gput_left:Nn |
%                  \tlp_gput_left:No |
%                  \tlp_put_right:Nn |
%                  \tlp_put_right:cc |
%                  \tlp_gput_right:Nn |
%                  \tlp_gput_right:No|
%                  \tlp_gput_right:cn|
%                  \tlp_gput_right:co|
% }
% \begin{syntax}
%   "\tlp_put_left:Nn" <tlp> "{" <token list> "}"
% \end{syntax}
% These functions will append <token list> to the left or right of
% <tlp>. Assigment is done either locally or globally and <token
% list> might be subject to expansion before assigment.
% \end{function}
%
% \begin{function}{%
%                  \tlp_set_eq:NN |
%                  \tlp_gset_eq:NN |
%                  \tlp_gset_eq:Nc |
%                  \tlp_gset_eq:cN |
%                  \tlp_gset_eq:cc |
% }
% \begin{syntax}
%    "\tlp_set_eq:NN" <tlp1> <tlp2>
% \end{syntax}
% Fast form for
% \begin{syntax}
%    "\tlp_set:No" <tlp1> "{" <tlp2> "}"
% \end{syntax}
% when <tlp2> is known to be a variable of type token list pointer.
% \end{function}
%
% \begin{function}{\tlp_to_str:N |
%                  \tlp_to_str:c
% }
% \begin{syntax}
%   "\tlp_to_str:N" <tlp>
% \end{syntax}
% This function returns the token list kept in <tlp> as a string list
% with all characters catcoded to `other'.
% \end{function}
%
% \subsection{Predicates and conditionals}
%
% \begin{function}{\tlp_if_empty_p:N}
% \begin{syntax}
%   "\tlp_if_empty_p:N" <tlp>
% \end{syntax}
% This predicate returns `true' if <tlp> is `empty' i.e., doesn't
% contain any tokens.
% \end{function}
%
% \begin{function}{\tlp_if_empty:NF|
%                  \tlp_if_empty:NTF|
%                  \tlp_if_empty:cTF}
% \begin{syntax}
%   "\tlp_if_empty:NF" <tlp> "{"<false code>"}"
% \end{syntax}
% Execute <false code> if <tlp> is empty.
% contain any tokens.
% \end{function}
%
% \begin{function}{\tlp_if_eq:NNF |
%                  \tlp_if_eq:cNF
%                  }
% \begin{syntax}
%   "\tlp_if_eq:NNF" <tlp1> <tlp2> "{"<false code>"}"
% \end{syntax}
% Execute <false code> if <tlp1> doesn't hold the same token list as
% <tlp2>.
% \end{function}
%
% \subsection{Variable and constants}
%
% \begin{variable}{\C_job_name_tlp}
% Constant that gets the `job name' assigned when \TeX{} starts.
% \begin{texnote}
% This is the new name for the primitive \tn{jobname}. It is a constant
% that will be set by \TeX{} and can not be overwritten by the package.
% Therefore the "C"
% \end{texnote}
% \end{variable}
%
% \begin{variable}{\c_empty_tlp}
% Constant that is always empty.
% \begin{texnote}
% This was named \tn{@empty} in \LaTeX2 and \tn{empty} in plain \TeX{}.
% \end{texnote}
% \end{variable}
%
% \begin{variable}{\c_relax_tlp}
% Constant holding the token that is assigned to a newly created control
% sequence by \TeX.
% \end{variable}
%
% \begin{variable}{%
%                  \l_tmpa_tlp |
%                  \l_tmpb_tlp |
%                  \g_tmpa_tlp |
%                  \g_tmpb_tlp
% }
% Scratch register for immediate use. They are not used by conditionals
% or predicate functions.
% \end{variable}
%
% \subsubsection{Internal functions}
%
% \begin{function}{\tlp_put_left:aux}
% Used by "\tlp_put_left:Nn" and its variants.
% \end{function}
%
% \begin{function}{\tlp_to_str:aux}
% Function used to implement "\tlp_to_str:N".
% \end{function}
%
% \StopEventually{}
%
%    \section{The code}
%
%    We start by ensuring that the required packages are loaded.
%    \begin{macrocode}
%<package&!check>\RequirePackage{l3basics}\par
%<package&check>\RequirePackage{l3chk}\par
%<*initex|package>
%    \end{macrocode}
%
% A token list pointer is a control sequence that holds tokens.  The
% interface is similar to that for token registers, but beware that
% the behavior vis \'a vis |\def:Npx| etc. \ldots{} is different.  (You
% see this comes from Denys' implementation.)
%
% We don't implement a |\tlp_use:n| function. Execution is done by
% calling the list.
%
%
% \begin{macro}{\tlp_new:Nn}
% \begin{macro}{\tlp_new:cn}
% \begin{macro}{\tlp_clear:N}
% \begin{macro}{\tlp_gclear:N}
%    We provide one allocation function (which checks that the name is
%    not used) and two clear functions that locally or globally clear
%    the token list. The allocation function has two arguments to
%    specify an initial value. This is the only way to give values to
%    constants.
%    \begin{macrocode}
%<*check>
\def_long_new:Npn \tlp_new:Nn #1#2{
  \chk_new_cs:N #1
  \chk_var_or_const:N #1
%    \end{macrocode}
%    The next line contains the token list assignments without any
%    checking for variable types etc.\ since we want to allow to
%    update constants here.
%    \begin{macrocode}
  \gdef_long:Npn #1{#2}}
%</check>
%<-check> \def_long_new:Npn \tlp_new:Nn #1#2 {\chk_new_cs:N #1 \gdef_long:Npn #1{#2}}
\def_new:Npn \tlp_new:cn #1 {\exp_after:NN \tlp_new:Nn \cs:w #1 \cs_end: }
\def_new:Npn \tlp_clear:N #1{\tlp_set_eq:NN #1\c_empty_tlp}
\def_new:Npn \tlp_gclear:N #1{\tlp_gset_eq:NN #1\c_empty_tlp}
%    \end{macrocode}
% \end{macro}
% \end{macro}
% \end{macro}
% \end{macro}
%
% \begin{macro}{\tlp_use:N}
% \begin{macro}{\tlp_use:c}
%    \begin{macrocode}
\def_new:Npn \tlp_use:N #1 {
  \if_meaning:NN #1 \scan_stop:
%    \end{macrocode}
%  If \m{tlp} equals |\scan_stop:| it is probably stemming from a
%  |\cs:w ... \cd_end:| that was created by mistake somewhere.
%    \begin{macrocode}
     \err_latex_bug:n {Token~list~pointer~ `\token_to_string:N #1'~ has~ an~ erroneous~ structure!}
  \else:
    \exp_after:NN #1
  \fi:
}
\def_new:Npn \tlp_use:c {\exp_args:Nc \tlp_use:N}
%    \end{macrocode}
% \end{macro}
% \end{macro}
%
% \begin{macro}{\tlp_clear:c}
% \begin{macro}{\tlp_gclear:c}
%    We also define the variants that construct the token list name
%    from a string.
%    \begin{macrocode}
\def_new:Npn \tlp_clear:c {\exp_args:Nc \tlp_clear:N}
\def_new:Npn \tlp_gclear:c {\exp_args:Nc \tlp_gclear:N}
%    \end{macrocode}
% \end{macro}
% \end{macro}
%
% \begin{macro}{\tlp_clear_new:N}
% \begin{macro}{\tlp_clear_new:c}
%    These macros check whether a token list exists. If it does it
%    is cleared, if it doesn't it is allocated.
%    \begin{macrocode}
%<*check>
\def_new:Npn \tlp_clear_new:N #1{
     \chk_var_or_const:N #1
     \if:w \cs_exist_p:N #1
       \tlp_clear:N #1
     \else:
       \tlp_new:Nn #1{}
     \fi:}
%</check>
%<-check>\let_new:NN \tlp_clear_new:N \tlp_clear:N
\def_new:Npn \tlp_clear_new:c {\exp_args:Nc \tlp_clear_new:N}
%    \end{macrocode}
% \end{macro}
% \end{macro}
%
% \begin{macro}{\tlp_gclear_new:N}
% \begin{macro}{\tlp_gclear_new:c}
%    These are the global versions of the above.
%    \begin{macrocode}
%<*check>
\def_new:Npn \tlp_gclear_new:N #1{
  \chk_var_or_const:N #1
  \if:w \cs_exist_p:N #1
    \tlp_gclear:N #1
  \else:
    \tlp_new:Nn #1{}
  \fi:}
%</check>
%<-check>\let_new:NN \tlp_gclear_new:N \tlp_gclear:N
\def_new:Npn \tlp_gclear_new:c {\exp_args:Nc \tlp_gclear_new:N}
%    \end{macrocode}
% \end{macro}
% \end{macro}
%
% \begin{macro}{\tlp_put_left:Nn}
% \begin{macro}{\tlp_put_left:No}
% \begin{macro}{\tlp_gput_left:Nn}
% \begin{macro}{\tlp_gput_left:No}
% \begin{macro}{\tlp_put_left:aux}
%    We can add tokens to the left (either globally or locally).
%    \begin{macrocode}
\def_long_new:Npn \tlp_put_left:Nn #1{\exp_after:NN
%    \end{macrocode}
%    We need expanding over a brace to ensure that if |#1| contains
%    just one token within braces the braces are preserved.
%    \begin{macrocode}
  \tlp_put_left:aux\exp_after:NN{#1}#1}
\def_new:Npn\tlp_put_left:No{\exp_args:NNo\tlp_put_left:Nn}
\def_new:Npn \tlp_gput_left:Nn {
%<*check>
   \pref_global_chk:
%</check>
%<-check> \pref_global:D
   \tlp_put_left:Nn}
\def_new:Npn\tlp_gput_left:No{\exp_args:NNo\tlp_gput_left:Nn}
\def_long_new:Npn \tlp_put_left:aux #1#2#3{\def:Npn #2{#3#1}
%    \end{macrocode}
%    We check the type afterwards to avoid conflicts with the use of
%    |\pref_global:D|.
%    \begin{macrocode}
%<*check>
        \chk_local_or_pref_global:N #2
%</check>
}
%    \end{macrocode}
% \end{macro}
% \end{macro}
% \end{macro}
% \end{macro}
% \end{macro}
%
% \begin{macro}{\tlp_put_right:Nn}
% \begin{macro}{\tlp_put_right:No}
% \begin{macro}{\tlp_put_right:Nx}
% \begin{macro}{\tlp_put_right:cc}
% \begin{macro}{\tlp_gput_right:Nn}
% \begin{macro}{\tlp_gput_right:No}
% \begin{macro}{\tlp_gput_right:cn}
% \begin{macro}{\tlp_gput_right:co}
% \begin{macro}{\tlp_gput_right:Nx}
%    These are variants of the functions above, but for adding tokens
%    to the right.
%    \begin{macrocode}
\def_long_new:Npn \tlp_put_right:Nn #1#2{\tlp_set:No #1{#1#2}}
\def_long_new:Npn \tlp_gput_right:Nn #1#2{\tlp_gset:No #1{#1#2}}
\def_new:Npn \tlp_gput_right:No {\exp_args:NNo \tlp_gput_right:Nn}
\def_new:Npn \tlp_gput_right:Nx {\exp_args:NNx \tlp_gput_right:Nn}
\def_new:Npn \tlp_gput_right:cn {\exp_args:Nc \tlp_gput_right:Nn}
\def_new:Npn \tlp_gput_right:co {\exp_args:Nco \tlp_gput_right:Nn}
\def_new:Npn \tlp_put_right:cc {\exp_args:Ncc \tlp_put_right:Nn}
\def_new:Npn \tlp_put_right:No {\exp_args:NNo \tlp_put_right:Nn}
\def_new:Npn \tlp_put_right:Nx {\exp_args:Nnx \tlp_put_right:Nn}
%    \end{macrocode}
% \end{macro}
% \end{macro}
% \end{macro}
% \end{macro}
% \end{macro}
% \end{macro}
% \end{macro}
% \end{macro}
% \end{macro}
%
%
% \begin{macro}{\tlp_set:Nn}
% \begin{macro}{\tlp_set:No}
% \begin{macro}{\tlp_set:Nf}
% \begin{macro}{\tlp_set:Nx}
% \begin{macro}{\tlp_set:cn}
% \begin{macro}{\tlp_set:co}
% \begin{macro}{\tlp_set:cx}
% \begin{macro}{\tlp_gset:Nn}
% \begin{macro}{\tlp_gset:No}
% \begin{macro}{\tlp_gset:Nx}
% \begin{macro}{\tlp_gset:cn}
% \begin{macro}{\tlp_gset:cx}
% \begin{macro}{\tlp_set_eq:NN}
% \begin{macro}{\tlp_gset_eq:NN}
% \begin{macro}{\tlp_gset_eq:Nc}
% \begin{macro}{\tlp_gset_eq:cN}
% \begin{macro}{\tlp_gset_eq:cc}
%    To set token lists to a specific value to type of functions are
%    available: |\tlp_set_eq:NN| takes two token-lists as its
%    arguments assign the first the contents of the second;
%    |\tlp_set:Nn| has as its second argument a `real' list of tokens.
%    One can view |\tlp_set_eq:NN| as a special form of |\tlp_set:No|
%    (which is not implemented). Both functions have global
%    counterparts.
%
%    During development we check if the token list that is being
%    assigned to exists. If not, a warning will be issued.
%    \begin{macrocode}
%<*check>
\def_long_new:Npn \tlp_set:Nn #1#2{
       \chk_exist_cs:N #1
       \def_long:Npn #1{#2}
%    \end{macrocode}
%    We use |\chk_local_or_pref_global:N| after the assignment to
%    allow constructs with |\pref_global_chk:|. But one should note
%    that this is less efficient then using the real global variant
%    since they are builtin.
%    \begin{macrocode}
        \chk_local_or_pref_global:N #1}
\def_new:Npn \tlp_set:No {\exp_args:NNo \tlp_set:Nn}
\def_new:Npn \tlp_set:Nf {\exp_args:NNf \tlp_set:Nn}
\def_long_new:Npn \tlp_set:Nx #1#2{
        \chk_exist_cs:N #1
        \def_long:Npx #1{#2}\chk_local:N #1}
\def_new:Npn \tlp_set:cn{\exp_args:Nc\tlp_set:Nn}
\def_new:Npn \tlp_set:co{\exp_args:Nco\tlp_set:Nn}
\def_new:Npn \tlp_set:cx{\exp_args:Ncx\tlp_set:Nn}
\def_long_new:Npn \tlp_gset:Nn #1#2{
        \chk_exist_cs:N #1
        \gdef_long:Npn #1{#2}\chk_global:N #1}
\def_new:Npn \tlp_gset:No {
        \exp_args:NNo \tlp_gset:Nn}
\def_long_new:Npn \tlp_gset:Nx #1#2{
        \chk_exist_cs:N #1
        \gdef_long:Npx #1{#2}\chk_global:N #1}
\def_long_new:Npn \tlp_gset:cn#1#2{
        \chk_exist_cs:c {#1}
        \gdef_long:cpn{#1}{#2}\chk_global:c {#1}}
\def_long_new:Npn \tlp_gset:cx#1#2{
        \chk_exist_cs:c {#1}
        \gdef_long:cpx{#1}{#2}\chk_global:c {#1}}
\def_new:Npn \tlp_set_eq:NN #1#2{
        \chk_exist_cs:N #1
        \let:NwN #1#2
        \chk_local_or_pref_global:N #1\chk_var_or_const:N #2}
\def_new:Npn \tlp_gset_eq:NN #1#2{
        \chk_exist_cs:N #1
        \glet:NN #1#2
        \chk_global:N #1\chk_var_or_const:N #2}
\def_new:Npn \tlp_gset_eq:Nc #1#2{
        \chk_exist_cs:N #1
        \glet:Nc #1{#2}
        \chk_global:N #1\chk_var_or_const:c {#2}}
\def_new:Npn \tlp_gset_eq:cN #1#2{
        \chk_exist_cs:c {#1}
        \glet:cN {#1}#2
        \chk_global:c {#1}\chk_var_or_const:N #2}
\def_new:Npn \tlp_gset_eq:cc #1#2{
        \chk_exist_cs:c {#1}
        \glet:cc {#1}{#2}
        \chk_global:c {#1}\chk_var_or_const:c {#2}}
%</check>
%    \end{macrocode}
%    For some functions like |\tlp_set:Nn| we need to define the
%    `non-check' version with  arguments since we want to allow
%    constructions like |\tlp_set:Nn\l_tmpa_tlp\foo| and so we can't
%    use the primitive \TeX{} command.
%    \begin{macrocode}
%<-check> \def_long_new:Npn\tlp_set:Nn#1#2{\def_long:Npn#1{#2}}
%<-check> \def_new:Npn \tlp_set:Nf {\exp_args:NNf \tlp_set:Nn}
%<-check> \def_long_new:Npn\tlp_set:Nx#1#2{\def_long:Npx#1{#2}}
%<-check> \def_long_new:Npn\tlp_gset:Nn#1#2{\gdef_long:Npn#1{#2}}
%<-check> \def_long_new:Npn\tlp_gset:Nx#1#2{\gdef_long:Npx#1{#2}}
%<-check> \def_long_new:Npn\tlp_gset:cn#1#2{\gdef_long:cpn{#1}{#2}}
%<-check> \def_long_new:Npn\tlp_gset:cx#1#2{\gdef_long:cpx{#1}{#2}}
%<-check> \def_new:Npn \tlp_set:cn{\exp_args:Nc\tlp_set:Nn}
%<-check> \def_new:Npn \tlp_set:co{\exp_args:Nco\tlp_set:Nn}
%<-check> \def_new:Npn \tlp_set:cx{\exp_args:Ncx\tlp_set:Nn}
%<-check> \def_new:Npn \tlp_set:No {\exp_args:NNo \tlp_set:Nn}
%<-check> \def_new:Npn \tlp_gset:No {\exp_args:NNo \tlp_gset:Nn}
%<-check> \let_new:NN\tlp_set_eq:NN \let:NN
%<-check> \let_new:NN\tlp_set_eq:Nc \let:Nc
%<-check> \let_new:NN\tlp_set_eq:cN \let:cN
%<-check> \let_new:NN\tlp_set_eq:cc \let:cc
%<-check> \let_new:NN\tlp_gset_eq:NN \glet:NN
%<-check> \let_new:NN\tlp_gset_eq:Nc \glet:Nc
%<-check> \let_new:NN\tlp_gset_eq:cN \glet:cN
%<-check> \let_new:NN\tlp_gset_eq:cc \glet:cc
%    \end{macrocode}
% \end{macro}
% \end{macro}
% \end{macro}
% \end{macro}
% \end{macro}
% \end{macro}
% \end{macro}
% \end{macro}
% \end{macro}
% \end{macro}
% \end{macro}
% \end{macro}
% \end{macro}
% \end{macro}
% \end{macro}
% \end{macro}
% \end{macro}
%
%
% \begin{macro}{\tlp_gset:Nc}
% \begin{macro}{\tlp_set:Nc}
%    These two functions are included because they are necessary in
%    Denys' implementations. The |:Nc| convention (see the expansion
%    module) is very unusual at first sight, but it works nicely
%    over all modules, so we would like to keep it.
%
%    Construct a control sequence on the fly from |#2| and save it in
%    |#1|.
%    \begin{macrocode}
\def_new:Npn \tlp_gset:Nc {
%<*check>
   \pref_global_chk:
%</check>
%<-check> \pref_global:D
   \tlp_set:Nc}
%    \end{macrocode}
%    |\pref_global_chk:| will turn the variable check in |\tlp_set:No|
%    into a  global check.
%    \begin{macrocode}
\def_new:Npn \tlp_set:Nc #1#2{\tlp_set:No #1{\cs:w#2\cs_end:}}
%    \end{macrocode}
% \end{macro}
% \end{macro}
%
%
% We also provide a few conditionals, both in expandable form (with
% |\c_true|) and in `brace-form', the latter are denoted by |TF| at the
% end, as explained elsewhere.
%
%
% \begin{macro}{\tlp_if_empty_p:N}
%   Returns |\c_true| iff the token list in the argument is empty.
%    \begin{macrocode}
\def_new:Npn \tlp_if_empty_p:N #1{
  \if_meaning:NN#1\c_empty_tlp \c_true \else: \c_false \fi:}
%    \end{macrocode}
% \end{macro}
%
%
% \begin{macro}{\tlp_if_empty:NF}
% \begin{macro}{\tlp_if_empty:NTF}
% \begin{macro}{\tlp_if_empty:cTF}
%    These functions check whether the token list in the argument is
%    empty and execute the proper code from their argument(s).
%    \begin{macrocode}
\def_new:Npn \tlp_if_empty:NF #1{
  \if_meaning:NN#1\c_empty_tlp
    \exp_after:NN\use_none:nn
  \fi:
  \use_arg_i:n}
\def_new:Npn \tlp_if_empty:NTF #1{
  \if_meaning:NN#1\c_empty_tlp
    \exp_after:NN\use_arg_i:nn
  \else:
    \exp_after:NN\use_arg_ii:nn
  \fi:}
\def_new:Npn \tlp_if_empty:cTF {\exp_args:Nc \tlp_if_empty:NTF}
%    \end{macrocode}
% \end{macro}
% \end{macro}
% \end{macro}
%
% \begin{macro}{\tlp_if_eq:NNF}
% \begin{macro}{\tlp_if_eq:cNF}
%    This function test whether the token lists that are in its first
%    two arguments are equal; if they are not |#3| is executed.
%    \begin{macrocode}
\def_new:Npn \tlp_if_eq:NNF #1#2{
 \if_meaning:NN#1#2
    \exp_after:NN\use_none:nn
 \fi:
 \use_arg_i:n}
\def_new:Npn \tlp_if_eq:cNF {\exp_args:Nc \tlp_if_eq:NNF}
%    \end{macrocode}
% \end{macro}
% \end{macro}
%
%
% \begin{macro}{\tlp_if_eq:NNTF}
% \begin{macro}{\tlp_if_eq:cNTF}
% \begin{macro}{\tlp_if_eq:NNT}
%    \begin{macrocode}
\def_new:Npn \tlp_if_eq:NNTF #1#2{
 \if_meaning:NN #1 #2
    \exp_after:NN \use_arg_i:nn
 \else:
    \exp_after:NN\use_arg_ii:nn
 \fi:
}
\def_new:Npn \tlp_if_eq:cNTF {\exp_args:Nc\tlp_if_eq:NNTF}
\def_new:Npn \tlp_if_eq:NNT #1#2{
 \if_meaning:NN #1 #2
    \exp_after:NN \use_arg_i:n
 \else:
    \exp_after:NN \use_none:n
 \fi:
}
%    \end{macrocode}
% \end{macro}
% \end{macro}
% \end{macro}
%
% \begin{macro}{\c_empty_tlp}
% \begin{macro}{\c_relax_tlp}
%    Two constants which are often used.
%    \begin{macrocode}
\tlp_new:Nn \c_empty_tlp {}
\tlp_new:Nn \c_relax_tlp {\scan_stop:}
%    \end{macrocode}
% \end{macro}
% \end{macro}

% \begin{macro}{\g_tmpa_tlp}
% \begin{macro}{\g_tmpb_tlp}
%    Global temporary token list pointers.
%    They are supposed to be set and used immediately,
%    with no delay between the definition and the use because you
%    can't count on other macros not to redefine them from under you.
%
%    \begin{macrocode}
\tlp_new:Nn \g_tmpa_tlp{}
\tlp_new:Nn \g_tmpb_tlp{}
%    \end{macrocode}
% \end{macro}
% \end{macro}
%

% \begin{macro}{\l_testa_tlp}
% \begin{macro}{\l_testb_tlp}
% \begin{macro}{\g_testa_tlp}
% \begin{macro}{\g_testb_tlp}
%    Global and local temporaries.  These are the ones for test
%    routines.  This means that one can safely use other temporaries
%    when calling test routines.
%    \begin{macrocode}
\tlp_new:Nn \l_testa_tlp {}
\tlp_new:Nn \l_testb_tlp {}
\tlp_new:Nn \g_testa_tlp {}
\tlp_new:Nn \g_testb_tlp {}
%    \end{macrocode}
% \end{macro}
% \end{macro}
% \end{macro}
% \end{macro}
%

% \begin{macro}{\l_tmpa_tlp}
% \begin{macro}{\l_tmpb_tlp}
%    These are local temporary token list pointers.
%    \begin{macrocode}
\tlp_new:Nn \l_tmpa_tlp{}
\tlp_new:Nn \l_tmpb_tlp{}
%    \end{macrocode}
% \end{macro}
% \end{macro}
%
%
% \begin{macro}{\tlp_to_str:N}
% \begin{macro}{\tlp_to_str:c}
% \begin{macro}{\tlp_to_str:aux}
%    These functions return the replacement text of a token list as a
%    string list with all characters catcoded to `other'.
%    \begin{macrocode}
\def_new:Npn \tlp_to_str:N {\exp_after:NN\tlp_to_str:aux
    \token_to_meaning:N}
\def_new:Npn \tlp_to_str:aux #1>{}
\def_new:Npn\tlp_to_str:c{\exp_args:Nc\tlp_to_str:N}
%    \end{macrocode}
% \end{macro}
% \end{macro}
% \end{macro}
%
%
%    Show token usage:
%    \begin{macrocode}
%</initex|package>
%<*showmemory>
\showMemUsage
%</showmemory>
%    \end{macrocode}
%
% \endinput
%
% $Log$
% Revision 1.15  2005/04/06 21:34:07  morten
% Fixed all tlp_set:XX so that they're \long. Added \tlp_set:Nf
%
% Revision 1.14  2005/03/22 23:20:43  morten
% Fixed \long versions according to Benjamin's suggestions. Slightly reorganized.
%
% Revision 1.13  2005/03/16 22:35:41  braams
% Added the tweaks necessary to be able to load with initex
%
% Revision 1.12  2005/03/11 21:43:58  braams
% Fixed the use of RCS information;
% Fixed a documentation typo
%

% \iffalse
%% File: l3doc.dtx Copyright (C) 1990-2009 LaTeX3 project
%%
%% It may be distributed and/or modified under the conditions of the
%% LaTeX Project Public License (LPPL), either version 1.3c of this
%% license or (at your option) any later version.  The latest version
%% of this license is in the file
%%
%%    http://www.latex-project.org/lppl.txt
%%
%% This file is part of the ``expl3 bundle'' (The Work in LPPL)
%% and all files in that bundle must be distributed together.
%%
%% The released version of this bundle is available from CTAN.
%%
%% -----------------------------------------------------------------------
%%
%% The development version of the bundle can be found at
%%
%%    http://www.latex-project.org/cgi-bin/cvsweb.cgi/
%%
%% for those people who are interested.
%%
%%%%%%%%%%%
%% NOTE: %%
%%%%%%%%%%%
%%
%%   Snapshots taken from the repository represent work in progress and may
%%   not work or may contain conflicting material!  We therefore ask
%%   people _not_ to put them into distributions, archives, etc. without
%%   prior consultation with the LaTeX Project Team.
%%
%% -----------------------------------------------------------------------
%<*driver>
\def\nameofplainTeX{plain}
\ifx\fmtname\nameofplainTeX\else
  \expandafter\begingroup
\fi
\input docstrip.tex
\askforoverwritefalse
\preamble


EXPERIMENTAL CODE

Do not distribute this file without also distributing the
source files specified above.

Do not distribute a modified version of this file.


\endpreamble
% stop docstrip adding \endinput
\postamble
\endpostamble
\generate{\file{l3doc.cls}{\from{l3doc.dtx}{class}}}
\generate{\file{l3doc.ist}{\from{l3doc.dtx}{docist}}}
\ifx\fmtname\nameofplainTeX
  \expandafter\endbatchfile
\else
  \expandafter\endgroup
\fi
%</driver>
%
%<*driver|class>
\RequirePackage{l3names}
%</driver|class>
%
% Need to protect the file metadata for any modules that load l3doc.
% This is restored after "\ProvideExplClass" below.
%    \begin{macrocode}
%<class>\let        \filenameOld        \filename
%<class>\let     \filenameextOld     \filenameext
%<class>\let        \filedateOld        \filedate
%<class>\let     \fileversionOld     \fileversion
%<class>\let \filedescriptionOld \filedescription
%    \end{macrocode}
%
%<*driver|class>
\GetIdInfo$Id$
          {L3 Experimental documentation class}
%</driver|class>
%
%<*driver>
\ProvidesFile{\filename.\filenameext}
  [\filedate\space v\fileversion\space\filedescription]
\documentclass{l3doc}
\begin{document}
\DocInput{l3doc.dtx}
\end{document}
%</driver>
%
% This isn't included in the typeset documentation because it's a bit ugly:
%<*class>
\ProvidesExplClass
  {\filename}{\filedate}{\fileversion}{\filedescription}
\RequirePackage{expl3}
\let        \filename        \filenameOld
\let     \filenameext     \filenameextOld
\let        \filedate        \filedateOld
\let     \fileversion     \fileversionOld
\let \filedescription \filedescriptionOld
%</class>
% \fi
%
% \title{The \pkg{l3doc} class\thanks{This file
%         has version number \fileversion, last
%         revised \filedate.}}
% \author{\Team}
% \date{\filedate}
% \maketitle
% \tableofcontents
%
% \begin{documentation}
%
% \section{Introduction}
%
% This is an ad-hoc class for documenting the \pkg{expl3} bundle,
% a collection of modules or packages that make up \LaTeX3's programming
% environment. Eventually it will replace the "ltxdoc" class for \LaTeX3,
% but not before the good ideas in \pkg{hypdoc}, \cls{xdoc2}, \pkg{docmfp}, and
% \cls{gmdoc} are incorporated.
%
% \textbf{It is even less stable than the main \pkg{expl3} packages. Use at own risk!}
%
% It is written as a `self-contained' docstrip file: executing
% "latex l3doc.dtx"
% will generate the "l3doc.cls" file and typeset this
% documentation; execute "tex l3doc.dtx" to only generate the ".cls" file.
%
% \section{Features of other packages}
%
% This class builds on the \pkg{ltxdoc} class and the \pkg{doc} package, but
% in the time since they were originally written some
% improvements and replacements have appeared that we would like to use as
% inspiration.
%
% These packages or classes are \pkg{hypdoc}, \pkg{docmfp}, \pkg{gmdoc},
% and \pkg{xdoc}. I have summarised them below in order to work out what
% sort of features we should aim at a minimum for \pkg{l3doc}.
%
% \subsection{The \pkg{hypdoc} package}
%
% This package provides hyperlink support for the \pkg{doc} package. I have
% included it in this list to remind me that cross-referencing between
% documentation and implementation of methods is not very good. (E.g., it
% would be nice to be able to automatically hyperlink the documentation for
% a function from its implementation and vice-versa.)
%
% \subsection{The \pkg{docmfp} package}
%
% \begin{itemize}
% \item Provides "\DescribeRoutine" and the "routine" environment (etc.)
%       for  MetaFont and MetaPost code.
% \item Provides "\DescribeVariable" and the "variable" environment (etc.)
%       for more general code.
% \item Provides "\Describe" and the "Code" environment (etc.) as a
%       generalisation of the above two instantiations.
% \item Small tweaks to the DocStrip system to aid non-\LaTeX\ use.
% \end{itemize}
%
% \subsection{The \pkg{xdoc2} package}
%
% \begin{itemize}
% \item Two-sided printing.
% \item "\NewMacroEnvironment", "\NewDescribeEnvironment"; similar idea
%       to \pkg{docmfp} but more comprehensive.
% \item Tons of small improvements.
% \end{itemize}
%
% \subsection{The \pkg{gmdoc} package}
%
% Radical re-implementation of \pkg{doc} as a package or class.
% \begin{itemize}
% \item Requires no "\begin{macrocode}" blocks!
% \item Automatically inserts "\begin{macro}" blocks!
% \item And a whole bunch of other little things.
% \end{itemize}
%
% \section{Problems \& Todo}
%
% Problems at the moment: (1)~not flexible in the types of things that can be documented; (2)~very nonstandard markup (using `"|"' as a delimiter and the odd `"/ (...)"' tags; (3)~no obvious link to the "\begin{macro}" function that's used analogously in the implementation.
%
% The environments "function" and "macro" should probably
% be renamed to something like \env{funcdoc} and \env{funcimpl},
% respectively.
%
% Furthermore, we need another `layer' of documentation commands to
% account for `user-macro' as opposed to `code-functions'; the \pkg{expl3}
% functions should be documented differently, probably, to the \pkg{xparse}
% user macros (at least in terms of indexing).
%
% In no particular order, a list of things to do:
% \begin{itemize}
% \item Rename \env{function}/\env{macro} environments to better describe
%       their use.
% \item Generalise \env{function}/\env{macro} for documenting `other things',
%       such as environment names, package options, even keyval options.
% \item Use \pkg{xparse}.
% \item New function like "\part" but for files (remove awkward `File' as "\partname").
% \item Something better to replace "\StopEventually"; I'm thinking two
%       environments \env{documentation} and \env{implementation} that can
%       conditionally typeset/ignore their material.
% \item Hyperlink documentation and implementation of macros (also see
%       the \textsc{dtx} file of \pkg{svn-multi} v2).
% \end{itemize}
%
% \section{Bugs}
%
% \begin{itemize}
% \item Spaces are ignored entirely within \env{function} and \env{macro}
%       arguments. This is just waiting for a convenient space-trimming
%       macro in \pkg{expl3}.
% \end{itemize}
%
% \section{Documentation}
%
% \subsection{Partitioning documentation and implementation}
%
% \pkg{doc} uses the "\OnlyDocumentation"/"\AlsoImplementation" macros
% to guide the use of "\StopEventually{}", which is intended to be placed
% to partition the documentation and implementation within a single DTX file.
% 
% This isn't very flexible, since it assumes that we \emph{always} want
% to print the documentation. For the \pkg{expl3} sources, I wanted to be
% be able to input DTX files in two modes: only displaying the documentation,
% and only displaying the implementation. For example:
%
% \begin{verbatim}
% \DisableImplementation
% \DocInput{l3basics,l3prg,...}
% \EnableImplementation
% \DisableDocumentation
% \DocInputAgain
% \end{verbatim}
%
% The idea being that the entire \pkg{expl3} bundle can be documented,
% with the implementation included at the back. Now, this isn't perfect,
% but it's a start.
%
% Use "\begin{documentation}...\end{documentation}" around the documentation,
% and "\begin{implementation}...\end{implementation}" around the implementation.
% The "\EnableDocumentation"/"\EnableImplementation" will cause them to be typeset
% when the DTX file is "\DocInput"; use "\DisableDocumentation"/"\DisableImplementation"
% to omit the contents of those environments.
%
% Note that \cmd\DocInput\ now takes comma-separated arguments, and \cmd\DocInputAgain\
% can be used to re-input all DTX files previously input in this way.
%
% \subsection{Describing functions in the documentation}
%
% \DescribeEnv{function}
% \DescribeEnv{syntax}
% Two heavily-used environments are defined to describe the syntax
% of \textsf{expl3} functions and variables.
% \begin{verbatim}
% \begin{function}{ list_of | functions }
%   \begin{syntax}
%     "\foo_bar:" \Arg{meta} <test1>
%   \end{syntax}
% <description>
% \end{function}
% \end{verbatim}
%
% \begin{function}{ list_of | functions }
%   \begin{syntax}
%     "\foo_bar:" \Arg{meta} <test1>
%   \end{syntax}
% <description>
% \end{function}
%
% Note that the list of functions use "|" as a separator.
%
% \bigskip
% \textbf{MAJOR PROPOSED CHANGE}\qquad
% I think we should probably discourage this sort of free-form markup in
% \LaTeX3 and change this to a plain ol' comma-separated list instead.
%
% Alternatively, we could provide programmatic means of looping through
% lists with arbitary separators and promote this kind of thing.
%
% \subsection{Describing functions in the implementation}
%
% \DescribeEnv{macro}
% The well-used environment from \LaTeXe\ for marking up the implementation
% of macros/functions remains the \env{macro} environment.
% One major change in \pkg{l3doc}: it now accepts comma-separated lists
% of functions, to avoid a very large number of consecutive "\end{macro}"
% statements.
% \begin{verbatim}
% % \begin{macro}{\foo:N,\foo:c}
% %   \begin{macrocode}
% ... code for \foo:N and \foo:c ...
% %   \end{macrocode}
% % \end{macro}
% \end{verbatim}
%
% \DescribeEnv{arguments}
% Within a \env{macro} environment, you may use the \env{arguments} environment
% to describe the arguments taken by the function(s). It behaves exactly
% like a list environment.
% \begin{verbatim}
% % \begin{macro}{\foo:nn,\foo:VV}
% % \begin{arguments}
% %   \item Name of froozle to be frazzled
% %   \item Name of muble to be jubled
% % \end{arguments}
% %   \begin{macrocode}
% ... code for \foo:nn and \foo:VV ...
% %   \end{macrocode}
% % \end{macro}
% \end{verbatim}
%
% \bigskip
% \textbf{OPTIONS FOR THE FUTURE}\qquad Any improvements to the markup
% for the \env{function} environment would be good to mirror in \env{macro}.
%
% Perhaps this would be a better syntax for describing arguments?
% \begin{verbatim}
% \begin{macro}{\foo:nn,foo:VV}
% \dArg{Name of froozle to be frazzled}
% \dArg{Name of mumble to be jumbled}
% ...
% \end{verbatim}
% I.e., get rid of the environment and do things like in, say, \pkg{fontspec}.
%
% \subsection{Keeping things consistent}
%
% Whenever a function is either documented or defined with \env{function}
% and \env{macro} respectively, its name is stored in a sequence for later
% processing.
%
% At the end of the document (i.e., after the \textsc{dtx} file has finished
% processing), the list of names is analysed to check whether all defined
% functions have been documented and vice versa. The results are printed
% in the console output.
%
% If you need to do more serious work with these lists of names, take a
% look at the implementation for the data structures and methods used to
% store and access them directly.
%
% \subsection{Documenting templates}
%
% The following macros are provided for documenting templates; might
% end up being something completely different but who knows.
% \begin{quote}\parskip=0pt\obeylines
% "\begin{TemplateInterfaceDescription}" \Arg{template type name}
% "  \TemplateArgument{none}{---}"
% \textsc{or one or more of these:}
% "  \TemplateArgument" \Arg{arg no} \Arg{meaning}
% \textsc{and}
% "\TemplateSemantics"
% "  " \meta{text describing the template type semantics}
% "\end{TemplateInterfaceDescription}"
% \end{quote}
%
% \begin{quote}\parskip=0pt\obeylines
% "\begin{TemplateDescription}" \Arg{template type name} \Arg{name}
% \textsc{one or more of these:}
% "  \TemplateKey" \marg{key name} \marg{type of key}
% "    "\marg{textual description of meaning}
% "    "\marg{default value if any}
% \textsc{and}
% "\TemplateSemantics"
% "  " \meta{text describing special additional semantics of the template}
% "\end{TemplateDescription}"
% \end{quote}
%
% \begin{quote}\parskip=0pt\obeylines
% "\begin{InstanceDescription}" \oarg{text to specify key column width (optional)}
% \hfill\marg{template type name}\marg{instance name}\marg{template name}
% \textsc{one or more of these:}
% "  \InstanceKey" \marg{key name} \marg{value}
% \textsc{and}
% "\InstanceSemantics"
% "  " \meta{text describing the result of this instance}
% "\end{InstanceDescription}"
% \end{quote}
%
% \end{documentation}
%
% \begin{implementation}
%
% \section{\pkg{l3doc} implementation}
%
%    \begin{macrocode}
%<*class>
%    \end{macrocode}
%
% The Guilty Parties.
%    \begin{macrocode}
\cs_new_nopar:Npn\Team{%
  The~\LaTeX3~Project\thanks{%
  Frank~Mittelbach,~Denys~Duchier,~Chris~Rowley,~
  Rainer~Sch\"opf,~Johannes~Braams,~Michael~Downes,~
  David~Carlisle,~Alan~Jeffrey,~Morten~H\o{}gholm,~Thomas~Lotze,~
  Javier~Bezos,~Will~Robertson,~Joseph~Wright}}
%    \end{macrocode}
%
% \subsection{Options}
%
%    \begin{macrocode}
\DeclareOption{a5paper}{\@latexerr{Option not supported}{}}
%    \end{macrocode}
%
%    \begin{macrocode}
\bool_new:N \g_doc_full_bool
\bool_set_false:N \g_doc_full_bool
\DeclareOption{full}{ \bool_set_true:N \g_doc_full_bool }
\DeclareOption{onlydoc}{ \bool_set_false:N \g_doc_full_bool }
%    \end{macrocode}
%
%    \begin{macrocode}
\bool_new:N \g_doc_checkfunc_bool
\bool_set_true:N \g_doc_checkfunc_bool
\DeclareOption{check}{ \bool_set_true:N \g_doc_checkfunc_bool }
\DeclareOption{nocheck}{ \bool_set_false:N \g_doc_checkfunc_bool }
%    \end{macrocode}
%
%    \begin{macrocode}
\DeclareOption*{\PassOptionsToClass{\CurrentOption}{article}}
\ExecuteOptions{full}
\ProcessOptions
%    \end{macrocode}
%
%
% \subsection{Class and package loading}
%
%    \begin{macrocode}
\LoadClass[a4paper]{article}
\RequirePackage{doc}
\RequirePackage{array,alphalph,color,fixltx2e,enumitem,trace,underscore}
%    \end{macrocode}
% 
%    \begin{macrocode}
\RequirePackage[T1]{fontenc}
\RequirePackage{lmodern,textcomp}
\RequirePackage{hypdoc}
%    \end{macrocode}
% Just want the "comment" environment from \pkg{verbatim}:
%    \begin{macrocode}
\let\doc@verbatim\verbatim
\let\enddoc@verbatim\endverbatim
\let\doc@@verbatim\@verbatim
\usepackage{verbatim}
\AtBeginDocument{%
  \let\verbatim\doc@verbatim
  \let\endverbatim\enddoc@verbatim
  \let\@verbatim\doc@@verbatim
}
%    \end{macrocode}
%
% \subsection{Configuration}
%
% \begin{macro}{\MakePrivateLetters}
%    \begin{macrocode}
\cs_set_nopar:Npn \MakePrivateLetters {
  \char_make_letter:N \@
  \char_make_letter:N \_
  \char_make_letter:N \:
}
%    \end{macrocode}
% \end{macro}
%
%    \begin{macrocode}
\setcounter{StandardModuleDepth}{1}
\@addtoreset{CodelineNo}{part}
\cs_set_nopar:Npn \theCodelineNo {
  \textcolor[gray]{0.5}{ \sffamily\tiny\arabic{CodelineNo} }
}
%    \end{macrocode}
%
%
%
% \subsection{Design}
%
% Increase the text width slightly so that width the standard fonts
% 72 columns of code may appear in a |macrocode| environment.
% Increase the marginpar width slightly, for long command names.
% And increase the left margin by a similar amount.
%    \begin{macrocode}
\setlength   \textwidth      { 385pt }
\addtolength \marginparwidth {  30pt }
\addtolength \oddsidemargin  {  20pt }
\addtolength \evensidemargin {  20pt }
%    \end{macrocode}
% (These were introduced when "article" was the documentclass, but
%  I've left them here for now to remind me to do something about them
%  later; we still have the problem of \emph{very long} command names.)
%
% Customise lists:
%    \begin{macrocode}
\cs_set_eq:NN \@@oldlist\list
\cs_set_nopar:Npn \list#1#2{\@@oldlist{#1}{#2\listparindent\z@}}
\setlength \parindent  { 0pt }
\setlength \itemindent { 0pt }
\setlength \parskip    { \medskipamount }
%    \end{macrocode}
%
% \subsection{Text markup}
%
%    Make "|" and |"| be `short verb' characters, but not in
%    the document preamble, where an active character may interfere
%    with packages that are loaded.
%    \begin{macrocode}
\AtBeginDocument {
  \MakeShortVerb \"
  \MakeShortVerb \|
}
%    \end{macrocode}
%
%    \begin{macrocode}
\providecommand*\eTeX{
  \if b\expandafter\@car\f@series\@nil\boldmath\fi
  $\m@th\varepsilon$-\TeX
}
%    \end{macrocode}
%
% \begin{macro}{\cmd,\cs}
% |\cmd{\foo}| Prints |\foo| verbatim. It may be used inside moving
% arguments. |\cs{foo}| also prints |\foo|, for those who prefer that
% syntax.
%    \begin{macrocode}
\cs_set_nopar:Npn \cmd #1 { \cs{\expandafter\cmd@to@cs\string#1} }
\cs_set_nopar:Npn \cmd@to@cs #1#2 { \char\number`#2\relax }
\DeclareRobustCommand \cs [1] { \texttt { \char`\\ #1 } }
%    \end{macrocode}
% \end{macro}
%
% \begin{macro}{\marg,\oarg,\parg}
%    |\marg{text}| prints \marg{text}, `mandatory argument'.\\
%    |\oarg{text}| prints \oarg{text}, `optional argument'.\\
%    |\parg{te,xt}| prints \parg{te,xt}, `picture mode argument'.
%    \begin{macrocode}
\providecommand\marg[1]{ \texttt{\char`\{} \meta{#1} \texttt{\char`\}} }
\providecommand\oarg[1]{ \texttt[ \meta{#1} \texttt] }
\providecommand\parg[1]{ \texttt( \meta{#1} \texttt) }
%    \end{macrocode}
% \end{macro}
%
% \begin{macro}{\m,\file,\env,\pkg,\cls}
% This list may change\dots this is just my preference for markup.
%    \begin{macrocode}
\cs_set_eq:NN \m    \meta
\cs_set_eq:NN \file \nolinkurl
\DeclareRobustCommand \env {\texttt}
\DeclareRobustCommand \pkg {\textsf}
\DeclareRobustCommand \cls {\textsf}
%    \end{macrocode}
% \end{macro}
%
% \begin{environment}{texnote}
%    \begin{macrocode}
\newenvironment{texnote}{
  \endgraf
  \vspace{3mm}
  \small\textbf{\TeX~hackers~note:}
}{
  \vspace{3mm}
}
%    \end{macrocode}
% \end{environment}
%
% \begin{macro}{\tn}
%    \begin{macrocode}
\newcommand\tn[1]{
  \texttt{\bslash #1}
  \index{TeX~and~LaTeX2e~commands\actualchar
          \string\TeX{}~and~\string\LaTeXe{}~commands:\levelchar
          #1\actualchar{\string\ttfamily\string\bslash{}#1}}}
%    \end{macrocode}
% \end{macro}
%
% \begin{environment}{documentation}
% \begin{environment}{implementation}
% \begin{macro}{\EnableDocumentation,\EnableImplementation}
% \begin{macro}{\DisableDocumentation,\DisableImplementation}
%    \begin{macrocode}
\newenvironment{documentation}{}{}
\newenvironment{implementation}{}{}
\newcommand\EnableDocumentation{%
  \renewenvironment{documentation}{}{}%
}
\newcommand\EnableImplementation{%
  \renewenvironment{implementation}{}{}%
}
\newcommand\DisableDocumentation{%
  \cs_set_eq:NN \documentation \comment
  \cs_set_eq:NN \enddocumentation \endcomment
}
\newcommand\DisableImplementation{%
  \cs_set_eq:NN \implementation \comment
  \cs_set_eq:NN \endimplementation \endcomment
}
%    \end{macrocode}
% \end{macro}
% \end{macro}
% \end{environment}
% \end{environment}
%
% \begin{environment}{arguments}
% This environment is designed to be used within a \env{macro} environment
% to describe the arguments of the macro/function.
%    \begin{macrocode}
\newenvironment{arguments}{
  \enumerate[
    nolistsep,
    label=\texttt{\#\arabic*}~:,
    labelsep=*,
  ]
}{
  \endenumerate
}
%    \end{macrocode}
% \end{environment}
%
% \begin{environment}{function}
% \begin{environment}{variable}
% Environment for documenting function(s).
% Stick the function names in a box. Use a "|" as delimiter and
% allow |<...>| to be used as markup for |\meta{...}|.
% Ignore spaces and sanitize with catcodes before reading argument.
%    \begin{macrocode}
\group_begin:
\char_make_active:N \<
\cs_gnew_nopar:Npn \function {
  \char_make_active:N \<
  \cs_set_eq:NN < \doc_open_meta:n
  \group_begin:
    \MakePrivateLetters
    \char_make_other:N \|
    \char_make_other:N \\
    \char_make_space:N \~
    \char_make_ignore:N \ % space
    \char_make_ignore:N \^^M
    \char_make_ignore:N \^^I
    \function_aux:n
}
\group_end:
%    \end{macrocode}
% And the "variable" function is exactly the same for now:
%    \begin{macrocode}
\cs_set_eq:NN \variable \function
%    \end{macrocode}
%
% \begin{macro}{\function_aux:n}
% \begin{arguments}
% \item Vertical bar--separated list of functions with optional metadata;
%       input has already been sanitised by catcode changes before reading
%       the argument.
% \end{arguments}
%    \begin{macrocode}
\cs_set_nopar:Npn \function_aux:n #1 {
    \cs_set_nopar:Npn \nextnewline{\cs_gset_nopar:Npn\nextnewline{\\}}
    \tl_gset_eq:NN \g_doc_macro_tl \c_empty_tl
    \bigskip\endgraf\noindent\ttfamily
    \tabular[b]{ | l @{} c | }
      \hline
      \doc_showmacro:w #1 | \q_stop \\
      \hline
    \endtabular
  \group_end:
}
%    \end{macrocode}
% \end{macro}
% \end{environment}
% \end{environment}
%
% \begin{macro}{\doc_showmacro:w}
% This function reads in a "|"-separated list, passing each item to
% the auxiliary function "\doc_showmacro_aux:w".
%    \begin{macrocode}
\cs_set_nopar:Npn \doc_showmacro:w #1 | {
  \tl_if_blank:nTF {#1} {
    \use_none:n
  }{
    \doc_showmacro_aux:w #1 / \q_stop
    \peek_meaning:NTF \q_stop { \use_none:n } { \doc_showmacro:w }
  }
}
%    \end{macrocode}
% \end{macro}
%
%    \begin{macrocode}
\bool_new:N \l_doc_meta_TF_bool
\bool_new:N \l_doc_meta_EXP_bool
%    \end{macrocode}
%
% \begin{macro}{\doc_showmacro_aux:w}
% This macro is passed one of:
% \begin{quote}
%   "\abc:xyz / (EXP) / \q_stop" \\
%   "\abc:xyz / \q_stop" \\
% \end{quote}
% We also have some code here to print out every documented macro at the end
% of the document.
% \begin{arguments}
% \item Function/macro/variable name \item Metadata tags (if any)
% \end{arguments}
%    \begin{macrocode}
\cs_new_nopar:Npn \doc_showmacro_aux:w #1 / #2 \q_stop {

  \tl_if_in:nnTF {#2} { (TF)  } {
    \bool_gset_true:N \l_doc_meta_TF_bool
  }{
    \bool_gset_false:N \l_doc_meta_TF_bool
  }
  \tl_if_in:nnTF {#2} { (EXP) } {
    \bool_gset_true:N \l_doc_meta_EXP_bool
  }{
    \bool_gset_false:N \l_doc_meta_EXP_bool
  }

  \bool_if:NTF \l_doc_meta_TF_bool {
    \doc_special_main_index:o { #1 TF }
    \seq_gput_right:Nx \g_doc_functions_seq { \tl_to_str:n { #1 TF } }
    \seq_gput_right:Nx \g_doc_functions_seq { \tl_to_str:n { #1 T  } }
    \seq_gput_right:Nx \g_doc_functions_seq { \tl_to_str:n { #1  F } }
  }{
    \doc_special_main_index:o { #1 }
    \seq_gput_right:Nx \g_doc_functions_seq { \tl_to_str:n { #1    } }
  }

  \doc_showmacro_aux_ii:w #1::\q_stop
}
%    \end{macrocode}
% \end{macro}
%
% \begin{macro}{\doc_showmacro_aux_ii:w}
% This macro is passed one of:
% \begin{quote}
%   "\showexplmacro "^^A
%    \makebox[\widthof{\texttt{123456}}]{\meta{name}}^^A
%                        " ::\q_stop" \\
%   "\showexplmacro \foo   ::\q_stop" \\
%   "\showexplmacro \foo:  ::\q_stop" \\
%   "\showexplmacro \foo:Z ::\q_stop" \\
% \end{quote}
% Notice that for "\foo", "#2" and "#3" are empty,\\
% for "\foo:", "#2" is empty,\\
% and for both "\foo:" and "\foo:x", "#3" is `":"' .
% \begin{arguments}
% \item Function name \item Possible arg.\ spec. \item Possible colon
% \end{arguments}
%    \begin{macrocode}
\cs_set_nopar:Npn \doc_showmacro_aux_ii:w #1:#2:#3 \q_stop {
  \nextnewline

  \tl_if_eq:xxTF {#1} {\g_doc_macro_tl} {
    \doc_typeset_aux:n
  }{
    \tl_gset:Nn \g_doc_macro_tl {#1}
    \use:n
  }
  { \g_doc_macro_tl }
  #3
  #2
  \bool_if:NT \l_doc_meta_TF_bool { \doc_typeset_TF: }
  &
  \bool_if:NT \l_doc_meta_EXP_bool {
    \hspace{\tabcolsep}
    $\star$
  }
}
%    \end{macrocode}
% \end{macro}
%
% \begin{environment}{syntax}
% Syntax block placed next to the list of functions to illustrate their use.
%    \begin{macrocode}
\newenvironment{syntax}{%
  \minipage[b]{.7\textwidth}%
    \cs_set_nopar:Npn \meta@font@select{\rmfamily\itshape} % (Will: I HATE italic cmtt!)
    \small\ttfamily\raggedright
    \obeyspaces\obeylines
}{%
  \endminipage
  \hfil\break
  \global\@ignoretrue
}
%    \end{macrocode}
% \end{environment}
%
% Perhaps these belong in \file{l3token}?
%    \begin{macrocode}
\tl_map_inline:nn {0123456789} { \cs_gset_eq:cN {char_other_#1} #1 }
%    \end{macrocode}
%
% \begin{macro}{\doc_open_meta:n,\doc_close_meta:n}
% This code turns all numbers within "<...>" markup to be set as subscripts.
% You can use escaped numbers to get the real thing (e.g., "\1" = `1').
%    \begin{macrocode}
\group_begin:
  \tl_map_inline:nn {0123456789} { \char_make_active:N #1 }
  \cs_gnew:Npn \doc_open_meta:n {
    \group_begin:
      \tl_map_function:nN {0123456789} \doc_assign_num:n
      \doc_close_meta:w
  }
  \cs_gnew:Npn \Arg {
    \texttt{ \char`\{ }
    \group_begin:
      \tl_map_function:nN {0123456789} \doc_assign_num:n
      \doc_close_Arg:n
  }
\group_end:
\cs_new_nopar:Npn \doc_close_meta:w #1> { \meta {#1} \group_end: }
\cs_new_nopar:Npn \doc_close_Arg:n #1 {
  \meta {#1}
  \group_end:
  \texttt{ \char`\} }
}
%    \end{macrocode}
% \end{macro}
%
% \begin{macro}{\doc_assign_num:n}
% This function takes a numeral (`0'), defines its escaped self to be equal
% to itself ("\0" $\to$ `0'), makes it active, and turns itself into a subscript
% instead (`0' $\to$ `${}_0$').
%    \begin{macrocode}
\cs_new_nopar:Npn \doc_assign_num:n #1 {
  \cs_set_eq:cc { \string #1 } { char_other_\string #1 }
  \char_make_active:N #1
  \cs_set_nopar:Npn #1 { \unskip \, $ {} \sb { \use:c { char_other_\string #1 } } $ }
}
%    \end{macrocode}
% \end{macro}
%
% \begin{environment}{macro}
% We want to extend the old definition to allow comma-separated lists of
% macros, rather than one at a time. keyval processing is very rudimentary;
% awaiting a more robust solution.
%    \begin{macrocode}
\renewcommand \macro [1][] {
  \bool_set_false:N \l_doc_macro_aux_bool
  \bool_set_false:N \l_doc_macro_TF_bool
  \cs_set_eq:NN \doc_macroname_prefix:n \use:n
  \cs_set_eq:NN \doc_macroname_suffix: \c_empty_tl

  \cs_set_nopar:Npn \KV_key_no_value_elt:n ##1 { \use:c {doc_macro_opt_##1:} }
  \KV_parse_space_removal_sanitize:n {#1}

  \group_begin:
    \MakePrivateLetters
    \char_make_letter:N \\
    \char_make_ignore:N \ % space
    \char_make_ignore:N \^^M
    \char_make_ignore:N \^^I
    \doc_macro_aux:n
}
%    \end{macrocode}
% After changing the catcodes, parse the arguments:
%    \begin{macrocode}
\cs_new_nopar:Npn \doc_macro_aux:n #1 {
  \group_end:
  \clist_map_inline:nn {#1} { \doc_macro_single {##1} }
}
%    \end{macrocode}
%
%    \begin{macrocode}
\bool_new:N \l_doc_macro_aux_bool
\bool_new:N \l_doc_macro_TF_bool
\cs_set_nopar:Npn \doc_macro_opt_aux: { \bool_set_true:N \l_doc_macro_aux_bool }
\cs_set_nopar:Npn \doc_macro_opt_TF:  { \bool_set_true:N \l_doc_macro_TF_bool }
%    \end{macrocode}
% \end{environment}
%
% \begin{environment}{doc_macro_single}
% Let's start to mess around with "doc"'s "macro" environment. See \file{doc.dtx}
% for a full explanation of the original environment. It's
% rather \emph{enthusiastically} commented.
% \begin{arguments}
% \item Macro/function/whatever name; input has already been sanitised.
% \end{arguments}
%    \begin{macrocode}
\cs_set_nopar:Npn \doc_macro_single #1 {

  \topsep\MacroTopsep
  \trivlist
  \cs_set_nopar:Npn \makelabel ##1 { \llap{##1} }
  \if@inlabel
    \cs_set_eq:NN \@tempa \@empty
    \count@ \macro@cnt
    \loop \ifnum\count@>\z@
      \cs_set_nopar:Npx \@tempa{\@tempa\hbox{\strut}}
      \advance\count@\m@ne
    \repeat
    \cs_set_nopar:Npx \makelabel ##1 {
      \llap{\vtop to\baselineskip {\@tempa\hbox{##1}\vss}}
    }
    \advance \macro@cnt \@ne
  \else
    \macro@cnt \@ne
  \fi

  \bool_if:NT \l_doc_macro_aux_bool {
    \cs_set_eq:NN \doc_macroname_prefix:n \doc_typeset_aux:n
  }
  \bool_if:NT \l_doc_macro_TF_bool {
    \cs_set_eq:NN \doc_macroname_suffix: \doc_typeset_TF:
  }

  \cs_set_nopar:Npx \@tempa {
    \exp_not:N \item [ \exp_not:N \doc_print_macroname:n {
      \tl_to_str:n {#1}
    }]
  } \@tempa
  \global\advance \c@CodelineNo \@ne

  \bool_if:NF \l_doc_macro_aux_bool {
    \bool_if:NTF \l_doc_macro_TF_bool {
      \seq_gput_right:Nx \g_doc_macros_seq { \tl_to_str:n { #1 TF } }
      \seq_gput_right:Nx \g_doc_macros_seq { \tl_to_str:n { #1 T  } }
      \seq_gput_right:Nx \g_doc_macros_seq { \tl_to_str:n { #1 F  } }
    }{
      \seq_gput_right:Nx \g_doc_macros_seq { \tl_to_str:n {#1} }
    }
  }
  \bool_if:NTF \l_doc_macro_TF_bool {
    \SpecialMainIndex{#1 TF}\nobreak
    \DoNotIndex{#1 TF}
  }{
    \SpecialMainIndex{#1}\nobreak
    \DoNotIndex{#1}
  }

  \global\advance \c@CodelineNo \m@ne
  \ignorespaces
}
%    \end{macrocode}
%
% \begin{macro}{\doc_print_macroname:n}
%    \begin{macrocode}
\cs_set_nopar:Npn \doc_print_macroname:n #1 {
  \strut \MacroFont
  \doc_macroname_prefix:n {#1} \doc_macroname_suffix: \ % space!
}
%    \end{macrocode}
% \end{macro}
% \end{environment}
%
% \begin{macro}{\doc_typeset_TF:,\doc_typeset_aux:n}
% Used by \cs{doc_macro_single} and \cs{doc_showmacro_aux_ii:w} to typeset
% conditionals and auxiliary functions.
%    \begin{macrocode}
\cs_set_nopar:Npn \doc_typeset_TF: {
  \color[gray]{0.5}
  \underline { \color{black} \itshape TF \kern-0.1em }
}
\cs_set_nopar:Npn \doc_typeset_aux:n #1 {
  {\color[gray]{0.5} #1}
}
%    \end{macrocode}
% \end{macro}
%
% \begin{macro}{\DescribeOption}
% For describing package options. Due to Joseph Wright.
% Name/usage might change soon.
%    \begin{macrocode}
\newcommand*{\DescribeOption}{
 \leavevmode
 \@bsphack
 \begingroup
   \MakePrivateLetters
   \Describe@Option
}
\newcommand*{\Describe@Option}[1]{
 \endgroup
 \marginpar{
   \raggedleft
   \PrintDescribeEnv{#1}
 }
 \SpecialOptionIndex{#1}
 \@esphack
 \ignorespaces
}
\newcommand*{\SpecialOptionIndex}[1]{
 \@bsphack
 \begingroup
   \HD@target
   \let\HDorg@encapchar\encapchar
   \edef\encapchar usage{
     \HDorg@encapchar hdclindex{\the\c@HD@hypercount}{usage}
   }
   \index{
     #1\actualchar{\protect\ttfamily#1} (option)
     \encapchar usage
   }
   \index{
     options:\levelchar#1\actualchar{\protect\ttfamily#1}
     \encapchar usage
   }
 \endgroup
 \@esphack
}
%    \end{macrocode}
% \end{macro}
%
% Here are some definitions for additional markup that will help to
% structure your documentation.
%
% \begin{environment}{danger}
% \begin{environment}{ddanger}
% \begin{syntax}
% |\begin{[d]danger}|\\
% dangerous code\\
% |\end{[d]danger}|
% \end{syntax}
%
% \begin{danger}
%   Provides a danger bend, as known from the \TeX{}book.
% \end{danger}
% The actual character from the font |manfnt|:
%    \begin{macrocode}
\font\manual=manfnt
\cs_set_nopar:Npn \dbend { {\manual\char127} }
%    \end{macrocode}
%
% Defines the single danger bend. Use it whenever there is a feature in your
% package that might be tricky to use.
% FIXME: Has to be fixed when in combination with a macro-definition.
%    \begin{macrocode}
\newenvironment {danger} {
  \begin{trivlist}\item[]\noindent
  \begingroup\hangindent=2pc\hangafter=-2
  \cs_set_nopar:Npn \par{\endgraf\endgroup}
  \hbox to0pt{\hskip-\hangindent\dbend\hfill}\ignorespaces
}{
  \par\end{trivlist}
}
%    \end{macrocode}
%
% \begin{ddanger}
%   Use the double danger bend if there is something which could cause serious
%   problems when used in a wrong way. Better the normal user does not know
%   about such things.
% \end{ddanger}
%    \begin{macrocode}
\newenvironment {ddanger} {
  \begin{trivlist}\item[]\noindent
  \begingroup\hangindent=3.5pc\hangafter=-2
  \cs_set_nopar:Npn \par{\endgraf\endgroup}
  \hbox to0pt{\hskip-\hangindent\dbend\kern2pt\dbend\hfill}\ignorespaces
}{
  \par\end{trivlist}
}
%    \end{macrocode}
% \end{environment}
% \end{environment}
%
% \subsection{Documenting templates}
%
%    \begin{macrocode}
\newenvironment{TemplateInterfaceDescription}[1]
  {\subsection{The template type `#1'}%
   \begingroup
   \@beginparpenalty\@M
   \description
   \def\TemplateArgument##1##2{\item[Arg: ##1]##2\par}%
   \def\TemplateSemantics{\enddescription\endgroup
       \subsubsection*{Semantics:}}%
  }
  {\par\bigskip}
%    \end{macrocode}
%
%    \begin{macrocode}
\newenvironment{TemplateDescription}[2]
  {\subsection{The template `#2' (type #1)}%
   \subsubsection*{Attributes:}%
   \begingroup
   \@beginparpenalty\@M
   \description
   \def\TemplateKey##1##2##3##4{\item[##1 (##2)]##3%
     \ifx\TemplateKey##4\TemplateKey\else
%         \hskip0ptplus3em\penalty-500\hskip 0pt plus 1filll Default:~##4%
         \hfill\penalty500\hbox{}\hfill Default:~##4%
         \nobreak\hskip-\parfillskip\hskip0pt\relax
     \fi
     \par}%
   \def\TemplateSemantics{\enddescription\endgroup
       \subsubsection*{Semantics \& Comments:}}%
  }
  {\par\bigskip}
%    \end{macrocode}
%
%    \begin{macrocode}
\newenvironment{InstanceDescription}[4][xxxxxxxxxxxxxxx]
  {\subsubsection{The instance `#3' (template #2/#4)}%
   \subsubsection*{Attribute values:}%
   \begingroup
   \@beginparpenalty\@M
   \def\InstanceKey##1##2{\>\textbf{##1}\>##2\\}%
   \def\InstanceSemantics{\endtabbing\endgroup
       \vskip-30pt\vskip0pt
       \subsubsection*{Layout description \& Comments:}}%
   \tabbing
   xxxx\=#1\=\kill
  }
  {\par\bigskip}
%    \end{macrocode}
%
% \subsection{Inheriting doc}
%
% Code here is taken from \pkg{doc}, stripped of comments and translated
% into \pkg{expl3} syntax. New features are added in various places.
%
% \begin{macro}{\StopEventually,\Finale,\AlsoImplementation,\OnlyDescription}
%    \begin{macrocode}
\bool_new:N \g_doc_implementation_bool
\cs_set_nopar:Npn \AlsoImplementation {
  \bool_set_true:N \g_doc_implementation_bool
  \cs_set:Npn \StopEventually ##1 {
    \@bsphack
    \cs_gset_nopar:Npn \Finale { ##1 \check@checksum }
    \init@checksum
    \@esphack
  }
}
\AlsoImplementation
\cs_set_nopar:Npn \OnlyDescription {
  \@bsphack
  \bool_set_false:N \g_doc_implementation_bool
  \cs_set:Npn \StopEventually ##1 { ##1 \endinput }
  \@esphack
}
\cs_set_eq:NN \Finale \relax
%    \end{macrocode}
% \end{macro}
%
%    \begin{macrocode}
\cs_set_nopar:Npn \partname{File}
%    \end{macrocode}
%
% \begin{macro}{\DocInput}
% From \pkg{doc}. Now accepts comma-list input (who has commas in filenames?).
%    \begin{macrocode}
\clist_new:N \g_docinput_clist
\cs_set:Npn \DocInput #1 {
  \clist_map_inline:nn {#1} {
    \clist_put_right:Nn \g_docinput_clist {##1}
    \MakePercentIgnore
    \input{##1}
    \MakePercentComment
  }
}
%    \end{macrocode}
% \end{macro}
%
% \begin{macro}{\DocInputAgain}
% Uses "\g_docinput_clist" to re-input whatever's already been "\DocInput"-ed until now.
% May be used multiple times.
%    \begin{macrocode}
\cs_set:Npn \DocInputAgain {
  \clist_map_inline:Nn \g_docinput_clist {
    \MakePercentIgnore
    \input{##1}
    \MakePercentComment
  }
}
%    \end{macrocode}
% \end{macro}
%
% \begin{macro}{\DocInclude}
% More or less exactly the same as |\include|, but uses |\DocInput|
% on a |dtx| file, not |\input| on a |tex| file.
%    \begin{macrocode}
\cs_set_nopar:Npn \partname{File}
%    \end{macrocode}
%
%    \begin{macrocode}
\newcommand*{\DocInclude}[1]{%
  \relax\clearpage
  \docincludeaux
  \IfFileExists{#1.fdd}{
    \cs_set_nopar:Npn \currentfile{#1.fdd}
  }{
    \cs_set_nopar:Npn \currentfile{#1.dtx}
  }
  \ifnum\@auxout=\@partaux
    \@latexerr{\string\include\space cannot~be~nested}\@eha
  \else
    \@docinclude #1
  \fi
}
%    \end{macrocode}
%
%    \begin{macrocode}
\cs_set_nopar:Npn \@docinclude #1 {
  \clearpage
  \immediate\write\@mainaux{\string\@input{#1.aux}}
  \@tempswatrue
  \if@partsw
    \@tempswafalse
    \cs_set_nopar:Npx \@tempb{#1}
    \@for\@tempa:=\@partlist\do{
      \ifx\@tempa\@tempb\@tempswatrue\fi
    }
  \fi
  \if@tempswa
    \cs_set_eq:NN \@auxout\@partaux
    \immediate\openout\@partaux #1.aux
    \immediate\write\@partaux{\relax}
    \cs_set_eq:NN \@ltxdoc@PrintIndex\PrintIndex
    \cs_set_eq:NN \PrintIndex\relax
    \cs_set_eq:NN \@ltxdoc@PrintChanges\PrintChanges
    \cs_set_eq:NN \PrintChanges\relax
    \cs_set_eq:NN \@ltxdoc@theglossary\theglossary
    \cs_set_eq:NN \@ltxdoc@endtheglossary\endtheglossary
    \part{\currentfile}
    {
      \cs_set_eq:NN \ttfamily\relax
      \cs_gset_nopar:Npx \filekey{\filekey, \thepart={\ttfamily\currentfile}}
    }
    \DocInput{\currentfile}
    \cs_set_eq:NN \PrintIndex\@ltxdoc@PrintIndex
    \cs_set_eq:NN \PrintChanges\@ltxdoc@PrintChanges
    \cs_set_eq:NN \theglossary\@ltxdoc@theglossary
    \cs_set_eq:NN \endtheglossary\@ltxdoc@endtheglossary
    \clearpage
    \@writeckpt{#1}
    \immediate\closeout\@partaux
  \else
    \@nameuse{cp@#1}
  \fi
  \cs_set_eq:NN \@auxout\@mainaux
}
%    \end{macrocode}
%
%    \begin{macrocode}
\cs_gset_nopar:Npn \codeline@wrindex #1 {
  \immediate\write\@indexfile {
    \string\indexentry{#1}
    {\filesep\number\c@CodelineNo}
  }
}
%    \end{macrocode}
% \end{macro}
%
%    \begin{macrocode}
\cs_set_eq:NN \filesep \@empty
%    \end{macrocode}
%
% \begin{macro}{\docincludeaux}
%    \begin{macrocode}
\cs_set_nopar:Npn \docincludeaux {
  \cs_set_nopar:Npn \thepart {\alphalph{part}}
  \cs_set_nopar:Npn \filesep {\thepart-}
  \cs_set_eq:NN \filekey\@gobble
  \g@addto@macro\index@prologue{
    \cs_gset_nopar:Npn\@oddfoot{
      \parbox{\textwidth}{
        \strut\footnotesize
        \raggedright{\bfseries File~Key:}~\filekey
      }
    }
    \cs_set_eq:NN \@evenfoot\@oddfoot
  }
  \cs_gset_eq:NN \docincludeaux\relax
  \cs_gset_nopar:Npn\@oddfoot{
    \expandafter\ifx\csname ver@\currentfile\endcsname\relax
      File~\thepart :~{\ttfamily\currentfile}~
    \else
      \GetFileInfo{\currentfile}
      File~\thepart :~{\ttfamily\filename}~
      Date:~\filedate\ % space
      Version~\fileversion
    \fi
    \hfill\thepage
  }
  \cs_set_eq:NN \@evenfoot \@oddfoot
}
%    \end{macrocode}
% \end{macro}
%
% \subsection{At end document}
%
% Print all defined and documented macros/functions.
%
%    \begin{macrocode}
\seq_new:N \g_doc_functions_seq
\seq_new:N \g_doc_macros_seq
%    \end{macrocode}
%
%    \begin{macrocode}
\iow_new:N   \g_write_func_stream
\iow_open:Nn \g_write_func_stream { \jobname.cmds }
%    \end{macrocode}
%
%    \begin{macrocode}
\cs_new_nopar:Npn \doc_show_functions_defined: {
  \bool_if:nT { \g_doc_implementation_bool && \g_doc_checkfunc_bool } {
    \typeout{ ======================================== ^^J }

    \tl_clear:N \l_tmpa_tl
    \seq_map_inline:Nn \g_doc_functions_seq {
      \seq_if_in:NnT \g_doc_macros_seq {##1} {
        \tl_put_right:Nn \l_tmpa_tl { ##1 ^^J }
        \iow_now:Nn \g_write_func_stream { ##1 }
      }
    }
    \iow_close:N \g_write_func_stream
    \doc_functions_typeout:n {
      Functions~both~documented~and~defined:^^J (In~order~of~being~documented)
    }

    \seq_map_inline:Nn \g_doc_functions_seq {
      \seq_if_in:NnF \g_doc_macros_seq {##1} {
        \tl_put_right:Nn \l_tmpa_tl { ##1 ^^J }
      }
    }
    \doc_functions_typeout:n { Functions~documented~but~not~defined: }

    \seq_map_inline:Nn \g_doc_macros_seq {
      \seq_if_in:NnF \g_doc_functions_seq {##1} {
        \tl_put_right:Nn \l_tmpa_tl { ##1 ^^J }
      }
    }
    \doc_functions_typeout:n { Functions~defined~but~not~documented: }

    \typeout{ ======================================== }
  }
}
\AtEndDocument{ \doc_show_functions_defined: }
%    \end{macrocode}
%
%    \begin{macrocode}
\cs_set_nopar:Npn \doc_functions_typeout:n #1 {
  \tl_if_empty:NF \l_tmpa_tl {
    \typeout{
      -------------------------------------- ^^J #1 ^^J
      -------------------------------------- ^^J \l_tmpa_tl
    }
    \tl_clear:N \l_tmpa_tl
  }
}
%    \end{macrocode}
%
% \subsection{Indexing}
%
% Fix index (for now):
%    \begin{macrocode}
\g@addto@macro\theindex{\MakePrivateLetters}
%    \end{macrocode}
%
%    \begin{macrocode}
\setcounter{IndexColumns}{2}
%    \end{macrocode}
%
% Set up the Index to use "\part"
%    \begin{macrocode}
\IndexPrologue{
  \part*{Index}
  \markboth{Index}{Index}
  \addcontentsline{toc}{part}{Index}
  The~italic~numbers~denote~the~pages~where~the~
  corresponding~entry~is~described,~
  numbers~underlined~point~to~the~definition,~
  all~others~indicate~the~places~where~it~is~used.
}
%    \end{macrocode}
%
%
% \begin{macro}{\doc_special_main_index:n,\doc_special_main_index:o,\hdpgindex}
% Heiko's replacement to play nicely with |hypdoc|:
%    \begin{macrocode}
\cs_set_nopar:Npn \doc_special_main_index:n #1 {
  \index{
    \@gobble#1
    \actualchar
    \string\verb\quotechar*\verbatimchar#1\verbatimchar
    \encapchar
    hdpgindex{\thepage}{usage}
  }
}
\cs_set_nopar:Npn \doc_special_main_index:o { \exp_args:No \doc_special_main_index:n }
%    \end{macrocode}
%    \begin{macrocode}
\cs_set_nopar:Npn \hdpgindex #1#2#3 {
  \csname\ifx\\#2\\relax\else#2\fi\endcsname{
    \hyperlink{page.#1}{#3}
  }
}
%    \end{macrocode}
% \end{macro}
%
%    \begin{macrocode}
\g@addto@macro \PrintIndex { \AtEndDocument{ \typeout{^^J
  ========================================^^J
  Generate~the~index~by~executing^^J
  \text_put_four_sp: makeindex~-s~l3doc.ist~-o~\jobname.ind~\jobname.idx^^J
  ========================================^^J
  }}
}
%    \end{macrocode}
%
% \subsection{Change history}
%
% Set the change history to use "\part".
% Allow control names to be hyphenated in here...
%    \begin{macrocode}
\GlossaryPrologue{
  \part*{Change~History}
  {\GlossaryParms\ttfamily\hyphenchar\font=`\-}
  \markboth{Change~History}{Change~History}
  \addcontentsline{toc}{part}{Change~History}
}
%    \end{macrocode}
%
%    \begin{macrocode}
\g@addto@macro \PrintChanges { \AtEndDocument{ \typeout{^^J
  ========================================^^J
  Generate~the~change~list~by~executing^^J
  \text_put_four_sp: makeindex~-s~gglo.ist~~-o~\jobname.gls~\jobname.glo^^J
  ========================================^^J
  }}
}
%    \end{macrocode}
%
%^^A The standard \changes command modified slightly to better cope with
%^^A this multiple file document.
%^^A\def\changes@#1#2#3{%
%^^A  \let\protect\@unexpandable@protect
%^^A  \edef\@tempa{\noexpand\glossary{#2\space\currentfile\space#1\levelchar
%^^A                                 \ifx\saved@macroname\@empty
%^^A                                   \space
%^^A                                   \actualchar
%^^A                                   \generalname
%^^A                                 \else
%^^A                                   \expandafter\@gobble
%^^A                                   \saved@macroname
%^^A                                   \actualchar
%^^A                                   \string\verb\quotechar*%
%^^A                                   \verbatimchar\saved@macroname
%^^A                                   \verbatimchar
%^^A                                 \fi
%^^A                                 :\levelchar #3}}%
%^^A  \@tempa\endgroup\@esphack}
%
% \subsection{Patch file}
%
% Will: Do we still need this?
%
% \begin{macro}{\includeltpatch}
% This command will be used to input the patch file
% if that file exists.
%    \begin{macrocode}
\cs_new_nopar:Npn \includeltpatch {
  \cs_set_nopar:Npn \currentfile {l3patch.ltx}
  \part{l3patch}
  {\cs_set_eq:NN \ttfamily\relax
    \cs_gset_nopar:Npx \filekey{\filekey, \thepart={\ttfamily\currentfile}}}
  Things~we~did~wrong\ldots
  \IndexInput{l3patch.ltx}}
%    \end{macrocode}
% Get the date from ltvers.dtx
%    \begin{macrocode}
\cs_set_nopar:Npn \patchdate{0}
\begingroup
  %%\cs_set_nopar:Npn \ProvidesFile#1\fmtversion#2{\date{#2}\endinput}
  \cs_set_nopar:Npn \ProvidesFile#1[#2#3]{\date{#2}\endinput}
  % \iffalse
%% File: l3vers.dtx Copyright (C) 1990-2005 LaTeX3 project
%%
%% It may be distributed and/or modified under the conditions of the
%% LaTeX Project Public License (LPPL), either version 1.3a of this
%% license or (at your option) any later version.  The latest version
%% of this license is in the file
%%
%%    http://www.latex-project.org/lppl.txt
%%
%% This file is part of the ``expl3 bundle'' (The Work in LPPL)
%% and all files in that bundle must be distributed together.
%%
%% The released version of this bundle is available from CTAN.
%%
%% -----------------------------------------------------------------------
%%
%% The development version of the bundle can be found at
%%
%%    http://www.latex-project.org/cgi-bin/cvsweb.cgi/
%%
%% for those people who are interested.
%%
%%%%%%%%%%%
%% NOTE: %%
%%%%%%%%%%%
%%
%%   Snapshots taken from the repository represent work in progress and may
%%   not work or may contain conflicting material!  We therefore ask
%%   people _not_ to put them into distributions, archives, etc. without
%%   prior consultation with the LaTeX Project Team.
%%
%% -----------------------------------------------------------------------
%% \fi
%
% \iffalse
%%% From File: l3vers.dtx
%
%<*driver>
% \fi
\ProvidesFile{l3vers.dtx}
             [2005/03/09 v0.1 LaTeX Kernel (Version Info)]
% \iffalse
\documentclass{ltxdoc}
\GetFileInfo{l3vers.dtx}
\title{\filename}
\date{\filedate}
 \author{%
  Johannes Braams\and
  David Carlisle\and
  Alan Jeffrey\and
  Leslie Lamport\and
  Frank Mittelbach\and
  Chris Rowley\and
  Rainer Sch\"opf}
\begin{document}
\maketitle
 \DocInput{\filename}
\end{document}
%</driver>
% \fi
%
% \CheckSum{46}
%
% \section{Version Identification}
% First we identify the date and version number of this release of
% \LaTeX, and set |\everyjob| so that it is printed at the start of
% every \LaTeX\ run.
%
% \StopEventually{}
%
%
% \begin{macro}{\fmtname}
% \begin{macro}{\fmtversion}
%    \begin{macrocode}
%<*3kernel>
\def\fmtname{LaTeX3}
\edef\fmtversion{--release--date--goes--here--}
%    \end{macrocode}
% \end{macro}
% \end{macro}
% 
% Check that the format being made is not too old.
% The error message complains about `more than 5 years'
% but in fact the error is not triggered until 65 months.
%
% This code is currently not activated as we don't know if we already
% got to the last official 2e version (due to staff shortage or due to
% a successor (think positive:-)).
%    \begin{macrocode}
\iftrue
\def\reserved@a#1/#2/#3\@nil{%
  \count@\year
  \advance\count@-#1\relax
  \multiply\count@ by 12\relax
  \advance\count@\month
  \advance\count@-#2\relax}
\expandafter\reserved@a\fmtversion\@nil
%    \end{macrocode}
% |\count@| is now the age of this file in months. Take a generous
% definition of `year' so this message is not generated too often.
%    \begin{macrocode}
\ifnum\count@>65
  \typeout{^^J%
!!!!!!!!!!!!!!!!!!!!!!!!!!!!!!!!!!!!!!!!!!!!!!!!!!!!!!!!!!!!!!!!!!^^J%
!  You are attempting to make a LaTeX format from a source file^^J%
!  That is more than five years old.^^J%
!^^J%
!  If you enter <return> to scroll past this message then the format^^J%
!  will be built, but please consider obtaining newer source files^^J%
!  before continuing to build LaTeX.^^J%
!!!!!!!!!!!!!!!!!!!!!!!!!!!!!!!!!!!!!!!!!!!!!!!!!!!!!!!!!!!!!!!!!!^^J%
}
   \errhelp{To avoid this error message, obtain new LaTeX sources.}
   \errmessage{LaTeX source files more than 5 years old!}
\fi
\let\reserved@a\relax
\fi
%    \end{macrocode}
%
% This startup banner may be further modified by the code in
% |ltfinal.dtx| if a patch file is present.
%    \begin{macrocode}
\everyjob{\typeout{\fmtname
                                     \space<\fmtversion>}}
\immediate\write16{\fmtname
                                     \space<\fmtversion>}
%</3kernel>
%    \end{macrocode}
%
% \Finale
%
\endinput
%
% $Log$
% Revision 1.1  2005/03/11 21:26:31  braams
% File copied from 2e for now, to hold the format version identification
%
  \cs_gset_nopar:Npx \@date{\@date}
  \cs_gset_eq:NN \X@date \@date
% Add the patch version if available.
  \cs_set:Npn \Xdef#1#2#3\def#4#5 {
    \cs_gset_nopar:Npx \X@date{#2}
    \cs_gset_nopar:Npx \patchdate{#5}
    \endinput
  }
  \InputIfFileExists{l3patch.ltx}{
    \cs_set_eq:NN \cs_set_nopar:Npn \Xdef
  }{
    \cs_gset_eq:NN \includeltpatch \relax
  }
\endgroup
\ifx\@date\X@date
   \cs_set_nopar:Npn \Xpatch{0}
   \ifx\patchdate\Xpatch\else
     \cs_set_nopar:Npx \@date {\@date\space Patch level \patchdate}
   \fi
\else
   \@warning{l3patch.ltx does not match l3vers.dtx!}
   \cs_set_eq:NN \includeltpatch \relax
\fi
%    \end{macrocode}
% \end{macro}
%
% \subsection{cfg}
%
%    \begin{macrocode}
\bool_if:NTF \g_doc_full_bool {
  \RecordChanges
  \CodelineIndex
  \EnableCrossrefs
  \AlsoImplementation
}{
  \CodelineNumbered
  \DisableCrossrefs
  \OnlyDescription
}
%    \end{macrocode}
%
% Input a local configuration file, if it exists.
%    \begin{macrocode}
\InputIfFileExists{l3doc.cfg}{
  \typeout{*************************************^^J
           *~Local~config~file~l3doc.cfg~used   ^^J
           *************************************}
}{}
%    \end{macrocode}
%
%    \begin{macrocode}
%</class>
%    \end{macrocode}
%
%
% \subsection{Makeindex configuration}
%
% The makeindex style "l3doc.ist" is used in place of the usual
% "gind.ist" to ensure that I is used in the sequence I J K
% not I II II, which would be the default makeindex behaviour.
%
% Will: Do we need this?
%
%    \begin{macrocode}
%<*docist>
actual '='
quote '!'
level '>'
preamble
"\n \\begin{theindex} \n \\makeatletter\\scan@allowedfalse\n"
postamble
"\n\n \\end{theindex}\n"
item_x1   "\\efill \n \\subitem "
item_x2   "\\efill \n \\subsubitem "
delim_0   "\\pfill "
delim_1   "\\pfill "
delim_2   "\\pfill "
% The next lines will produce some warnings when
% running Makeindex as they try to cover two different
% versions of the program:
lethead_prefix   "{\\bfseries\\hfil "
lethead_suffix   "\\hfil}\\nopagebreak\n"
lethead_flag       1
heading_prefix   "{\\bfseries\\hfil "
heading_suffix   "\\hfil}\\nopagebreak\n"
headings_flag       1

% and just for source3:
% Remove R so I is treated in sequence I J K not I II III
page_precedence "rnaA"
%</docist>
%    \end{macrocode}
%
% \section{Testing}
%
% \begin{function}{\example_foo:N|\example_foo:c}
% \begin{syntax}
%   "\example_foo:N" <arg1>
% \end{syntax}
% <0123456789> <\0\1\2\3\4\5\6\7\8\9>  \Arg{arg1} ":("
% \end{function}
%
% \begin{function}{ \foo | \foo: | \foo:z | \barrz: }
% \begin{syntax}
%   "\example_foo:N" <arg1>
% \end{syntax}
% <0123456789> <\0\1\2\3\4\5\6\7\8\9>
% \end{function}
%
% \begin{function}{\foo:N / (TF) | \foo_if:c / (TF) (EXP)}
% Test.
% \end{function}
%
% \begin{function}{ \bar / (EXP) | \bar: / (EXP)  | \bar:z / (EXP)  | }
% \begin{syntax}
%   "\example_foo:N" <arg1>
% \end{syntax}
% <0123456789> <\0\1\2\3\4\5\6\7\8\9>
% \end{function}
%
% \begin{macro}{ \foo , \foo: , \foo:z }
% Testing.
% \end{macro}
%
% \bigskip\bigskip
%
% \begin{macro}[aux]{ \foo_aux: }
% Testing.
% \end{macro}
%
% \bigskip\bigskip
%
% \begin{macro}[TF]{ \foo_if:c }
% Testing.
% \end{macro}
%
% \bigskip\bigskip
%
% \begin{macro}{\c_minus_one}
% \begin{macro}{\c_zero}
% \begin{macro}{\c_one}
% \begin{macro}{\c_two}
% \begin{macro}{\c_three}
% \begin{macro}{\c_four}
% \begin{macro}{\c_five}
% \begin{macro}{\c_six}
% \begin{macro}{\c_seven}
% \begin{macro}{\c_eight}
% \begin{macro}{\c_nine}
% \begin{macro}{\c_ten}
% \begin{macro}{\c_eleven}
% \begin{macro}{\c_sixteen}
% \begin{macro}{\c_thirty_two}
% \begin{macro}{\c_hundred_one}
% \begin{macro}{\c_twohundred_fifty_five}
% \begin{macro}{\c_twohundred_fifty_six}
% \begin{macro}{\c_thousand}
% \begin{macro}{\c_ten_thousand}
% \begin{macro}{\c_ten_thousand_one}
% \begin{arguments}
% \item name
% \item parameters
% \end{arguments}
% Another test.
% \end{macro}
% \end{macro}
% \end{macro}
% \end{macro}
% \end{macro}
% \end{macro}
% \end{macro}
% \end{macro}
% \end{macro}
% \end{macro}
% \end{macro}
% \end{macro}
% \end{macro}
% \end{macro}
% \end{macro}
% \end{macro}
% \end{macro}
% \end{macro}
% \end{macro}
% \end{macro}
% \end{macro}
%
%
% \subsection{Macros}
% \raggedright
% \ExplSyntaxOn
% \seq_map_inline:Nn \g_doc_macros_seq { `\texttt{#1}' \quad }
% \ExplSyntaxOff
%
% \subsection{Functions}
% \ExplSyntaxOn
% \seq_map_inline:Nn \g_doc_functions_seq { `\texttt{#1}' \quad }
% \ExplSyntaxOff
%
% \end{implementation}
%
% \PrintIndex
%
% \endinput

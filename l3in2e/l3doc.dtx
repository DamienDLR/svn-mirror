% \iffalse
%% File: l3doc.dtx Copyright (C) 1990-2008 LaTeX3 project
%%
%% It may be distributed and/or modified under the conditions of the
%% LaTeX Project Public License (LPPL), either version 1.3c of this
%% license or (at your option) any later version.  The latest version
%% of this license is in the file
%%
%%    http://www.latex-project.org/lppl.txt
%%
%% This file is part of the ``expl3 bundle'' (The Work in LPPL)
%% and all files in that bundle must be distributed together.
%%
%% The released version of this bundle is available from CTAN.
%%
%% -----------------------------------------------------------------------
%%
%% The development version of the bundle can be found at
%%
%%    http://www.latex-project.org/cgi-bin/cvsweb.cgi/
%%
%% for those people who are interested.
%%
%%%%%%%%%%%
%% NOTE: %%
%%%%%%%%%%%
%%
%%   Snapshots taken from the repository represent work in progress and may
%%   not work or may contain conflicting material!  We therefore ask
%%   people _not_ to put them into distributions, archives, etc. without
%%   prior consultation with the LaTeX Project Team.
%%
%% -----------------------------------------------------------------------
%<*driver>
\def\nameofplainTeX{plain}
\ifx\fmtname\nameofplainTeX\else
  \expandafter\begingroup
\fi
\input docstrip.tex
\askforoverwritefalse
\preamble


EXPERIMENTAL CODE

Do not distribute this file without also distributing the
source files specified above.

Do not distribute a modified version of this file.


\endpreamble
% stop docstrip adding \endinput
\postamble
\endpostamble
\keepsilent
\generate{\file{l3doc.cls}{\from{l3doc.dtx}{class}}}

\ifx\fmtname\nameofplainTeX
  \expandafter\endbatchfile
\else
  \expandafter\endgroup
\fi
%</driver>
%
%<*driver|class>
\RequirePackage{l3names}
%</driver|class>
%
% Need to protect the file metadata for any modules that load l3doc.
% This is restored after "\ProvideExplClass" below.
%    \begin{macrocode}
%<class>\let        \filenameOld        \filename
%<class>\let     \filenameextOld     \filenameext
%<class>\let        \filedateOld        \filedate
%<class>\let     \fileversionOld     \fileversion
%<class>\let \filedescriptionOld \filedescription
%    \end{macrocode}
%
%<*driver|class>
\GetIdInfo$Id: l3doc.dtx 824 2008-09-17 14:21:11Z will $
          {L3 Experimental documentation class}
%</driver|class>
%
%<*driver>
\ProvidesFile{\filename.\filenameext}
  [\filedate\space v\fileversion\space\filedescription]
\documentclass[full]{l3doc}
\begin{document}
\DocInput{\filename.\filenameext}
\end{document}
%</driver>
%
% This isn't included in the typeset documentation because it's a bit ugly:
%<*class>
\ProvidesExplClass
  {\filename}{\filedate}{\fileversion}{\filedescription}
\RequirePackage{expl3}
\let        \filename        \filenameOld
\let     \filenameext     \filenameextOld
\let        \filedate        \filedateOld
\let     \fileversion     \fileversionOld
\let \filedescription \filedescriptionOld
%</class>
% \fi
%
% \title{The \textsf{l3doc} class\thanks{This file
%         has version number \fileversion, last
%         revised \filedate.}}
% \author{\Team}
% \date{\filedate}
% \maketitle
%
% \section{Introduction}
%
% This is an ad-hoc class for documenting the \textsf{expl3} bundle,
% a collection of modules or packages that make up \LaTeX3's programming 
% environment. Eventually it will replace the "ltxdoc" class for \LaTeX3, 
% but not before the good ideas in "hypdoc", "xdoc"("2"), "mfpdoc", and "gmdoc" are 
% incorporated.
%
% It is written as a `self-contained' docstrip file: executing 
% "latex l3doc.dtx" 
% will generate the "l3doc.cls" file and typeset this
% documentation; execute "tex l3doc.dtx" to only generate the ".cls" file.
%
% For now, this package is written in \LaTeXe\ syntax, but that will no doubt
% change\dots.
%    \begin{macrocode}
%<*class>
\ExplSyntaxOff
%    \end{macrocode}
%
% The Guilty Parties.
%    \begin{macrocode}
\def\Team{%
  The \LaTeX3 Project\thanks{%
  Frank Mittelbach, Denys Duchier, Chris Rowley,
  Rainer Sch\"opf, Johannes Braams, Michael Downes,
  David Carlisle, Alan Jeffrey, Morten H\o gholm, Thomas Lotze,
  Javier Bezos, Will Robertson}}
%    \end{macrocode}
%
% \section{Options}                       
%
%    \begin{macrocode}
\DeclareOption{a5paper}{\@latexerr{Option not supported}{}}
\DeclareOption{full}{\AtBeginDocument{\input{l3full.cfg}}}
\DeclareOption*{\PassOptionsToClass{\CurrentOption}{article}}
\ProcessOptions
%    \end{macrocode}
%
% \section{Class and package loading}
%    \begin{macrocode}
\LoadClass{ltxdoc}
\RequirePackage{color,fixltx2e,trace,url}
\RequirePackage[T1]{fontenc}
\RequirePackage{lmodern}
\RequirePackage{hypdoc}
%    \end{macrocode}
% hypdoc kills |\function|'s sanitising of "_", for some reason, 
% so we load "underscore" (for now) to work around this:
%    \begin{macrocode}
\RequirePackage{underscore}
%    \end{macrocode}
%
% \section{Configuration}
% Input a local configuration file, if it exists.
%    \begin{macrocode}
\InputIfFileExists{l3doc.cfg}
           {\typeout{*************************************^^J%
                     * Local config file l3doc.cfg used^^J%
                     *************************************}}
           {}
%    \end{macrocode}
%
%    \begin{macrocode}
\renewcommand\MakePrivateLetters{%
  \makeatletter
  \catcode`\_=11\relax
  \catcode`\:=11\relax}
%    \end{macrocode}
%
%    \begin{macrocode}
\CodelineNumbered
\EnableCrossrefs
\setcounter{StandardModuleDepth}{1}
%    \end{macrocode}
%
% \section{Design}
%
% Increase the text width slightly so that width the standard fonts
% 72 columns of code may appear in a |macrocode| environment.
% Increase the marginpar width slightly, for long command names.
% And increase the left margin by a similar amount.
%    \begin{macrocode}
\setlength   \textwidth      { 385pt }
\addtolength \marginparwidth {  30pt }
\addtolength \oddsidemargin  {  20pt }
\addtolength \evensidemargin {  20pt }
%    \end{macrocode}
% (These were introduced when "article" was the documentclass, but
%  I've left them here for now to remind me to do something about them
%  later; we still have the problem of \emph{very long} command names.)
%
% Customise lists:
%    \begin{macrocode}
\let\@@oldlist\list
\def\list#1#2{\@@oldlist{#1}{#2\listparindent\z@}}
\setlength \parindent  { 0pt }
\setlength \itemindent { 0pt }
\setlength \parskip    { \medskipamount }
%    \end{macrocode}
%
% \section{Text markup}
%
%    Make "|" and |"| be `short verb' characters, but not in
%    the document preamble, where an active character may interfere
%    with packages that are loaded.
%    \begin{macrocode}
\AtBeginDocument{%
  \MakeShortVerb\"%
  \MakeShortVerb\|%
}
%    \end{macrocode}
%
%    \begin{macrocode}
\newcommand\tn[1]{%
  \texttt{\bslash #1}%
  \index{TeX and LaTeX2e commands\actualchar
          \string\TeX{} and \string\LaTeXe{} commands:\levelchar
          #1\actualchar{\string\ttfamily\string\bslash{}#1}}}
%    \end{macrocode}
%
%    \begin{macrocode}
\providecommand*\eTeX{%
  \if b\expandafter\@car\f@series\@nil\boldmath\fi
  $\m@th\varepsilon$-\TeX}
%    \end{macrocode}
%
%    \begin{macrocode}
\let\m\meta
\let\file\url
%    \end{macrocode}
%
%    \begin{macrocode}
\newcommand\Arg[1]{\texttt{\{}\,\meta{#1}\,\texttt{\}}}
%    \end{macrocode}
%
% \begin{environment}{texnote}
%    \begin{macrocode}
\newenvironment{texnote}{%
  \endgraf
  \vspace{3mm}%
  \small\textbf{\TeX{}hackers~note:}%
}{%
  \vspace{3mm}%
}
%    \end{macrocode}
% \end{environment}
%    
% stick the function names in a box. Use an active "|" as delimiter and 
% allow |<...>| to be used as markup for |\meta{...}|
%    \begin{macrocode}
\begingroup
\catcode`\|=\active
\catcode`\<=\active
%    \end{macrocode}
%
% \begin{environment}{function}
% \begin{environment}{variable}
% Environment for documenting function(s).
%    \begin{macrocode}
\gdef\function{%
  \catcode`\<=\active\def<##1>{\meta{##1}}%
  \bigskip
  \begingroup
  \@makeother\\
  \catcode`\~=10
  \catcode`\ =9
  \catcode`\^^M=9
  \catcode`\^^I=9
  \catcode`\|=\active
  \@function}
%    \end{macrocode}
% And the "variable" function is exactly the same:
%    \begin{macrocode}
\global\let\variable\function
%    \end{macrocode}
% \end{environment}
% \end{environment}
% 
%    \begin{macrocode}
\gdef\@function#1{%
  \ifx\@function#1\@function\else
    \funcindex#1| |%
  \fi
  \def\nextnewline{\gdef\nextnewline{\\}}
  \global\let\macroname\@empty
%    \end{macrocode}
% 
%    \begin{macrocode}
  \endgraf
  \ttfamily
  \noindent\hspace*{-2\parindent}%
  \tabular[b]{ @{\vline\hskip\tabcolsep}
               l @{} c
               @{\hskip\tabcolsep\vline} }%
    \hline
    \showexplmacros#1|\@nil
    \crcr
    \hline
    \noalign{\kern-2pt}%
  \endtabular
  \endgroup
}
%    \end{macrocode}
%
% index fuction lists (could be better...)
%    \begin{macrocode}
\ExplSyntaxOn
\gdef:Npn \funcindex #1| {
  \tlist_if_blank:nTF {#1} {
%    \end{macrocode}
% at the end, look for extra "|" and remove it:
%    \begin{macrocode}
    \@ifnextchar|{\@gobble}{}
  }{
    \def\yyy{#1}
    \expandafter\vindex\meaning\yyy\@nil  
    \funcindex
  }
}
\ExplSyntaxOff
%    \end{macrocode}
%
% \begin{macro}{\showexplmacros}
% Will: I wanted to colourise the macros shown in |{function}| blocks;
% this is the ugly code that does that.
%    \begin{macrocode}
\gdef\showexplmacros#1|{%
  \showexplmacro#1::\@nil
  \@ifnextchar\@nil{\@gobble}{\showexplmacros}%
}
%    \end{macrocode}
% \end{macro}
% 
%    \begin{macrocode}
\gdef\showexplmacro#1:#2:#3\@nil{%
  \gdef\newmacroname{#1}%
  \gdef\macrosuffix{#2}%
  \parse@meta#2/\@nil
  \ifx\newmacroname\@empty
    \expandafter\@gobble
  \else
    \expandafter\@firstofone
  \fi{%
    \nextnewline
    \ifx\newmacroname\macroname
      \color[gray]{0.5}%
    \else
      \global\let\macroname\newmacroname
      \color{black}%
    \fi
    \detokenize{#1}\unskip
    \ifx\macrosuffix\@empty
      #3% in case we gobbled a : we wanted
    \else
      \color{black}:\before@slash\unskip
    \fi
    &%
%    \end{macrocode}
% This is `informational' part of the macro; at the moment to show if it's
% expandable or not.
%    \begin{macrocode}
    \def\@tempa{{(EXP)}}%
    \expandafter\expandafter\expandafter\in@\expandafter\@tempa\after@slash
    \ifin@
      \hspace{\tabcolsep}$\star$%
    \else
    \fi}%
}
%    \end{macrocode}
%
%
% End the scope of having active "|" and "<".
%    \begin{macrocode}
\endgroup
%    \end{macrocode}
%
%    \begin{macrocode}
\def\parse@meta#1/#2\@nil{%
  \gdef\before@slash{#1}%
  \gdef\after@slash{{#2}}}
%    \end{macrocode}
%
% \begin{environment}{syntax}
% Syntax block placed next to the list of functions to illustrate their use.
%    \begin{macrocode}
\newenvironment{syntax}{%
  \minipage[b]{.7\textwidth}%
    \def\meta@font@select{\rmfamily\itshape} % (Will: I HATE italic cmtt!)
    \small
    \ttfamily
    \raggedright
    \obeyspaces
    \obeylines
}{%
  \endminipage
  \hfil\break
  \global\@ignoretrue
}
%    \end{macrocode}
% \end{environment}
%
% \begin{environment}{macros}
% We want to extend the old definition to allow comma-separated lists of
% macros, rather than one at a time. (Will: like most things I write, this
% code is pretty hacky. No effort is made to normalise the argument after
% tokenisation.)
%    \begin{macrocode}
\newenvironment{macros}[1]{%
  \let\doc@beginmacros\@empty
  \let\doc@endmacros\@empty
  \@for\@ii:=\@empty#1\do{%
    \expandafter\macros@addto\expandafter{\@ii}%
  }%
  \doc@beginmacros
}{%
  \doc@endmacros
}
%    \end{macrocode}
% Helper to make expansion bearable.
%    \begin{macrocode}
\newcommand\macros@addto[1]{%
  \g@addto@macro\doc@beginmacros{\begin{macro}{#1}}%
  \g@addto@macro\doc@endmacros{\end{macro}}%
}
%    \end{macrocode}
% \end{environment}
%
% \section{Everything else}
%
% Will: To be honest I'm not entirely sure if the following code is working or
% usable or relevant any more. It may well be, I just don't know exactly
% what it's for\dots
%
%    \begin{macrocode}
\def\vindex#1>#2\@nil{%
  \edef\temp{\foo#2<>\@nil}%
  \ifx\temp\space\else
  \expandafter\SpecialMainIndexB\expandafter{\temp}\fi}
%    \end{macrocode}
% remove <foo> from index (quick hack, could be better)
%    \begin{macrocode}
\def\foo#1<#2>#3\@nil{%
 #1\ifx\foo#3\foo
   \else#2\expandafter\foo\fi#3\@nil}
%\def\foo#1<#2>#3\@nil{#1<#2>#3}
\def\foo#1<#2>#3\@nil{#1}
%    \end{macrocode}
%
%
% \begin{macro}{\SpecialMainIndexB}
% Heiko's replacement to play nicely with |hypdoc|:
%    \begin{macrocode}
\def\SpecialMainIndexB#1{%
  \index{%
    \@gobble#1%
    \actualchar
    \string\verb\quotechar*\verbatimchar#1\verbatimchar
    \encapchar
    hdpgindex{\thepage}{usage}%
  }%
}
\def\hdpgindex#1#2#3{%
  \csname\ifx\\#2\\relax\else#2\fi\endcsname{%
    \hyperlink{page.#1}{#3}%
  }%
}
%    \end{macrocode}
% \end{macro}
%
% This command will be used to input the patch file
% if that file exists.
%    \begin{macrocode}
\newcommand{\includeltpatch}{%
  \def\currentfile{l3patch.ltx}
  \part{l3patch}
  {\let\ttfamily\relax
    \xdef\filekey{\filekey, \thepart={\ttfamily\currentfile}}}%
  Things we did wrong\ldots
  \IndexInput{l3patch.ltx}}
%    \end{macrocode}
% Get the date from ltvers.dtx
%    \begin{macrocode}
\def\patchdate{0}
\begingroup
   %%\def\ProvidesFile#1\fmtversion#2{\date{#2}\endinput}
   \def\ProvidesFile#1[#2#3]{\date{#2}\endinput}
   % \iffalse
%% File: l3vers.dtx Copyright (C) 1990-2005 LaTeX3 project
%%
%% It may be distributed and/or modified under the conditions of the
%% LaTeX Project Public License (LPPL), either version 1.3a of this
%% license or (at your option) any later version.  The latest version
%% of this license is in the file
%%
%%    http://www.latex-project.org/lppl.txt
%%
%% This file is part of the ``expl3 bundle'' (The Work in LPPL)
%% and all files in that bundle must be distributed together.
%%
%% The released version of this bundle is available from CTAN.
%%
%% -----------------------------------------------------------------------
%%
%% The development version of the bundle can be found at
%%
%%    http://www.latex-project.org/cgi-bin/cvsweb.cgi/
%%
%% for those people who are interested.
%%
%%%%%%%%%%%
%% NOTE: %%
%%%%%%%%%%%
%%
%%   Snapshots taken from the repository represent work in progress and may
%%   not work or may contain conflicting material!  We therefore ask
%%   people _not_ to put them into distributions, archives, etc. without
%%   prior consultation with the LaTeX Project Team.
%%
%% -----------------------------------------------------------------------
%% \fi
%
% \iffalse
%%% From File: l3vers.dtx
%
%<*driver>
% \fi
\ProvidesFile{l3vers.dtx}
             [2005/03/09 v0.1 LaTeX Kernel (Version Info)]
% \iffalse
\documentclass{ltxdoc}
\GetFileInfo{l3vers.dtx}
\title{\filename}
\date{\filedate}
 \author{%
  Johannes Braams\and
  David Carlisle\and
  Alan Jeffrey\and
  Leslie Lamport\and
  Frank Mittelbach\and
  Chris Rowley\and
  Rainer Sch\"opf}
\begin{document}
\maketitle
 \DocInput{\filename}
\end{document}
%</driver>
% \fi
%
% \CheckSum{46}
%
% \section{Version Identification}
% First we identify the date and version number of this release of
% \LaTeX, and set |\everyjob| so that it is printed at the start of
% every \LaTeX\ run.
%
% \StopEventually{}
%
%
% \begin{macro}{\fmtname}
% \begin{macro}{\fmtversion}
%    \begin{macrocode}
%<*3kernel>
\def\fmtname{LaTeX3}
\edef\fmtversion{--release--date--goes--here--}
%    \end{macrocode}
% \end{macro}
% \end{macro}
% 
% Check that the format being made is not too old.
% The error message complains about `more than 5 years'
% but in fact the error is not triggered until 65 months.
%
% This code is currently not activated as we don't know if we already
% got to the last official 2e version (due to staff shortage or due to
% a successor (think positive:-)).
%    \begin{macrocode}
\iftrue
\def\reserved@a#1/#2/#3\@nil{%
  \count@\year
  \advance\count@-#1\relax
  \multiply\count@ by 12\relax
  \advance\count@\month
  \advance\count@-#2\relax}
\expandafter\reserved@a\fmtversion\@nil
%    \end{macrocode}
% |\count@| is now the age of this file in months. Take a generous
% definition of `year' so this message is not generated too often.
%    \begin{macrocode}
\ifnum\count@>65
  \typeout{^^J%
!!!!!!!!!!!!!!!!!!!!!!!!!!!!!!!!!!!!!!!!!!!!!!!!!!!!!!!!!!!!!!!!!!^^J%
!  You are attempting to make a LaTeX format from a source file^^J%
!  That is more than five years old.^^J%
!^^J%
!  If you enter <return> to scroll past this message then the format^^J%
!  will be built, but please consider obtaining newer source files^^J%
!  before continuing to build LaTeX.^^J%
!!!!!!!!!!!!!!!!!!!!!!!!!!!!!!!!!!!!!!!!!!!!!!!!!!!!!!!!!!!!!!!!!!^^J%
}
   \errhelp{To avoid this error message, obtain new LaTeX sources.}
   \errmessage{LaTeX source files more than 5 years old!}
\fi
\let\reserved@a\relax
\fi
%    \end{macrocode}
%
% This startup banner may be further modified by the code in
% |ltfinal.dtx| if a patch file is present.
%    \begin{macrocode}
\everyjob{\typeout{\fmtname
                                     \space<\fmtversion>}}
\immediate\write16{\fmtname
                                     \space<\fmtversion>}
%</3kernel>
%    \end{macrocode}
%
% \Finale
%
\endinput
%
% $Log$
% Revision 1.1  2005/03/11 21:26:31  braams
% File copied from 2e for now, to hold the format version identification
%
   \xdef\@date{\@date}
   \global\let\X@date=\@date
% Add the patch version if available.
   \long\def\Xdef#1#2#3\def#4#5{%
    \xdef\X@date{#2}%
    \xdef\patchdate{#5}%
    \endinput}%
   \InputIfFileExists{l3patch.ltx}
    {\let\def\Xdef}{\global\let\includeltpatch\relax}
\endgroup
%    \end{macrocode}
%    \begin{macrocode}
\ifx\@date\X@date
   \def\Xpatch{0}
   \ifx\patchdate\Xpatch\else
     \edef\@date{\@date\space Patch level \patchdate}
   \fi
\else
   \@warning{l3patch.ltx does not match l3vers.dtx!}
   \let\includeltpatch\relax
\fi
%    \end{macrocode}
%
%    \begin{macrocode}
%</class>
%    \end{macrocode}
% \Finale
%
% \endinput

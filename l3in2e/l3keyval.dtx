% \iffalse
%% File: l3keyval.dtx Copyright (C) 2006-2009 LaTeX3 project
%%
%% It may be distributed and/or modified under the conditions of the
%% LaTeX Project Public License (LPPL), either version 1.3c of this
%% license or (at your option) any later version.  The latest version
%% of this license is in the file
%%
%%    http://www.latex-project.org/lppl.txt
%%
%% This file is part of the ``expl3 bundle'' (The Work in LPPL)
%% and all files in that bundle must be distributed together.
%%
%% The released version of this bundle is available from CTAN.
%%
%% -----------------------------------------------------------------------
%%
%% The development version of the bundle can be found at
%%
%%    http://www.latex-project.org/svnroot/experimental/trunk/
%%
%% for those people who are interested.
%%
%%%%%%%%%%%
%% NOTE: %%
%%%%%%%%%%%
%%
%%   Snapshots taken from the repository represent work in progress and may
%%   not work or may contain conflicting material!  We therefore ask
%%   people _not_ to put them into distributions, archives, etc. without
%%   prior consultation with the LaTeX Project Team.
%%
%% -----------------------------------------------------------------------
%<*driver|package>
\RequirePackage{l3names}
%</driver|package>
%\fi
\GetIdInfo$Id$
       {L3 Experimental keyval processing}
%\iffalse
%<*driver>
%\fi
\ProvidesFile{\filename.\filenameext}
  [\filedate\space v\fileversion\space\filedescription]
%\iffalse
\documentclass[full]{l3doc}
\begin{document}
\DocInput{l3keyval.dtx}
\end{document}
%</driver>
% \fi
%
%
% \title{The \textsf{l3keyval} package\thanks{This file
%         has version number \fileversion, last
%         revised \filedate.}\\
% Key-value parsing}
% \author{\Team}
% \date{\filedate}
% \maketitle
%
% \begin{documentation}
% 
% A key--value list is input of the form
%\begin{verbatim}
%  KeyOne = ValueOne ,
%  KeyTwo = ValueTwo ,
%  KeyThree          ,
%\end{verbatim} 
% where each key--value pair is separated by a comma from the rest of
% the list, and each key--value pair does not necessarily contain an
% equals sign or a value! Processing this type of input correctly 
% requires a number of careful steps, to correctly account for
% braces, spaces and the category codes of separators.
% 
% This module provides the low-level machinery for processing arbitrary
% key--value lists. The \pkg{l3keys} module provides a higher-level
% interface for managing run-time settings using key--value input,
% while other parts of \LaTeX3 also use key--value input based on
% \pkg{l3keyval} (for example the \pkg{xtemplate} module).
% 
%\section{Features of \pkg{l3keyval}} 
%
% As \pkg{l3keyval} is a low-level module, its functions are
% restricted to converting a \meta{keyval list} into keys and values
% for further processing. Each key and value (or key alone)
% has to be processed further by a function provided when
% \pkg{l3keyval} is called. Typically, this will be \emph{via}
% one of the \cs{KV_process\ldots} functions:
%\begin{verbatim}
%  \KV_process_space_removal_sanitize:NNn
%    \my_processor_function_one:n
%    \my_processor_function_two:nn
%    { <keyval list> }
%\end{verbatim} 
% The two processor functions here handle the cases where there is
% only a key, and where there is both a key and value, respectively.
% 
% \pkg{l3keyval} parses key--value lists in a manner that does not
% double "#" tokens or expand any input. The module has processor
% functions which will sanitize the category codes of \texttt{=}
% and \texttt{,} tokens (for use in the document body) as well
% as faster versions which do not do this (for use inside code
% blocks). Spaces can be removed from each end of the key and
% value (again for the document body), again with faster code
% to be used where this is not necessary. Values which are wrapped
% in braces will have exactly one set removed, meaning that
%\begin{verbatim}
%  key = {value here},
%\end{verbatim} 
% and 
%\begin{verbatim}
%  key = value here,
%\end{verbatim} 
% are treated as identical (assuming that space removal is in force).
% \pkg{l3keyval} 
% 
%\section{Functions for keyval processing}
%
% The \pkg{l3keyval} module should be accessed \emph{via} a small
% set of external functions. These correctly set up the module 
% internals for use by other parts of \LaTeX3.
% 
% In all cases, two functions have to be supplied by the programmer
% to apply to the items from the <keyval list> after \pkg{l3keyval}
% has separated out the entries. The first function should take
% one argument, and will receive the names of keys for which no
% value was supplied. The second function should take two arguments:
% a key name and the associated value. 
%
%\begin{function}{\KV_process_space_removal_sanitize:NNn}
%  \begin{syntax}
%    "\KV_process_space_removal_sanitize:NNn"
%    ~~~~<function 1> <function 2> \Arg{keyval list}
%  \end{syntax}
%  Parses the <keyval list> splitting it into keys and associated
%  values. Spaces are removed from the ends of both the key and
%  value by this function, and the category codes of non-braced
%  \texttt{=} and \texttt{,} tokens are normalised so that 
%  parsing is `category code safe'.  After parsing is completed, 
%  <function 1> is used to process keys without values and 
%  <function 2> deals with keys which have associated values. 
%\end{function}
%
%\begin{function}{\KV_process_space_removal_no_sanitize:NNn}
%  \begin{syntax}
%    "\KV_process_space_removal_no_sanitize:NNn"
%    ~~~~<function 1> <function 2> \Arg{keyval list}
%  \end{syntax}
%  Parses the <keyval list> splitting it into keys and associated
%  values. Spaces are removed from the ends of both the key and
%  value by this function, but category codes are not normalised.  
%  After parsing is completed, <function 1> is used to process keys 
%  without values and <function 2> deals with keys which have 
%  associated values. 
%\end{function}
%
%\begin{function}{\KV_process_no_space_removal_no_sanitize:NNn}
%  \begin{syntax}
%    "\KV_process_no_space_removal_no_sanitize:NNn"
%    ~~~~<function 1> <function 2> \Arg{keyval list}
%  \end{syntax}
%  Parses the <keyval list> splitting it into keys and associated
%  values. Spaces are \emph{not} removed from the ends of
%  the key and value, and category codes are \emph{not} normalised.  
%  After parsing is completed, <function 1> is used to process keys 
%  without values and <function 2> deals with keys which have 
%  associated values. 
%\end{function}
%
% \begin{variable}{\l_KV_remove_one_level_of_braces_bool}
%   This boolean controls whether or not one level of braces is
%   stripped from the key and value. The default value for this
%   boolean is \texttt{true} so that exactly one level of braces is
%   stripped. For certain applications it is desirable to keep the
%   braces in which case the programmer just has to set the boolean
%   false temporarily. Only applicable when spaces are being
%   removed.
% \end{variable}
%
%\section{Internal functions}
%
% The remaining functions provided by \pkg{l3keyval} do not
% have any protection for nesting of one call to the module 
% inside another. They should therefore not be called directly by
% other modules.
%
% \begin{function}{\KV_parse_no_space_removal_no_sanitize:n}
%   \begin{syntax}
%     "\KV_parse_no_space_removal_no_sanitize:n" \Arg{keyval list}
%   \end{syntax}
%   Parses the keys and values, passing the results to
%   \cs{KV_key_no_value_elt:n} and \cs{KV_key_value_elt:nn} as
%   appropriate. Spaces are not removed in the parsing process and
%   the category codes of \texttt{=} and \texttt{,} are not
%   normalised.
% \end{function}
%
% \begin{function}{\KV_parse_space_removal_no_sanitize:n}
%   \begin{syntax}
%     "\KV_parse_space_removal_no_sanitize:n" \Arg{keyval list}
%   \end{syntax}
%   Parses the keys and values, passing the results to
%   \cs{KV_key_no_value_elt:n} and \cs{KV_key_value_elt:nn} as
%   appropriate. Spaces are removed in the parsing process from the
%   ends of the key and value, but the category codes of \texttt{=} 
%   and \texttt{,} are not normalised.
% \end{function}
%
% \begin{function}{\KV_parse_space_removal_sanitize:n}
%   \begin{syntax}
%     "\KV_parse_space_removal_sanitize:n" \Arg{keyval list}
%   \end{syntax}
%   Parses the keys and values, passing the results to
%   \cs{KV_key_no_value_elt:n} and \cs{KV_key_value_elt:nn} as
%   appropriate. Spaces are removed in the parsing process from the
%   ends of the key and value and the category codes of \texttt{=} 
%   and \texttt{,} are normalised at the outer level 
%   (\emph{i.e}.~only unbraced tokens are affected).
% \end{function}
%
% \begin{function}{
%     \KV_key_no_value_elt:n |
%     \KV_key_value_elt:nn
% }
%   \begin{syntax}
%     "\KV_key_no_value_elt:n" \Arg{key} \\
%     "\KV_key_value_elt:n"    \Arg{key} \Arg{value}
%   \end{syntax}
%   Used by \cs{KV_parse\ldots} functions to further process
%   keys with no values and keys with values, respectively. The
%   standard definitions are error functions: the programmer should
%   provide appropriate definitions for both at point of use.
% \end{function}
% 
%\section{Variables and constants}
%
%\begin{variable}{\c_KV_single_equal_sign_tl} 
%  Constant token list to make finding \texttt{=} faster.
%\end{variable}
%
%\begin{variable}{ 
%  \l_KV_tmpa_tl | 
%  \l_KV_tmpb_tl
%}
%  Scratch token lists.
%\end{variable}
% 
%\begin{variable}{
%  \l_KV_parse_toks   | 
%  \l_KV_currkey_toks |
%  \l_KV_currval_toks 
%}
% Token registers for various parts of the parsed input.
%\end{variable}
%
% \end{documentation}
%
% \begin{implementation}
%
% \section{\pkg{l3keyval} implementation}
%
% \begin{function}{ \KV_sanitize_outerlevel_active_equals:N |
%                   \KV_sanitize_outerlevel_active_commas:N }
% \begin{syntax}
% "\KV_sanitize_outerlevel_active_equals:N" <tl var.>
% \end{syntax}
% Replaces catcode other "=" and "," within a <tl var.> with active characters.
% \end{function}
%
% \begin{function}{ \KV_remove_surrounding_spaces:nw |
%                   \KV_remove_surrounding_spaces_auxi:w / (EXP) }
% \begin{syntax}
%
% "\KV_remove_surrounding_spaces:nw" <toks> <token list> "\q_nil"
% "\KV_remove_surrounding_spaces_auxi:w" <token list> \verb*" Q"$\sb3$
% \end{syntax}
%   Removes a possible
%   leading space plus a possible ending space from a <token list>.
%   The first version (which is not used in the code) stores it in <toks>.
% \end{function}
%
% \begin{function}{ \KV_add_value_element:w | \KV_set_key_element:w }
% \begin{syntax}
% "\KV_set_key_element:w" <token list> "\q_nil"
% "\KV_add_value_element:w" "\q_stop" <token list> "\q_nil"
% \end{syntax}
% Specialised functions to strip spaces from their input and set the
% token registers "\l_KV_currkey_toks" or "\l_KV_currval_toks"
% respectively.
% \end{function}
%
% \begin{function}{ \KV_split_key_value_current:w                      |
%                   \KV_split_key_value_space_removal:w                |
%                   \KV_split_key_value_space_removal_detect_error:wTF |
%                   \KV_split_key_value_no_space_removal:w             }
% \begin{syntax}
% "\KV_split_key_value_current:w" \dots
% \end{syntax}
% These functions split keyval lists into chunks depending which 
% sanitising method is being used. "\KV_split_key_value_current:w" is "\cs_set_eq:NN"
% to whichever is appropriate.
% \end{function}
%
% \subsection{Module code}
%
%    We start by ensuring that the required packages are loaded.
%    \begin{macrocode}
%<*package>
\ProvidesExplPackage
  {\filename}{\filedate}{\fileversion}{\filedescription}
\package_check_loaded_expl:
%</package>
%<*initex|package>
%    \end{macrocode}
%
%
% \begin{macro}{\l_KV_tmpa_tl}
% \begin{macro}{\l_KV_tmpb_tl}
% \begin{macro}{\c_KV_single_equal_sign_tl}
% Various useful things.
%    \begin{macrocode}
\tl_new:N  \l_KV_tmpa_tl
\tl_new:N  \l_KV_tmpb_tl
\tl_const:Nn \c_KV_single_equal_sign_tl { = }
%    \end{macrocode}
% \end{macro}
% \end{macro}
% \end{macro}
%
% \begin{macro}{\l_KV_parse_toks}
% \begin{macro}{\l_KV_currkey_toks}
% \begin{macro}{\l_KV_currval_toks}
% Some more useful things.
%    \begin{macrocode}
\toks_new:N \l_KV_parse_toks
\toks_new:N \l_KV_currkey_toks
\toks_new:N \l_KV_currval_toks
%    \end{macrocode}
% \end{macro}
% \end{macro}
% \end{macro}
% 
%\begin{macro}{\l_KV_level_int}
% This is used to track how deeply nested calls to the keyval processor
% are, so that the correct functions are always in use.
%    \begin{macrocode}
\int_new:N \l_KV_level_int
%    \end{macrocode}
%\end{macro} 
%
% \begin{macro}{\l_KV_remove_one_level_of_braces_bool}
% A boolean to control 
%    \begin{macrocode}
\bool_new:N \l_KV_remove_one_level_of_braces_bool
\bool_set_true:N \l_KV_remove_one_level_of_braces_bool
%    \end{macrocode}
% \end{macro}
% 
%\begin{macro}{\KV_process_space_removal_sanitize:NNn}
%\begin{macro}{\KV_process_space_removal_no_sanitize:NNn}
%\begin{macro}{\KV_process_no_space_removal_no_sanitize:NNn}
%\begin{macro}[aux]{\KV_process_aux:NNNn}
% The wrapper function takes care of assigning the appropriate 
% \texttt{elt} functions before and after the parsing step. In that
% way there is no danger of a mistake with the wrong functions being
% used.
%    \begin{macrocode}
\cs_new_protected_nopar:Npn \KV_process_space_removal_sanitize:NNn {
  \KV_process_aux:NNNn \KV_parse_space_removal_sanitize:n
}
\cs_new_protected_nopar:Npn \KV_process_space_removal_no_sanitize:NNn {
  \KV_process_aux:NNNn \KV_parse_space_removal_no_sanitize:n
}
\cs_new_protected_nopar:Npn \KV_process_no_space_removal_no_sanitize:NNn {
  \KV_process_aux:NNNn \KV_parse_no_space_removal_no_sanitize:n
}
\cs_new_protected:Npn \KV_process_aux:NNNn #1#2#3#4 {
  \cs_set_eq:cN 
    { KV_key_no_value_elt_ \int_use:N \l_KV_level_int :n }
    \KV_key_no_value_elt:n 
  \cs_set_eq:cN 
    { KV_key_value_elt_ \int_use:N \l_KV_level_int :nn }
    \KV_key_value_elt:nn
  \cs_set_eq:NN \KV_key_no_value_elt:n #2
  \cs_set_eq:NN \KV_key_value_elt:nn #3
  \int_incr:N \l_KV_level_int 
  #1 {#4}  
  \int_decr:N \l_KV_level_int
  \cs_set_eq:Nc \KV_key_no_value_elt:n 
    { KV_key_no_value_elt_ \int_use:N \l_KV_level_int :n } 
  \cs_set_eq:Nc \KV_key_value_elt:nn
    { KV_key_value_elt_ \int_use:N \l_KV_level_int :nn }
}
%    \end{macrocode}
%\end{macro} 
%\end{macro} 
%\end{macro} 
%\end{macro} 
%
% \begin{macro}{\KV_sanitize_outerlevel_active_equals:N}
% \begin{macro}{\KV_sanitize_outerlevel_active_commas:N}
%   Some functions for sanitizing top level equals and commas. Replace
%   |=|$\sb{13}$ and |,|$\sb{13}$ with |=|$\sb{12}$ and |,|$\sb{12}$
%   resp.
%    \begin{macrocode}
\group_begin:
\char_set_catcode:nn{`\=}{13}
\char_set_catcode:nn{`\,}{13}
\char_set_lccode:nn{`\8}{`\=}
\char_set_lccode:nn{`\9}{`\,}
\tl_to_lowercase:n{\group_end:
\cs_new_protected_nopar:Npn \KV_sanitize_outerlevel_active_equals:N #1{
  \tl_replace_all_in:Nnn #1 = 8
}
\cs_new_nopar:Npn \KV_sanitize_outerlevel_active_commas:N #1{
  \tl_replace_all_in:Nnn #1 , 9
}
}
%    \end{macrocode}
% \end{macro}
% \end{macro}
%
%
%
%
%
% \begin{macro}{\KV_remove_surrounding_spaces:nw}
% \begin{macro}{\KV_remove_surrounding_spaces_auxi:w}
% \begin{macro}[aux]{\KV_remove_surrounding_spaces_auxii:w}
% \begin{macro}{\KV_set_key_element:w}
% \begin{macro}{\KV_add_value_element:w}
%   The macro |\KV_remove_surrounding_spaces:nw| removes a possible
%   leading space plus a possible ending space from its second
%   argument and stores it in the token register |#1|.
%
%   Based on Around the Bend No.~15 but with some enhancements. For
%   instance, this definition is purely expandable.
%
%   We use a funny token |Q|$\sb3$ as a delimiter.
%    \begin{macrocode}
\group_begin:
\char_set_catcode:nn{`\Q}{3}
%    \end{macrocode}
%    \begin{macrocode}
\cs_gnew:Npn\KV_remove_surrounding_spaces:nw#1#2\q_nil{
%    \end{macrocode}
% The idea in this processing is to use a Q with strange catcode to
% remove a trailing space. But first, how to get this expansion going?
%
% If you have read the fine print in the \textsf{l3expan} module,
% you'll know that the |f| type expansion will expand until the first
% non-expandable token is seen and if this token is a space, it will
% be gobbled. Sounds useful for removing a leading space but we also
% need to make sure that it does nothing but removing that space!
% Therefore we prepend the argument to be trimmed with an
% |\exp_not:N|. Now why is that? |\exp_not:N| in itself is an
% expandable command so will allow the |f| expansion to continue. If
% the first token in the argument to be trimmed is a space, it will be
% gobbled and the expansion stop. If the first token isn't a space,
% the |\exp_not:N| turns it temporarily into |\scan_stop:| which is
% unexpandable. The processing stops but the token following directly
% after |\exp_not:N| is now back to normal.
%
% The function here allows you to insert arbitrary functions in the
% first argument but they should all be with an |f| type
% expansion. For the application in this module, we use
% |\toks_set:Nf|.
%
% Once the expansion has been kick-started, we apply
% |\KV_remove_surrounding_spaces_auxi:w| to the replacement text of
% |#2|, adding a leading |\exp_not:N|. Note that no braces are
% stripped off of the original argument.
%    \begin{macrocode}
  #1{\KV_remove_surrounding_spaces_auxi:w \exp_not:N#2Q~Q}
}
%    \end{macrocode}
% |\KV_remove_surrounding_spaces_auxi:w| removes a trailing space if
% present, then calls |\KV_remove_surrounding_spaces_auxii:w| to clean
% up any leftover bizarre Qs. In order for
% |\KV_remove_surrounding_spaces_auxii:w| to work properly we need to
% put back a Q first.
%    \begin{macrocode}
\cs_gnew:Npn\KV_remove_surrounding_spaces_auxi:w#1~Q{
  \KV_remove_surrounding_spaces_auxii:w #1 Q
}
%    \end{macrocode}
% Now all that is left to do is remove a leading space which should be
% taken care of by the function used to initiate the expansion. Simply
% return the argument before the funny Q.
%    \begin{macrocode}
\cs_gnew:Npn\KV_remove_surrounding_spaces_auxii:w#1Q#2{#1}
%    \end{macrocode}
% 
% Here are some specialized versions of the above. They do exactly
% what we want in one go. First trim spaces from the value and then
% put the result surrounded in braces onto |\l_KV_parse_toks|.
%    \begin{macrocode}
\cs_gnew_protected:Npn\KV_add_value_element:w\q_stop#1\q_nil{
  \toks_set:Nf\l_KV_currval_toks { 
    \KV_remove_surrounding_spaces_auxi:w \exp_not:N#1Q~Q
  }
  \toks_put_right:No\l_KV_parse_toks{
    \exp_after:wN {\toks_use:N \l_KV_currval_toks}
  }
}
%    \end{macrocode}
% When storing the key we firstly remove spaces plus the prepended
% |\q_no_value|.
%    \begin{macrocode}
\cs_gnew_protected:Npn\KV_set_key_element:w#1\q_nil{
  \toks_set:Nf\l_KV_currkey_toks
  {
    \exp_last_unbraced:NNo \KV_remove_surrounding_spaces_auxi:w
      \exp_not:N \use_none:n #1Q~Q
  }
%    \end{macrocode}
% Afterwards we gobble an extra level of braces if that's what we are
% asked to do.  
%    \begin{macrocode}
  \bool_if:NT \l_KV_remove_one_level_of_braces_bool 
  {
    \exp_args:NNo \toks_set:No \l_KV_currkey_toks {
      \exp_after:wN \KV_add_element_aux:w 
        \toks_use:N \l_KV_currkey_toks \q_nil
    }
  }
}
\group_end:
%    \end{macrocode}
% \end{macro}
% \end{macro}
% \end{macro}
% \end{macro}
% \end{macro}
% 
% \begin{macro}[aux]{\KV_add_element_aux:w}
%   A helper function for fixing braces around keys and values.
%    \begin{macrocode}
\cs_new:Npn \KV_add_element_aux:w#1\q_nil{#1}
%    \end{macrocode}
% \end{macro}
%
%
% Parse a list of keyvals, put them into list form with entries like
% |\KV_key_no_value_elt:n{key1}| and |\KV_key_value_elt:nn{key2}{val2}|.
%
% \begin{macro}[aux]{\KV_parse_sanitize_aux:n}
%   The slow parsing algorithm sanitizes active commas and equal signs
%   at the top level first. Then uses |#1| as inspector of each
%   element in the comma list.
%    \begin{macrocode}
\cs_new_protected:Npn \KV_parse_sanitize_aux:n #1 {
  \group_begin:
    \toks_clear:N \l_KV_parse_toks
    \tl_set:Nx \l_KV_tmpa_tl { \exp_not:n {#1} }
    \KV_sanitize_outerlevel_active_equals:N \l_KV_tmpa_tl
    \KV_sanitize_outerlevel_active_commas:N \l_KV_tmpa_tl
    \exp_last_unbraced:NNV \KV_parse_elt:w \q_no_value 
      \l_KV_tmpa_tl , \q_nil ,
%    \end{macrocode}
% We evaluate the parsed keys and values outside the group so the
% token register is restored to its previous value.
%    \begin{macrocode}
  \exp_last_unbraced:NV \group_end:
  \l_KV_parse_toks
}
%    \end{macrocode}
% \end{macro}
%
%
% \begin{macro}[aux]{\KV_parse_no_sanitize_aux:n}
%   Like above but we don't waste time sanitizing. This is probably
%   the one we will use for preamble parsing where catcodes of |=| and
%   |,| are as expected!
%    \begin{macrocode}
\cs_new_protected:Npn \KV_parse_no_sanitize_aux:n #1{
  \group_begin:
    \toks_clear:N \l_KV_parse_toks
    \KV_parse_elt:w \q_no_value #1 , \q_nil ,
  \exp_last_unbraced:NV \group_end:
  \l_KV_parse_toks
}
%    \end{macrocode}
% \end{macro}
%
%
% \begin{macro}[aux]{\KV_parse_elt:w}
%   This function will always have a |\q_no_value| stuffed in as the
%   rightmost token in |#1|. In case there was a blank entry in the
%   comma separated list we just run it again. The |\use_none:n| makes
%   sure to gobble the quark |\q_no_value|. A similar test is made to
%   check if we hit the end of the recursion.
%    \begin{macrocode}
\cs_set:Npn \KV_parse_elt:w #1,{
  \tl_if_blank:oTF{\use_none:n #1}
  { \KV_parse_elt:w \q_no_value }
  {
    \quark_if_nil:oF {\use_ii:nn #1 }
%    \end{macrocode}
% If we made it to here we can start parsing the key and value. When
% done try, try again.
%    \begin{macrocode}
    {
      \KV_split_key_value_current:w #1==\q_nil
      \KV_parse_elt:w \q_no_value
    }
  }
}
%    \end{macrocode}
% \end{macro}
%
%
% \begin{macro}{\KV_split_key_value_current:w}
% The function called to split the keys and values.
%    \begin{macrocode}
\cs_new:Npn \KV_split_key_value_current:w {\ERROR}
%    \end{macrocode}
% \end{macro}
%
% We provide two functions for splitting keys and values. The reason
% being that most of the time, we should probably be in the special
% coding regime where spaces are ignored. Hence it makes no sense to
% spend time searching for extra space tokens and we can do the
% settings directly. When comparing these two versions (neither doing
% any sanitizing) the |no_space_removal| version is more than 40\%
% faster than |space_removal|.
%
% It is up to functions like |\DeclareTemplate| to check which catcode
% regime is active and then pick up the version best suited for it.
%
%
%
% \begin{macro}{\KV_split_key_value_space_removal:w}
% \begin{macro}{\KV_split_key_value_space_removal_detect_error:wTF}
% \begin{macro}[aux]{\KV_split_key_value_space_removal_aux:w}
%   The code below removes extraneous spaces around the keys and
%   values plus one set of braces around the entire value.
%
%   Unlike the version to be used when spaces are ignored, this one
%   only grabs the key which is everything up to the first = and save
%   the rest for closer inspection. Reason is that if a user has
%   entered |mykey={{myval}},| then the outer braces have already been
%   removed before we even look at what might come after the key. So
%   this is slightly more tedious (but only slightly) but at least it
%   always removes only one level of braces.
%    \begin{macrocode}
\cs_new_protected:Npn \KV_split_key_value_space_removal:w #1 = #2\q_nil{
%    \end{macrocode}
%   First grab the key.
%    \begin{macrocode}
  \KV_set_key_element:w#1\q_nil
%    \end{macrocode}
% Then we start checking. If only a key was entered, |#2| contains
% |=| and nothing else, so we test for that first.
%    \begin{macrocode}
  \tl_set:Nx\l_KV_tmpa_tl{\exp_not:n{#2}}
  \tl_if_eq:NNTF\l_KV_tmpa_tl\c_KV_single_equal_sign_tl
%    \end{macrocode}
% Then we just insert the default key.
%    \begin{macrocode}
  {
    \toks_put_right:No\l_KV_parse_toks{
      \exp_after:wN \KV_key_no_value_elt:n 
      \exp_after:wN {\toks_use:N\l_KV_currkey_toks}
    }
  }
%    \end{macrocode}
% Otherwise we must take a closer look at what is left. The remainder
% of the original list up to the comma is now stored in |#2| plus an
% additional |==|, which wasn't gobbled during the initial reading of
% arguments.  If there is an error then we can see at least one more
% |=| so we call an auxiliary function to check for this.
%    \begin{macrocode}
  {
    \KV_split_key_value_space_removal_detect_error:wTF#2\q_no_value\q_nil
    {\KV_split_key_value_space_removal_aux:w \q_stop #2}
    { \msg_kernel_error:nn { keyval } { misplaced-equals-sign } }
  }
}
%    \end{macrocode}
% The error test.
%    \begin{macrocode}
\cs_new_protected:Npn
  \KV_split_key_value_space_removal_detect_error:wTF#1=#2#3\q_nil{
    \tl_if_head_eq_meaning:nNTF{#3}\q_no_value
}
%    \end{macrocode}
% Now we can start extracting the value. Recall that |#1| here starts
% with |\q_stop| so all braces are still there! First we try to see
% how much is left if we gobble three brace groups from |#1|. If |#1|
% is empty or blank, all three quarks are gobbled. If |#1| consists of
% exactly one token or brace group, only the latter quark is left.
%    \begin{macrocode}
\cs_new:Npn \KV_val_preserve_braces:NnN #1#2#3{{#2}}
\cs_new_protected:Npn\KV_split_key_value_space_removal_aux:w #1=={
  \tl_set:Nx\l_KV_tmpa_tl{\exp_not:o{\use_none:nnn#1\q_nil\q_nil}}
  \toks_put_right:No\l_KV_parse_toks{
    \exp_after:wN \KV_key_value_elt:nn 
    \exp_after:wN {\toks_use:N\l_KV_currkey_toks}
  }
%    \end{macrocode}
% If there a blank space or nothing at all, |\l_KV_tmpa_tl| is now
% completely empty.
%    \begin{macrocode}
    \tl_if_empty:NTF\l_KV_tmpa_tl
%    \end{macrocode}
% We just put an empty value on the stack.
%    \begin{macrocode}
  { \toks_put_right:Nn\l_KV_parse_toks{{}} }
  {
%    \end{macrocode}
% If there was exactly one brace group or token in |#1|,
% |\l_KV_tmpa_tl| is now equal to |\q_nil|. Then we can just pick it
% up as the second argument of |#1|. This will also take care of any
% spaces which might surround it.
%    \begin{macrocode}
    \quark_if_nil:NTF\l_KV_tmpa_tl
    {
      \bool_if:NTF \l_KV_remove_one_level_of_braces_bool
      {
        \toks_put_right:No\l_KV_parse_toks{
          \exp_after:wN{\use_ii:nnn #1\q_nil}
        } 
      }
      {
        \toks_put_right:No\l_KV_parse_toks{
          \exp_after:wN{\KV_val_preserve_braces:NnN #1\q_nil}
        }
      }
    }
%    \end{macrocode}
% Otherwise we grab the value.
%    \begin{macrocode}
    { \KV_add_value_element:w #1\q_nil }
  }
}
%    \end{macrocode}
% \end{macro}
% \end{macro}
% \end{macro}
%
% \begin{macro}{\KV_split_key_value_no_space_removal:w}
%   This version is for when in the special coding regime where spaces
%   are ignored so there is no need to do any fancy space hacks,
%   however fun they may be. Since there are no spaces, a set of
%   braces around a value is automatically stripped by \TeX.
%    \begin{macrocode}
\cs_new_protected:Npn \KV_split_key_value_no_space_removal:w #1#2=#3=#4\q_nil{
  \tl_set:Nn\l_KV_tmpa_tl{#4}
  \tl_if_empty:NTF \l_KV_tmpa_tl
  {
    \toks_put_right:Nn\l_KV_parse_toks{\KV_key_no_value_elt:n{#2}}
  }
  {
    \tl_if_eq:NNTF\c_KV_single_equal_sign_tl\l_KV_tmpa_tl
    {
      \toks_put_right:Nn\l_KV_parse_toks{\KV_key_value_elt:nn{#2}{#3}}
    }
    { \msg_kernel_error:nn { keyval } { misplaced-equals-sign } }
  }
}
%    \end{macrocode}
% \end{macro}
%
%
%
% \begin{macro}{\KV_key_no_value_elt:n}
% \begin{macro}{\KV_key_value_elt:nn}
%    \begin{macrocode}
\cs_new:Npn \KV_key_no_value_elt:n #1{\ERROR}
\cs_new:Npn \KV_key_value_elt:nn #1#2{\ERROR}
%    \end{macrocode}
% \end{macro}
% \end{macro}
%
%
% 
%
% \begin{macro}{\KV_parse_no_space_removal_no_sanitize:n}
%   Finally we can put all the things
%   together. |\KV_parse_no_space_removal_no_sanitize:n| is the
%   version that disallows unmatched conditional and does no space
%   removal.
%    \begin{macrocode}
\cs_new_protected_nopar:Npn \KV_parse_no_space_removal_no_sanitize:n {
  \cs_set_eq:NN \KV_split_key_value_current:w \KV_split_key_value_no_space_removal:w
  \KV_parse_no_sanitize_aux:n
}
%    \end{macrocode}
% \end{macro}
%
% \begin{macro}{\KV_parse_space_removal_sanitize:n}
% \begin{macro}{\KV_parse_space_removal_no_sanitize:n}
%   The other varieties can be defined in a similar manner. For the
%   version needed at the document level, we can use this one.
%    \begin{macrocode}
\cs_new_protected_nopar:Npn \KV_parse_space_removal_sanitize:n {
  \cs_set_eq:NN \KV_split_key_value_current:w \KV_split_key_value_space_removal:w
  \KV_parse_sanitize_aux:n
}
%    \end{macrocode}
% For preamble use by the non-programmer this is probably best.
%    \begin{macrocode}
\cs_new_protected_nopar:Npn \KV_parse_space_removal_no_sanitize:n {
  \cs_set_eq:NN \KV_split_key_value_current:w \KV_split_key_value_space_removal:w
  \KV_parse_no_sanitize_aux:n 
}
%    \end{macrocode}
% \end{macro}
% \end{macro}
%
%    \begin{macrocode}
\msg_kernel_new:nnnn { keyval } { misplaced-equals-sign }
  {Misplaced equals sign in key--value input \msg_line_context:}
  {
    I am trying to read some key--value input but found two equals
    signs\\%
    without a comma between them.%
}  
%    \end{macrocode}
%
%
%    \begin{macrocode}
%</initex|package>
%    \end{macrocode}
% 
%    \begin{macrocode}
%<*showmemory>
\showMemUsage
%</showmemory>
%    \end{macrocode}
%
% \end{implementation}
% \PrintIndex
%
% \endinput

% \iffalse
%% File: l3messages.dtx Copyright (C) 1990-2009 LaTeX3 project
%%
%% It may be distributed and/or modified under the conditions of the
%% LaTeX Project Public License (LPPL), either version 1.3c of this
%% license or (at your option) any later version.  The latest version
%% of this license is in the file
%%
%%    http://www.latex-project.org/lppl.txt
%%
%% This file is part of the ``expl3 bundle'' (The Work in LPPL)
%% and all files in that bundle must be distributed together.
%%
%% The released version of this bundle is available from CTAN.
%%
%% -----------------------------------------------------------------------
%%
%% The development version of the bundle can be found at
%%
%%    http://www.latex-project.org/cgi-bin/cvsweb.cgi/
%%
%% for those people who are interested.
%%
%%%%%%%%%%%
%% NOTE: %%
%%%%%%%%%%%
%%
%%   Snapshots taken from the repository represent work in progress and may
%%   not work or may contain conflicting material!  We therefore ask
%%   people _not_ to put them into distributions, archives, etc. without
%%   prior consultation with the LaTeX Project Team.
%%
%% -----------------------------------------------------------------------
%
%<*driver|package>
\RequirePackage{l3names}
%</driver|package>
%\fi
\GetIdInfo$Id$
          {L3 Experimental LaTeX Messages module}
%\iffalse
%<*driver>
%\fi
\ProvidesFile{\filename.\filenameext}
  [\filedate\space v\fileversion\space\filedescription]
%\iffalse
\documentclass[full]{l3doc}
\begin{document}
\DocInput{\filename.\filenameext}
\end{document}
%</driver>
% \fi
%
% \begin{documentation}
%
% \section{Communicating with the user}
%
%    Sometimes it is necesary to pass information back to the user
%    about what is going on. The information can be just that,
%    information, or it can be a warning that something might not
%    happen to his expectation. It could also be that something has
%    gone awry and that processing can't reliably continue without
%    some help from the user. In such a case an error is signalled.
%    When things are really bad, processing may have to stop as there
%    is no way to enter additional commands that put things right
%    again. In such a case we have a fatal error and the \LaTeX\ run
%    will be aborted.
%
% \subsection{Displaying the information}
%
%    First of all we need a couple of fairly low level functions that
%    deal with the job of passing the information to the user.
%
%    Real information is usually only written to the log file, while
%    warnings are displayed on the screen as well.
%  \begin{function}{%
%                     \err_info:nn |
%                     \err_warn:nn |
%                     \err_info_noline:nn |
%                     \err_warn_noline:nn
%                     }
%    \begin{syntax}
%      "\err_info:nn" \Arg{message} \Arg{continuation}
%    \end{syntax}
%    The <message> will be written to the log file. When it contains
%    the command "\err_newline:" a line break will occur and the new
%    line will start with the <continuation>. The function
%    "\err_warn:nn" writes the message to the terminal as well.
%  \end{function}
%    When an erroneous situation is encountered, a message is
%    displayed and the user is given the opportunity to enter some
%    additional code in an attempt to put things right. He may first
%    ask for some help, in which case some extra text will be
%    displayed to him.
%  \begin{function}{\err_interrupt:NNw}
%    \begin{syntax}
%      "\err_interrupt:NNw" <err id> <label> <more args>
%    \end{syntax}
%    This function signals a user error by searching the error file
%    denoted by <err id> for an error message associated with <label>,
%    i.e., specified by a corresponding "\err_interrupt_new:NNpnnn"
%    command. Depending on the arguments specified as
%    <param> when the error message was defined, further arguments are
%    read. Then the error message is displayed as explained in
%    "\err_interrupt_new:NNpnnn".
%  \end{function}
%
%    Finally, when something really serious occurs, \LaTeX\ will tell
%    the user about it and abort the run.
%  \begin{function}{\err_fatal:nn|\err_fatal_noline:nn}
%    \begin{syntax}
%      "\err_fatal:nn" \Arg{message} \Arg{continuation}
%    \end{syntax}
%    Just displays the <message> and then aborts the \LaTeX\ run.
%  \end{function}
%
%  \begin{function}{\err_newline:}
%    \begin{syntax}
%      "\err_newline:"
%    \end{syntax}
%    Is used to break an informational, warning or error message up
%    into multiple lines. May be defined in such a way that the new
%    line starts with a standard <continuation>. A normal line break in such
%    messages  can be achieved with "\iow_newline:" from the l3iow module.
%  \end{function}
%
%
% \begin{function}{\text_put_sp: | \text_put_four_sp:}
% \begin{syntax}
% "\text_put_sp:"
% \end{syntax}
% Is used to insert a literal space into an info/warning/error message.
% Necessary after control sequences and for otherwise padding when regular
% spaces are being gobbled or collapsed.
% \end{function}
% 
%
% \subsection{Storing the information}
%
%    The informational and warning messages are usually short and can
%    be stored as part of a macro; but error messages need to be more
%    verbose. Therefor error messages are stored in external files
%    which are read and searched for the correct error message at the
%    time of the error. In this way it is possible to write extensive
%    help texts without cluttering \TeX{}'s main memory. 
%
% \subsubsection{Dealing with the error file}
%
%  \begin{function}{\err_file_new:Nn}
%    \begin{syntax}
%      "\err_file_new:Nn" <err id> \Arg{err file name}
%    \end{syntax}
%    Opens a new error file to write errors to. <err id> is a unique
%    identifier for the external <err file name>. By convention <err
%    id> is  declared as a constant (i.e., starts with "\c_") und ends
%    with "_tl". If this command is issued while some other error
%    file is open we get an internal error message.
%  \end{function}
%
%  \begin{function}{\err_file_close:N}
%    \begin{syntax}
%      "\err_file_close:N" <err id>
%    \end{syntax}
%    Closes the currently open error file and checks that it matches
%    <err id>, i.e., that everything is alright in the code.
%  \end{function}
%
% \subsubsection{Declaring an error message in the error file}
%  \begin{function}{\err_interrupt_new:NNpnnn}
%    \begin{syntax}
%      "\err_interrupt_new:NNpnnn"  <err id> <label> <param>
%                               \Arg{short msg}
%                               \Arg{long msg}
%                               \Arg{recovery code}
%    \end{syntax}
%    This function declares an new error message which can be
%    addressed via "\err_interrupt:NNw". The pair (<err id>, <label>)
%    has to be unique where <label> can be some otherwise arbitrary
%    token (usually the function name in which the error routine is
%    called. Actually, the pair (<err id>, expansion of <label>) has
%    to be unique since for reasons of speed, tests are carried out
%    using "\if_meaning:w".
%
%    <param> specifies the extra arguments that will be
%    supplied to the error routine when "\err_interrupt:NNw" is
%    called. These arguments can be used within <short msg>, <long
%    msg>, and/or <recovery code> to provide further information to
%    the user. They are denoted with "#1", "#2", etc.\ within these
%    arguments.
%
%    The <short msg> is displayed directly on the terminal if the
%    error occurs, <long msg> is displayed when the user types "h" in
%    response to the error prompt of \TeX{}, and <recovery code> is
%    executed afterwards.  This means that <recovery code> is inserted
%    after any deletions or insertions given by the user. All three
%    arguments are expanded while they are written to the error file,
%    therefore one has to prevent expansion of tokens with
%    "\token_to_str:N" that should be expanded when the error is
%    triggered.
%  \end{function}
%
% \subsection{Internal functions}
%
%  \begin{function}{\err_display_aux:w}
%    This function is constructed on the fly while reading the error file.
%    It grabs following arguments from the code (if any) and then displays
%    the error message and inserts the <recovery code>.
%  \end{function}
%
%  \begin{function}{\err_msgline_aux:Npnnn}
%    \begin{syntax}
%   "\err_msgline_aux:Npnnn"  <label> <param> \Arg{short}
%                             \Arg{long msg} \Arg{recovery code}
%   \end{syntax}
%    Function written in front of every error message on the error file. It
%    will be executed when the error file is read back in comparing <label>
%    to "\l_err_label_token". If they are the same, "\err_display_aux:w"
%    will be defined and the reading process will stop.
%  \end{function}
%
%  \begin{function}{\err_message:x}
%    \begin{syntax}
%      "\err_message:x" \Arg{error message}
%    \end{syntax}
%    Function that directly triggers \TeX{}'s error handler. It should
%    not be used directly. 
%    \begin{texnote}
%    This is the \LaTeX3 name for the \tn{errormessage} primitive.
%    \end{texnote}
%  \end{function}
%
% \begin{function}{\io_show_file_lineno:}
%    A function to add the number of the line and the name of the file
%    to a message as an indication of where the message was triggered. 
% \end{function}
% 
%
% \subsection{Kernel specific functions}
%
%    For a number of the functions described above specific variants
%    are provided that are used in the kernel of \LaTeX3.
%
%    \begin{function}{%
%                     \err_kernel_info:n |
%                     \err_kernel_warn:n |
%                     \err_kernel_fatal:n |
%                     \err_kernel_info_noline:n |
%                     \err_kernel_warn_noline:n |
%                     \err_kernel_fatal_noline:n
%                     }
%    \begin{syntax}
%      "\err_kernel_info:n" \Arg{message}
%    \end{syntax}
%    \end{function}
%  \begin{function}{%
%                  \err_kernel_interrupt:Nw |
%                  \err_kernel_interrupt_new:NNnnn |
%    }
%    Abbrivations for writing and accessing kernel error messages that
%    go to the error file "\c_kernel_err_tl". 
%  \end{function}
%
%  \begin{function}{\err_latex_bug:x}
%    \begin{syntax}
%      "\err_latex_bug:x" \Arg{error message}
%    \end{syntax}
%    Creates an internal error message. This is intended to be used in
%    places that should not be reached in normal operation. Something is
%    wrong with the code. 
%  \end{function}
%
% \subsection{Variables and constants}
%
%    \begin{variable}{\c_iow_err_stream}
%    Output stream used to access the error files during their
%    generation. 
%    \end{variable}
%
%    \begin{variable}{\c_kernel_err_tl}
%    Identifier denoting the kernel error file. (Its contents is the
%    name of the external file.)
%    \end{variable}
%
%    \begin{variable}{\g_err_curr_fname}
%    Global variable containing the name of the currently open error
%    file. Empty when no such file is open for writing.
%    \end{variable}
%
%    \begin{variable}{\tex_errorcontextlines:D}
%    Variable determining the amount of macro expansion contents shown
%    to the user when an error is triggered. \LaTeX3 sets this to -1
%    since to the average user this contents is of no interest.
%    \begin{texnote}
%    This is the \LaTeX3 name for the \TeX3 primitive
%    \tn{errorcontextlines}.
%    \end{texnote}
%    \end{variable}
%
%    \begin{variable}{\g_err_help_toks}
%    Token register that holds the message that will be shown if the
%    user types "h" in response to an error message that was produced
%    by "\err_message:x". 
%    \begin{texnote}
%    This is the \LaTeX3 name for the \TeX\ primitive \tn{errhelp}.
%    \end{texnote}
%    \end{variable}
%
%    \begin{variable}{\l_err_label_token}
%    Variable holding the <label> to look up in an error file.
%    \end{variable}
%
%    \begin{variable}{\g_file_curr_name_tl}
%    This variable is used to store the name of the file currently
%    being processed. 
%    \end{variable}
%
% \end{documentation}
%
% \begin{implementation}
%
% \section{\pkg{l3messages} implementation}
%
%    \begin{macrocode}
%<*package>
\ProvidesExplPackage
  {\filename}{\filedate}{\fileversion}{\filedescription}
\package_check_loaded_expl:
%</package>
%<*initex|package>
%    \end{macrocode}
%
% \subsubsection{Code to be moved to other modules}
%
% \begin{macro}{\g_file_curr_name_tl}
%    This variable is used to store the name of the file currently
%    being processed. It should be part of the code that defines the
%    higher level I/O commands.
%    \begin{macrocode}
\tl_new:Nn \g_file_curr_name_tl {no~file}
%    \end{macrocode}
% \end{macro}
%
% \begin{macro}{\err_message:x}
%    The \LaTeX3 name for a \TeX\ primitive. This should perhaps move
%    to \texttt{l3names.dtx}.
%    \begin{macrocode}
\cs_new_eq:NN \err_message:x \tex_errmessage:D
%    \end{macrocode}
% \end{macro}
%
% \begin{macro}{\text_put_sp:}
% \begin{macro}{\text_put_four_sp:}
%    We need these functions for certain error and warning messages
%    right away. They put one and four spaces into the message stream.
%    \begin{macrocode}
\cs_new_nopar:Npn \text_put_sp: {~}
\cs_new_nopar:Npn \text_put_four_sp: { \text_put_sp: \text_put_sp: 
                                  \text_put_sp: \text_put_sp: }
%    \end{macrocode}
% \end{macro}
% \end{macro}
%
%  \begin{macro}{\io_show_file_lineno:}
%    A function to add the number of the line and the name of the file
%    to a message as an indication of where the message was triggered. 
%    \begin{macrocode}
\cs_new_nopar:Npn \io_show_file_lineno:{
  on~line~\toks_use:N\tex_inputlineno:D\text_put_sp:~
  of~file~\g_file_curr_name_tl}
%    \end{macrocode}
%  \end{macro}
%
% \subsubsection{Variables and constants}
%
%  \begin{macro}{\g_err_help_toks}
%    A token register to store the help text for an error message in.
%    \begin{macrocode}
\cs_set_eq:NwN \g_err_help_toks \tex_errhelp:D
%    \end{macrocode}
%  \end{macro}
%
% \begin{macro}{\l_err_label_token}
%    This will hold the current error label.
%    \begin{macrocode}
\cs_new_nopar:Npn \l_err_label_token {}
%    \end{macrocode}
% \end{macro}
%
% \begin{macro}{\tex_errorcontextlines:D}
%    Since we are producing our own error and help messages we can
%    turn off the nasty stack information coming from \TeX{}'s
%    stomach.
%    \begin{macrocode}
\int_set:Nn\tex_errorcontextlines:D\c_minus_one
%    \end{macrocode}
% \end{macro}
%
%  \subsubsection{Displaying the information}
%
%    Here we define the fairly low level commands needed to
%    communicate with the user.
%
%  \begin{macro}{\err_info:nn}
%  \begin{macro}{\err_warn:nn}
%    Write a message to the log file ("\err_info:nn") or to both the
%    log file and the terminal ("\err_warn:nn").
%    \begin{macrocode}
\cs_new_nopar:Npn \err_info:nn #1#2{
%    \end{macrocode}
%    Make sure that the \emph{continuation} is part of "\err_newline:".
%    \begin{macrocode}
  \cs_set_nopar:Npn\err_newline:{\iow_newline:#2}
%    \end{macrocode}
%    Then write the message.
%    \begin{macrocode}
  \iow_log:x {#1~\io_show_file_lineno:}}
\cs_new_nopar:Npn \err_warn:nn #1#2{
  \cs_set_nopar:Npn\err_newline:{\iow_newline:#2}
  \iow_term:x {#1~\io_show_file_lineno:}}
%    \end{macrocode}
%  \end{macro}
%  \end{macro}
%
%  \begin{macro}{\err_info_noline:nn}
%  \begin{macro}{\err_warn_noline:nn}
%    These variants of the above two functions don't add the
%    linenumber to the message.
%    \begin{macrocode}
\cs_new_nopar:Npn \err_info_noline:nn #1#2{
  \cs_set_nopar:Npn\err_newline:{\iow_newline:#2}
  \iow_log:x {#1}}
\cs_new_nopar:Npn \err_warn_noline:nn #1#2{
  \cs_set_nopar:Npn\err_newline:{\iow_newline:#2}
  \iow_term:x {#1}}
%    \end{macrocode}
%  \end{macro}
%  \end{macro}
%
% \begin{macro}{\err_interrupt:NNw}
%    The function that is called when some error
%    occurs in the code. It takes at least two arguments, the
%    \m{errfile} which is a token list that holds the name of the file
%    where the error message should be fetched from, and the label to
%    identify the error message in the error file. However, it may
%    have additional arguments that are picked up by the error handler
%    extracted from the error file. This is specified in the third
%    argument to |\err_interrupt_new:NNpnnn|.
%    \begin{macrocode}
\cs_new_nopar:Npn \err_interrupt:NNw #1#2{
  \cs_set_eq:NN \l_err_label_token #2
  \group_begin:
%    \end{macrocode}
%    For some reason we get some |\par|s into the file if we use the
%    current definition of |\iow_now_buffer_safe:Nn| to write the messages.
%    This is probably a consequence of using token registers to
%    prohibit the expansion of code.
%    \begin{macrocode}
    \cs_set_eq:NN \par \prg_do_nothing:
%    \end{macrocode}
%    We want to ensure that we are not in programmer's mode (no spaces)
%    but we want to switch on internal naming conventions.
%    \begin{macrocode}
    \ExplSyntaxOff
    \ExplSyntaxNamesOn
%    \end{macrocode}
%    We better clear all short references that are active, otherwise
%    we may get surprising results.
%    \begin{macrocode}
    %\clearshortrefmaps
    \tex_input:D #1~\err_display_aux:w}
%    \end{macrocode}
% \end{macro}
%
% \begin{macro}{\err_fatal:nn}
% \begin{macro}{\err_fatal_noline:nn}
%    Write a message to the log file and to the terminal.
%    \begin{macrocode}
\cs_new_nopar:Npn \err_fatal:nn #1#2{
%    \end{macrocode}
%    Make sure that the \emph{continuation} is part of "\err_newline".
%    \begin{macrocode}
  \cs_set_nopar:Npn\err_newline:{\iow_newline:#2}
%    \end{macrocode}
%    Then write the message.
%    \begin{macrocode}
  \iow_term:x {#1~\io_show_file_lineno:}
%    \end{macrocode}
%    Finally abort the \LaTeX{} run.
%    \begin{macrocode}
  \tex_end:D
  }
%    \end{macrocode}
%    A variant that doesn't include the line number where the error
%    occured.
%    \begin{macrocode}
\cs_new_nopar:Npn \err_fatal_noline:nn #1#2{
  \cs_set_nopar:Npn\err_newline:{\iow_newline:#2}
  \iow_term:x {#1}
  \tex_end:D
  }
%    \end{macrocode}
% \end{macro}
% \end{macro}
%
% \begin{macro}{\err_newline:}
%    |\err_newline:| is used to introduce a new line in an error
%    message. I would like to use |\\| but this would mean
%    redefinition which should be avoided to make the error message
%    the last action before control is given to the user (otherwise
%    something like |\group_end:| would interfere with
%    insertions/deletions by the user). "\err_newline:" will be
%    redefined by the various functions displaying messages to include
%    the correct \emph{continuation}.
%    \begin{macrocode}
\cs_new_nopar:Npn \err_newline: {^^J}
%    \end{macrocode}
% \end{macro}
%
% \subsubsection{Dealing with the error file}
%
% This section contains code that combines Michaels original thoughts
% on the the subject with Denys' further ideas.
%
% \begin{macro}{\c_iow_err_stream}
%    Error messages are logged using the output stream
%    |\c_iow_err_stream|.
%    \begin{macrocode}
\iow_new:N \c_iow_err_stream
%    \end{macrocode}
% \end{macro}
%
% \begin{macro}{\g_err_curr_fname}
%    A nick name for the currently open error file. It is empty
%    if no error file is currently open.
%    \begin{macrocode}
\tl_new:Nn \g_err_curr_fname{}
%    \end{macrocode}
% \end{macro}
%
% \begin{macro}{\err_file_new:Nn}
%    This function defines a new error file. The first argument is a
%    token list which should hold the name of the error file, the second
%    argument is the name of the error file. The token list should be
%    a constant defined by this function.
%    \begin{macrocode}
\cs_new_nopar:Npn \err_file_new:Nn #1#2{
   \tl_if_empty:NF\g_err_curr_fname
      {\err_latex_bug:x{Unclosed~error~file~`\g_err_curr_fname'}}
   \iow_open:Nn \c_iow_err_stream {#2}
   \err_kernel_info:n{Errorfile~`#2'~opened~for~output}
   \tl_gset:Nn \g_err_curr_fname{#2}
   \tl_new:Nn #1{#2}}
%    \end{macrocode}
% \end{macro}
%
% \begin{macro}{\err_file_close:N}
%    This function closes the current error file.
%    \begin{macrocode}
\cs_new_nopar:Npn \err_file_close:N#1{
%    \end{macrocode}
%    Before we close the stream, we write out a final error handler
%    that catches mismatch within error message labels and their
%    calls. Actually this should be integrated into
%    |\err_file_new:Nn|, too.
%    \begin{macrocode}
  \tl_if_eq:NNF#1\g_err_curr_fname
     {\err_latex_bug:x{You~closed~the~wrong~error~file~`#1'.~
         Open~is~`\g_err_curr_fname'.}}
  \iow_now_buffer_safe:Nn \c_iow_err_stream {\err_latex_bug:x{Didn't~find~the~
        correct~error~message~to~show.\iow_newline:
        Was~searching~for~a~function~
        with~the~following~meaning:\iow_newline:
          \token_to_str:N\token_to_meaning:N
          \token_to_str:N\l_err_label_token}
%    \end{macrocode}
%    The |\group_end:| here matches the one from |\err_interrupt:NNw| that
%    is used to hide changes to |\par| etc.
%    \begin{macrocode}
      \group_end:}
  \iow_close:N \c_iow_err_stream
  \err_kernel_info:n{Errorfile~`\g_err_curr_fname'~closed}
  \tl_gset_eq:NN\g_err_curr_fname\c_empty_tl
}
%    \end{macrocode}
% \end{macro}
%
% \subsubsection{Declaring an error message in the error file}
%
% \begin{macro}{\err_interrupt_new:NNpnnn}
%    This function declares a new error message.
%    |\err_interrupt_new:NNpnnn| \m{errfile} \m{errlabel} \m{param}
%    \Arg{errmsg} \Arg{helpmsg} \Arg{code}. That error message is fetched
%    from the error file \m{errfile}. The label to search for is
%    \m{errlabel}, the error handler has \m{param} arguments
%    (actually \m{param} + 3 since the \m{errmsg}, \m{helpmsg} and
%    \m{code} are also arguments), and \m{errmsg} is the message to
%    display, \m{helpmsg} is the message that is displayed when the
%    user enters |h|, while \m{code} is extra code to perform when the
%    error occurs. \m{code} is perhaps not necessary, we will see.
%
%    We have to check that the label associated with the error message
%    is unique. This means that its replacement text (labels are
%    simply arbitrary functions) is different from the replacement
%    text of any other label in the same error set.
%    \begin{macrocode}
\cs_new_nopar:Npn \err_interrupt_new:NNpnnn #1{
%    \end{macrocode}
%    Both \m{errmsg} and \m{code} might contain hashmarks denoting
%    arguments to the error handler.
%    \begin{macrocode}
  \group_begin:
    \char_set_catcode:nn{`\#}{12}
%    \end{macrocode}
%    We also have to check that output goes to the correct error file.
%    \begin{macrocode}
    \if_meaning:w#1\g_err_curr_fname
    \else:
      \err_latex_bug:x{
        Error~text~goes~to~wrong~err~file:~
          `\g_err_curr_fname'~is~open~but~you~requested~
          `#1'}
    \fi:
    \err_interrupt_new_aux:Np}
%    \end{macrocode}
%    \begin{macrocode}
\cs_new:Npn \err_interrupt_new_aux:Np #1#2# {
  \err_interrupt_new_aux:w #1 {#2}
}
\cs_new:Npn \err_interrupt_new_aux:w #1#2#3#4#5{
    \iow_now_buffer_safe:Nn \c_iow_err_stream
        {\err_msgline_aux:Npnnn #1{#2}{#3}{#4}{#5}\prg_do_nothing:}
  \group_end:}
%    \end{macrocode}
% \end{macro}
%
% \begin{macro}{\err_msgline_aux:Npnnn}
%    This function is executed when an error file is read back by
%    |\err_interrupt:NNw|.  It compares its first argument against
%    |\l_err_label_token| using |\if_meaning:w| and if this fails the
%    function does nothing; otherwise it defines |\err_display_aux:w| in
%    a way that it will pick up the arguments (if any) from the code and
%    generates a suitable error message.
%    \begin{macrocode}
\cs_new_nopar:Npn \err_msgline_aux:Npnnn #1#2#3#4#5{
  \if_meaning:w #1 \l_err_label_token
%    \end{macrocode}
%    At the moment we simply use the old \LaTeX{} error code and
%    |\renewcommand| to generate the error handler.  After displaying
%    the error message we insert error code this can be manipulated by
%    the user with the deletion/insertion facility of \TeX{}'s error
%    mechanism.
%
%    The |\group_end:| at the very beginning matches the
%    |\group_begin:| when the file starts.
%    \begin{macrocode}
        \cs_set_nopar:Npn\err_display_aux:w #2 {
           \group_end:
           \toks_gset:Nx\g_err_help_toks{#4}
           \iow_term:x{LaTeX~error~\io_show_file_lineno:.\iow_newline:
               \text_put_sp:\text_put_four_sp: \text_put_sp:
               See~LaTeX~manual~for~explanation.\iow_newline:
               \text_put_sp:\text_put_four_sp: \text_put_sp:
               Type~\text_put_sp: H~<return>~\text_put_sp: for~
               immediate~help.}
           \err_message:x{#3}
           #5
        }
        \tex_endinput:D
   \fi:
}
%    \end{macrocode}
% \end{macro}
%
%
% \begin{macro}{\err_display_aux:w}
%    We should make sure that this function is definable.
%    \begin{macrocode}
\cs_new_nopar:Npn \err_display_aux:w {}
%    \end{macrocode}
% \end{macro}
%
% \subsubsection{Kernel specific functions}
%
% \begin{macro}{\err_latex_bug:x}
%    This will show internal errors. Appears in \pkg{l3basics}
%    but we could probably get away with moving it here.
%    \begin{macrocode}
%*<bootstrap>
\cs_set_nopar:Npn \err_latex_bug:x #1 {
   \iow_term:x{This~is~a~LaTeX~bug!~Check~coding!}\tex_errmessage:D{#1}}
%/<bootstrap>
%    \end{macrocode}
% \end{macro}
%
% \begin{macro}{\err_kernel_interrupt:Nw}
%    |\err_kernel_interrupt:Nw | is just the abbreviation to read from
%    the standard system error file.
%    \begin{macrocode}
\cs_new_nopar:Npn \err_kernel_interrupt:Nw {\err_interrupt:NNw \c_kernel_err_tl}
%    \end{macrocode}
% \end{macro}
%
% \begin{macro}{\err_kernel_interrupt_new:NNnnn}
%    To ease the coding in case of system messages that should all go
%    to one and the same error file (if!) we also have the following
%    function.
%    \begin{macrocode}
\cs_new_nopar:Npn \err_kernel_interrupt_new:NNnnn {
        \err_interrupt_new:NNpnnn \c_kernel_err_tl}
%    \end{macrocode}
% \end{macro}
%
%  \begin{macro}{\err_kernel_info:n}
%  \begin{macro}{\err_kernel_warn:n}
%  \begin{macro}{\err_kernel_fatal:n}
%  \begin{macro}{\err_kernel_info_noline:n}
%  \begin{macro}{\err_kernel_warn_noline:n}
%  \begin{macro}{\err_kernel_fatal_noline:n}
%    These variants are specific for the \LaTeX\ kernel.
%    \begin{macrocode}
\cs_new_nopar:Npn \err_kernel_info:n #1 {
  \err_info:nn {LaTeX~Info:~#1}
               {\text_put_four_sp:\text_put_four_sp:\text_put_four_sp:}
  }
\cs_new_nopar:Npn \err_kernel_warn:n #1 {
  \err_warn:nn {LaTeX~Warning:~#1}
               {\text_put_sp:\text_put_sp:\text_put_sp:
                \text_put_four_sp:\text_put_four_sp:\text_put_four_sp:}
  }
\cs_new_nopar:Npn \err_kernel_fatal:n #1 {
  \err_fatal:nn {LaTeX~Fatal:~#1}
                {\text_put_sp:
                 \text_put_four_sp:\text_put_four_sp:\text_put_four_sp:}
  }
\cs_new_nopar:Npn \err_kernel_info_noline:n #1 {
  \err_info_noline:nn {LaTeX~Info:~#1}
               {\text_put_four_sp:\text_put_four_sp:\text_put_four_sp:}
  }
\cs_new_nopar:Npn \err_kernel_warn_noline:n #1 {
  \err_warn_noline:nn {LaTeX~Warning:~#1}
               {\text_put_sp:\text_put_sp:\text_put_sp:
                \text_put_four_sp:\text_put_four_sp:\text_put_four_sp:}
  }
\cs_new_nopar:Npn \err_kernel_fatal_noline:n #1 {
  \err_fatal_noline:nn {LaTeX~Fatal:~#1}
                {\text_put_sp:
                 \text_put_four_sp:\text_put_four_sp:\text_put_four_sp:}
  }
%</initex|package>
%    \end{macrocode}
%    At a later stage variants may be provided for what in \LaTeXe{}
%    used to be called document classes and packages.
%  \end{macro}
%  \end{macro}
%  \end{macro}
%  \end{macro}
%  \end{macro}
%  \end{macro}
%
% \begin{macro}{\c_kernel_err_tl}
%    Most error messages will go to the system error file; it's name is
%    stored in |\c_kernel_err_tl|.
%    \begin{macrocode}
%<initex>\err_file_new:Nn \c_kernel_err_tl {ltxkernel.err}
%<package>\err_file_new:Nn \c_kernel_err_tl {l3in2e.err}
%    \end{macrocode}
% \end{macro}
%
%    The code below is a temporary implementation of a few of
%    \LaTeX209 error messages with the new syntax. They are only
%    included in the package as the \LaTeX3 kernel will certainly have
%    it's own error message definitions that differ from \LaTeXe's way
%    of signalling errors. These primarily serve as an example on how
%    to use this concept of dealing with errors.
%
%    First we declare a couple of helper macros that contain texts
%    that are used frequently throughout \LaTeX.
%    \begin{macrocode}
%<*package>
\cs_set_nopar:Npn\err_help_ignored: {
  Your~command~was~ignored.\iow_newline:
  Type \text_put_sp: I~<command>~<return>
  \text_put_sp: to~replace~it~with~another~command,\iow_newline:
  or~\text_put_sp: <return> \text_put_sp: to~continue~without~it.}

\cs_set_nopar:Npn\err_help_textlost: {
  You've~lost~some~text.\text_put_sp: \err_help_return_or_X:}

\cs_set_nopar:Npn\err_help_return_or_X: {
  Try~typing\text_put_sp: <return>
  \text_put_sp: to~proceed.\iow_newline:
  If~that~doesn't~work,~type
  \text_put_sp: X~<return>\text_put_sp: to~quit.}

\cs_set_nopar:Npn\err_help_trouble: {
  You're~ in~ trouble~ here.
  \text_put_sp:\err_help_return_or_X:}
%    \end{macrocode}
%
%    Below are the definitions of the complete messages
%    \begin{macrocode}

\err_kernel_interrupt_new:NNnnn\cs_if_free_p:N{1}
   {Command~name~`\tex_string:D#1'~already~used}
   {You~tried~to~define~a~command~which~already~has~
     a~meaning.\iow_newline:
     If~you~really~want~to~redefine~it~try~
     \token_to_str:N\cs_set:Nn\text_put_sp:
     or~similar~next~time.\iow_newline:
     For~this~run~I~will~ignore~your~definition.}
   {}

\err_kernel_interrupt_new:NNnnn\newline{0}
   {There's~no~line~here~to~end}
   {You~tried~to~end~a~line~at~a~place~where~I~thought~
     we~were~already~between~paragraphs.}
   {}

\err_kernel_interrupt_new:NNnnn\newcnt{0}
   {No~such~counter}
   {The~counter~name~mentioned~in~the~operation~is~not~
     known~to~me.\iow_newline:
     Check~the~spelling.}
   {}

\err_kernel_interrupt_new:NNnnn\nodocument{0}
    {Missing~\token_to_str:N\begin{document}}
    {\err_help_trouble:}
    {}

\err_kernel_interrupt_new:NNnnn\badmath{0}
    {Bad~math~environment~delimiter}
    {\err_help_ignored:}
    {}

\err_kernel_interrupt_new:NNnnn\toodeep{0}
    {Too~deeply~nested}
    {\err_help_trouble:}
    {}

\err_kernel_interrupt_new:NNnnn\badpoptabs{0}
    {\token_to_str:N\pushtabs \text_put_sp: 
      and~\token_to_str:N\poptabs
      \text_put_sp: don't~match}
    {\err_help_trouble:}
    {}

\err_kernel_interrupt_new:NNnnn\badtab{0}
    {Undefined~tab~position}
    {\err_help_trouble:}
    {}

\err_kernel_interrupt_new:NNnnn\preamerr{}
    {\if_case:w #1~Illegal~character\or:
      Missing~@-exp\or: Missing~p-arg\fi:\text_put_sp:
      in~array~arg}
    {\err_help_trouble:}
    {}

\err_kernel_interrupt_new:NNnnn\badlinearg{}
    {Bad~\token_to_str:N\line
      \text_put_sp: or~\token_to_str:N\vector
      \text_put_sp: argument}
    {\err_help_textlost:}
    {}

\err_kernel_interrupt_new:NNnnn\parmoderr{0}
    {Not~in~outer~par~mode}
    {\err_help_textlost:}
    {}

\err_kernel_interrupt_new:NNnnn\fltovf{0}
    {Too~many~unprocessed~floats}
    {\err_help_textlost:}
    {}

\err_kernel_interrupt_new:NNnnn\badcrerr{0}
    {Bad~use~of~\token_to_str:N\\}
    {\err_help_return_or_X:}
    {}

\err_kernel_interrupt_new:NNnnn\noitemerr{0}
    {Something's~wrong--perhaps~a~missing~
      \token_to_str:N\item}
    {\err_help_return_or_X:}
    {}

\err_kernel_interrupt_new:NNnnn\notprerr{0}
    {Can~be~used~only~in~preamble}
    {\err_help_ignored:}
    {}

\err_file_close:N\c_kernel_err_tl
%<package>
%    \end{macrocode}
%
%
%    Show token usage:
%    \begin{macrocode}
%<*showmemory>
\showMemUsage
%</showmemory>
%    \end{macrocode}
%
% \end{implementation}
% \PrintIndex
%
% \endinput

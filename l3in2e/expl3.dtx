% \iffalse
%% File: expl3.dtx Copyright (C) 1990-2008 LaTeX3 project
%%
%% It may be distributed and/or modified under the conditions of the
%% LaTeX Project Public License (LPPL), either version 1.3c of this
%% license or (at your option) any later version.  The latest version
%% of this license is in the file
%%
%%    http://www.latex-project.org/lppl.txt
%%
%% This file is part of the ``expl3 bundle'' (The Work in LPPL)
%% and all files in that bundle must be distributed together.
%%
%% The released version of this bundle is available from CTAN.
%%
%% -----------------------------------------------------------------------
%%
%% The development version of the bundle can be found at
%%
%%    http://www.latex-project.org/cgi-bin/cvsweb.cgi/
%%
%% for those people who are interested.
%%
%%%%%%%%%%%
%% NOTE: %%
%%%%%%%%%%%
%%
%%   Snapshots taken from the repository represent work in progress and may
%%   not work or may contain conflicting material!  We therefore ask
%%   people _not_ to put them into distributions, archives, etc. without
%%   prior consultation with the LaTeX Project Team.
%%
%% -----------------------------------------------------------------------
%
%<*driver|package>
\RequirePackage{l3names}
%</driver|package>
%\fi
\GetIdInfo$Id$
       {L3 Experimental code bundle wrapper}%
%\iffalse
%<*driver>
%\fi
\ProvidesFile{\filename.\filenameext}
  [\filedate\space v\fileversion\space\filedescription]
%\iffalse
\documentclass{l3doc}
\usepackage{shortvrb,hypdoc}
\MakeShortVerb{\|}
\hyphenation{para-meters para-meter}
% I have found at least 3 other hyphenations of this in various refs,
% but I did also find this, which is my personal favourite ---chris
\hyphenation{ignore ignored ignores}
\begin{document}
\DocInput{\filename.\filenameext}
\end{document}
%</driver>
% \fi
% 
% \providecommand\m[1]{$\langle$\textit{#1}$\rangle$}
% \providecommand\acro[1]{\textsc{\MakeLowercase{#1}}}
% \providecommand\ie{i.e.,~}
% \providecommand\eg{e.g.,~}
% \providecommand\pkg[1]{\textsf{#1}}
% \providecommand\cmd[1]{\texttt{\string#1}}
% \newenvironment{arg-description}{%
%   \begin{itemize}\def\makelabel##1{\hss\llap{\bfseries##1}}}{\end{itemize}}
% 
% \title{The \textsf{expl3} package and philosophy\thanks{This file
%         has version number \fileversion, last
%         revised \filedate.}}
% \author{\Team}
% \date{\filedate}
%
% \maketitle
% \begin{abstract}
% 
% This paper gives a brief introduction to a new set of programming
% conventions that have been designed to meet the requirements of
% implementing large scale \TeX\ macro programming projects such as
% \LaTeX.
% 
% The main features of the system described are:
% \begin{itemize}
% \item classification of the macros (or, in \LaTeX{} terminology,
%   commands) into \LaTeX{} functions and \LaTeX{} parameters, and also
%   into modules containing related commands;
% \item  a systematic naming scheme based on these
%   classifications;
% \item  a simple mechanism for controlling the expansion of a function's
% arguments.
% \end{itemize}
% This system is being used as the basis for 
% \TeX{} programming within the \LaTeX3 project.
% Note that the language is not intended for either
% document mark-up or style specification.
%
% An earlier description of the \LaTeX3 programming language is
% available from \url{http://www.latex-project.org/papers}.
% \end{abstract}
% 
% 
% \section{Introduction}
% 
% This paper describes the conventions for a \TeX-based programming
% language which is intended to provide a more consistent and rational
% environment for the construction of large scale systems, such as
% \LaTeX, using \TeX{} macros.
% 
% Variants of this language have been in use by The \LaTeX3 Project Team
% since around 1990 but the syntax specification to be outlined here
% should \emph{not} be considered final. This is an experimental
% language thus many aspects, such as the syntax conventions and naming
% schemes, may (and probably will) change as more experience is gained
% with using the language in practice.
% 
% The next section shows where this language fits into a complete
% \TeX-based document processing system.  We then describe the major
% features of the syntactic structure of command names, including the
% argument specification syntax used in function names.
% 
% The practical ideas behind this argument syntax will be explained,
% together with the semantics of the expansion control mechanism and the
% interface used to define variant forms of functions.  The paper also
% discusses some advantages of the syntax for parameter names.
% 
% As we shall demonstrate, the use of a structured naming scheme and of
% variant forms for functions greatly improves the readability of the
% code and hence also its reliability.  Moreover, experience has shown
% that the longer command names which result from the new syntax do not
% make the process of \emph{writing} code significantly harder
% (especially when using a reasonably intelligent editor).
% 
% The final section gives some details of our plans to distribute parts
% of this system. Although the paper is now a little dated, more general 
% information concerning the work of the \LaTeX3 Project can be found 
% in Ref.~\cite{tub:MR97-1}.
% 
% 
% \section{Languages and interfaces}
% \label{sec:langs}
% 
% It is possible to identify several distinct languages related to the
% various interfaces that are needed in a \TeX-based document processing
% system.  This section looks at those we consider most important for
% the \LaTeX3 system.
% 
% \begin{description}
% \item[Document mark-up] This comprises those commands (often called tags)
%   that are to embedded in the document (the |.tex| file).
% 
%   It is generally accepted that such mark-up should be essentially
%   \emph{declarative}.
%   It may be traditional \TeX-based mark-up such as
%   \LaTeXe, as described in~\cite{A-W:LLa94} and~\cite{A-W:GMS94},
%   or a mark-up language defined via \acro{SGML} or \acro{XML}.
%   
%   One problem with more traditional \TeX\ coding conventions (as
%   described in~\cite{A-W:K-TB}) is that the names and syntax of \TeX's
%   primitive formatting commands are ingeniously designed to be
%   `natural' when used directly by the author as document mark-up or in
%   macros.  Ironically, the ubiquity (and widely recognised
%   superiority) of logical mark-up has meant that such explicit
%   formatting commands are almost never needed in documents or in
%   author-defined macros.  Thus they are used almost exclusively by
%   \TeX{} programmers to define higher-level commands; and their
%   idiosyncratic syntax is not at all popular with this community.
%   Moreover, many of them have names that could be very useful as
%   document mark-up tags were they not pre-empted as primitives (\eg
%   |\box| or |\special|).
%   
% \item[Designer interface] This relates a (human) typographic
%   designer's specification for a document to a program that `formats
%   the document'.  It should ideally use a declarative language that
%   facilitates expression of the relationship and spacing rules specified
%   for the layout of the various document elements.
% 
%   This language is not embedded in document text and it will be very
%   different in form to the document mark-up language.  For
%   \acro{SGML}-based systems the \acro{DSSSL} language may come to play
%   this role.  For \LaTeX, this level was almost completely missing
%   from \LaTeX2.09; \LaTeXe\ made some improvements in this area but it
%   is still the case that implementing a design specification in
%   \LaTeX\ requires far more `low-level' coding than is acceptable.
% \item[Programmer interface] 
%   This language is the implementation
%   language within which the basic typesetting functionality is
%   implemented, building upon the primitives of \TeX\ (or a
%   successor program).
%   It may also be used to implement the previous
%   two languages `within' \TeX, as in the current \LaTeX\ system.
% \end{description}
% 
% Only the last of these three interfaces is covered by this paper,
% which describes a system aimed at providing a suitable basis for
% coding large scale projects in \TeX{} (but this should not preclude its
% use for smaller projects).  Its main distinguishing features are
% summarised here.
% 
% \begin{itemize}
% \item A consistent naming scheme for all commands, including \TeX\
%   primitives.
% \item The classification of commands as \LaTeX{} functions or \LaTeX{}
%   parameters, and also their division into modules according to their
%   functionality.
% \item A simple mechanism for controlling argument expansion.
% \item Provision of a set of core \LaTeX{} functions that is sufficient
%   for handling programming constructs such as queues, sets,
%   stacks, property lists.
% \item A \TeX{} programming environment in which, for example, all
%   white space is ignored.
% \end{itemize}
% 
% \section{The naming scheme}
% \label{sec:scheme}
% 
% The naming conventions for this programming language distinguish
% between \textit{functions} and \textit{parameters}. Functions can have
% arguments and they are executed.  Parameters can be assigned values
% and they are used in arguments to functions; they are not directly
% executed but are manipulated by mutator and accessor functions.
% Functions and parameters with a related functionality (for example
% accessing counters, or manipulating token-lists, etc.)\ are collected
% together into a
% \textit{module}.
% 
% 
% Note that all these terms are only \LaTeX{} terminology and are not,
% for example, intended to indicate that the commands have these
% properties when considered in the context of basic \TeX{} or in any
% more general programming context.
% 
% 
% \subsection{Examples}
% \label{sec:ex}
% 
% Before giving the details of the naming scheme, here are a few typical
% examples to indicate the flavour of the scheme; first some parameter
% names.
% \begin{quote}
% |\l_tmpa_box| is a local parameter (hence the~|l_| prefix)
% corresponding to a box register.\\
% |\g_tmpa_int| is a global parameter (hence the~|g_| prefix)
% corresponding to an integer register (\ie a \TeX{} count register).\\
% |\c_empty_toks|
% is the constant~(|c_|) token register parameter that is for ever empty.
% \end{quote}
% Now here is an example of a typical function name.
% 
% |\seq_push:Nn| is the function which puts the token list specified by
% its second argument onto the stack specified by its first argument.
% The different natures of the two arguments are indicated by the~|:Nn|
% suffix. The first argument must be a single token which `names'
% the stack parameter: such single-token arguments are denoted~|N|.
% The second argument is a normal \TeX\ `undelimited argument', which
% may either be a single token or a balanced, brace-delimited token
% list (which we shall here call a \textit{braced token list}): the~|n|
% denotes such a `normal' argument form. The name of the function
% indicates it belongs to the |seq| module.
% 
%^^A |\seq_push:cn| would be similar to the above, but in this case the~|c|
%^^A means that the stack-name is specified in the first argument by a
%^^A token list that expands, using |\csname...|, to a control sequence that
%^^A is the \emph{name} of the stack parameter.
%^^A 
%^^A \noindent
%^^A The names of these two functions also indicate that they are in the
%^^A module called |seq|.
% 
% 
% \subsection{Formal syntax of the conventions}
% \label{sec:namesyn}
% 
% We shall now look in more detail at the syntax of these names.\\
% The syntax of parameter names is as follows:
%  \begin{quote}
%    |\|\m{access}|_|\m{module}|_|\m{description}|_|\m{type}
%  \end{quote}
% The syntax of function names is as follows:
%  \begin{quote}
%    |\|\m{module}|_|\m{description}|:|\m{arg-spec}
%  \end{quote}
% 
% 
% \subsection{Modules and descriptions}
% \label{sec:modules}
% 
% The syntax of all names contains
% \begin{quote}
%   \m{module} and \m{description}:  
% \end{quote}
% these both give information about the command.
% 
% A \textit{module} is a collection of 
% closely related functions and parameters.
% Typical module names include~|int| for integer parameters
% and related functions,~|seq| for sequences and~|box| for boxes.
% 
% Packages providing new programming functionality will add new modules
% as needed; the programmer can choose any unused name, consisting
% of letters only, for a module.
% 
% The \textit{description} gives more detailed information about the
% function or parameter, and provides a unique name for it.  It should
% consist of letters and, possibly,~|_|~characters.
% 
% As a semi-formalized concept the letter |g| is sometimes used to
% prefix certain parts of the \m{description}
% to mark the function as ``globally acting'', e.g., |\int_set:Nn| is
% a local operation while |\int_gset:Nn| is a global operation. This
% of course goes hand in hand with when to use |\l_| and |\g_|
% variable prefixes.
%
% \subsection{Parameters: access and type}
% \label{sec:parms}
% 
% The \m{access} part of the name describes how the parameter can be
% accessed.  Parameters are primarily classified as local, global or
% constant (there are further, more technical, classes).  This
% \textit{access} type appears as a code at the beginning of the name;
% the codes used include:\footnote{The following have also been
% considered but are not currently used:
% \begin{arg-description}
% \item[C] A constant according to \TeX's rules. Can not be changed at
% all.
% \item[L] A local variable that is usually set (and/or reset) by
% \TeX{} itself.
% \item[G] A global variable that is usually set (and/or reset by \TeX).
% \item[R] A  variable that is set (and changed) by \TeX{} and can not
% be changed by in the code (read-only).
% \end{arg-description}
% }
% \begin{arg-description}
% \item[c]
%   constants (global parameters whose value should not be changed);
% \item[g]
%   parameters whose value should only be set globally;
% \item[l]
%   parameters whose value should only be set locally.
% \end{arg-description}
% 
% The \m{type} will normally (except when introducing a new data-type)
% be in the list of available \textit{data-types}; these include the
% primitive \TeX\ data-types, such as the various registers, but to
% these will be added data-types built within the \LaTeX{} programming
% system.
% 
% Here are some typical data-type names:\footnote{The following types
% have been discussed but are not currently used:
% \begin{description}
% \item[factor] Another integer value type. Used for things where the
% parameter is used as a factor for something else.
% \item[status] The sort of boolean stuff \TeX{} provides. Essentially
% an integer with the meaning |0| = `off' and other values may or may
% not have sensible meanings.
% \item[pen] Another integer describing penalties.
% \item[dem] The demerits.
% \item[char] An integer denoting a character.
% \item[muskip] A math unit.
% \end{description}
% }
% \begin{description}
% \item[int] integer-valued count register;
% \item[tlp] token list pointers: placeholders for token lists;
% \item[toks] token register;
% \item[box] box register;
% \item[skip] `rubber' lengths;
% \item[dim] `rigid' lengths;
% \item[num] A `fake' integer type using only macros. Useful for
% setting up allocation routines;
% \item[seq] `sequence': a data-type used to implement lists 
%   (with access at both ends) and stacks;
% \item[plist] property list;
% \item[clist] comma separated list;
% \item[stream] an input or output stream (for reading from or writing to, respectively);
% \item[bool] either true or false.
% \end{description}
% When the \m{type} and \m{module} are identical (as often happens in
% the more basic modules) the \m{module} part is often omitted for
% aesthetic reasons.
% 
% 
% \subsection{Functions: basic argument specifications}
% \label{sec:args}
% 
%  Function names end with an \m{arg-spec} after a colon.  This
%  gives an indication of the types of argument that a function takes,
%  and provides a convenient method of naming similar functions that
%  differ only in their argument forms (see the next section for
%  examples).
% 
%  The \m{arg-spec} consists of a (possibly empty) list of characters,
%  each denoting one argument of the function.  It is important to
%  understand that `argument' here refers to the effective argument of
%  the \LaTeX{} function, not to an argument at the \TeX-level.  Indeed,
%  the top level \TeX\ macro that has this name typically has no
%  arguments.  This is an extension of the existing \LaTeX\ convention
%  where one says that |\section| has an optional argument and a
%  mandatory argument, whereas the \TeX\ macro |\section| actually has
%  zero parameters at the \TeX\ level, it merely calls an internal \LaTeX\
%  command which in turn calls others that look ahead for star forms and
%  optional arguments.
%
%  All functions have a base form with arguments using one of the
%  following argument specifiers.
% \begin{arg-description}
% \item[n]  Unexpanded token or braced token list.\\
%   This is a standard \TeX\ undelimited macro argument.
% \item[N]  Single token (unlike~|n|, the argument must \emph{not} be
%   surrounded by braces).\\
%   A typical example of a command taking an~|N|
%   argument is~|\def|, in which the command being defined must be
%   unbraced.
%  \item[p]  Primitive \TeX\ parameter specification.\\
%    This can be something simple like~|#1#2#3|, but may use arbitrary
%    delimited argument syntax such as: |#1,#2\q_stop#3|. This is used
%    when defining functions.
%  \item[T,F]
%    These are special cases of~|n| arguments, used for the
%    true and false code in conditional commands.
% \end{arg-description}
% There are two other specifiers with more general meanings:
% \begin{arg-description}
% \item[D] This means: \textbf{Do not
%     use}.\footnote{% One could perhaps envisage an extended system
%     which allocated letters to denote the various primitive argument
%     types available in \TeX, however it seems that this just
%     complicates the system without adding any real benefit, as these
%     primitives would never be used in production code, as higher
%     level packages should offer a better interface. Thus the
%     following letters, although they were considered have not been
%     used. ``D'' is used in most cases in preference.
% \begin{arg-description}
% \item[i] Denotes an integer in \TeX{} notation (which might be a
% register or \dots).
%
% \item[d] Denotes a dimension in \TeX{} notation.
%
% \item[g] Denotes a glue in \TeX{} notation.
%
% \item[m] Denotes an muglue or mukern in \TeX{} notation.
%
% \item[b] Denotes a box specification in \TeX{} notation (again
% something pretty arbitrary).
%
% \item[r] Denotes a rule specification in \TeX{} notation.
%
% \end{arg-description}}  This special case is used
%   for \TeX\ primitives and other commands that are provided for use
%   only while bootstrapping the \LaTeX\ kernel.  If the \TeX\ primitive
%   needs to be used in other contexts it will be given an alternative,
%   more appropriate, name with a useful argument specification.  The
%   argument syntax of these is often weird, in the sense described next.
%  \item[w] This means that the argument syntax is `weird' in that it
%    does not follow any standard rule.  It is used for functions with
%    arguments that take non standard forms: examples are \TeX-level
%    delimited arguments and the boolean tests needed after certain
%    primitive |\if|\ldots\ commands.
% \end{arg-description}
% 
% In case of "n" arguments that consist of a
% single token the surrounding braces can be omitted in nearly all
% situations---functions that force the use of braces even for single
% token arguments are explicitly mentioned. For example, "\seq_gpush:Nn"
% is a function that takes two arguments, the first is a single token
% (the sequence) and the second may consist of several tokens surrounded
% by braces.
%
% This concept of argument specification makes it easy to read the code
% and should be followed when defining new functions.
%
%
% 
% 
% \section{Expansion control}
% \label{sec:exp}
%
% Let's take a look at some typical operations one might want to
% perform. Suppose we maintain a stack of open files and we use the
% stack |\g_io_file_name_seq| to keep track of them (|io| is the
% input-output file module). The basic operation
% here is to push a name onto this stack which could be done by the
% operation 
% \begin{quote}
%   "\seq_gpush:Nn"                           \\
%   "   \g_io_file_name_seq"                   \\
%   "   {#1}"
% \end{quote}
% where |#1| is the filename. In other words, this operation would
% push the file name as is. 
%
% However, we might face a situation where
% the filename is stored in a register of some sort, say
% |\l_io_curr_file_tlp|. In this case we want to retrieve the value of
% the register. If we simply use
% \begin{quote}
%   "\seq_gpush:Nn"                           \\
%   "   \g_io_file_name_seq"                   \\
%   "   \l_io_curr_file_tlp "
% \end{quote}
% we will not get the value of the register pushed onto the stack,
% only the register name itself. Instead a suitable number of
% "\exp_after:NN" would be necessary (together with extra braces) to
% change the order of execution, i.e.
% \begin{quote}
%   "\exp_after:NN"                              \\
%   "   \seq_gpush:Nn"                           \\
%   "\exp_after:NN"                              \\
%   "      \g_io_file_name_seq"                   \\
%   "\exp_after:NN"                              \\
%   "      {\l_io_curr_file_tlp}"
% \end{quote}
%
% The above example is probably the simplest case but already shows
% how the code changes to something difficult to
% understand. Furthermore there is an assumption in this: That the
% storage bin reveals its contents after exactly one
% expansion. Relying on this means that you cannot do proper checking
% plus you have to know exactly how a storage bin acts in order to get
% the correct number of expansions.  Therefore \LaTeX3 provides the
% programmer with a general scheme that keeps the code compact and
% easy to understand.
%
% To denote that some argument to a function needs special treatment one
% just uses different letters in the argument part of the function to
% mark the desired behavior. In the above example one would write
% \begin{quote}
%   "\seq_gpush:NV"                           \\
%   "   \g_io_file_name_seq "                   \\
%   "   \l_io_curr_file_tlp "
% \end{quote}
% to achieve the desired effect. Here the "V" is for ``retrieve the
% value of the register'' (the second argument) before passing it to
% the base function.
%
%
% The following letters can be used to denote special treatment of
% arguments before passing it to the base function:
% \begin{description}
% \item[c] Character string used as a command name.\\ The argument (a
%   token or braced token list) must, when fully expanded, produce a
%   sequence of characters which is then used to construct a command
%   name (via~|\csname|, |\endcsname|). This command name is the
%   single token that is passed to the function as the argument. Hence
%   \begin{quote}
%   "\seq_gpush:cV {g_file_name_stack} \l_tmpa_tlp"
%   \end{quote}
%   is equivalent to 
%   \begin{quote}
%   "\seq_gpush:NV \g_file_name_stack \l_tmpa_tlp".
%   \end{quote}
%
% \item[V]  Value of a register.\\
%   This means that the register in question is returned, be it an
%   integer, a length type register, a macro storage register or
%   similar. The value is returned to the function as a braced token list.
%
% \item[v] Value of a register, constructed from a character string
%   used as a command name.\\
%   This is a combination of |c| and |V| which first constructs a
%   control sequence from the argument and then returns the value.
%
% \item[x]  Fully-expanded token or braced token list.\\
%   This means that the argument is expanded as in the replacement
%   text of an~|\edef|, and the expansion is passed to the function as
%   a braced token list. This means that expansion takes place until
%   only unexpandable tokens are left.
%
% \item[o]  One-level-expanded token or braced token list.\\  
%   This means that the argument is expanded one level, as by
%   |\expandafter|, and the expansion is passed to the function as a braced
%   token list.  Note that if the original argument is a braced
%   token list then only the first token in that list is expanded.
%
%  \item[f] Almost the same as the |x| type except here the token list
%  is expanded fully until the first unexpandable token is found and
%  the rest is left unchanged. Note that if this function finds a
%  space at the beginning of the argument it will gobble it and not
%  expand the next argument.
% \end{description}
%
% \subsection{Possible additions}
%
% Here are three new expansion types that may be useful but I'm not
% sure yet. Only time will tell\dots{} Proper documentation of these
% functions is postponed until later.
% \begin{description}
% \item[d] This is pretty much like the |o| type except the token list
%   is expanded twice before being passed on. (|d| is for double.)
%   It is often useful in conjunction with a forced expansion.
% 
% \item[E] Sometimes you need to unpack a token list or something else
%   but you don't want it to add the braces that the |o| type
%   does. This is where you usually wind up with a lot of
%   |\exp_after:NN|s and we would like to avoid that. This type works
%   quite well with the other syntax but it won't work in certain
%   circumstances: Since the generic expansion functions read their
%   arguments when the expanded code is shuffled around, this type
%   will have a problem if the last token you want to expand once is
%   |\token_to_str:N| \emph{and} you're in an argument expansion
%   process involving arguments in braces such as the |n| and |o| type
%   arguments. If you stick to functions invloving only |N| and |E|
%   everything will work just fine. (|E| is for expanded, single
%   token.)
%
% \item[e] Same as above but the argument must be given in braces.
% \end{description}
%
% 
% \subsection{Simpler means better}
% \label{sec;simpler}
% 
% Anyone who programs in \TeX\ is frustratingly familiar with the
% problem of arranging that arguments to functions are suitably expanded
% before the function is called.  To illustrate how expansion control
% can bring instant relief to this problem we shall consider two
% examples copied from \texttt{latex.ltx}.
% 
% \begin{verbatim}
%   \global
%   \expandafter
%     \expandafter
%   \expandafter
%       \let
%   \expandafter
%     \reserved@a
%   \csname \curr@fontshape \endcsname
% \end{verbatim}
% This first piece of code is in essence simply a
% global |\let|. However, the token to be defined is obtained by
% expanding |\reserved@a| one level; and, worse, the token to which it
% is to be let is obtained by fully expanding |\curr@fontshape| and then
% using the characters produced by that expansion to construct a command
% name.  The result is a mess of interwoven |\expandafter| and~|\csname|
% beloved of all \TeX\ programmers, and the code is essentially
% unreadable.
% 
% Using the conventions and functionality outlined here, the task would
% be achieved with code such as this:
% \begin{verbatim}
%   \glet:Vc \g_reserved_a_tlp
%              \l_current_font_shape_tlp
% \end{verbatim}
% The command |\glet:Vc| is a global~|\let| that returns the value of
% the register given as its
% first argument, and generates a command name out of its second
% argument, before making the definition. This produces code that
% is far more readable and more likely to be correct first time.
% 
% Here is the second example.
% \begin{verbatim}
%   \expandafter
%     \in@
%   \csname sym#3%
%     \expandafter
%       \endcsname
%     \expandafter
%       {%
%     \group@list}%
% \end{verbatim}
% This piece of code is part of the definition of another function. It
% first produces two things: a token list, by expanding |\group@list| once;
% and a token whose name comes from~`|sym#3|'.  Then the function~|\in@|
% is called and this tests if its first argument occurs in the token list
% of its second argument.
% 
% Again we can improve enormously on the code.  First we shall rename
% the function~|\in@| according to our conventions.  A function such as
% this but taking two normal `\texttt{n}' arguments might reasonably be
% named |\seq_test_in:Nn|; thus the variant function we need will be
% defined with the appropriate argument types and its name will be
% |\seq_test_in:cV|.  Now this code fragment will be simply:
% \begin{verbatim}
%   \seq_test_in:cV {sym#3} \l_group_seq
% \end{verbatim}
% Note that, in addition to the lack of |\expandafter|, the space after
% the~|}| will be silently ignored since all white space is ignored in
% this programming environment.
% 
% 
% \subsection{New functions from old}
% \label{sec:newfunc}
% 
% For many common functions the \LaTeX3 kernel will provide variants
% with a range of argument forms, and similarly it is expected that
% extension packages providing new functions will make them available in
% the all the commonly needed forms.
% 
% However, there will be occasions where it is necessary to construct a
% new such variant form; therefore the expansion module provides a
% straightforward mechanism for the creation of functions with any
% required argument type, starting from a function that takes `normal'
% \TeX\ undelimited arguments.
% 
% To illustrate this let us suppose you have a `base function'
% |\demo_cmd:nnn| that takes three normal arguments, and that you need
% to construct the variant |\demo_cmd:cnx|, for which the first argument
% is used to construct the \emph{name} of a command, whilst the third
% argument must be fully expanded before being passed to
% |\demo_cmd:nnn|.
% To produce the variant form from the base form, simply use this:
% \begin{verbatim}
%   \exp_def_form:nnn {demo_cmd} {nnn} {cnx} 
% \end{verbatim}
% This defines the variant form so that you can then write, for example:
% \begin{verbatim}
%   \demo_cmd:cnx {abc} {pq} {\rst \xyz }
% \end{verbatim}
% rather than \ldots\ well, something like this!
% \begin{verbatim}
%   \def \tempa {{pq}}%
%   \edef \tempb {\rst \xyz}%
%   \expandafter
%     \demo@cmd:nnn 
%   \csname abc%
%     \expandafter
%       \expandafter
%     \expandafter
%         \endcsname
%     \expandafter
%       \tempa
%     \expandafter
%       {%
%     \tempb
%       }%
% \end{verbatim}
% 
% Another example: you may wish to declare a function
% |\demo_cmd_b:xcxcx|, a variant of an existing function
% |\demo_cmd_b:nnnnn|, that fully
% expands arguments 1,~3 and~5, and produces commands to pass as
% arguments 2 and~4 using~|\csname|.
% The definition you need is simply
% \begin{verbatim}
%   \exp_def_form:nnn
%     {demo_cmd_b} {nnnnn} {xcxcx}
% \end{verbatim}
% 
% This extension mechanism is written so that if the same new form of 
% some existing command is implemented by two extension packages then the
% two definitions will be identical and thus no conflict will occur.
% 
% 
% \section{Parameter assignments and accessor functions}
% \label{sec:access}
% 
% \subsection{Checking assignments}
% \label{sec:check}
% 
% One of the advantages of having a consistent scheme is that the system
% can provide more extensive error-checking and debugging facilities.
% For example, an accessor function that makes a \emph{global}
% assignment of a value to a parameter can check that it is not passed
% the name of a \emph{local} parameter as that argument: it does this by
% checking that the name starts with~|\g_|.
% 
% Such checking is probably too slow for normal use, but the code can
% have hooks built in that allow a format to be made in which all
% functions perform this kind of check.
% 
% A typical section of the source^^A
% \footnote{This code uses the \textsf{docstrip}
% system described in~\cite{A-W:GMS94}, Section~14.3.}
% for such code might look like this
% (recall that all white space is ignored):
% 
% \begin{verbatim}
%   %<*!check>
%   \let_new:NN 
%     \toks_gset:Nn \tex_global:D
%   %</!check>
%   %<*check>
%   \def_new:Npn
%     \toks_gset:Nn #1
%       {
%       \chk_global:N #1
%       \tex_global:D #1
%       }
%   %</check>
% \end{verbatim}
% In the above code the function |\toks_gset:Nn| takes a single
% token~(|N|) specifying a token register, and globally sets it to the
% value passed in the second argument. 
% 
% A typical use of it would be:
% \begin{verbatim}
%   \toks_gset:Nn \g_xxx_toks {<some value>}
% \end{verbatim}
% In the normal definition, |\toks_gset:Nn| can be simply~|\let|
% to~|\global| because the primitive \TeX{} token register does not
% require any explicit assignment function:
% this is done by the |%<*!check>| code above.
% 
% The alternative definition first checks that the argument
% passed as~|#1| is the name of a global parameter and raises an error
% if it is not.  It does this by taking apart the command name passed
% as~|#1| and checking that it starts~|\g_|.
% 
% \subsection{Consistency}
% \label{sec:cons}
% 
% The primitive \TeX\ syntax for register assignments has a very minimal
% syntax and, apart from box functions, there are no explicit functions
% for assigning values to these registers.
% 
% This makes it impossible to implement alternative data-types with a
% syntax that is both consistent and at all similar to the syntax for
% the primitives; moreover, it encourages a coding style that is very
% error prone.
% 
% As in the |\toks_gset:Nn| example given above, all \LaTeX\ data-types
% are provided with explicit functions for assignment and for use, even
% when these have essentially empty definitions.  This allows for better
% error-checking as described above; it also allows the construction of
% further data-types with a similar interface, even when the
% implementation of the associated functions is very complex.
% 
% For example, the `fake-integer' (\texttt{num}) data-type mentioned
% above will appear at the \LaTeX{} programming level to be exactly like
% the data-type based on primitive count registers; however, internally
% it makes no use of count registers.  Typical functions in this module
% are illustrated here.
% 
% \begin{verbatim}
% \num_new:N \l_tmpa_num
% \end{verbatim}
% This declares the local parameter |\l_example_num| as a fake-counter.
% 
% \begin{verbatim}
% \num_add:Nn \l_example_num \c_thirty_two
% \end{verbatim}
% This increments the value of this fake-counter by 32.  
% 
% 
% \section{The experimental distribution}
% \label{sec:dist}
% 
% The initial implementations of a \LaTeX\ programming language using
% this kind of syntax remain unreleased (and not completely functional);
% they partly pre-date \LaTeXe!  The planned distribution will provide a
% subset of the functionality of those implementations, in the form of
% packages to be used on top of \LaTeXe.
% 
% The intention is to allow experienced \TeX\ programmers to experiment
% with the system and to comment on the interface. This means that
% \textbf{\itshape the interface will change}. No part of this system,
% including the name of anything, should be relied upon as being
% available in a later release.  Please do \emph{experiment} with these
% packages, but do \emph{not} use them for code that you expect to keep
% unchanged over a long period.
% 
% In view of the intended experimental use for this distribution we
% shall, in the first instance, produce only a few modules for use with
% \LaTeXe.  These will set up the conventions and the basic functionality
% of, for example, the expansion mechanism; they will also implement some
% of the basic programming constructs, such as token-lists and sequences.
% They are intended only to give a flavour of the code: the full \LaTeX3
% kernel will provide a very rich set of programming constructs so that
% packages can efficiently share code, in contrast with the situation in
% the current \LaTeX\ where every large package must implement its own
% version of queues, stacks, etc., as necessary.
% 
% At time of writing the \textsf{expl3} bundle consists of the
% following modules.
% 
% This distribution will also contain the \LaTeX\ source for the latest
% version of this document, a docstrip install file and three small test
% files.
% 
% \begin{description}
% \providecommand\explpkg[2]{\item[#1]#2}
% \explpkg{l3basics}{
% This contains the basic definition modules used
% by the other packages.
% }
% \explpkg{l3box}{
% Primitives for dealing with boxes.
% }
% \explpkg{l3calc}{
% A re-implementation of the \LaTeXe\ package \pkg{calc} package in 
% \pkg{expl3} that provides extended methods for numeric and dimensional 
% calculations and assignments.
% }
% \explpkg{l3chk}{  
% A module that provides functionality comparable to
% \LaTeX's \cmd{\newcommand} and \cmd{\renewcommand}, and also the
% extra level of 
% checking for ensure internal consistency in the code.
% }
% \explpkg{l3clist}{
% Methods for manipulating comma-separated token lists.
% }
% \explpkg{l3expan}{ 
% This is the argument expansion module discussed above.
% }
% \explpkg{l3int}{
% This implements the integer data-type \texttt{int}.
% }
% \explpkg{l3io}{
% A module providing low level input and output functions.
% }
% \explpkg{l3keyval}{
% A re-implementation of the \LaTeXe\ package \pkg{keyval}; 
% provides low-level macros for dealing with lists of the form
% \texttt{\{ key1=val1 , key2=val2 \}}.
% }
% \explpkg{l3messages}{
% Communicating with the user.
% }
% \explpkg{l3names}{ 
% This sets up the basic naming scheme and renames all
% the \TeX\ primitives.  If it is loaded with the option
% \texttt{[removeoldnames]} then the old primitive names such as~\cmd{\box}
% become \emph{undefined} and are thus available for user
% definition. Caution: use of this option will certainly break existing
% \TeX\ code!
% }
% \explpkg{l3num}{ 
% This implements the `fake integer' datatype \texttt{num}.
% }
% \explpkg{l3precom}{
% A `pre-compilation' module that provides functions dealing with pointer
% creation and handling, and using external files to record the state of the current
% definitions.
% }
% \explpkg{l3prg}{
% Program control structures such as boolean data type |bool|, generic
% do-while loops, case-switches, sorting routines and stepwise loops. 
% }
% \explpkg{l3prop}{
% This implements the data-type for `property lists' that are used, in
% particular, for storing key/value pairs.
% }
% \explpkg{l3quark}{ 
% A `quark' is a command that is defined to expand to
% itself!  Therefore they must never be expanded as this will generate
% infinite recursion; they do however have many uses, \eg as
% special markers and delimiters within code.
% }
% \explpkg{l3seq}{
% This implements data-types such as queues and stacks.
% }
% \explpkg{l3skip}{
% Implements the `rubber length' datatype \texttt{skip} and the
% `rigid length' datatype |dim|. 
% }
% \explpkg{l3tlp}{
% This implements a basic data-type, called a \textit{token-list
% pointer} (|tlp|), used for storing named token lists: these are
% essentially \TeX{} macros with no arguments.
% }
% \explpkg{l3token}{
% Analysing token lists and token streams, including peeking ahead to
% see what's coming next and inspecting tokens to detect which kind
% they are.
% }
% \explpkg{l3toks}{
% A data-type corresponding to \TeX's primitive token registers.
% }
% \explpkg{l3xref}{
% Data structure for low-level document cross-referencing. This
% provided the foundation for Heiko Oberdiek's \pkg{zref} package.
% }
% \end{description}
% 
% \begin{thebibliography}{1}
% 
% \bibitem{A-W:K-TB}
% Donald E Knuth
% \newblock {\em The {\TeX}book}.
% \newblock Addison-Wesley, Reading, Massachusetts, 1984.
% 
% \bibitem{A-W:GMS94}
% Goossens, Mittelbach and Samarin.
% \newblock {\em The {\LaTeX} Companion}.
% \newblock Addison-Wesley, Reading, Massachusetts, 1994.
% 
% \bibitem{A-W:LLa94}
% Leslie Lamport.
% \newblock {\em {\LaTeX:} A Document Preparation System}.
% \newblock Addison-Wesley, Reading, Massachusetts, second edition, 1994.
% 
% \bibitem{tub:MR97-1}
% Frank Mittelbach and Chris Rowley.
% \newblock `The {\LaTeX3} Project'.
% \newblock {\em {TUG}boat}, 
% Vol.\,18, No.\,3, pp.\,195--198, 1997.
% 
% \end{thebibliography}
% 
% % \StopEventually{}
% \subsection{Implementation of \texttt{expl3.sty}}
% Well, it's not complicated |:)|
%
%<*package>
%    \begin{macrocode}
\ProvidesExplPackage
  {\filename}{\filedate}{\fileversion}{\filedescription}
%    \end{macrocode}
% 
% The \pkg{l3chk} package is omitted since it is only used for conditional 
% processing with full error-checking turned on. Most users will generally 
% not need to do this\dots
%
%    \begin{macrocode}
\RequirePackage{
  l3basics,
  l3box,
  l3calc,
  l3clist,
  l3expan,
  l3int,
  l3io,
  l3keyval,
  l3messages,
  l3num,
  l3precom,
  l3prg,
  l3prop,
  l3quark,
  l3seq,
  l3skip,
  l3tlp,
  l3token,
  l3toks,
  l3xref
}
%    \end{macrocode}
%</package>
%
% \endinput
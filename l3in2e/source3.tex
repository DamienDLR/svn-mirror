% \iffalse
%% File: source3.dtx Copyright (C) 1990-2009 LaTeX3 project
%%
%% It may be distributed and/or modified under the conditions of the
%% LaTeX Project Public License (LPPL), either version 1.3c of this
%% license or (at your option) any later version.  The latest version
%% of this license is in the file
%%
%%    http://www.latex-project.org/lppl.txt
%%
%% This file is part of the ``expl3 bundle'' (The Work in LPPL)
%% and all files in that bundle must be distributed together.
%%
%% The released version of this bundle is available from CTAN.
%%
%% -----------------------------------------------------------------------
%%
%% The development version of the bundle can be found at
%%
%%    http://www.latex-project.org/cgi-bin/cvsweb.cgi/
%%
%% for those people who are interested.
%%
%%%%%%%%%%%
%% NOTE: %%
%%%%%%%%%%%
%%
%%   Snapshots taken from the repository represent work in progress and may
%%   not work or may contain conflicting material!  We therefore ask
%%   people _not_ to put them into distributions, archives, etc. without
%%   prior consultation with the LaTeX Project Team.
%%
%% -----------------------------------------------------------------------
%% \fi

% This document will typeset the LaTeX3 sources as a single document.
% This will produce quite a large file (roughly ??? pages) and may
% take a long time on a slow machine.

\documentclass{l3doc}
\listfiles

\begin{document}

\title{The \LaTeX3 Sources}
\author{\Team}


\pagenumbering{roman}
\maketitle

\begin{abstract}

\parindent=0pt
\parskip=\baselineskip

\noindent This is the reference documentation for the \pkg{expl3} 
programming environment. The \pkg{expl3} modules set up an experimental
naming scheme for \LaTeX\ commands, which allow the \LaTeX\ programmer
to systematically name functions and variables, and specify the argument
types of functions.

The \TeX\ and \eTeX\ primitives are all given a new name according to
these conventions. However, in the main direct use of the primitives is
not required or encouraged: the \pkg{expl3} modules define an
independent low-level \LaTeX3 programming language.

At present, the \pkg{expl3} modules are designed to be loaded on top of
\LaTeXe. In time, a \LaTeX3 format will be produced based on this code.
This allows the code to be used in \LaTeXe\ packages \emph{now} while a
stand-alone \LaTeX3 is developed.

\begin{bfseries}
  While \pkg{expl3} is still experimental, the bundle is now regarded as
  broadly stable. The syntax conventions and functions provided are now
  ready for wider use. There may still be changes to some functions, but
  these will be minor when compared to the scope of \pkg{expl3}.

  New modules will be added to the distributed version of \pkg{expl3} as
  they reach maturity.
\end{bfseries}

\end{abstract}

\clearpage

{\def\\{:}% fix "newlines" in the ToC
\tableofcontents}

\clearpage
\pagenumbering{arabic}

%%%%%%%%%%%%%%%%%%%%%%%%%%%%%%%%%%%%%%%%%%%%%%%%%%%%%%%%%%%%

% Each of the following \DocInput lines includes a file with extension
% .dtx. Each of these files may be typeset separately. For instance
%   pdflatex l3box.dtx
% will typeset the source of the LaTeX3 box commands. If you use the
% Makefile, the index will be generated automatically; e.g.,
%   make doc F=l3box
%
% If this file is processed, each of these separate dtx files will be
% contained as a part of a single document.

\makeatletter
\def\partname{Part}
\def\maketitle{\part{\@title}}
\let\thanks\@gobble
\let\DelayPrintIndex\PrintIndex
\let\PrintIndex\@empty
\makeatother

\part{Documentation overview: conventions used}

This document is intended to act as a comprehensive reference manual 
for the \pkg{expl3} language. A general guide to the \LaTeX3 
programming language is found in \href{expl3.pdf}{expl3.pdf}. 

\LaTeX3 does not use \texttt{@} as a ``letter'' for defining
internal macros.  Instead, the symbols |_| and \texttt{:}
are used in internal macro names to provide structure. The name of
each \emph{function} is divided into logical units using \texttt{_}, 
while \texttt{:} separates the \emph{name} of the function from the 
\emph{argument specifier} (``arg-spec''). This describes the arguments 
expected by the function. In most cases, each argument is represented 
by a single letter. The complete list of arg-spec letters for a function
is referred to as the \emph{signature} of the function.

This document is typeset with the experimental \pkg{l3doc} class; 
several conventions are used to help describe the features of the code.
A number of conventions are used here to make the documentation clearer.

Each group of related functions is given in a box. For a function with
a ``user'' name, this might read:
\begin{function}{
    \ExplSyntaxOn |
    \ExplSyntaxOff
  }
  \begin{syntax}
    "\ExplSyntaxOn"
  \end{syntax}
  The description of how the function works would appear here. The 
  syntax of the function is shown in mono-spaced text to the right of 
  the box. In this case, the function takes no arguments and so the 
  name of the function is simply reprinted.
\end{function}

For programming functions, which use \texttt{_} and \texttt{:} in their
name. If two related functions are given with identical names but 
different argument specifiers, these are termed \emph{variants} of each 
other, and the latter functions are printed in grey to show this more 
clearly. They will carry out the same function but will take different
types of argument:
\begin{function}{
    \seq_new:N |
    \seq_new:c
  }
  \begin{syntax}
    "\seq_new:N" <sequence>
  \end{syntax}
  When a number of variants are described, the arguments are usually
  illustrated only for the base function. Here, <sequence> indicates 
  that \cs{seq_new:N} expects the name of a sequence. From the argument
  specifier, \cs{seq_new:c} also expects a sequence name, but as a 
  name rather than as a control sequence. Each argument given in the
  illustration should be described in the following text.
\end{function}

Some functions are fully expandable, which allows it to be used within 
an \texttt{x}-type argument (in plain \TeX\ terms, inside an \cs{edef}).
These fully expandable functions are indicated in the documentation by 
a star:
\begin{function}{
    \cs_to_str:N / (EXP)
  }
  \begin{syntax}
    "\cs_to_str:N" <cs>
  \end{syntax}
  As with other functions, some text should follow which explains how
  the function works. Usually, only the star will indicate that the 
  function is expandable. In this case, the function expects a <cs>, 
  shorthand for a <control sequence>. 
\end{function}

Conditional (\texttt{if}) functions are defined in three variants, with
\texttt{T}, \texttt{F} and \texttt{TF} argument specifiers. This allows
them to be used for different `true'/`false' branches, depending on 
which outcome the conditional is being used to test. To indicate this 
without repetition, this information is given in a shortened form:
\begin{function}{
    \xetex_if_engine: / (TF) (EXP)
  }
  \begin{syntax}
    "\xetex_if_engine:TF" <true code> <false code>
  \end{syntax}
  The underlining and italic of \texttt{TF} indicates that 
  \cs{xetex_if_engine:T}, \cs{xetex_if_engine:F} and 
  \cs{xetex_if_engine:TF} are all available. Usually, the illustration 
  will use the \texttt{TF} variant, and so both <true code>
  and <false code> will be shown. The two variant forms \texttt{T} and
  \texttt{F} take only <true code> and <false code>, respectively.
  Here, the star also shows that this function is expandable.
  With some minor exceptions, \emph{all} conditional functions in the 
  \pkg{expl3} modules should be defined in this way.
\end{function}

Variables, constants and so on are described in a similar manner:
\begin{variable}{
    \l_tmpa_tl
  }
  A short piece of text will describe the variable: there is no 
  illustration in this case.
\end{variable}

In some cases, the function is similar to one in \LaTeXe\ or plain \TeX.
In these cases, the text will include an extra `\textbf{\TeX{}hackers 
note}' section:
\begin{function}{
    \token_to_str:N / (EXP)
  }
  \begin{syntax}
    "\token_to_str:N" <token> 
  \end{syntax}
  The normal description text.
  \begin{texnote}
    Detail for the experienced \TeX\ or \LaTeXe\ programmer. In this 
    case, it would point out that this function is the \TeX\ primitive
    \cs{string}.
  \end{texnote}
\end{function}

\DisableImplementation

\DocInput{l3names.dtx}

\DocInput{l3basics.dtx}
\DocInput{l3expan.dtx}
\DocInput{l3prg.dtx}
\DocInput{l3quark.dtx}
\DocInput{l3token.dtx}

\DocInput{l3int.dtx}
\DocInput{l3num.dtx}
\DocInput{l3intexpr.dtx}
\DocInput{l3skip.dtx}

\DocInput{l3tl.dtx}
\DocInput{l3toks.dtx}
\DocInput{l3seq.dtx}
\DocInput{l3clist.dtx}
\DocInput{l3prop.dtx}

\DocInput{l3io.dtx}
\DocInput{l3msg.dtx}
\DocInput{l3box.dtx}
\DocInput{l3xref.dtx}
\DocInput{l3keyval.dtx}
\DocInput{l3calc.dtx}
\DocInput{l3file.dtx}

% \DocInput{l3precom.dtx}
% \DocInput{l3alloc.dtx}
% \DocInput{l3chk.dtx}

\part{Implementation}
\def\maketitle{}
\EnableImplementation
\DisableDocumentation
\DocInputAgain

%% \DocInput{l3vers.dtx}   % Current version date

\includeltpatch       % Corrections distributed after the full release

\clearpage
\pagestyle{headings}

% Make TeX shut up.
\hbadness=10000
\newcount\hbadness
\hfuzz=\maxdimen

\PrintChanges
\clearpage

\begingroup
\def\endash{--}
\catcode`\-\active
\def-{\futurelet\temp\indexdash}
\def\indexdash{\ifx\temp-\endash\fi}

\DelayPrintIndex
\endgroup

\end{document}



% \iffalse
%% File: l3basics.dtx Copyright (C) 1990-2005 LaTeX3 project
%%
%% It may be distributed and/or modified under the conditions of the
%% LaTeX Project Public License (LPPL), either version 1.3a of this
%% license or (at your option) any later version.  The latest version
%% of this license is in the file
%%
%%    http://www.latex-project.org/lppl.txt
%%
%% This file is part of the ``expl3 bundle'' (The Work in LPPL)
%% and all files in that bundle must be distributed together.
%%
%% The released version of this bundle is available from CTAN.
%%
%% -----------------------------------------------------------------------
%%
%% The development version of the bundle can be found at
%%
%%    http://www.latex-project.org/cgi-bin/cvsweb.cgi/
%%
%% for those people who are interested.
%%
%%%%%%%%%%%
%% NOTE: %%
%%%%%%%%%%%
%%
%%   Snapshots taken from the repository represent work in progress and may
%%   not work or may contain conflicting material!  We therefore ask
%%   people _not_ to put them into distributions, archives, etc. without
%%   prior consultation with the LaTeX Project Team.
%%
%% -----------------------------------------------------------------------
%^^A The next definition of \next is needed in the initex environment
%^^A otherwise the format can not be loaded.
%^^A but they shouldn't be seen by LaTeX when typesetting the code...
%
%<initex>\tex_def:D\next{\CodeStop\xnext}
%<initex>\tex_def:D\xnext#1~#2.dtx,v~#3~#4~#5~#6~#7$#8{%^^A }$
%<initex>  \tex_immediate:D\tex_write:D-1{#2;~v#3,~#4;~#8}%
%<initex>  \CodeStart
%<initex>  }
%<*!initex>
%\fi
\def\next#1: #2.dtx,v #3 #4 #5 #6 #7$#8{%^^A emacs is confused by this}$
  \def\fileversion{#3}%
  \def\filedate{#4}%
%\iffalse
%</!initex>
%<*dtx>
%\fi
     \ProvidesFile{#2.dtx}[#4 v#3 #8]%
%\iffalse
%</dtx>
%<package> \ProvidesPackage{#2}[#4 v#3 #8]%
%<driver>  \ProvidesFile{#2.drv}[#4 v#3 #8]%
%<*!initex>
%\fi
}
%\iffalse
%</!initex>
%\fi
\next$Id$
       {L3 Experimental basic definitions}
%
% \iffalse
%<*driver>
\documentclass{l3doc}

\begin{document}
\DocInput{l3basics.dtx}
\end{document}
%</driver>
% \fi
%
% \title{The \textsf{l3basics} package\thanks{This file
%         has version number \fileversion, last
%         revised \filedate.}\\
% Basic Definitions}
% \author{\Team}
% \date{\filedate}
% \maketitle
%
%
% As the name suggest this package holds some basic definitions which
% are needed by most or all other packages in this set.
%
% \section{Basics}
%
% Here we describe those functions that used all over the place. With
% that we mean functions dealing with the construction and testing of
% control sequences. Furthermore the basic parts of conditional
% processing are covered; conditional processing dealing with specific data types is described in the modules specific for the respective data types.
%
% \subsection{Predicates and conditionals}
% 
% \subsubsection{Primitive conditionals} 
%
% The \eTeX\ engine itself provides many different conditionals. Some
% expand whatever comes after them and others don't. Hence the names
% for these underlying functions will often contain a |:w| part but
% higher level functions are often available. See for instance
% |\int_compare_p:nNn| which is a wrapper for |\if_num:w|.
%
% Certain conditionals deal with specific data types like boxes and
% fonts and are described there. The ones described below are either
% the universal conditionals or deal with control sequences. We will
% prefix primitive conditionals with |\if_|.
%
% \begin{function}{%
%                  \if_true:  |
%                  \if_false: |
%                  \else:         |
%                  \fi: |
%                  \reverse_if:N |
% }
% \begin{syntax}
%   "\if_true:" <true code> "\else:" <false code> "\fi:" \\
%   "\if_false:" <true code> "\else:" <false code> "\fi:" \\
%   "\reverse_if:N" <primitive conditional>
% \end{syntax}
% "\if_true:" always executes <true code>, while "\if_false:" always
% executes <false code>. "\reverse_if:N" reverses any two-way primitive
% conditional. "\else:" and "\fi:" delimit the branches of the conditional.
% \end{function}
%
% \begin{function}{%
%                  \if_meaning:NN |
%                  \if_cs_meaning_eq:NN |
%                  \if_token_eq:NN |
% }
% \begin{syntax}
%   "\if_meaning:NN" <cs1> <cs2> <true code> "\else:" <false code> "\fi:" 
%   "\if_cs_meaning_eq:NN" <cs1> <cs2> <true code> "\else:" <false code> "\fi:" 
%   "\if_token_eq:NN" <token1> <token2> <true code> "\else:" <false code> "\fi:" 
% \end{syntax}
% "\if_meaning:NN" executes <true code> when the replacement text,
% i.e., the expansion of <cs1> and <cs2> are the same, otherwise it
% executes <false code>. However this name isn't really that
% good. What the \TeX\ primitive does is compare two tokens to see if
% they are equal. Hence this is actually a "token" functions. A
% similar argument applies to the situation where it is used to
% compare control sequences, where it is the meaning being compared.
% Something to be cleaned up at some point.
% \end{function}
%
% \begin{function}{%
%                  \if:w          |
%                  \if_catcode:w   |
% }
% \begin{syntax}
%   "\if:w" <token1> <token2> <true code> "\else:" <false code> "\fi:" \\
%   "\if_catcode:w" <token1> <token2> <true code> "\else:" <false code> "\fi:"
% \end{syntax}
% These conditionals will expand any following tokens until two
% unexpandable tokens are left. If you wish to prevent this expansion,
% prefix the token in question with "\exp_not:N". "\if_catcode:w"
% tests if the category codes of the two tokens are the same whereas
% "\if:w" tests if the character codes are identical.
% \end{function}
%
% \begin{function}{%
%                  \if_cs_exist:N  |
%                  \if_cs_exist:w   |
% }
% \begin{syntax}
%   "\if_cs_exist:N" <cs> <true code> "\else:" <false code> "\fi:" \\
%   "\if_cs_exist:w" <tokens> "\cs_end:" <true code> "\else:" <false
%   code> "\fi:"
% \end{syntax}
% Check if <cs> appears in the hash table or if the control sequence
% that can be formed from <tokens> appears in the hash table. The
% latter function does not turn the control sequence in question into
% "\scan_stop:"! This can be useful when dealing with control
% sequences which cannot be entered as a single token.
% \end{function}
%
% \begin{function}{
%     \if_mode_horizontal: |
%     \if_mode_vertical: |
%     \if_mode_math: |
%     \if_mode_inner: |
% }
% \begin{syntax}
%   "\if_horizontal_mode:" <true code> "\else:" <false code> "\fi:"
% \end{syntax}
% Execute <true code> if currently in horizontal mode, otherwise
% execute <false code>. Similar for the other functions.
% \end{function}
%
% \subsubsection{Non-primitive conditionals}
%
% \begin{function}{%
%                  \cs_if_eq_p:NN |
% }
% \begin{syntax}
%   "\cs_if_eq_p:NN" <cs1> <cs2>
% \end{syntax}
% Returns `true' if <cs1> and <cs2> are textually the same, i.e.\ have
% the same name, otherwise it returns `false'.
% \end{function}
%
%
% \begin{function}{%
%                  \cs_free_p:N |
% }
% \begin{syntax}
%   "\cs_free_p:N" <cs>
% \end{syntax}
% Returns `true' if <cs> is either undefined or equal to "\scan_stop:".
% However, it returns `false' if <cs> is textually "\c_undefined" (the
% constantly undefined function), or  textually "\scan_stop:".
% \end{function}
%
% \begin{function}{%
%                  \cs_free:NTF |
%                  \cs_free:NF |
%                  \cs_free:NT |
%                  \cs_free:cF |
%                  \cs_free:cTF |
% }
% \begin{syntax}
%    "\cs_free:NTF" <cs> "{"<true code>"}{"<false code>"}"
% \end{syntax}
% These functions check if <cs> is free and then execute either <true
% code> or <false code>.
% \begin{texnote}
% The conditional "\cs_free:cTF" is the \LaTeX3 implementation of the
% \LaTeX2 function \tn{@ifundefined}. The other functions haven't been
% around before.
% \end{texnote}
% \end{function}
%
% \begin{function}{%
%                  \cs_exist_p:N |
% }
% \begin{syntax}
%   "\cs_exist_p:N" <cs>
% \end{syntax}
% This function does the opposite of "\cs_free_p:N".
% \end{function}
%
% \begin{function}{%
%                  \cs_really_free:cTF |
%                  \cs_really_free:cF |
%                  \cs_really_free:cT |
% }
% \begin{syntax}
%    "\cs_really_free:cTF" "{"<tokens>"}" "{"<true code>"}{"<false code>"}"
% \end{syntax}
% Similar to "\cs_free:cTF" but does not put anything previously
% undefined into the hash table. Useful for special control sequences
% like "\foo/bar" which cannot be entered as one token.
% \end{function}
%
% \begin{function}{%
%                  \chk_new_cs:N |
% }
% \begin{syntax}
%   "\chk_new_cs:N" <cs>
% \end{syntax}
% This function checks that <cs> is so far either undefined or equals
% "\scan_stop:" (the function that is assigned to newly created
% control sequences by \TeX{} when "\cs:w" "..." "\cs_end:" is
% used).
% \end{function}
%
% \begin{function}{%
%                  \chk_exist_cs:N |
%                  \chk_exist_cs:c |
% }
% \begin{syntax}
%   "\chk_exist_cs:N" <cs>
% \end{syntax}
% This function checks that <cs> is defined. If it is not an error
% is generated.
% \end{function}
%
%
% \begin{variable}{%
%                  \c_true | \c_false |
% }
% \begin{syntax}\end{syntax}
% Constants that represent `true' or `false', respectively. Used to
% implement predicates.
% \end{variable}
%
%
%
% \subsection{Selecting and discarding tokens from the input stream}
%
%  The conditional processing could not have been implemented without
%  being able to gobble and select which tokens to use from the input
%  stream.
%
% \begin{function}{%
%                  \use_none:n |
%                  \use_none:nn |
%                  \use_none:nnn |
%                  \use_none:nnnn |
% }
% \begin{syntax}
%    "\use_none:n"  "{"<arg1>"}"\\
%    "\use_none:nn" "{"<arg1>"}{"<arg2>"}"
% \end{syntax}
%  These functions gobble the tokens or brace groups from the input
%  stream.
% \end{function}
%
%
% \begin{function}{\use_arg_i:n}
% \begin{syntax}
%   "\use_arg_i:n"  "{" <code1> "}"
% \end{syntax}
% Function that executes the next argument after removing the
% surrounding braces. Used to implement conditionals.
% \end{function}
%
% \begin{function}{%
%                  \use_arg_i:nn |
%                  \use_arg_ii:nn |
% }
% \begin{syntax}
%   "\use_arg_i:nn"  "{" <code1> "}{" <code2> "}"
% \end{syntax}
% Functions that execute the first or second argument respectively,
% after removing the surrounding braces. Primarily used to implement
% conditionals.
% \end{function}
%
% \begin{function}{%
%                  \use_arg_i:nnn |
%                  \use_arg_ii:nnn |
%                  \use_arg_iii:nnn |
% }
% \begin{syntax}
%   "\use_arg_i:nnn"  "{" <arg1> "}{" <arg2> "}{" <arg3> "}"
% \end{syntax}
% Functions that pick up one of three arguments and execute them after
% removing the surrounding braces. Should be described somewhere else.
% \end{function}
%
%
%
%
%
%
%
%
% \subsection{Internal functions}
%
% \begin{function}{%
%                  \cs:w |
%                  \cs_end: |
% }
% \begin{syntax}
%   "\cs:w" <tokens> "\cs_end:"
% \end{syntax}
% This is the \TeX{} internal way of generating a  control sequence from
% some token list. <tokens> get expanded and must ultimately result in a
% sequence of characters.
% \begin{texnote}
% These functions are the primitives \tn{csname} and \tn{endcsname}.
% "\cs:w" is considered weird because it expands tokens until it reaches
% "\cs_end:".
% \end{texnote}
% \end{function}
%
% \begin{function}{\pref_global:D |
%                  \pref_long:D |
%                  \pref_robust:D |
% }
% \begin{syntax}
%   "\pref_global:D" "\def:Npn"
% \end{syntax}
% Prefix functions that can be used in front of some definition
% functions (namely \ldots). The result of prefixing a function
% definition with "\pref_global:D" makes the definition global,
% "\pref_long:D" change the argument scanning mechanism so that it
% allows "\par" tokens in the argument of the prefixed function,
% and "\pref_robust:D" makes the definition robust in "\write"s etc.
%
%
% None of these internal functions should be used by a programmer since
% the necessary combinations are all available as separate function,
% e.g., "\def_long:Npn" is internally implemented as "\pref_long:D"
% "\def:Npn".
% \begin{texnote}
% These prefixes are the primitives \tn{global}, \tn{long}, and
% \tn{protected}.
% The \tn{outer} isn't used at all within \LaTeX3
% because \ldots
% \end{texnote}
% \end{function}
%
%
%
% \begin{function}{\io_put_log:x |
%                  \io_put_term:x |
%                  \io_put_deferred:Nx |
% }
% \begin{syntax}
% "\io_put_log:x" "{"<message>"}"
% "\io_put_deferred:Nx" <write_stream> "{"<message>"}"
% \end{syntax}
%  Writes <message> to either to log or the terminal.
% \end{function}
%
% \subsection{Defining functions}
%
% There are two types of function definitions in \LaTeX3:  versions
% that check if the function name is still unused, and versions that
% simply make the definition. The later are used for internal scratch
% functions that get new meanings all over the place.
%
% (Parts of this module are a mess, as far as naming conventions are
% concerned.)
%
% \begin{texnote}
% While \TeX{} makes all definition functions directly available to the
% user \LaTeX3 hides them very carefully to avoid the problems with
% definitions that are overwritten accidentally. Many functions that are in
% \TeX{} a combination of prefixes and definition functions are provided
% as individual functions.
% \end{texnote}
%
% \subsubsection{Defining new functions}
%
% A definition of a new function can be done locally and globally. Currently
% nearly all function definitions are done locally on top level, in
% other words they are global but don't show it. Therefore I think it may
% be better to remove the local variants in the future and declare all
% checked function definitions global.
%
% \begin{function}{\def_new:Npn |
%                  \def_new:Npx |
%                  \def_new:cpn |
%                  \def_new:cpx
% }
% \begin{syntax}
%   "\def_new:Npn" <cs> <parms> "{" <code> "}"
% \end{syntax}
% Defines a new function, making sure that <cs> is unused so far.
% <parms> may consist of arbitrary parameter specification in \TeX{}
% syntax. It is under the responsibility of the programmer to name the
% new function according to the rules laid out in the previous section.
% <code> is either passed literally or may be subject to expansion
% (under the "x" variants).
% \end{function}
%
% \begin{function}{\gdef_new:Npn
% }
% \begin{syntax}
%   "\gdef_new:Npn" <cs> <parms> "{" <code> "}"
% \end{syntax}
% Like "\def_new:Npn" but defines the new function globally. See
% comments above.
% \end{function}
%
% \begin{function}{\def_long_new:Npn |
% }
% \begin{syntax}
%   "\def_long_new:Npn" <cs> <parms> "{" <code> "}"
% \end{syntax}
% Defines a function that may contain "\par" tokens in the argument(s)
% when called. This is not allowed for normal functions.
% \end{function}
%
% \begin{function}{\let_new:NN |
%                  \let_new:cN |
%                  \let_new:Nc |
%                  \glet_new:NN|
%                  \glet_new:cN }
% \begin{syntax}
%   "\let_new:NN" <cs1> <cs2>
% \end{syntax}
% Gives the function <cs1> the current meaning of <cs2>. Again, we may
% do this always globally.
% \end{function}
%
% \subsubsection{Undefining functions}
%
% \begin{function}{ \cs_gundefine:N
% }
% \begin{syntax}
%   "\cs_gundefine:N" <cs>
% \end{syntax}
% Undefines the control sequence.
% \end{function}
%
% \subsubsection{Defining internal functions (no checks)}
%
% Besides the function definitions that check whether or not their
% argument is an unused function we need function definitions that
% overwrite currently used definitions. The following functions are
% provided for this purpose.
%
% \begin{function}{\def:Npn |
%                  \def:Npx |
%                  \def:cpn |
%                  \def:cpx |
% }
% \begin{syntax}
%   "\def:Npn" <cs> <parms> "{" <code> "}"
% \end{syntax}
% Like "\def_new:Npn" etc.\ but does not check the <cs> name.
% \begin{texnote}
% "\def:Npn" is the \LaTeX3 name for \TeX{}'s \tn{def} and "\def:Npx"
% corresponds to the primitive \tn{edef}. The "\def:cpn" function was
% known in \LaTeX2 as \tn{@namedef}. "\def:cpx" has no equivalent.
% \end{texnote}
% \end{function}
%
% \begin{function}{\gdef:Npn |
%                  \gdef:Npx |
%                  \gdef:cpn |
%                  \gdef:cpx |
% }
% \begin{syntax}
%   "\gdef:Npn" <cs> <parms> "{" <code> "}"
% \end{syntax}
% Like "\def:Npn" but defines the <cs> globally.
% \begin{texnote}
% "\gdef:Npn" and "\gdef:Npx" are known to \TeX{}hackers as \tn{gdef}
% and \tn{xdef}.
% \end{texnote}
% \end{function}
%
% \begin{function}{\def:No |
%                  \gdef:No
% }
% \begin{syntax}
%   "\def:No" <cs> "{" <code> "}"
% \end{syntax}
% Local and global variant that expands code once before defining <cs>.
% The function may not take <parms> as the others do. Perhaps this
% should be changed.
% \end{function}
%
% \begin{function}{\def_long:Npn |
%                  \def_long:Npx |
%                  \def_long:cpn |
%                  \def_long:cpx |
% }
% \begin{syntax}
%   "\def_long:Npn" <cs> <parms> "{" <code> "}"
% \end{syntax}
% Like "\def:Npn" but allows "\par" tokens in the arguments of the
% function being defined.
% \end{function}
%
% \begin{function}{\gdef_long:Npn |
%                  \gdef_long:Npx |
%                  \gdef_long:cpn |
%                  \gdef_long:cpx |
% }
% \begin{syntax}
%   "\gdef_long:Npn" <cs> <parms> "{" <code> "}"
% \end{syntax}
% Global variant of "\def_long:Npn".
% \end{function}
%
% \begin{function}{\def_robust:Npn |
%                  \def_robust_long:Npn |
%                  \def_robust_new:Npn |
% }
% \begin{syntax}
%   "\def_robust:Npn" <cs> <parms> "{" <code> "}"
% \end{syntax}
% Naturally robust macro that won't expand in an |x| type argument.
% This also comes as a |long| version. If you for some reason want to
% expand it inside an |x| type expansion, prefix it with
% |\exp_after:NN \use_noop:|.
% \end{function}
%
%
% \begin{function}{\let:NN |
%                  \let:cN  |
%                  \let:Nc  |
%                  \let:cc  |
%                  \glet:NN |
%                  \glet:cN |
%                  \glet:Nc |
%                  \glet:cc
%                 }
% \begin{syntax}
%   "\let:cN" <cs1> <cs2>
% \end{syntax}
% Gives the function <cs1> the current meaning of <cs2>. Again, we may
% always do this globally.
% \end{function}
%
% \begin{function}{\let:NwN}
% \begin{syntax}
%   "\let:NwN"  <cs1> <cs2>
%   "\let:NwN"  <cs1> "=" <cs2>
% \end{syntax}
% These functions assign the meaning of <cs2> locally or globally to the
% function <cs1>. Because the \TeX{} primitive operation is being used
% which may have an equal sign and (a certain number of) spaces between
% <cs1> and <cs2> the name contains a "w". (Not happy about this
% convention!).
% \begin{texnote}
% "\let:NwN" is the \LaTeX3 name for \TeX{}'s \tn{let}.
% \end{texnote}
% \end{function}
%
%
%
% \subsection{The innards of a function}
%
% \begin{function}{\cs_to_str:N}
% \begin{syntax}
%   "\cs_to_str:N" <cs>
% \end{syntax}
% This function return the name of <cs> as a sequence of letters with
% the escape character removed.
% \end{function}
%
% \begin{function}{\token_to_string:N}
% \begin{syntax}
%   "\token_to_string:N" <arg>
% \end{syntax}
% This function return the name of <arg> as a sequence of letters
% including the escape character.
% \end{function}
%
% \begin{function}{\token_to_meaning:N}
% \begin{syntax}
%   "\token_to_meaning:N" <arg>
% \end{syntax}
% This function returns the type and definition of <arg> as a sequence
% of letters.
% \end{function}
%
% Other functions regarding arbitrary tokens can be found in the \textsf{l3token} module.
%
%  \subsection{Grouping and scanning}
%
% \begin{function}{\scan_stop:}
% \begin{syntax}
%   "\scan_stop:"
% \end{syntax}
% This function stops \TeX's scanning ahead when ending a number.
% \begin{texnote}
% This is the \TeX{} primitive \tn{relax} renamed.
% \end{texnote}
% \end{function}
%
%
% \begin{function}{\group_begin:|
%                  \group_end:}
% \begin{syntax}
%   "\group_begin:" <...> "\group_end:"
% \end{syntax}
% Encloses <...> inside a group.
% \begin{texnote}
% These are the \TeX{} primitives \tn{begingroup} and \tn{endgroup}
% renamed.
% \end{texnote}
% \end{function}
%
%
% \subsection{Engine specific definitions}
%
% \begin{function}{\engine_aleph:TF}
% \begin{syntax}
%   "\engine_aleph:TF" "{"<true code>"}" "{"<false code>"}"
% \end{syntax}
% This function detects if we're running an Aleph based format. This
% is particularly useful when allocating registers.
% \end{function}
%
%
%
% \StopEventually{}
%
%
% \section{Implementation}
%    We start by ensuring that the required packages are loaded.  In
%    this case there is a boot-strapping problem since, to use much of
%    the code in this file, the `l3expan' package needs to be loaded;
%    but this file must be loaded first.
%    \begin{macrocode}
%<package>\RequirePackage{l3names}\par
%    \end{macrocode}
%
% \subsection{Renaming some \TeX{} primitives (again)}
%
% Having given all the tex primitives a consistent name, we need to
% give sensible names to the ones we actually want to use.
% These will be defined as needed in the appropriate modules, but
% do a few now, just to get started.\footnote{This renaming gets expensive
% in terms of csname usage, an alternative scheme would be to just use
% the ``tex\ldots D'' name in the cases where no good alternative exists.}
%    \begin{macrocode}
%<*initex|package>
\tex_let:D \let:NwN            \tex_let:D
\let:NwN   \def:Npn            \tex_def:D
\let:NwN   \gdef:Npn           \tex_gdef:D
\let:NwN   \def:Npx            \tex_edef:D
\let:NwN   \gdef:Npx           \tex_xdef:D
%    \end{macrocode}
% Then some conditionals.
%    \begin{macrocode}
\let:NwN   \if_true:           \tex_iftrue:D
\let:NwN   \if_false:          \tex_iffalse:D
\let:NwN   \else:              \tex_else:D
\let:NwN   \fi:                \tex_fi:D
\let:NwN   \reverse_if:N       \etex_unless:D
\let:NwN   \if:w               \tex_if:D
\let:NwN   \if_catcode:w       \tex_ifcat:D
%    \end{macrocode}
% Some different names for |\ifx|.\footnote{MH: Clean up at some point}
%    \begin{macrocode}
\let:NwN   \if_meaning:NN      \tex_ifx:D
\let:NwN   \if_token_eq:NN     \tex_ifx:D
\let:NwN   \if_cs_meaning_eq:NN\tex_ifx:D
%    \end{macrocode}
% \TeX{} lets us detect some if its modes.
%    \begin{macrocode}
\let:NwN   \if_mode_math:      \tex_mmode:D
\let:NwN   \if_mode_horizontal:\tex_hmode:D
\let:NwN   \if_mode_vertical:  \tex_vmode:D
\let:NwN   \if_mode_inner:     \tex_ifinner:D
%    \end{macrocode}
% 
%    \begin{macrocode}
\let:NwN   \if_cs_exist:N      \etex_ifdefined:D
\let:NwN   \if_cs_exist:w      \etex_ifcsname:D
%    \end{macrocode}
% 
%    \begin{macrocode}
\let:NwN   \io_put_deferred:Nx \tex_write:D
\let:NwN   \token_to_meaning:N \tex_meaning:D
\let:NwN   \token_to_string:N  \tex_string:D
\let:NwN   \cs:w               \tex_csname:D
\let:NwN   \cs_end:            \tex_endcsname:D
%    \end{macrocode}
%  The three |\exp_| functions are used in the \textsf{l3expan} module
%  where they are described.
%    \begin{macrocode}
\let:NwN   \exp_after:NN       \tex_expandafter:D
\let:NwN   \exp_not:N          \tex_noexpand:D
\let:NwN   \exp_not:n          \etex_unexpanded:D
%    \end{macrocode}
%  The next three are basic functions for which there also exist
%  versions that are safe inside alignments. These safe versions are
%  defined in the \textsf{l3prg} module.
%    \begin{macrocode}
\let:NwN   \scan_stop:         \tex_relax:D
\let:NwN   \group_begin:       \tex_begingroup:D
\let:NwN   \group_end:         \tex_endgroup:D
%    \end{macrocode}
%
%    These following names are temporary and should be removed as soon
%    as possible (April 1998).
%    \begin{macrocode}
\let:NwN   \the_internal:D     \tex_the:D
\let:NwN   \pref_global:D      \tex_global:D
%    \end{macrocode}
% "\pref_long:D" has been documented for years but didn't exist\dots{}
% Added it and the robustness prefix.
%    \begin{macrocode}
\let:NwN   \pref_long:D        \tex_long:D
\let:NwN   \pref_robust:D      \etex_protected:D
%    \end{macrocode}
%
%
%
% \subsection {Defining and checking (new) functions}
%
%  \begin{macro}{\c_minus_one}
%  \begin{macro}{\c_sixteen}
%    We need the constants |\c_minus_one| and |\c_sixteen| now for
%    writing information to the log and the terminal but the
%    rest are defined in the \textsf{l3num} module -- at least for the
%    ones that can be defined with |\tex_chardef:D| or
%    |\tex_mathchardef:D|. Otherwise the \textsf{l3int} module is
%    required but it can't be used until the allocation has been set
%    up properly! The actual allocation mechanism is in
%    \textsf{l3alloc} and as \TeX{} wants to reserve count registers
%    0--9, the first available one is~10 so we use that for
%    |\c_minus_one|.
%    \begin{macrocode}
%<*!initex>
\let:NwN \c_minus_one\m@ne
%</!initex>
%<*!package>
\tex_countdef:D \c_minus_one = 10 \scan_stop:
\c_minus_one = -1 \scan_stop:
%</!package>
\tex_chardef:D \c_sixteen = 16\scan_stop:
%    \end{macrocode}
%  \end{macro}
%  \end{macro}
%
%    We provide two kinds of functions that can be used to define
%    control sequences. On the one hand we have functions that check
%    if their argument doesn't already exist, they are called
%    |\..._new|. The second type of defining functions doesn't check
%    if the argument is already defined.
%
%    Before we can define them, we need some auxiliary macros that
%    allow us to generate error messages. The definitions here are
%    only temporary, they will be redefined later on.
%
%  \begin{macro}{\io_put_log:x}
%  \begin{macro}{\io_put_term:x}
%    We define a routine to write only to the log file. And a
%    similar one for writing to both the log file and the terminal.
%
%    \begin{macrocode}
\def:Npn \io_put_log:x{
      \tex_immediate:D\io_put_deferred:Nx \c_minus_one }
\def:Npn \io_put_term:x{
      \tex_immediate:D\io_put_deferred:Nx \c_sixteen }
%    \end{macrocode}
%  \end{macro}
%  \end{macro}
%
% \begin{macro}{\err_latex_bug:n}
%    This will show internal errors.
%    \begin{macrocode}
\def:Npn\err_latex_bug:n#1{
   \io_put_term:x{This~is~a~LaTeX~bug!~Check~coding!}\tex_errmessage:D{#1}}
%    \end{macrocode}
% \end{macro}
%
%  \begin{macro}{\cs_record_meaning:N}
%    This macro will be used later on for tracing purposes. But we
%    need some more modules to define it, so we just give some dummy
%    definition here.
%    \begin{macrocode}
%<*trace>
\def:Npn \cs_record_meaning:N#1{}
%</trace>
%    \end{macrocode}
%  \end{macro}
%
% \subsection{Checking new definitions}
%
% \begin{macro}{\chk_new_cs:N}
%    This command is called by |\def_new:Npn| and |\let_new:NN| etc.\
%    to make sure that the argument sequence is not already in use. If
%    it is, an error is signalled.
%    It checks if \m{csname} is undefined or
%    |\scan_stop:|. Otherwise an error message is issued.
%    \begin {macrocode}
\def:Npn \chk_new_cs:N #1{
  \if_meaning:NN #1\c_undefined
  \else:
    \if_meaning:NN #1\scan_stop:
    \else:
      \err_latex_bug:n {Command~name~`\token_to_string:N #1'~
                            already~defined!~
                        Current~meaning:~\token_to_meaning:N #1
                      }
    \fi:
  \fi:
%<*trace>
  \cs_record_meaning:N#1
%     \io_put_term:x{Defining~\token_to_string:N #1~on~%}
  \io_put_log:x{Defining~\token_to_string:N #1~on~
                line~\tex_the:D \tex_inputlineno:D}
%</trace>
}
%    \end{macrocode}
% \end {macro}
%
% \begin{macro}{\cs_exist_p:N}
%    Expands into |\c_true| if the control sequence given as its
%    argument \emph{is} in use.
%    \begin{macrocode}
\def:Npn \cs_exist_p:N #1{
  \if:w \cs_free_p:N #1
    \c_false
  \else:
    \c_true \fi:}
%    \end{macrocode}
% \end{macro}
%
% \begin{macro}{\chk_exist_cs:N }
% \begin{macro}{\chk_exist_cs:c }
%    This function issues a warning message when the control sequence
%    in its argument does not exist.
%    \begin{macrocode}
\def:Npn \chk_exist_cs:N #1 {
  \if:w \cs_exist_p:N #1
  \else:
    \err_latex_bug:n{Command~ `\token_to_string:N #1'~
                     not~ yet~ defined!}
  \fi:}
\def:Npn \chk_exist_cs:c #1 {
  \exp_after:NN \chk_exist_cs:N \cs:w #1\cs_end: }
%    \end{macrocode}
% \end{macro}
% \end{macro}
%
% \begin{macro}{\cs_free_p:N}
%    Expands into |\c_true| if the control sequence given as its
%    argument is not yet in use.  Note that we make sure to expand
%    into |\c_false | if the control sequence is textually
%    |\c_undefined| or |\scan_stop:|, so that we don't end up
%    (re)defining them.
%    \begin{macrocode}
\def:Npn \cs_free_p:N #1{
  \if_meaning:NN#1\c_undefined
    \if:w\cs_if_eq_p:NN #1\c_undefined \c_false
    \else: \c_true \fi:
  \else:
    \if_meaning:NN#1\scan_stop:
      \if:w\cs_if_eq_p:NN #1\scan_stop: \c_false
       \else: \c_true \fi:
     \else:
       \c_false \fi:\fi:}
%    \end{macrocode}
% \end{macro}
%
% \subsection{More new definitions}
%
%  \begin{macro}{\let:NN}
%  \begin{macro}{\let:cN}
%  \begin{macro}{\let:Nc}
%  \begin{macro}{\let:cc}
%    These macros allow us to copy the definition of a control sequence
%    to another control sequence.
%    \begin{macrocode}
\tex_long:D\def:Npn \let:NN #1{
%    \end{macrocode}
%    The |=| sign allows us to define funny char tokens like .|=|
%    itself or \verb*= = with this function. For the definition of
%    |\c_space_chartok{~}| to work we need the |~| after the |=|
%    \begin{macrocode}
                              \let:NwN #1=~}
\def:Npn\let:cN #1 {\exp_after:NN\let:NwN\cs:w#1\cs_end:=~}
\def:Npn\let:Nc{\exp_args:NNc\let:NN}
\def:Npn\let:cc{\exp_args:Ncc\let:NN}
%    \end{macrocode}
%  \end{macro}
%  \end{macro}
%  \end{macro}
%  \end{macro}
%
%
% \begin {macro}{\def_new:Npn}
% \begin {macro}{\def_new:Npx}
% \begin {macro}{\let_new:NN}
% \begin {macro}{\let_new:cN}
% \begin {macro}{\let_new:Nc}
%    These are like |\def:Npn| and |\let:NN|, but they first check that
%    the argument command is not already in use. You may use
%    |\tex_global:D|, |\tex_long:D|, and |\tex_outer:D| as
%    prefixes.
%     \begin {macrocode}
\def:Npn \def_new:Npn #1{\chk_new_cs:N #1
                         \def:Npn #1}
\def:Npn \def_new:Npx #1{\chk_new_cs:N #1
                         \def:Npx #1}
\def_new:Npn \let_new:NN #1{\chk_new_cs:N #1
                              \let:NN #1}
\def_new:Npn \let_new:cN {\exp_args:Nc \let_new:NN}
\def_new:Npn \let_new:Nc {\exp_args:NNc \let_new:NN}
%    \end{macrocode}
% \end {macro}
% \end {macro}
% \end {macro}
% \end {macro}
% \end {macro}
%
% \begin {macro}{\def:cpn}
% \begin {macro}{\def:cpx}
% \begin {macro}{\def_new:cpn}
% \begin {macro}{\def_new:cpx}
%    Like |\def:Npn| and |\def_new:Npn|, except that the first
%    argument consists of the sequence of characters that should be
%    used to form the name of the desired control sequence ( the |c|
%    stands for csname argument, see the expansion module.).
%
%    |\def:cpn|\m{string}\m{rep-text} will turn \m{string} into a
%    csname and then assign \m {rep-text} to it by using |\def:Npn|.
%    This means that there might be a parameter string between the two
%    arguments.
%    \begin{macrocode}
\def_new:Npn \def:cpn #1{\exp_after:NN
                             \def:Npn
                             \cs:w #1\cs_end:}
\def_new:Npn \def:cpx #1{\exp_after:NN
                             \def:Npx
                             \cs:w #1\cs_end:}
\def_new:Npn \def_new:cpn #1{\exp_after:NN
                                  \def_new:Npn
                                  \cs:w #1\cs_end:}
\def_new:Npn \def_new:cpx #1{\exp_after:NN
                                  \def_new:Npx
                                  \cs:w #1\cs_end:}
%    \end{macrocode}
% \end {macro}
% \end {macro}
% \end {macro}
% \end {macro}
%
%
%  \begin{macro}{\def:No}
%    |\def:No| expands its second argument one time before making
%    the definition.
%    \begin{macrocode}
\def_new:Npn \def:No{\exp_args:NNo\def:Npn}
%    \end{macrocode}
%  \end{macro}
%
% \begin {macro}{\def_long:Npn}
% \begin {macro}{\def_long_new:Npn}
% \begin {macro}{\def_long:cpn}
% \begin {macro}{\def_long:Npx}
% \begin{macro}{\def_long:cpx}
%    |\def_long:Npn| stands for |\tex_long:D| |\def:Npn|.
%    |\def_long:Npx| expands its second argument.
%    \begin{macrocode}
\def_new:Npn \def_long:Npn {\tex_long:D\def:Npn}
\def_new:Npn \def_long_new:Npn #1{\chk_new_cs:N #1
                                 \def_long:Npn #1}
\def_new:Npn \def_long:cpn #1{\exp_after:NN
                              \def_long:Npn
                              \cs:w #1\cs_end:}
\def_new:Npn \def_long:Npx {\tex_long:D\def:Npx}
\def_new:Npn \def_long:cpx #1{
  \exp_after:NN\def_long:Npx\cs:w #1\cs_end:
}
%    \end{macrocode}
% \end {macro}
% \end {macro}
% \end {macro}
% \end {macro}
% \end{macro}
%
% \begin {macro}{\glet:NN}
% \begin {macro}{\glet:cN}
% \begin {macro}{\glet:Nc}
% \begin {macro}{\glet:cc}
% \begin {macro}{\glet_new:NN}
% \begin {macro}{\glet_new:cN}
% \begin {macro}{\gdef_new:Npn}
% \begin {macro}{\gdef:cpn}
% \begin {macro}{\gdef:cpx}
% \begin {macro}{\gdef:No}
%  These are global versions of some of the previously defined functions.
%    \begin{macrocode}
\def_new:Npn \glet:NN {\tex_global:D \let:NN}
\def_new:Npn \glet:Nc {\exp_args:NNc \glet:NN}
\def_new:Npn \glet:cN {\exp_args:Nc \glet:NN}
\def_new:Npn \glet:cc {\exp_args:Ncc \glet:NN}
\def_new:Npn \glet_new:NN #1{\chk_new_cs:N #1
                               \tex_global:D\let:NN #1}
\def_new:Npn \glet_new:cN {\exp_args:Nc \glet_new:NN}
\def_new:Npn \gdef_new:Npn #1{\chk_new_cs:N #1
                              \gdef:Npn #1}
\def_new:Npn \gdef:cpn {\tex_global:D \def:cpn}
\def_new:Npn \gdef:cpx {\tex_global:D \def:cpx}
\def_new:Npn \gdef:No  {\exp_args:NNo\gdef:Npn}
%    \end{macrocode}
% \end {macro}
% \end {macro}
% \end {macro}
% \end {macro}
% \end {macro}
% \end {macro}
% \end {macro}
% \end {macro}
% \end {macro}
% \end {macro}
%
% \begin {macro}{\gdef_long:Npn}
% \begin {macro}{\gdef_long:Npx}
% \begin {macro}{\gdef_long:cpn}
% \begin {macro}{\gdef_long:cpx}
%    |\gdef_long:Npn| stands for |\tex_long:D| |\tex_global:D| |\def:Npn|.
%    |\gdef_long:Npx| expands its second argument.
%    \begin {macrocode}
\def_new:Npn \gdef_long:Npn {\tex_long:D\gdef:Npn}
\def_new:Npn \gdef_long:Npx {\tex_long:D\gdef:Npx}
\def_new:Npn \gdef_long:cpn {\tex_global:D \def_long:cpn}
\def_new:Npn \gdef_long:cpx {\tex_global:D \def_long:cpx}
%    \end{macrocode}
% \end {macro}
% \end {macro}
% \end {macro}
% \end {macro}
%
%
%  \begin{macro}{\def_robust:Npn}
%  \begin{macro}{\def_robust_long:Npn}
%  \begin{macro}{\def_robust_new:Npn}
%  Robust commands.
%    \begin{macrocode}
\def_new:Npn \def_robust:Npn {\pref_robust:D \def:Npn}
\def_new:Npn \def_robust_long:Npn {\pref_robust:D \def_long:Npn}
\def_new:Npn \def_robust_new:Npn #1 {
  \chk_new_cs:N #1
  \def_robust:Npn #1
}
%    \end{macrocode}
%  \end{macro}
%  \end{macro}
%  \end{macro}
%
%
% \subsection{Further checking}
%
% \begin{macro}{\cs_free:NTF}
% \begin{macro}{\cs_free:NT}
% \begin{macro}{\cs_free:NF}
%    The old |\@ifundefined| of \LaTeX{} 2.09 is re-implemented in the
%    function |\cs_free:cTF|, again in a way that |\else:| and |\fi:|
%    are removed. In this implementation this is absolutely
%    necessary because functions inside the conditional parts expect
%    to read further input from outside the conditional.  Actually,
%    the first part of the code below is more general, since it checks
%    \m{csnames} directly and therefore allows both |\scan_stop:| and
%    |\c_undefined|.
%    \begin{macrocode}
\def_long_new:Npn \cs_free:NTF #1{\if:w\cs_free_p:N #1
  \exp_after:NN \use_arg_i:nn \else:
  \exp_after:NN \use_arg_ii:nn \fi:}
\def_long_new:Npn \cs_free:NT #1{\if:w\cs_free_p:N #1
  \exp_after:NN \use_arg_i:n \else:
  \exp_after:NN \use_none:n \fi:}
\def_long_new:Npn \cs_free:NF #1{\if:w\cs_free_p:N #1
  \exp_after:NN \use_none:n \else:
  \exp_after:NN \use_arg_i:n \fi:}
%    \end{macrocode}
% \end{macro}
% \end{macro}
% \end{macro}
%
% \begin{macro}{\cs_free:cTF}
% \begin{macro}{\cs_free:cT}
% \begin{macro}{\cs_free:cF}
%    We have to implement the |c| variants `by hand' because a different
%    test is necessary and I don't want the overhead for the test with
%    |\if:w|.  What a mistake Don made by making this a
%    feature of |\cs:w|. If I'm not totally mistaken this
%    feature alone has cost him more than 600\$ for bug-checks. %^^A$
%    \begin{macrocode}
\def_long_new:Npn \cs_free:cTF #1{
  \exp_after:NN
  \if_meaning:NN \cs:w#1\cs_end: \scan_stop:
    \exp_after:NN \use_arg_i:nn
  \else:
    \exp_after:NN \use_arg_ii:nn
  \fi:}
\def_long_new:Npn \cs_free:cF #1{\exp_after:NN
  \if_meaning:NN \cs:w#1\cs_end: \scan_stop:
  \exp_after:NN \use_none:nn \fi: \use_arg_i:n}
\def_long_new:Npn \cs_free:cT #1{\exp_after:NN
  \if_meaning:NN \cs:w#1\cs_end: \scan_stop:
  \else:
  \exp_after:NN \use_none:nn \fi: \use_arg_i:n}
%    \end{macrocode}
% \end{macro}
% \end{macro}
% \end{macro}
%
% \begin{macro}{\cs_really_free:cTF}
% \begin{macro}{\cs_really_free:cT}
% \begin{macro}{\cs_really_free:cF}
%   These versions are for special control sequences that can only be
%   formed through |\cs:w ... \cs_end:|
%    \begin{macrocode}
\def_long_new:Npn \cs_really_free:cTF #1{
  \if_cs_defined:w #1\cs_end:
    \exp_after:NN \use_arg_ii:nn
  \else:
    \exp_after:NN \use_arg_i:nn
  \fi:}
\def_long_new:Npn \cs_free:cT #1{
  \if_cs_defined:w #1\cs_end:
  \exp_after:NN \use_none:nn \fi: \use_arg_i:n}
\def_long_new:Npn \cs_free:cF #1{
  \if_cs_defined:w #1\cs_end:
  \else:
  \exp_after:NN \use_none:nn \fi: \use_arg_i:n}
%    \end{macrocode}
% \end{macro}
% \end{macro}
% \end{macro}
%
%
% \subsection{Freeing memory}
%
% \begin{macro}{\cs_gundefine:N }
%    The following function is used to free the main memory from the
%    definition of some function that isn't in use any longer.
%    \begin{macrocode}
\def_new:Npn \cs_gundefine:N #1{\glet:NN #1\c_undefined}
%    \end{macrocode}
% \end{macro}
%
%
%  \subsection{Engine specific definitions}
%
%  \begin{macro}{\engine_aleph:TF}
%  In some cases it will be useful to know which engine we're running.
%    \begin{macrocode}
\def_new:Npn \engine_aleph:TF {
  \etex_ifdefined:D \aleph_textdir:D
    \exp_after:NN \use_arg_i:nn
  \else:
    \exp_after:NN \use_arg_ii:nn
  \fi:
}
%    \end{macrocode}
%  \end{macro}
%
% \subsection{Gobbling tokens from input}
%
% \begin{macro}{\use_none:n}
% \begin{macro}{\use_none:nn}
% \begin{macro}{\use_none:nnn}
% \begin{macro}{\use_none:nnnn}
%    To gobble tokens from the input we use a standard naming
%    convention: the number of tokens gobbled is given by the number
%    of |n|'s following the |:| in the name.
%    \begin{macrocode}
\def_long_new:Npn \use_none:n #1{}
\def_long_new:Npn \use_none:nn #1#2{}
\def_long_new:Npn \use_none:nnn #1{\use_none:nn}
\def_long_new:Npn \use_none:nnnn
    {\exp_after:NN\use_none:nn \use_none:nn}
%    \end{macrocode}
% \end{macro}
% \end{macro}
% \end{macro}
% \end{macro}
%
%
% \subsection{Selecting tokens}
%
% \begin{macro}{\use_arg_i:n}
%    This macro grabs its argument and returns it back to the input
%    (with outer braces removed).
%    \begin{macrocode}
\def_long_new:Npn \use_arg_i:n #1{#1}
%    \end{macrocode}
% \end{macro}
%
% \begin{macro}{\use:c}
% \begin{macro}{\cs_use:c}
% \begin{macro}{\use:cc}
%    This macro grabs its argument and returns a csname from it.
%    \begin{macrocode}
\def_new:Npn \use:c #1{\cs:w #1\cs_end:}
\def_new:Npn \cs_use:c #1 { \cs:w#1\cs_end: }
%    \end{macrocode}
%    THE NAME IS COMPLETELY WRONG!!!!!
%    \begin{macrocode}
\def_new:Npn \use:cc #1#2
  {\cs:w #1\exp_after:NN\cs_end:\cs:w #2\cs_end:}
%    \end{macrocode}
% \end{macro}
% \end{macro}
% \end{macro}
%
% \begin{macro}{\use_arg_i:nn}
% \begin{macro}{\use_arg_ii:nn}
%    These macros are needed to provide functions with true and false
%    cases, as introduced by Michael some time ago. By using
%    |\exp_after:NN| |\use_arg_i:nn | |\else:| constructions it
%    is possible to write code where the true or false case is able to
%    access the following tokens from the input stream, which is not
%    possible if the |\c_true| syntax is used.
%    \begin{macrocode}
\def_long_new:Npn \use_arg_i:nn #1#2{#1}
\def_long_new:Npn \use_arg_ii:nn #1#2{#2}
%    \end{macrocode}
% \end{macro}
% \end{macro}
%
%
%
%  \begin{macro}{\use_arg_i:nnn}
%  \begin{macro}{\use_arg_ii:nnn}
%  \begin{macro}{\use_arg_iii:nnn}
%    We also need something for picking up arguments from a longer
%    list.
%    \begin{macrocode}
\def_long_new:Npn\use_arg_i:nnn#1#2#3{#1}
\def_long_new:Npn\use_arg_ii:nnn#1#2#3{#2}
\def_long_new:Npn\use_arg_iii:nnn#1#2#3{#3}
%    \end{macrocode}
%  \end{macro}
%  \end{macro}
%  \end{macro}
%
%
%
% \subsection{Scratch functions}
%
% \begin{macro}{\gtmp:w}
%    This function is for global scratch definitions that are used
%    immediately afterwards. It should be used when we need a function
%    that operates on input, i.e.\ has arguments. If we want to save
%    only some tokens for later use, token-list scratch variables
%    should be used.
%    \begin{macrocode}
\def_new:Npn \gtmp:w {}
%    \end{macrocode}
% \end{macro}
%
% \begin{macro}{\tmp:w}
%    This is a local version of the previous function.
%    \begin{macrocode}
\def_new:Npn \tmp:w {}
%    \end{macrocode}
% \end{macro}
%
% \begin{macro}{\use_noop:}
%    I don't think this function belongs here, but one place is as
%    good as any other. I want to use this function when I want to
%    express `no operation'.
%    \begin{macrocode}
\def_new:Npn \use_noop: {}
%    \end{macrocode}
% \end{macro}
%
% \section{Strings and input stream token lists}
%
% \begin{macro}{\cs_to_str:N}
%   This converts a control sequence into the character string of its
%   name, removing the leading escape character.
%    \begin{macrocode}
\def_new:Npn \cs_to_str:N {\exp_after:NN\use_none:n \token_to_string:N}
%    \end{macrocode}
% \end{macro}
%
%
% \begin{macro}{\str_if_eq_p:nn}
%   Takes 2 lists of characters as arguments and expands into
%   |\c_true| if they are equal, and |\c_false| otherwise. I'm
%    pretty sure that the non expandable version above is faster, but
%    it's not expandable so we also include this one.
%    \begin{macrocode}
\def_new:Npn \str_if_eq_p:nn #1#2{\str_if_eq_p_aux:w #1\scan_stop:\\#2\scan_stop:\\}
\def_new:Npn \str_if_eq_p_aux:w #1#2\\#3#4\\{
  \if_meaning:NN#1#3
    \if_meaning:NN#1\scan_stop:\c_true \else:
    \if_meaning:NN#3\scan_stop:\c_false \else:
    \str_if_eq_p_aux:w #2\\#4\\\fi:\fi:
  \else:\c_false \fi:}
%    \end{macrocode}
% \end{macro}
%
% \begin{macro}{\cs_if_eq_p:NN}
%   An application of the above function, already streamlined for
%   speed, so I put it in here.  It takes two control sequences as
%   arguments and expands into true iff they have the same name.
%   We make it long in case one of them is \cs{par}!
%    \begin{macrocode}
\def_long_new:Npn \cs_if_eq_p:NN #1#2{
  \exp_after:NN\exp_after:NN
  \exp_after:NN\str_if_eq_p_aux:w
  \exp_after:NN\token_to_string:N
  \exp_after:NN#1
  \exp_after:NN\scan_stop:
  \exp_after:NN\\
  \token_to_string:N#2\scan_stop:\\}
%    \end{macrocode}
% \end{macro}
%
% \begin{macro}{\cs_if_eq:NNT}
% \begin{macro}{\cs_if_eq:cNT}
% Check if two control sequences are identical.
%    \begin{macrocode}
\def:Npn \cs_if_eq:NNT #1#2 {
  \if_meaning:NN #1#2
    \exp_after:NN \use_arg_i:n
  \else:
    \exp_after:NN \use_none:n
  \fi:
}
\def:Npn \cs_if_eq:cNT {\exp_args:Nc \cs_if_eq:NNT}
%    \end{macrocode}
% \end{macro}
% \end{macro}
%
%  Finally some code that is needed as we
%  do not distribute the file module at the moment (so we simply define
%  the needed function via an existing \LaTeX{} command) and some other
%  stuff which was set up elsewhere, in undistributed modules.
%    \begin{macrocode}
\def_new:Npn\file_not_found:nTF #1#2#3{\IfFileExists{#1}{#3}{#2}}
%    \end{macrocode}
%
%
% \section{Predicates and conditionals}
% \label{sec:predicates}
%
% \LaTeX3 has three concepts for conditional flow processing:
% \begin{enumerate}
%
% \item
%   Functions that carry out a test an then execute, depending on its
% result, either the code supplied in the \m{true arg} or the \m{false
% arg}. These arguments are denoted with "T" and "F" repectively. An
% example would be
% \begin{quote}
%  "\cs_free:cTF{abc}{...}{...}"
% \end{quote}
% a function that will turn the first argument into a control sequence
% (since its marked as "c") then checks whether this control sequence is
% still free and then depending on the result carry out the code in the
% second argument (true case) or in the third argument (false case).
%
% \item
%   Functions that return a special type of boolean value which can be
% tested by the function "\if:w". All functions of this type
% have names that end with "_p" in the description part. For example
% \begin{quote}
%  "\cs_free_p:N"
% \end{quote}
% would be a predicate function for the same type of test as the
% function above. It would return `true' if its argument (a single token
% denoted by "N") is still free for definition. It would be used in
% constructions like
% \begin{quote}
%  "\if:w \cs_free_p:N \l_foo_bar ... \else: ... \fi:"
% \end{quote}
%
% \item
%   Actually there is a third one, namely the original concept used in
% plain \TeX{}. This belongs to the second form but needs further
% thoughts.
% \end{enumerate}
%
% Important to note is that conditionals with \m{true code} and/or
% \m{false code} are always defined in a way that the code of the chosen
% alternative can operate on following tokens in the input stream while
% the predicate implementations always have an "\else:" or "\fi:"
% interfering. This can be important in scanner implementations.
%
% \subsection{Predicate implementation}
%
%    I think Michael originated the idea of expandable boolean tests.
%    At first these were supposed to expand into either \texttt{TT} or
%    \texttt{TF} to be tested using |\if:w| but this was later changed
%    to \texttt{00} and \texttt{01}, so they can be used in logical
%    operations (see the \textsf{l3prg} module).
%
% \begin{macro}{\c_true}
% \begin{macro}{\c_false}
%    Here are the canonical boolean values.
%    \begin{macrocode}
\def_new:Npn \c_true  {00}
\def_new:Npn \c_false {01}
%    \end{macrocode}
% \end{macro}
% \end{macro}
%
%    \begin{macrocode}
%</initex|package>
%<*showmemory>
\showMemUsage
%</showmemory>
%    \end{macrocode}
%
% \endinput
%
%  $Log$
%  Revision 1.21  2005/10/27 22:35:13  morten
%  Cleaned up a bit and moved some code to l3num.
%
%  Revision 1.20  2005/07/12 16:10:24  morten
%  Allow \cs_if_eq_p:NN to accept \par. (Yes, real life experience)
%
%  Revision 1.19  2005/04/06 21:34:55  morten
%  Added more functions. Moved \engine_aleph:TF from l3prg
%
%  Revision 1.18  2005/03/25 23:52:24  braams
%  The definition of \next needs to be seen at initex time,
%  not at typesetting time
%
%  Revision 1.17  2005/03/22 23:28:46  morten
%  Moved some code to l3prg, removed obsolete code, cleaned up documentation.
%
%  Revision 1.16  2005/03/16 22:31:44  braams
%  Changed definition of \c_zero and \c_minus_one
%
%  Revision 1.15  2005/03/15 23:16:27  braams
%  Some tweaking with the docstrip guards to get the documentation to
%  come out right.
%
%  Revision 1.14  2005/03/15 22:50:00  braams
%  Made loadable by initex as the second file
%
%  Revision 1.13  2005/03/11 21:32:21  braams
%  Fixed the use of RCS information;
%  moved \RequirePackage after \StopEventually
%

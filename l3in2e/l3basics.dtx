% \iffalse
%% File: l3basics.dtx Copyright (C) 1990-1998 LaTeX3 project
%
%<*dtx>
          \ProvidesFile{l3basics.dtx}
%</dtx>
%<package>\NeedsTeXFormat{LaTeX2e}
%<package>\ProvidesPackage{l3basics}
%<driver> \ProvidesFile{l3basics.drv}
% \fi
%         \ProvidesFile{l3basics.dtx}
          [1998/04/14 v1.0f L3 Experimental basic definitions]
%
% \iffalse
%<*driver>
\documentclass{l3doc}

\begin{document}
\DocInput{l3basics.dtx}
\end{document}
%</driver>
% \fi
%
%<package>\RequirePackage{l3names}
%
% \GetFileInfo{l3basics.dtx}
% \title{The \textsf{l3basics} package\thanks{This file
%         has version number \fileversion, last
%         revised \filedate.}\\
% Basic Definitions}
% \author{\Team}
% \date{\filedate}
% \maketitle
%
%
% As the name suggest this package holds some basic definitions which
% are needed by most or all other packages in this set.
%
%
% 
% \section{Renaming some \TeX{} primitives (again)}
%
% Having given all the tex primitives a consistent name, we need to
% give sensible names to the ones we actually want to use.
% These will be defined as needed in the apropriate modules, but
% do a few now, just to get started.\footnote{This renaming gets expensive
% in terms of csname usage, an alternative scheme would be to just use
% the ``tex\ldots D'' name in the cases where no good alternative exists.}
%    \begin{macrocode}
%<*package>
\tex_let:D \let:NwN            \tex_let:D
\let:NwN   \def:Npn            \tex_def:D
\let:NwN   \gdef:Npn           \tex_gdef:D
\let:NwN   \def:Npx            \tex_edef:D
\let:NwN   \gdef:Npx           \tex_xdef:D
\let:NwN   \if:w               \tex_if:D
\let:NwN   \if_num:w           \tex_ifnum:D
\let:NwN   \if_meaning:NN      \tex_ifx:D
\let:NwN   \else:              \tex_else:D
\let:NwN   \fi:                \tex_fi:D
\let:NwN   \io_put_deferred:Nx \tex_write:D
\let:NwN   \token_to_meaning:N \tex_meaning:D
\let:NwN   \token_to_string:N  \tex_string:D
\let:NwN   \cs:w               \tex_csname:D
\let:NwN   \cs_end:            \tex_endcsname:D
\let:NwN   \exp_after:NN       \tex_expandafter:D
\let:NwN   \scan_stop:         \tex_relax:D
\let:NwN   \exp_not:N          \tex_noexpand:D
%    \end{macrocode}
%
% \section {Defining (new) functions}
%
%    We need some constants now, that should actually all be defined
%    together. We try to avoid using count registers for constants as
%    much as possible.
%
%    \begin{macrocode}
\let:NwN\c_minus_one\m@ne
\tex_chardef:D \c_one     = 1\scan_stop:
\tex_chardef:D \c_sixteen = 16\scan_stop:
\tex_mathchardef:D \c_two_hundred_fifty_six = 256\scan_stop:
%    \end{macrocode}
%
%    We provide two kinds of functions that can be used to define
%    control sequences. On the one hand we have functions that check
%    if their argument doesn't already exist, they are called
%    |\..._new|. The second type of defining functions doesn't check
%    if the argument is already defined.
%
%    Before we can define them, we need some auxiliary macros that
%    allow us to generate error messages. The definitions here are 
%    only temporary, they will be redefined later on.
%
%  \begin{macro}{\io_put_log:x}
%  \begin{macro}{\io_put_term:x}
%    We define a routine to write only to the log file. And a
%    similar one for writing to both the log file and the terminal.
%
%    \begin{macrocode}
\def:Npn \io_put_log:x{
      \tex_immediate:D\io_put_deferred:Nx \c_minus_one }
\def:Npn \io_put_term:x{
      \tex_immediate:D\io_put_deferred:Nx \c_sixteen }
%    \end{macrocode}
%  \end{macro}
%  \end{macro}
%
% \begin{macro}{\err_latex_bug:n}
%    This will show internal errors.
%    \begin{macrocode}
\def:Npn\err_latex_bug:n#1{
   \io_put_term:x{This~is~a~LaTeX~bug!~Check~coding!}\tex_errmessage:D{#1}}
%    \end{macrocode}
% \end{macro}
%
% \begin{macro}{\chk_new_cs:N}
%    This command is called by |\def_new:Npn| and |\let_new:NN| etc.\ 
%    to make sure that the argument sequence is not already in use. If
%    it is, an error is signalled.
%    \begin {macrocode}
\def:Npn \chk_new_cs:N #1{
     \if_meaning:NN #1\c_undefined
     \else:
       \if_meaning:NN #1\scan_stop:
       \else:
         \err_latex_bug:n {Command~name~`\token_to_string:N #1'~
                               already~defined!~
                           Current~meaning:~\token_to_meaning:N #1
                          }
       \fi:
     \fi:
%<*trace>
     \cs_record_meaning:N#1
%     \io_put_term:x{Defining~\token_to_string:N #1~on~
     \io_put_log:x{Defining~\token_to_string:N #1~on~
                         line~\tex_the:D \tex_inputlineno:D}
%</trace>
   }
%    \end{macrocode}
% \end {macro}
%
%
%  \begin{macro}{\cs_record_meaning:N}
%    This macro will be used later on for tracing purposes. But we
%    need some more modules to define it, so we just give some dummy
%    definition here.
%    \begin{macrocode}
%<*trace>
\def:Npn \cs_record_meaning:N#1{}
%</trace>
%    \end{macrocode}
%  \end{macro}
%
%  \begin{macro}{\let:NN}
%  \begin{macro}{\let:cN}
%  \begin{macro}{\let:Nc}
%  \begin{macro}{\let:cc}
%    These macros allow us to copy the definition of a control sequence
%    to another control sequence. Maybe we should implement a few more
%    of these.
%
%    \begin{macrocode}
\tex_long:D\def:Npn \let:NN #1{
%    \end{macrocode}
%    The |=| sign allows us to define funny char tokens like .|=|
%    itself or \verb*= = with this function. For the definition of
%    |\c_space_chartok{~}| to work we need the |~| after the |=|
%
%    \begin{macrocode}
                              \let:NwN #1=~}
\def:Npn\let:cN #1 {\exp_after:NN\let:NwN\cs:w#1\cs_end:=~}
\def:Npn\let:Nc{\exp_args:NNc\let:NN}
\def:Npn\let:cc{\exp_args:Ncc\let:NN}
%    \end{macrocode}
%  \end{macro}
%  \end{macro}
%  \end{macro}
%  \end{macro}
%
%
% \begin {macro}{\def_new:Npn}
% \begin {macro}{\def_new:Npx}
% \begin {macro}{\let_new:NN}
%    These are like |\def:Npn| and |\let:NN|, but they first check that
%    the argument command is not already in use. You may use
%    |\tex_global:D|, |\tex_long:D|, and |\tex_outer:D| as
%    prefixes.
%     \begin {macrocode}
\def:Npn \def_new:Npn #1{\chk_new_cs:N #1
                         \def:Npn #1}
\def:Npn \def_new:Npx #1{\chk_new_cs:N #1
                         \def:Npx #1}
\def_new:Npn \let_new:NN #1{\chk_new_cs:N #1
                              \let:NN #1}
%    \end{macrocode}
% \end {macro}
% \end {macro}
% \end {macro}
%
% \begin {macro}{\def:cpn}
% \begin {macro}{\def:cpx}
% \begin {macro}{\def_new:cpn}
% \begin {macro}{\def_new:cpx}
%    Like |\def:Npn| and |\def_new:Npn|, except that the first
%    argument consists of the sequence of characters that should be
%    used to form the name of the desired control sequence ( the |c|
%    stands for csname argument, see the expansion module.).
%
%    |\def:cpn|\m{string}\m{rep-text} will turn \m{string} into a
%    csname and then assign \m {rep-text} to it by using |\def:Npn|.
%    This means that there might be a parameter string between the two
%    arguments.
%    \begin {macrocode}
\def_new:Npn \def:cpn #1{\exp_after:NN
                             \def:Npn
                             \cs:w #1\cs_end:}
\def_new:Npn \def:cpx #1{\exp_after:NN
                             \def:Npx
                             \cs:w #1\cs_end:}
\def_new:Npn \def_new:cpn #1{\exp_after:NN
                                  \def_new:Npn
                                  \cs:w #1\cs_end:}
\def_new:Npn \def_new:cpx #1{\exp_after:NN
                                  \def_new:Npx
                                  \cs:w #1\cs_end:}
%    \end{macrocode}
% \end {macro}
% \end {macro}
% \end {macro}
% \end {macro}
%
%
%  \begin{macro}{\def:No}
%    |\def:No| expands its second argument one time before making 
%    the definition.
%    \begin{macrocode}
\def_new:Npn \def:No{\exp_args:NNo\def:Npn}
%    \end{macrocode}
%  \end{macro}
%
% \begin {macro}{\def_long:Npn}
% \begin {macro}{\def_long_new:Npn}
% \begin {macro}{\def_long:cpn}
% \begin {macro}{\def_long:Npx}
%    |\def_long:Npn| stands for |\tex_long:D| |\def:Npn|.
%    |\def_long:Npx| expands its second argument.
%    \begin {macrocode}
\def_new:Npn \def_long:Npn {\tex_long:D\def:Npn}
\def_new:Npn \def_long_new:Npn #1{\chk_new_cs:N #1
                                 \def_long:Npn #1}
\def_new:Npn \def_long:cpn #1{\exp_after:NN
                              \def_long:Npn
                              \cs:w #1\cs_end:}
\def_new:Npn \def_long:Npx {\tex_long:D\def:Npx}
%    \end{macrocode}
% \end {macro}
% \end {macro}
% \end {macro}
% \end {macro}
%
% \begin {macro}{\glet:NN}
% \begin {macro}{\glet_new:NN}
% \begin {macro}{\gdef_new:Npn}
% \begin {macro}{\gdef:cpn}
% \begin {macro}{\gdef:cpx}
% \begin {macro}{\gdef:No}
%  These are global versions of some of the previosly defined functions.
%    \begin {macrocode}
\def_new:Npn \glet:NN {\tex_global:D \let:NN}
\def_new:Npn \glet_new:NN #1{\chk_new_cs:N #1
                               \tex_global:D\let:NN #1}
\def_new:Npn \gdef_new:Npn #1{\chk_new_cs:N #1
                              \gdef:Npn #1}
\def_new:Npn \gdef:cpn {\tex_global:D \def:cpn}
\def_new:Npn \gdef:cpx {\tex_global:D \def:cpx}
\def_new:Npn \gdef:No  {\exp_args:NNo\gdef:Npn}
%    \end{macrocode}
% \end {macro}
% \end {macro}
% \end {macro}
% \end {macro}
% \end {macro}
% \end {macro}

% \begin {macro}{\gdef_long:Npn}
% \begin {macro}{\gdef_long:Npx}
%    |\gdef_long:Npn| stands for |\tex_long:D| |\tex_global:D| |\def:Npn|.
%    |\gdef_long:Npx| expands its second argument.
%    \begin {macrocode}
\def_new:Npn \gdef_long:Npn {\tex_long:D\gdef:Npn}
\def_new:Npn \gdef_long:Npx {\tex_long:D\gdef:Npx}
%    \end{macrocode}
% \end {macro}
% \end {macro}
%
% \begin {macro}{\gfuturelet:NNN}
%    This is a global version of |\let_peek_after:NNN|
%    \begin {macrocode}
\def_new:Npn \gfuturelet:NNN{\tex_global:D \let_peek_after:NNN}
%    \end{macrocode}
% \end {macro}
%
%
%
% \subsection{Freeing memory}
%
% \begin{macro}{\gundefine:N }
%    The following function is used to free the main memory from the
%    definition of some function that isn't in use any longer.
%    \begin{macrocode}
\def_new:Npn \gundefine:N #1{\glet:NN #1\c_undefined}
%    \end{macrocode}
% \end{macro}
%
% \subsection{Gobbling tokens from input}
%
% \begin{macro}{\use_none:n}
% \begin{macro}{\use_none:nn}
% \begin{macro}{\use_none:nnn}
% \begin{macro}{\use_none:nnnn}
%    To gobble tokens from the input we use a standard naming
%    convention: the number of tokens gobbled is given by the number
%    of |n|'s following the |:| in the name.
%    \begin{macrocode}
\def_long_new:Npn \use_none:n #1{}
\def_long_new:Npn \use_none:nn #1#2{}
\def_long_new:Npn \use_none:nnn #1{\use_none:nn}
\def_long_new:Npn \use_none:nnnn
    {\exp_after:NN\use_none:nn \use_none:nn}
%    \end{macrocode}
% \end{macro}
% \end{macro}
% \end{macro}
% \end{macro}
%
%
% \subsection{Selecting tokens}
%
% \begin{macro}{\use:n}
%    This macro grabs its argument and returns it back to the input
%    (with outer braces removed).
%    \begin{macrocode}
\def_long_new:Npn \use:n #1{#1}
%    \end{macrocode}
% \end{macro}
%
% \begin{macro}{\use:c}
% \begin{macro}{\use:cc}
%    This macro grabs its argument and returns a csname from it.
%    \begin{macrocode}
\def_new:Npn \use:c #1{\cs:w #1\cs_end:}

%    \end{macrocode}
%    THE NAME IS COMPLETELY WRONG!!!!!
%    \begin{macrocode}
\def_new:Npn \use:cc #1#2
  {\cs:w #1\exp_after:NN\cs_end:\cs:w #2\cs_end:}
%    \end{macrocode}
% \end{macro}
% \end{macro}
%
% \begin{macro}{\use_choice_i:nn}
% \begin{macro}{\use_choice_ii:nn}
%    These macros are needed to provide functions with true and false
%    cases, as introduced by Michael some time ago. By using
%    |\exp_after:NN| |\use_choice_i:nn | |\else:| constructions it
%    is possible to write code where the true or false case is able to
%    access the following tokens from the input stream, which is not
%    possible if the |\c_true| syntax is used.
%    \begin{macrocode}
\def_long_new:Npn \use_choice_i:nn #1#2{#1}
\def_long_new:Npn \use_choice_ii:nn #1#2{#2}
%    \end{macrocode}
% \end{macro}
% \end{macro}
%
%
%
%  \begin{macro}{\use_choice_i:nnn}
%  \begin{macro}{\use_choice_ii:nnn}
%  \begin{macro}{\use_choice_iii:nnn}
%    We also need something for picking up arguments from a longer
%    list.  
%    \begin{macrocode}
\def_long_new:Npn\use_choice_i:nnn#1#2#3{#1}
\def_long_new:Npn\use_choice_ii:nnn#1#2#3{#2}
\def_long_new:Npn\use_choice_iii:nnn#1#2#3{#3}
%    \end{macrocode}
%  \end{macro}
%  \end{macro}
%  \end{macro}
%
%
% \subsection{Scratch functions}
%
% \begin{macro}{\gtmp:w}
%    This function is for global scratch definitions that are used
%    immediately afterwards. It should be used when we need a function
%    that operates on input, i.e.\ has arguments. If we want to save
%    only some tokens for later use, token-list scratch variables
%    should be used.
%    \begin{macrocode}
\def_new:Npn \gtmp:w {}
%    \end{macrocode}
% \end{macro}
%
% \begin{macro}{\tmp:w}
%    This is a local version of the previous function.
%    \begin{macrocode}
\def_new:Npn \tmp:w {}
%    \end{macrocode}
% \end{macro}
%
% \begin{macro}{\use_noop:}
%    I don't think this function belongs here, but one place is as
%    good as any other. I want to use this function when I want to
%    express `no operation'.
%    \begin{macrocode}
\def_new:Npn \use_noop: {}
%    \end{macrocode}
% \end{macro}
%
% \section{Strings and input stream token lists}
%
% \begin{macro}{\cs_to_str:N}
%   This converts a control sequence into the character string of its 
%   name, removing the leading escape character.
%    \begin{macrocode}
\def_new:Npn \cs_to_str:N {\exp_after:NN\use_none:n \token_to_string:N}
%    \end{macrocode}
% \end{macro}
%
% \begin{macro}{\tlist_eq:nnTF}
% \begin{macro}{\tlist_eq:onTF}
%    |\tlist_eq:nnTF| tests whether the first two arguments are equal and
%    executes either its third or fourth argument.
%    |\tlist_eq:onTF| expands its first argument once.
%    \begin{macrocode}
\def_new:Npn \tlist_eq:nnTF #1#2{
  \tlp_gset:Nn \g_testa_tlp {#1}
  \tlp_gset:Nn \g_testb_tlp {#2}
  \if_meaning:NN\g_testa_tlp \g_testb_tlp
    \exp_after:NN\use_choice_i:nn \else:
    \exp_after:NN\use_choice_ii:nn \fi:}
\def_new:Npn \tlist_eq:onTF {\exp_args:No \tlist_eq:nnTF}
%    \end{macrocode}
% \end{macro}
% \end{macro}
%
%
% \begin{macro}{\tlist_empty:nTF}
% \begin{macro}{\tlist_empty:nF}
%    These test whether a token list is empty and act accordingly.
%    \begin{macrocode}
\def_new:Npn \tlist_empty:nTF #1{\tlp_gset:Nn \g_testa_tlp {#1}
  \if_meaning:NN\g_testa_tlp \c_empty_tlp
      \exp_after:NN \use_choice_i:nn \else:
      \exp_after:NN \use_choice_ii:nn \fi:}
\def_new:Npn \tlist_empty:nF #1{\tlp_gset:Nn \g_testa_tlp {#1}
  \if_meaning:NN\g_testa_tlp \c_empty_tlp
      \exp_after:NN \use_none:nn \fi: \use:n}
%    \end{macrocode}
% \end{macro}
% \end{macro}
%
%
% \begin{macro}{\str_eq_p:nn}
%   Takes 2 lists of characters as arguments and expands into
%   |\c_true| if they are equal, and |\c_false| otherwise. I'm
%    pretty sure that the non expandable version above is faster, but
%    it's not expandable so we also include this one.
%    \begin{macrocode}
\def_new:Npn \str_eq_p:nn #1#2{\str_eq_p_aux:w #1\scan_stop:\\#2\scan_stop:\\}
\def_new:Npn \str_eq_p_aux:w #1#2\\#3#4\\{
  \if_meaning:NN#1#3
    \if_meaning:NN#1\scan_stop:\c_true \else:
    \if_meaning:NN#3\scan_stop:\c_false \else:
    \str_eq_p_aux:w #2\\#4\\\fi:\fi:
  \else:\c_false \fi:}
%    \end{macrocode}
% \end{macro}
%
% \begin{macro}{\cs_eq_p:NN}
%   An application of the above function, already streamlined for
%   speed, so I put it in here.  It takes two control sequences as
%   arguments and expands into true iff they have the same name.
%    \begin{macrocode}
\def:Npn \cs_eq_p:NN #1#2{
  \exp_after:NN\exp_after:NN
  \exp_after:NN\str_eq_p_aux:w
  \exp_after:NN\token_to_string:N
  \exp_after:NN#1
  \exp_after:NN\scan_stop:
  \exp_after:NN\\
  \token_to_string:N#2\scan_stop:\\}
%    \end{macrocode}
% \end{macro}
%
% \section{Predicates and conditionals}
% \label{sec:predicates}
%
% \LaTeX3 has three concepts for conditional flow processing:
% \begin{enumerate}
%
% \item
%   Functions that carry out a test an then execute, depending on its
% result, either the code supplied in the <true arg> or the <false
% arg>. These arguments are denoted with "T" and "F" repectively. An
% example would be
% \begin{quote}
%  "\cs_free:cTF{abc}{...}{...}"
% \end{quote}
% a function that will turn the first argument into a control sequence
% (since its marked as "c") then checks whether this control sequence is
% still free and then depending on the result carry out the code in the
% second argument (true case) or in the third argument (false case).
%
% \item
%   Functions that return a special type of boolean value which can be
% tested by the function "\if:w". All functions of this type
% have names that end with "_p" in the description part. For example
% \begin{quote}
%  "\cs_free_p:N"
% \end{quote}
% would be a predicate function for the same type of test as the
% function above. It would return `true' if its argument (a single token
% denoted by "N") is still free for definition. It would be used in
% constructions like
% \begin{quote}
%  "\if:w \cs_free_p:N \l_foo_bar ... \else: ... \fi:"
% \end{quote}
%
% \item
%   Actually there is a third one, namely the original concept used in
% plain \TeX{}. This belongs to the second form but needs further
% thoughts.
% \end{enumerate}
%
% Important to note is that conditionals with <true code> and/or <false
% code> are always defined in a way that the code of the chosen
% alternative can operate on following tokens in the input stream while
% the predicate implementations always have an "\else:" or "\fi:"
% interfering. This can be important in scanner implementations.
%
% \subsection{Internal functions}
%
% \begin{function}{%
%                  \use_choice_i:nn |
%                  \use_choice_ii:nn |
% }
% \begin{syntax}
%   "\use_choice_i:nn"  "{" <code1> "}{" <code2> "}"
% \end{syntax}
% Functions that execute the first or second argument respectively,
% after removing the surrounding braces. Used to implement conditionals.
% \end{function}
%
% \begin{function}{%
%                  \use_choice_i:nnn |
%                  \use_choice_ii:nnn |
%                  \use_choice_iii:nnn |
% }
% \begin{syntax}
%   "\use_choice_i:nnn"  "{" <arg1> "}{" <arg2> "}{" <arg2> "}"
% \end{syntax}
% Functions that pick up one of three arguments and execute them after
% removing the surrounding braces. Should be described somewhere else.
% \end{function}
%
% \begin{function}{\use:n}
% \begin{syntax}
%   "\use:n"  "{" <code1> "}"
% \end{syntax}
% Function that executes the next argument after removing the
% surrounding braces. Used to implement conditionals.
% \end{function}
%
% \subsection{Predicates}
%
% \subsection{Variables}
%
% \begin{variable}{%
%                  \c_true | \c_false |
% }
%  Constants that represend `true' or `false', respectively. Used to
% implement predicates.
% \end{variable}
%
% \begin{variable}{\l_testa_tlp |
%                  \l_testb_tlp |
%                  \g_testa_tlp |
%                  \g_testb_tlp |
% }
% All conditionals and predicates never use tmp variables. Instead they
% use the following reserved variables.
% \end{variable}
%
% \subsection{Predicate implementation}
%
%    I think Michael originated the idea of expandable boolean tests.
%    They must expand into either \texttt{TT} or \texttt{TF} and are
%    tested using |\if:w|.
%
%
% \begin{macro}{\c_true}
% \begin{macro}{\c_false}
%    Here are the canonical boolean values.
%    \begin{macrocode}
\def_new:Npn \c_true  {TT}
\def_new:Npn \c_false {TF}
%    \end{macrocode}
% \end{macro}
% \end{macro}
%
%    \begin{macrocode}
%</package>
%<*showmemory>
\showMemUsage
%</showmemory>
%    \end{macrocode}
%    

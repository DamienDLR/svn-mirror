% \iffalse
%% File: l3basics.dtx Copyright (C) 1990-2008 LaTeX3 project
%%
%% It may be distributed and/or modified under the conditions of the
%% LaTeX Project Public License (LPPL), either version 1.3c of this
%% license or (at your option) any later version.  The latest version
%% of this license is in the file
%%
%%    http://www.latex-project.org/lppl.txt
%%
%% This file is part of the ``expl3 bundle'' (The Work in LPPL)
%% and all files in that bundle must be distributed together.
%%
%% The released version of this bundle is available from CTAN.
%%
%% -----------------------------------------------------------------------
%%
%% The development version of the bundle can be found at
%%
%%    http://www.latex-project.org/svnroot/experimental/trunk/
%%
%% for those people who are interested.
%%
%%%%%%%%%%%
%% NOTE: %%
%%%%%%%%%%%
%%
%%   Snapshots taken from the repository represent work in progress and may
%%   not work or may contain conflicting material!  We therefore ask
%%   people _not_ to put them into distributions, archives, etc. without
%%   prior consultation with the LaTeX Project Team.
%%
%% -----------------------------------------------------------------------
%<*driver|package>
\RequirePackage{l3names}
%</driver|package>
%\fi
\GetIdInfo$Id$
       {L3 Experimental basic definitions}
%\iffalse
%<*driver>
%\fi
\ProvidesFile{\filename.\filenameext}
  [\filedate\space v\fileversion\space\filedescription]
%\iffalse
\documentclass[full]{l3doc}
\begin{document}
\DocInput{l3basics.dtx}
\end{document}
%</driver>
% \fi
%
%
% \title{The \textsf{l3basics} package\thanks{This file
%         has version number \fileversion, last
%         revised \filedate.}\\
% Basic Definitions}
% \author{\Team}
% \date{\filedate}
% \maketitle
%
% \begin{documentation}
%
% As the name suggest this package holds some basic definitions which
% are needed by most or all other packages in this set.
%
% Here we describe those functions that are used all over the place. With
% that we mean functions dealing with the construction and testing of
% control sequences. Furthermore the basic parts of conditional
% processing are covered; conditional processing dealing with specific
% data types is described in the modules specific for the respective
% data types.
%
% \section{Predicates and conditionals}
% \label{sec:predicates}
%
% \LaTeX3 has three concepts for conditional flow processing:
% \begin{description}
%
% \item[Branching conditionals]
% Functions that carry out a test and then execute, depending on its
% result, either the code supplied in the \m{true arg} or the \m{false
% arg}.
% These arguments are denoted with "T" and "F" repectively. An
% example would be
% \begin{quote}
%  "\cs_if_free:cTF{abc}" \Arg{true code} \Arg{false code}
% \end{quote}
% a function that will turn the first argument into a control sequence
% (since it's marked as "c") then checks whether this control sequence
% is still free and then depending on the result carry out the code in
% the second argument (true case) or in the third argument (false
% case).
%
% These type of functions are known as `conditionals'; whenever a "TF"
% function is defined it will usually be accompanied by "T" and "F"
% functions as well. These are provided for convenience when the
% branch only needs to go a single way. Package writers are free to
% choose which types to define but the kernel definitions will always
% provide all three versions.
%
% Important to note is that these branching conditionals with \m{true
%   code} and/or \m{false code} are always defined in a way that the
% code of the chosen alternative can operate on following tokens in
% the input stream.
%
% These conditional functions may or may not be fully expandable, but if they 
% are expandable they will be accompanied by a `predicate' for the same test 
% as described below.
%
% \item[Predicates]
%   `Predicates' are functions that return a special type of boolean
%   value which can be tested by the function "\if_predicate:w"
%   or in the boolean expression parser. All functions of this type
%   are expandable and have names that end with "_p" in the
%   description part.  For example,
% \begin{quote}
%  "\cs_if_free_p:N"
% \end{quote}
% would be a predicate function for the same type of test as the
% conditional described above. It would return `true' if its argument (a 
% single token denoted by "N") is still free for definition. It would be used 
% in constructions like
% \begin{quote}
%   "\if_predicate:w \cs_if_free_p:N \l_tmpz_tl" \m{true code} "\else:"
%   \m{false code} "\fi:"
% \end{quote}
% or in expressions utilizing the boolean logic parser:
% \begin{quote}
%   "\bool_if:nTF {" \\
%   "  \cs_if_free_p:N \l_tmpz_tl || \cs_if_free_p:N \g_tmpz_tl " \\
%   "}"
%   \Arg{true code} \Arg{false code} 
% \end{quote}
% 
% Like their branching cousins, predicate functions ensure that all
% underlying primitive "\else:" or "\fi:" have been removed before
% returning the boolean true or false values.\footnote{If defined
%   using the interface provided.}
%
% For each predicate defined, a `predicate conditional' will also exist that
% behaves like a conditional described above.
%
% \item[Primitive conditionals]
% There is a third variety of conditional, which is the original concept used in
% plain \TeX{} and \LaTeX. Their use is discouraged in \pkg{expl3} (although
% still used in low-level definitions) because they are more fragile and
% in many cases require more expansion control (hence more code) than the two 
% types of conditionals described above.
% \end{description}
%
%
% \subsection{Primitive conditionals}
%
% The \eTeX\ engine itself provides many different conditionals. Some
% expand whatever comes after them and others don't. Hence the names
% for these underlying functions will often contain a |:w| part but
% higher level functions are often available. See for instance
% |\intexpr_compare_p:nNn| which is a wrapper for |\if_num:w|.
%
% Certain conditionals deal with specific data types like boxes and
% fonts and are described there. The ones described below are either
% the universal conditionals or deal with control sequences. We will
% prefix primitive conditionals with |\if_|.
% 
% \begin{function}{ \if_true:     / (EXP) |
%                   \if_false:    / (EXP) |
%                   \or:        / (EXP) |
%                   \else:        / (EXP) |
%                   \fi:          / (EXP) |
%                   \reverse_if:N / (EXP) }
% \begin{syntax}
%   "\if_true:" <true code> "\else:" <false code> "\fi:" \\
%   "\if_false:" <true code> "\else:" <false code> "\fi:" \\
%   "\reverse_if:N" <primitive conditional>
% \end{syntax}
% "\if_true:" always executes <true code>, while "\if_false:" always
% executes <false code>. "\reverse_if:N" reverses any two-way primitive
% conditional. "\else:" and "\fi:" delimit the branches of the
% conditional. "\or:" is used in case switches, see \pkg{l3intexpr}
% for more.
% \begin{texnote}
% These are equivalent to their corresponding \TeX\ primitive 
% conditionals; |\reverse_if:N| is \eTeX's |\unless|.
% \end{texnote}
% \end{function}
% 
% \begin{function}{ \if_meaning:w / (EXP) }
% \begin{syntax}
%   "\if_meaning:w" <arg1> <arg2> <true code> "\else:" <false code> "\fi:"
% \end{syntax}
% "\if_meaning:w" executes <true code> when <arg1> and <arg2> are the same, 
% otherwise it executes <false code>. 
% <arg1> and <arg2> could be functions, variables, tokens; in all cases the
% \emph{unexpanded} definitions are compared.
% \begin{texnote}
% This is \TeX's |\ifx|.
% \end{texnote}
% \end{function}
%
% \begin{function}{ \if:w          / (EXP) |
%                   \if_charcode:w / (EXP) |
%                   \if_catcode:w  / (EXP) }
% \begin{syntax}
%   "\if:w" <token1> <token2> <true code> "\else:" <false code> "\fi:" \\
%   "\if_catcode:w" <token1> <token2> <true code> "\else:" <false
%   code> "\fi:"
% \end{syntax}
% These conditionals will expand any following tokens until two
% unexpandable tokens are left. If you wish to prevent this expansion,
% prefix the token in question with "\exp_not:N". "\if_catcode:w"
% tests if the category codes of the two tokens are the same whereas
% "\if:w" tests if the character codes are
% identical. "\if_charcode:w" is an alternative name for "\if:w".
% \end{function}
%
%
% \begin{function}{\if_predicate:w / (EXP)}
% \begin{syntax}
%  "\if_predicate:w" <predicate> <true code> "\else:" <false code> "\fi:"
% \end{syntax}
% This function takes a predicate function and
% branches according to the result.  (In practice this function would also
% accept a single boolean variable in place of the <predicate> but to make the
% coding clearer this should be done through "\if_bool:N".)
% \end{function}
%
%
%
% \begin{function}{\if_bool:N / (EXP)}
% \begin{syntax}
%  "\if_bool:N" <boolean> <true code> "\else:" <false code> "\fi:"
% \end{syntax}
% This function takes a boolean variable and
% branches according to the result.
% \end{function}
%
%
% \begin{function}{ \if_cs_exist:N  / (EXP) |
%                   \if_cs_exist:w  / (EXP) }
% \begin{syntax}
%   "\if_cs_exist:N" <cs> <true code> "\else:" <false code> "\fi:" \\
%   "\if_cs_exist:w" <tokens> "\cs_end:" <true code> "\else:" <false
%   code> "\fi:"
% \end{syntax}
% Check if <cs> appears in the hash table or if the control sequence
% that can be formed from <tokens> appears in the hash table. The
% latter function does not turn the control sequence in question into
% "\scan_stop:"! This can be useful when dealing with control
% sequences which cannot be entered as a single token.
% \end{function}
%
% \begin{function}{
%     \if_mode_horizontal: / (EXP) |
%     \if_mode_vertical:   / (EXP) |
%     \if_mode_math:       / (EXP) |
%     \if_mode_inner:      / (EXP) }
% \begin{syntax}
%   "\if_mode_horizontal:" <true code> "\else:" <false code> "\fi:"
% \end{syntax}
% Execute <true code> if currently in horizontal mode, otherwise
% execute <false code>. Similar for the other functions.
% \end{function}
%
% \subsection{Non-primitive conditionals}
%
% \begin{function}{ \cs_if_eq_name_p:NN }
% \begin{syntax}
%   "\cs_if_eq_name_p:NN" <cs1> <cs2>
% \end{syntax}
% Returns `true' if <cs1> and <cs2> are textually the same, i.e.\ have
% the same name, otherwise it returns `false'.
% \end{function}
%
% \begin{function}{ \cs_if_eq_p:NN / (EXP) |
%                   \cs_if_eq_p:cN / (EXP) |
%                   \cs_if_eq_p:Nc / (EXP) |
%                   \cs_if_eq_p:cc / (EXP) |
%                   \cs_if_eq:NN / (TF)(EXP) |
%                   \cs_if_eq:cN / (TF)(EXP) |
%                   \cs_if_eq:Nc / (TF)(EXP) |
%                   \cs_if_eq:cc / (TF)(EXP) }
% \begin{syntax}
%    "\cs_if_eq_p:NNTF" <cs1> <cs2>
%    "\cs_if_eq:NNTF" <cs1> <cs2> \Arg{true code} \Arg{false code}
% \end{syntax}
% These functions check if <cs1> and <cs2> have same meaning.
% \end{function}
%
%
% \begin{function}{ \cs_if_free_p:N / (EXP)   |
%                   \cs_if_free_p:c / (EXP)   |
%                   \cs_if_free:N / (TF)(EXP) |
%                   \cs_if_free:c / (TF)(EXP) }
% \begin{syntax}
%   "\cs_if_free_p:N" <cs>
%   "\cs_if_free:NTF" <cs> \Arg{true code} \Arg{false code}
% \end{syntax}
% Returns `true' if <cs> is either undefined or equal to
% "\tex_relax:D" (the function that is assigned to newly created
% control sequences by \TeX{} when "\cs:w" "..." "\cs_end:" is
% used). In addition to this, `true' is only returned if <cs> does not
% have a signature equal to "D", i.e., `do not use' functions are not
% free to be redefined.
% \end{function}
%
% \begin{function}{ \cs_if_exist_p:N / (EXP)   |
%                   \cs_if_exist_p:c / (EXP)   |
%                   \cs_if_exist:N / (TF)(EXP) |
%                   \cs_if_exist:c / (TF)(EXP) }
% \begin{syntax}
%   "\cs_if_exist_p:N" <cs>
%   "\cs_if_exist:NTF" <cs> \Arg{true code} \Arg{false code}
% \end{syntax}
% These functions check if <cs> exists, i.e., if <cs> is present in
% the hash table and is not the primitive "\tex_relax:D".
% \end{function}
%
% \begin{function}{ \cs_if_do_not_use_p:N / (EXP) }
% \begin{syntax}
%   "\cs_if_do_not_use_p:N" <cs>
% \end{syntax}
% These functions check if <cs> has the arg spec "D" for `\emph{do not use}'.
% There are no "TF"-type conditionals for this function as it is only used
% internally and not expected to be widely used. (For now, anyway.)
% \end{function}
%
% \begin{function}{ 
%   \chk_if_free_cs:N|
%   \chk_if_free_cs:c
% }
% \begin{syntax}
%   "\chk_if_free_cs:N" <cs>
% \end{syntax}
% This function checks that <cs> is \m{free} according to the criteria
% for "\cs_if_free_p:N" above. If not, an error is generated.
% \end{function}
%
% \begin{function}{ \chk_if_exist_cs:N |
%                   \chk_if_exist_cs:c }
% \begin{syntax}
%   "\chk_if_exist_cs:N" <cs>
% \end{syntax}
% This function checks that <cs> is defined. If it is not an error
% is generated.
% \end{function}
%
% \begin{variable}{ \c_true_bool | \c_false_bool }
% \begin{syntax}
% \end{syntax}
% Constants that represent `true' or `false', respectively. Used to
% implement predicates.
% \end{variable}
%
% \subsection{Applications}
%
% \begin{function}{ \str_if_eq_p:nn / (EXP) }
% \begin{syntax}
%   "\str_if_eq_p:nn" \Arg{string1} \Arg{string2}
% \end{syntax}
% Expands to `true' if <string1> is the same as <string2>, 
% otherwise `false'. Ignores spaces within the strings.
% \end{function}
%
% \begin{function}{ \str_if_eq_var_p:nf / (EXP) }
% \begin{syntax}
%   "\str_if_eq_var_p:nf" \Arg{string1} \Arg{string2}
% \end{syntax}
% A variant of "\str_if_eq_p:nn" which has the advantage of 
% obeying spaces in at least the second argument. 
% \end{function}
%
%
% \section{Control sequences}
%
% \begin{function}{ \cs:w / (EXP) | \cs_end: / (EXP) }
% \begin{syntax}
%   "\cs:w" <tokens> "\cs_end:"
% \end{syntax}
% This is the \TeX{} internal way of generating a control sequence from
% some token list. <tokens> get expanded and must ultimately result in a
% sequence of characters.
% \begin{texnote}
% These functions are the primitives \tn{csname} and \tn{endcsname}.
% "\cs:w" is considered weird because it expands tokens until it reaches
% "\cs_end:".
% \end{texnote}
% \end{function}
%
% \begin{function}{ \cs_show:N |
%                   \cs_show:c }
% \begin{syntax}
%   "\cs_show:N" <cs>
%   "\cs_show:c" \Arg{arg}
% \end{syntax}
% This function shows in the console output the \emph{meaning} of the control 
% sequence <cs> or that created by <arg>.
% \begin{texnote}
%   This is \TeX's |\show| and associated csname version of it.
% \end{texnote}
% \end{function}
%
% \begin{function}{ \cs_meaning:N / (EXP) |
%                   \cs_meaning:c / (EXP) }
% \begin{syntax}
%   "\cs_meaning:N" <cs>
%   "\cs_meaning:c" \Arg{arg}
% \end{syntax}
% This function expands to the \emph{meaning} of the control sequence <cs> or 
% that created by <arg>.
% \begin{texnote}
%   This is \TeX's |\meaning| and associated csname version of it.
% \end{texnote}
% \end{function}
%
% \section{Selecting and discarding tokens from the input stream}
%
%  The conditional processing cannot be implemented without
%  being able to gobble and select which tokens to use from the input
%  stream.
% 
%
% \begin{function}{ \use:n    / (EXP) |
%                   \use:nn   / (EXP) |
%                   \use:nnn  / (EXP) |
%                   \use:nnnn / (EXP) }
% \begin{syntax}
%   "\use:n"  \Arg{arg}
% \end{syntax}
% Functions that returns all of their arguments to the input stream after 
% removing the surrounding braces around each argument.
% \begin{texnote}
% "\use:n" is \LaTeXe's "\@firstofone"/"\@iden".
% \end{texnote}
% \end{function}
%
% \begin{function}{ \use:c / (EXP) }
% \begin{syntax}
%   "\use:c"  \Arg{cs}
% \end{syntax}
% Function that returns to the input stream the control sequence created from
% its argument. Requires two expansions before a control sequence is returned.
% \begin{texnote}
% "\use:c" is \LaTeXe's "\@nameuse".
% \end{texnote}
% \end{function}
%
% \begin{function}{ \use:x }
% \begin{syntax}
%   "\use:x"  \Arg{expandable tokens}
% \end{syntax}
% Function that fully expands its argument before passing it to the input
% stream. Contents of the argument must be fully expandable.
% \begin{texnote}
% Lua\TeX\ provides "\expanded" which performs this operation in an expandable
% manner, but we cannot assume this behaviour on all platforms yet.
% \end{texnote}
% \end{function}
%
% \begin{function}{ \use_none:n         / (EXP) |
%                   \use_none:nn        / (EXP) |
%                   \use_none:nnn       / (EXP) |
%                   \use_none:nnnn      / (EXP) |
%                   \use_none:nnnnn     / (EXP) |
%                   \use_none:nnnnnn    / (EXP) |
%                   \use_none:nnnnnnn   / (EXP) |
%                   \use_none:nnnnnnnn  / (EXP) |
%                   \use_none:nnnnnnnnn / (EXP) }
% \begin{syntax}
%    "\use_none:n"  \Arg{arg1}\\
%    "\use_none:nn" \Arg{arg1} \Arg{arg2}
% \end{syntax}
%  These functions gobble the tokens or brace groups from the input
%  stream.
% \begin{texnote}
% "\use_none:n", "\use_none:nn", "\use_none:nnnn" are \LaTeXe's 
% "\@gobble", "\@gobbletwo", and "\@gobblefour".
% \end{texnote}
% \end{function}
%
% \begin{function}{ \use_i:nn  / (EXP) |
%                   \use_ii:nn / (EXP) }
% \begin{syntax}
%   "\use_i:nn"  \Arg{code1} \Arg{code2}
% \end{syntax}
% Functions that execute the first or second argument respectively,
% after removing the surrounding braces. Primarily used to implement
% conditionals.
% \begin{texnote}
% These are \LaTeXe's "\@firstoftwo" and "\@secondoftwo", respectively.
% \end{texnote}
% \end{function}
%
% \begin{function}{ \use_i:nnn   / (EXP) |
%                   \use_ii:nnn  / (EXP) |
%                   \use_iii:nnn / (EXP) }
% \begin{syntax}
%   "\use_i:nnn"  \Arg{arg1} \Arg{arg2} \Arg{arg3}
% \end{syntax}
% Functions that pick up one of three arguments and execute them after
% removing the surrounding braces.
% \begin{texnote}
% \LaTeXe\ has only "\@thirdofthree".
% \end{texnote}
% \end{function}
%
% \begin{function}{ \use_i:nnnn   / (EXP) |
%                   \use_ii:nnnn  / (EXP) |
%                   \use_iii:nnnn / (EXP) |
%                   \use_iv:nnnn  / (EXP) }
% \begin{syntax}
%   "\use_i:nnnn"  \Arg{arg1} \Arg{arg2} \Arg{arg3} \Arg{arg4}
% \end{syntax}
% Functions that pick up one of four arguments and execute them after
% removing the surrounding braces.
% \end{function}
%
% \subsection{Extending the interface}
%
% \begin{function}{ \use_i_ii:nnn / (EXP) }
% \begin{syntax}
%   "\use_i_ii:nnn"  \Arg{arg1} \Arg{arg2} \Arg{arg3}
% \end{syntax}
% This function used in the expansion module reads three arguments and
% returns (without braces) the first and second argument while
% discarding the third argument. 
%
% If you wish to select multiple arguments while discarding others,
% use a syntax like this. Its definition is
% \begin{verbatim}
%   \cs_set:Npn \use_i_ii:nnn #1#2#3 {#1#2}
% \end{verbatim}
% \end{function}
% 
% \subsection{Selecting tokens from delimited arguments}
%
% A different kind of function for selecting tokens from the token
% stream are those that use delimited arguments.
%
% \begin{function}{ \use_none_delimit_by_q_nil:w            / (EXP) |
%                   \use_none_delimit_by_q_stop:w           / (EXP) |
%                   \use_none_delimit_by_q_recursion_stop:w / (EXP) }
% \begin{syntax}
%    "\use_none_delimit_by_q_nil:w" <balanced text> "\q_nil"
% \end{syntax}
% Gobbles <balanced text>. Useful in gobbling the remainder in a list
% structure or terminating recursion.
% \end{function}
%
% \begin{function}{ \use_i_delimit_by_q_nil:nw            / (EXP) |
%                   \use_i_delimit_by_q_stop:nw           / (EXP) |
%                   \use_i_delimit_by_q_recursion_stop:nw / (EXP) }
% \begin{syntax}
%    "\use_i_delimit_by_q_nil:nw" \Arg{arg} <balanced text> "\q_nil"
% \end{syntax}
% Gobbles <balanced text> and executes <arg> afterwards. This can also
% be used to get the first item in a token list.
% \end{function}
%
%
% \begin{function}{ \use_i_after_fi:nw     / (EXP) |
%                   \use_i_after_else:nw   / (EXP) |
%                   \use_i_after_or:nw     / (EXP) |
%                   \use_i_after_orelse:nw / (EXP) }
% \begin{syntax}
%    "\use_i_after_fi:nw" \Arg{arg} "\fi:"
%    "\use_i_after_else:nw" \Arg{arg} "\else:" <balanced text> "\fi:"
%    "\use_i_after_or:nw" \Arg{arg} "\or:" <balanced text> "\fi:"
%    "\use_i_after_orelse:nw" \Arg{arg} "\or:/\else:" <balanced text> "\fi:"
% \end{syntax}
% Executes <arg> after executing closing out
% "\fi:". "\use_i_after_orelse:nw" can be used anywhere where
% "\use_i_after_else:nw" or "\use_i_after_or:nw" are used.
% \end{function}
%
%
% \section{That which belongs in other modules but needs to be defined earlier}
%
% \begin{function}{ \exp_after:wN / (EXP) }
% \begin{syntax}
% "\exp_after:wN" <token1> <token2>
% \end{syntax}
% Expands <token2> once and then continues processing from <token1>.
% \begin{texnote}
% This is \TeX's \tn{expandafter}.
% \end{texnote} 
% \end{function}
%
% \begin{function}{ \exp_not:N / (EXP) | \exp_not:n / (EXP) }
% \begin{syntax}
% "\exp_not:N" <token>
% "\exp_not:n" \Arg{tokens}
% \end{syntax}
% In an expanding context, this function prevents <token> or <tokens> from
% expanding.
% \begin{texnote}
% These are \TeX's \tn{noexpand} and \eTeX's \tn{unexpanded}, respectively.
% \end{texnote}
% \end{function}
%
% \begin{function}{ \prg_do_nothing: / (EXP) }
% This is as close as we get to a null operation or no-op.
% \begin{texnote}
% Definition as in \LaTeX's \tn{\@empty} but not used for the same thing.
% \end{texnote}
% \end{function}
%
% \begin{function}{ \iow_log:x       |
%                   \iow_term:x      |
%                   \iow_shipout_x:Nn }
% \begin{syntax}
% "\iow_log:x" \Arg{message}
% "\iow_shipout_x:Nn" <write_stream> \Arg{message}
% \end{syntax}
%  Writes <message> to either to log or the terminal.
% \end{function}
%
% \begin{function}{ \msg_kernel_bug:x }
% \begin{syntax}
% "\msg_kernel_bug:x" \Arg{message}
% \end{syntax}
% Internal function for calling errors in our code.
% \end{function}
%
% \begin{function}{ \cs_record_meaning:N }
% Placeholder for a function to be defined by \file{l3chk}.
% \end{function}
%
% \begin{variable}{ \c_minus_one | \c_zero | \c_sixteen }
% Numeric constants.
% \end{variable}
% 
% 
%
% \section{Defining functions}
%
%
% There are two types of function definitions in \LaTeX3:  versions
% that check if the function name is still unused, and versions that
% simply make the definition. The latter are used for internal scratch
% functions that get new meanings all over the place.
%
% For each type there is an additional choice to be made: Does the
% function to be defined contain delimited arguments? The answer in
% 99\% of the cases is no. For this type the programmer will know the
% number of arguments and in most cases use the argument signature to
% signal this, e.g., "\foo_bar:nnn" presumably takes three
% arguments. We therefore also provide functions that automatically
% detect how many arguments are required and construct the parameter
% text on the fly.
%
% A definition of a new function can be done locally and globally. Currently
% nearly all function definitions are done locally on top level, in
% other words they are global but don't show it. Therefore I think it may
% be better to remove the local variants in the future and declare all
% checked function definitions global.
%
% \begin{texnote}
% While \TeX{} makes all definition functions directly available to the
% user \LaTeX3 hides them very carefully to avoid the problems with
% definitions that are overwritten accidentally. Many functions that are in
% \TeX{} a combination of prefixes and definition functions are provided
% as individual functions.
% \end{texnote}
%
% A slew of functions are defined in the following sections for defining
% new functions. Here's a quick summary to get an idea of what's available:
% {\centering 
%  "\cs_"("g")("new"/"set")("_protected")("_nopar")":"("N"/"c")("p")("n"/"x")
%  \par}
% That stands for, respectively, the following variations:
% \begin{description}
% \item[\texttt g] Global or local;
% \item[\texttt{new}/\texttt{set}] 
%    Define a new function or re-define an existing one;
% \item[\texttt{protected}] 
%    Prevent expansion of the function in "x" arguments;
% \item[\texttt{nopar}] Restrict the argument(s) from containing "\par";
% \item[\texttt N/\texttt c] Either a control sequence or a `csname';
% \item[\texttt p] 
%    Either the a primitive \TeX\ argument or the number of arguments
%    is detected from the argument signature, i.e., |\foo:nnn| is
%    assumed to have three arguments |#1#2#3|;
% \item[\texttt n/\texttt x] Either an unexpanded or an expanded definition.
% \end{description}
% That adds up to 128 variations (!). However, the system is very
% logical and only a handful will usually be required often.
%
%
%
% \subsection{Defining new functions using primitive parameter text}
%
%
% \begin{function}{\cs_new:Npn |
%                  \cs_new:Npx |
%                  \cs_new:cpn |
%                  \cs_new:cpx
% }
% \begin{syntax}
%   "\cs_new:Npn" <cs> <parms> \Arg{code}
% \end{syntax}
% Defines a function that may contain "\par" tokens in the argument(s)
% when called. This is not allowed for normal functions.
% \end{function}
%
% \begin{function}{\cs_new_nopar:Npn |
%                  \cs_new_nopar:Npx |
%                  \cs_new_nopar:cpn |
%                  \cs_new_nopar:cpx
% }
% \begin{syntax}
%   "\cs_new_nopar:Npn" <cs> <parms> \Arg{code}
% \end{syntax}
% Defines a new function, making sure that <cs> is unused so far.
% <parms> may consist of arbitrary parameter specification in \TeX{}
% syntax. It is under the responsibility of the programmer to name the
% new function according to the rules laid out in the previous section.
% <code> is either passed literally or may be subject to expansion
% (under the "x" variants).
% \end{function}
%
% \begin{function}{\cs_new_protected:Npn |
%                  \cs_new_protected:Npx |
%                  \cs_new_protected:cpn |
%                  \cs_new_protected:cpx
% }
% \begin{syntax}
%   "\cs_new_protected:Npn" <cs> <parms> \Arg{code}
% \end{syntax}
% Defines a function that is both robust and may contain "\par" tokens
% in the argument(s) when called.
% \end{function}
%
% \begin{function}{\cs_new_protected_nopar:Npn |
%                  \cs_new_protected_nopar:Npx |
%                  \cs_new_protected_nopar:cpn |
%                  \cs_new_protected_nopar:cpx
% }
% \begin{syntax}
%   "\cs_new_protected_nopar:Npn" <cs> <parms> \Arg{code}
% \end{syntax}
% Defines a function that does not expand when inside an |x| type
% expansion.
% \end{function}
%
%
%
%
%
%
%
% \subsection{Defining new functions using the signature}
%
%
% \begin{function}{\cs_new:Nn |
%                  \cs_new:Nx |
%                  \cs_new:cn |
%                  \cs_new:cx
% }
% \begin{syntax}
%   "\cs_new:Nn" <cs> \Arg{code}
% \end{syntax}
% Defines a new function, making sure that <cs> is unused so far. The
% parameter text is automatically detected from the length of the
% function signature. If <cs> is missing a colon in its name, an error
% is raised.  It is under the responsibility of the programmer to name
% the new function according to the rules laid out in the previous
% section.  <code> is either passed literally or may be subject to
% expansion (under the "x" variants).
% \begin{texnote}
% Internally, these use \TeX's "\long". These forms are
% recommended for low-level definitions as experience has shown that "\par"
% tokens often turn up in programming situations that wouldn't have been 
% expected.
% \end{texnote}
% \end{function}
%
% \begin{function}{\cs_new_nopar:Nn |
%                  \cs_new_nopar:Nx |
%                  \cs_new_nopar:cn |
%                  \cs_new_nopar:cx
% }
% \begin{syntax}
%   "\cs_new_nopar:Nn" <cs>  \Arg{code}
% \end{syntax}
% Version of the above in which "\par" is not allowed to appear within the 
% argument(s) of the defined functions.
% \end{function}
%
% \begin{function}{\cs_new_protected:Nn |
%                  \cs_new_protected:Nx |
%                  \cs_new_protected:cn |
%                  \cs_new_protected:cx
% }
% \begin{syntax}
%   "\cs_new_protected:Nn" <cs>  \Arg{code}
% \end{syntax}
% Defines a function that is both robust and may contain "\par" tokens
% in the argument(s) when called.
% \end{function}
%
% \begin{function}{\cs_new_protected_nopar:Nn |
%                  \cs_new_protected_nopar:Nx |
%                  \cs_new_protected_nopar:cn |
%                  \cs_new_protected_nopar:cx
% }
% \begin{syntax}
%   "\cs_new_protected_nopar:Nn" <cs>  \Arg{code}
% \end{syntax}
% Defines a function that does not expand when inside an |x| type
% expansion. "\par" is not allowed in the argument(s) of the defined function.
% \end{function}
%
%
%
% \subsection{Defining functions using primitive parameter text}
%
%
% Besides the function definitions that check whether or not their
% argument is an unused function we need function definitions that
% overwrite currently used definitions. The following functions are
% provided for this purpose.
%
% \begin{function}{\cs_set:Npn |
%                  \cs_set:Npx |
%                  \cs_set:cpn |
%                  \cs_set:cpx |
% }
% \begin{syntax}
%   "\cs_set:Npn" <cs> <parms> \Arg{code}
% \end{syntax}
% Like "\cs_set_nopar:Npn" but allows "\par" tokens in the arguments of the
% function being defined.
% \begin{texnote}
% These are equivalent to \TeX's "\long\def" and so on. These forms are
% recommended for low-level definitions as experience has shown that "\par"
% tokens often turn up in programming situations that wouldn't have been 
% expected.
% \end{texnote}
% \end{function}
%
% \begin{function}{\cs_gset:Npn |
%                  \cs_gset:Npx |
%                  \cs_gset:cpn |
%                  \cs_gset:cpx |
% }
% \begin{syntax}
%   "\cs_gset:Npn" <cs> <parms> \Arg{code}
% \end{syntax}
% Global variant of "\cs_set:Npn".
% \end{function}
%
% \begin{function}{\cs_set_nopar:Npn |
%                  \cs_set_nopar:Npx |
%                  \cs_set_nopar:cpn |
%                  \cs_set_nopar:cpx |
% }
% \begin{syntax}
%   "\cs_set_nopar:Npn" <cs> <parms> \Arg{code}
% \end{syntax}
% Like "\cs_new_nopar:Npn" etc.\ but does not check the <cs> name.
% \begin{texnote}
% "\cs_set_nopar:Npn" is the \LaTeX3 name for \TeX{}'s \tn{def} and 
% "\cs_set_nopar:Npx" corresponds to the primitive \tn{edef}. The 
% "\cs_set_nopar:cpn" function was known in \LaTeX2 as \tn{@namedef}. 
% "\cs_set_nopar:cpx" has no equivalent.
% \end{texnote}
% \end{function}
%
% \begin{function}{\cs_gset_nopar:Npn |
%                  \cs_gset_nopar:Npx |
%                  \cs_gset_nopar:cpn |
%                  \cs_gset_nopar:cpx |
% }
% \begin{syntax}
%   "\cs_gset_nopar:Npn" <cs> <parms> \Arg{code}
% \end{syntax}
% Like "\cs_set_nopar:Npn" but defines the <cs> globally.
% \begin{texnote}
% "\cs_gset_nopar:Npn" and "\cs_gset_nopar:Npx"
% are \TeX's \tn{gdef} and \tn{xdef}.
% \end{texnote}
% \end{function}
%
%
% \begin{function}{\cs_set_protected:Npn |
%                  \cs_set_protected:Npx |
%                  \cs_set_protected:cpn |
%                  \cs_set_protected:cpx |
% }
% \begin{syntax}
%   "\cs_set_protected:Npn" <cs> <parms> \Arg{code}
% \end{syntax}
% Naturally robust macro that won't expand in an |x| type argument.
% These varieties allow |\par| tokens in the arguments of the
% function being defined.
% \end{function}
%
% \begin{function}{\cs_gset_protected:Npn |
%                  \cs_gset_protected:Npx |
%                  \cs_gset_protected:cpn |
%                  \cs_gset_protected:cpx |
% }
% \begin{syntax}
%   "\cs_gset_protected:Npn" <cs> <parms> \Arg{code}
% \end{syntax}
% Global versions of the above functions.
% \end{function}
%
% \begin{function}{\cs_set_protected_nopar:Npn |
%                  \cs_set_protected_nopar:Npx |
%                  \cs_set_protected_nopar:cpn |
%                  \cs_set_protected_nopar:cpx |
% }
% \begin{syntax}
%   "\cs_set_protected_nopar:Npn" <cs> <parms> \Arg{code}
% \end{syntax}
% Naturally robust macro that won't expand in an |x| type argument.
% If you want for some reason to expand it inside an |x| type expansion, 
% prefix it with |\exp_after:wN \prg_do_nothing:|.
% \end{function}
%
% \begin{function}{\cs_gset_protected_nopar:Npn |
%                  \cs_gset_protected_nopar:Npx |
%                  \cs_gset_protected_nopar:cpn |
%                  \cs_gset_protected_nopar:cpx |
% }
% \begin{syntax}
%   "\cs_gset_protected_nopar:Npn" <cs> <parms> \Arg{code}
% \end{syntax}
% Global versions of the above functions.
% \end{function}
%
%
% \subsection{Defining functions using the signature (no checks)}
%
% As above but now detecting the parameter text from inspecting the
% signature.
%
% \begin{function}{\cs_set:Nn |
%                  \cs_set:Nx |
%                  \cs_set:cn |
%                  \cs_set:cx |
% }
% \begin{syntax}
%   "\cs_set:Nn" <cs>  \Arg{code}
% \end{syntax}
% Like "\cs_set_nopar:Nn" but allows "\par" tokens in the arguments of the
% function being defined.
% \end{function}
%
% \begin{function}{\cs_gset:Nn |
%                  \cs_gset:Nx |
%                  \cs_gset:cn |
%                  \cs_gset:cx |
% }
% \begin{syntax}
%   "\cs_gset:Nn" <cs>  \Arg{code}
% \end{syntax}
% Global variant of "\cs_set:Nn".
% \end{function}
%
% \begin{function}{\cs_set_nopar:Nn |
%                  \cs_set_nopar:Nx |
%                  \cs_set_nopar:cn |
%                  \cs_set_nopar:cx |
% }
% \begin{syntax}
%   "\cs_set_nopar:Nn" <cs>  \Arg{code}
% \end{syntax}
% Like "\cs_new_nopar:Nn" etc.\ but does not check the <cs> name.
% \end{function}
%
% \begin{function}{\cs_gset_nopar:Nn |
%                  \cs_gset_nopar:Nx |
%                  \cs_gset_nopar:cn |
%                  \cs_gset_nopar:cx |
% }
% \begin{syntax}
%   "\cs_gset_nopar:Nn" <cs> \Arg{code}
% \end{syntax}
% Like "\cs_set_nopar:Nn" but defines the <cs> globally.
% \end{function}
%
% \begin{function}{\cs_set_protected:Nn |
%                  \cs_set_protected:cn |
%                  \cs_set_protected:Nx |
%                  \cs_set_protected:cx |
% }
% \begin{syntax}
%   "\cs_set_protected:Nn" <cs>  \Arg{code}
% \end{syntax}
% Naturally robust macro that won't expand in an |x| type argument.
% These varieties also allow |\par| tokens in the arguments of the
% function being defined.
% \end{function}
%
% \begin{function}{\cs_gset_protected:Nn |
%                  \cs_gset_protected:cn |
%                  \cs_gset_protected:Nx |
%                  \cs_gset_protected:cx |
% }
% \begin{syntax}
%   "\cs_gset_protected:Nn" <cs> \Arg{code}
% \end{syntax}
% Global versions of the above functions.
% \end{function}
%
% \begin{function}{\cs_set_protected_nopar:Nn |
%                  \cs_set_protected_nopar:cn |
%                  \cs_set_protected_nopar:Nx |
%                  \cs_set_protected_nopar:cx |
% }
% \begin{syntax}
%   "\cs_set_protected_nopar:Nn" <cs> \Arg{code}
% \end{syntax}
% Naturally robust macro that won't expand in an |x| type argument.
% This also comes as a |long| version. If you for some reason want to
% expand it inside an |x| type expansion, prefix it with
% |\exp_after:wN \prg_do_nothing:|.
% \end{function}
%
% \begin{function}{\cs_gset_protected_nopar:Nn |
%                  \cs_gset_protected_nopar:cn |
%                  \cs_gset_protected_nopar:Nx |
%                  \cs_gset_protected_nopar:cx |
% }
% \begin{syntax}
%   "\cs_gset_protected_nopar:Nn" <cs>  \Arg{code}
% \end{syntax}
% Global versions of the above functions.
% \end{function}
%
%
%
% \subsection{Undefining functions}
%
% \begin{function}{
%   \cs_undefine:N  |
%   \cs_undefine:c  |
%   \cs_gundefine:N |
%   \cs_gundefine:c
% }
% \begin{syntax}
%   "\cs_gundefine:N" <cs>
% \end{syntax}
% Undefines the control sequence locally or globally.
% In a global context, this is useful for reclaiming a small amount of
% memory but shouldn't often be needed for this purpose.
% In a local context, this can be useful if you need to clear a definition
% before applying a short-term modification to something.
% \end{function}
%
%
% \subsection{Copying function definitions}
%
% \begin{function}{ \cs_new_eq:NN  |
%                   \cs_new_eq:cN  |
%                   \cs_new_eq:Nc  |
%                   \cs_new_eq:cc  }
% \begin{syntax}
%   "\cs_new_eq:NN" <cs1> <cs2>
% \end{syntax}
% Gives the function <cs1> locally or globally the current meaning of <cs2>. 
% If <cs1> already exists then an error is called.
% \end{function}
%
%
% \begin{function}{\cs_set_eq:NN  |
%                  \cs_set_eq:cN  |
%                  \cs_set_eq:Nc  |
%                  \cs_set_eq:cc  |
%                  \cs_gset_eq:NN |
%                  \cs_gset_eq:cN |
%                  \cs_gset_eq:Nc |
%                  \cs_gset_eq:cc }
% \begin{syntax}
%   "\cs_set_eq:cN" <cs1> <cs2>
% \end{syntax}
% Gives the function <cs1> the current meaning of <cs2>. Again, we may
% always do this globally.
% \end{function}
%
% \begin{function}{\cs_set_eq:NwN}
% \begin{syntax}
%   "\cs_set_eq:NwN"  <cs1> <cs2>
%   "\cs_set_eq:NwN"  <cs1> "=" <cs2>
% \end{syntax}
% These functions assign the meaning of <cs2> locally or globally to the
% function <cs1>. Because the \TeX{} primitive operation is being used
% which may have an equal sign and (a certain number of) spaces between
% <cs1> and <cs2> the name contains a "w". (Not happy about this
% convention!).
% \begin{texnote}
% "\cs_set_eq:NwN" is the \LaTeX3 name for \TeX{}'s \tn{let}.
% \end{texnote}
% \end{function}
%
% \subsection{Internal functions}
%
%
% \begin{function}{ \pref_global:D    |
%                   \pref_long:D      |
%                   \pref_protected:D }
% \begin{syntax}
%   "\pref_global:D" "\cs_set_nopar:Npn"
% \end{syntax}
% Prefix functions that can be used in front of some definition
% functions (namely \ldots). The result of prefixing a function
% definition with "\pref_global:D" makes the definition global,
% "\pref_long:D" change the argument scanning mechanism so that it
% allows "\par" tokens in the argument of the prefixed function,
% and "\pref_protected:D" makes the definition robust in "\write"s etc.
%
% None of these internal functions should be used by a programmer since
% the necessary combinations are all available as separate function,
% e.g., "\cs_set:Npn" is internally implemented as "\pref_long:D"
% "\cs_set_nopar:Npn".
% \begin{texnote}
%   These prefixes are the primitives \tn{global}, \tn{long}, and
%   \tn{protected}.  The \tn{outer} prefix isn't used at all within \LaTeX3
%   because \ldots (it causes more hassle than it's worth? It's nevery proved
%   useful in any meaningful way?)
% \end{texnote}
% \end{function}
%
% \section{The innards of a function}
%
% \begin{function}{\cs_to_str:N / (EXP)}
% \begin{syntax}
%   "\cs_to_str:N" <cs>
% \end{syntax}
% This function returns the name of <cs> as a sequence of letters with
% the escape character removed.
% \end{function}
%
% \begin{function}{\token_to_str:N / (EXP) | \token_to_str:c / (EXP)}
% \begin{syntax}
%   "\token_to_str:N" <arg>
% \end{syntax}
% This function return the name of <arg> as a sequence of letters
% including the escape character.
% \begin{texnote}
%   This is \TeX's \tn{string}.
% \end{texnote}
% \end{function}
%
% \begin{function}{\token_to_meaning:N / (EXP)}
% \begin{syntax}
%   "\token_to_meaning:N" <arg>
% \end{syntax}
% This function returns the type and definition of <arg> as a sequence
% of letters.
% \begin{texnote}
%   This is \TeX's \tn{meaning}.
% \end{texnote}
% \end{function}
%
%
% \begin{function}{\cs_get_function_name:N / (EXP) |
%     \cs_get_function_signature:N / (EXP)}
% \begin{syntax}
% "\cs_get_function_name:N" "\"<fn>":"<args>
% \end{syntax}
% The "name" variant strips off the leading escape character and the
% trailing argument specification (including the colon) to return
% <fn>. The "signature" variants does the same but returns the
% signature <args> instead.
% \end{function}
%
%
% \begin{function}{\cs_split_function:NN / (EXP)}
% \begin{syntax}
% "\cs_split_function:NN" "\"<fn>":"<args> <post process>
% \end{syntax}
% Strips off the leading escape character, splits off the signature
% without the colon, informs whether or not a colon was present and
% then prefixes these results with <post process>, i.e., <post
% process>\Arg{name}\Arg{signature}\meta{true}/\meta{false}. For
% example, "\cs_get_function_name:N" is nothing more than
% "\cs_split_function:NN" "\"<fn>":"<args> "\use_i:nnn".
% \end{function}
%
%
% \begin{function}{\cs_get_arg_count_from_signature:N / (EXP)}
% \begin{syntax}
% "\cs_get_arg_count_from_signature:N" "\"<fn>":"<args>
% \end{syntax}
% Returns the number of chars in <args>, signifying the number of arguments
% that the function uses.
% \end{function}
%
%
% Other functions regarding arbitrary tokens can be found in the
% \textsf{l3token} module.
%
%  \section{Grouping and scanning}
%
% \begin{function}{\scan_stop:}
% \begin{syntax}
%   "\scan_stop:"
% \end{syntax}
% This function stops \TeX's scanning ahead when ending a number.
% \begin{texnote}
% This is the \TeX{} primitive \tn{relax} renamed.
% \end{texnote}
% \end{function}
%
%
% \begin{function}{\group_begin:|
%                  \group_end:}
% \begin{syntax}
%   "\group_begin:" <...> "\group_end:"
% \end{syntax}
% Encloses <...> inside a group.
% \begin{texnote}
% These are the \TeX{} primitives \tn{begingroup} and \tn{endgroup}
% renamed.
% \end{texnote}
% \end{function}
%
% \begin{function}{\group_execute_after:N}
% \begin{syntax}
%   "\group_execute_after:N" <token>
% \end{syntax}
% Adds <token> to the list of tokens to be inserted after the 
% current group ends (through an explicit or implicit "\group_end:").
% \begin{texnote}
% This is \TeX's \tn{aftergroup}.
% \end{texnote}
% \end{function}
%
%
% \section{Checking the engine}
%
% \begin{function}{\xetex_if_engine: / (TF)(EXP)}
% \begin{syntax}
%   "\xetex_if_engine:TF" \Arg{true code} \Arg{false code}
% \end{syntax}
% This function detects if we're running a Xe\TeX-based format.
% \end{function}
%
% \begin{function}{\luatex_if_engine: / (TF)(EXP)}
% \begin{syntax}
%   "\luatex_if_engine:TF" \Arg{true code} \Arg{false code}
% \end{syntax}
% This function detects if we're running a Lua\TeX-based format.
% \end{function}
%
% \begin{variable}{\c_xetex_is_engine_bool|\c_luatex_is_engine_bool}
% Boolean variables used for the above functions.
% \end{variable}
%
% \end{documentation}
%
% \begin{implementation}
%
% \section{\pkg{l3basics} implementation}
%
% We need \textsf{l3names} to get things going but we actually need it very
% early on, so it is loaded at the very top of the file \file{l3basics.dtx}.
% Also, most of the code below won't run until \textsf{l3expan} has been
% loaded.
%
% \subsection{Renaming some \TeX{} primitives (again)}
%
% \begin{macro}{\cs_set_eq:NwN}
% Having given all the tex primitives a consistent name, we need to
% give sensible names to the ones we actually want to use.
% These will be defined as needed in the appropriate modules, but
% do a few now, just to get started.\footnote{This renaming gets expensive
% in terms of csname usage, an alternative scheme would be to just use
% the ``tex\ldots D'' name in the cases where no good alternative exists.}
%    \begin{macrocode}
%<*package>
\ProvidesExplPackage
  {\filename}{\filedate}{\fileversion}{\filedescription}
\package_check_loaded_expl:
%</package>
%<*initex|package>
\tex_let:D \cs_set_eq:NwN            \tex_let:D
%    \end{macrocode}
% \end{macro}
% \begin{macro}{\if_true:}
% \begin{macro}{\if_false:}
% \begin{macro}{\or:}
% \begin{macro}{\else:}
% \begin{macro}{\fi:}
% \begin{macro}{\reverse_if:N}
% \begin{macro}{\if:w}
% \begin{macro}{\if_bool:N}
% \begin{macro}{\if_predicate:w}
% \begin{macro}{\if_charcode:w}
% \begin{macro}{\if_catcode:w}
% Then some conditionals.
%    \begin{macrocode}
\cs_set_eq:NwN   \if_true:           \tex_iftrue:D
\cs_set_eq:NwN   \if_false:          \tex_iffalse:D
\cs_set_eq:NwN   \or:                \tex_or:D
\cs_set_eq:NwN   \else:              \tex_else:D
\cs_set_eq:NwN   \fi:                \tex_fi:D
\cs_set_eq:NwN   \reverse_if:N       \etex_unless:D
\cs_set_eq:NwN   \if:w               \tex_if:D
\cs_set_eq:NwN   \if_bool:N          \tex_ifodd:D
\cs_set_eq:NwN   \if_predicate:w     \tex_ifodd:D
\cs_set_eq:NwN   \if_charcode:w      \tex_if:D
\cs_set_eq:NwN   \if_catcode:w       \tex_ifcat:D
%    \end{macrocode}
% \end{macro}
% \end{macro}
% \end{macro}
% \end{macro}
% \end{macro}
% \end{macro}
% \end{macro}
% \end{macro}
% \end{macro}
% \end{macro}
% \end{macro}
%
% \begin{macro}{\if_meaning:w}
%    \begin{macrocode}
\cs_set_eq:NwN   \if_meaning:w       \tex_ifx:D
%    \end{macrocode}
% \end{macro}
%
% \begin{macro}{\if_mode_math:}
% \begin{macro}{\if_mode_horizontal:}
% \begin{macro}{\if_mode_vertical:}
% \begin{macro}{\if_mode_inner:}
% \TeX{} lets us detect some if its modes.
%    \begin{macrocode}
\cs_set_eq:NwN   \if_mode_math:       \tex_ifmmode:D
\cs_set_eq:NwN   \if_mode_horizontal: \tex_ifhmode:D
\cs_set_eq:NwN   \if_mode_vertical:   \tex_ifvmode:D
\cs_set_eq:NwN   \if_mode_inner:      \tex_ifinner:D
%    \end{macrocode}
% \end{macro}
% \end{macro}
% \end{macro}
% \end{macro}
% \begin{macro}{\if_cs_exist:N}
% \begin{macro}{\if_cs_exist:w}
%    \begin{macrocode}
\cs_set_eq:NwN   \if_cs_exist:N      \etex_ifdefined:D
\cs_set_eq:NwN   \if_cs_exist:w      \etex_ifcsname:D
%    \end{macrocode}
% \end{macro}
% \end{macro}
%
% \begin{macro}{\exp_after:wN}
% \begin{macro}{\exp_not:N}
% \begin{macro}{\exp_not:n}
%  The three |\exp_| functions are used in the \textsf{l3expan} module
%  where they are described.
%    \begin{macrocode}
\cs_set_eq:NwN   \exp_after:wN       \tex_expandafter:D
\cs_set_eq:NwN   \exp_not:N          \tex_noexpand:D
\cs_set_eq:NwN   \exp_not:n          \etex_unexpanded:D
%    \end{macrocode}
% \end{macro}
% \end{macro}
% \end{macro}
% 
% \begin{macro}{\iow_shipout_x:Nn}
% \begin{macro}{\token_to_meaning:N}
% \begin{macro}{\token_to_str:N}
% \begin{macro}{\token_to_str:c}
% \begin{macro}{\cs:w}
% \begin{macro}{\cs_end:}
% \begin{macro}{\cs_meaning:N}
% \begin{macro}{\cs_meaning:c}
% \begin{macro}{\cs_show:N}
% \begin{macro}{\cs_show:c}
%    \begin{macrocode}
\cs_set_eq:NwN   \iow_shipout_x:Nn    \tex_write:D
\cs_set_eq:NwN   \token_to_meaning:N \tex_meaning:D
\cs_set_eq:NwN   \token_to_str:N     \tex_string:D
\cs_set_eq:NwN   \cs:w               \tex_csname:D
\cs_set_eq:NwN   \cs_end:            \tex_endcsname:D
\cs_set_eq:NwN   \cs_meaning:N       \tex_meaning:D
\tex_def:D \cs_meaning:c {\exp_args:Nc\cs_meaning:N}
\cs_set_eq:NwN   \cs_show:N          \tex_show:D
\tex_def:D \cs_show:c {\exp_args:Nc\cs_show:N}
\tex_def:D \token_to_str:c {\exp_args:Nc\token_to_str:N}
%    \end{macrocode}
% \end{macro}
% \end{macro}
% \end{macro}
% \end{macro}
% \end{macro}
% \end{macro}
% \end{macro}
% \end{macro}
% \end{macro}
% \end{macro}
%
% \begin{macro}{\scan_stop:}
% \begin{macro}{\group_begin:}
% \begin{macro}{\group_end:}
%  The next three are basic functions for which there also exist
%  versions that are safe inside alignments. These safe versions are
%  defined in the \textsf{l3prg} module.
%    \begin{macrocode}
\cs_set_eq:NwN   \scan_stop:         \tex_relax:D
\cs_set_eq:NwN   \group_begin:       \tex_begingroup:D
\cs_set_eq:NwN   \group_end:         \tex_endgroup:D
%    \end{macrocode}
% \end{macro}
% \end{macro}
% \end{macro}
% \begin{macro}{\group_execute_after:N}
%    \begin{macrocode}
\cs_set_eq:NwN \group_execute_after:N \tex_aftergroup:D      
%    \end{macrocode}
% \end{macro}
% \begin{macro}{\pref_global:D}
% \begin{macro}{\pref_long:D}
% \begin{macro}{\pref_protected:D}
%    \begin{macrocode}
\cs_set_eq:NwN   \pref_global:D      \tex_global:D
\cs_set_eq:NwN   \pref_long:D        \tex_long:D
\cs_set_eq:NwN   \pref_protected:D   \etex_protected:D
%    \end{macrocode}
% \end{macro}
% \end{macro}
% \end{macro}
%
%
%
% \subsection {Defining functions}
%
%
% We start by providing functions for the typical definition
% functions. First the local ones.
%
% \begin{macro}{\cs_set_nopar:Npn}
% \begin{macro}{\cs_set_nopar:Npx}
% \begin{macro}{\cs_set:Npn}
% \begin{macro}{\cs_set:Npx}
% \begin{macro}{\cs_set_protected_nopar:Npn}
% \begin{macro}{\cs_set_protected_nopar:Npx}
% \begin{macro}{\cs_set_protected:Npn}
% \begin{macro}{\cs_set_protected:Npx}
%   All assignment functions in \LaTeX3 should be naturally robust;
%   after all, the \TeX\ primitives for assignments are and it can be
%   a cause of problems if others aren't.
%    \begin{macrocode}
\cs_set_eq:NwN   \cs_set_nopar:Npn            \tex_def:D
\cs_set_eq:NwN   \cs_set_nopar:Npx            \tex_edef:D
\pref_protected:D \cs_set_nopar:Npn \cs_set:Npn {
  \pref_long:D \cs_set_nopar:Npn
}
\pref_protected:D \cs_set_nopar:Npn \cs_set:Npx {
  \pref_long:D \cs_set_nopar:Npx
}
\pref_protected:D \cs_set_nopar:Npn \cs_set_protected_nopar:Npn {
  \pref_protected:D \cs_set_nopar:Npn
}
\pref_protected:D \cs_set_nopar:Npn \cs_set_protected_nopar:Npx {
  \pref_protected:D \cs_set_nopar:Npx
}
\cs_set_protected_nopar:Npn \cs_set_protected:Npn {
  \pref_protected:D \pref_long:D \cs_set_nopar:Npn
}
\cs_set_protected_nopar:Npn \cs_set_protected:Npx {
  \pref_protected:D \pref_long:D \cs_set_nopar:Npx
}
%    \end{macrocode}
% \end{macro}
% \end{macro}
% \end{macro}
% \end{macro}
% \end{macro}
% \end{macro}
% \end{macro}
% \end{macro}
%
% \begin{macro}{\cs_gset_nopar:Npn}
% \begin{macro}{\cs_gset_nopar:Npx}
% \begin{macro}{\cs_gset:Npn}
% \begin{macro}{\cs_gset:Npx}
% \begin{macro}{\cs_gset_protected_nopar:Npn}
% \begin{macro}{\cs_gset_protected_nopar:Npx}
% \begin{macro}{\cs_gset_protected:Npn}
% \begin{macro}{\cs_gset_protected:Npx}
%   Global versions of the above functions.
%    \begin{macrocode}
\cs_set_eq:NwN   \cs_gset_nopar:Npn           \tex_gdef:D
\cs_set_eq:NwN   \cs_gset_nopar:Npx           \tex_xdef:D
\cs_set_protected_nopar:Npn \cs_gset:Npn {
  \pref_long:D \cs_gset_nopar:Npn
}
\cs_set_protected_nopar:Npn \cs_gset:Npx {
  \pref_long:D \cs_gset_nopar:Npx
}
\cs_set_protected_nopar:Npn \cs_gset_protected_nopar:Npn {
  \pref_protected:D \cs_gset_nopar:Npn
}
\cs_set_protected_nopar:Npn \cs_gset_protected_nopar:Npx {
  \pref_protected:D \cs_gset_nopar:Npx
}
\cs_set_protected_nopar:Npn \cs_gset_protected:Npn {
  \pref_protected:D \pref_long:D \cs_gset_nopar:Npn
}
\cs_set_protected_nopar:Npn \cs_gset_protected:Npx {
  \pref_protected:D \pref_long:D \cs_gset_nopar:Npx
}
%    \end{macrocode}
% \end{macro}
% \end{macro}
% \end{macro}
% \end{macro}
% \end{macro}
% \end{macro}
% \end{macro}
% \end{macro}
%
%
%
% \subsection{Selecting tokens}
%
% \begin{macro}{\use:c}
%    This macro grabs its argument and returns a csname from it.
%    \begin{macrocode}
\cs_set:Npn \use:c #1 { \cs:w#1\cs_end: }
%    \end{macrocode}
% \end{macro}
%
% \begin{macro}{\use:x}
% Fully expands its argument and passes it to the input stream.
% Uses "\cs_tmp:" as a scratch register but does not affect it.
%    \begin{macrocode}
\cs_set_protected:Npn \use:x #1 {
  \group_begin:
    \cs_set:Npx \cs_tmp: {#1}
    \exp_after:wN
  \group_end:
  \cs_tmp:
}
%    \end{macrocode}
% \end{macro}
%
% \begin{macro}{\use:n}
% \begin{macro}{\use:nn}
% \begin{macro}{\use:nnn}
% \begin{macro}{\use:nnnn}
%    These macro grabs its arguments and returns it back to the input
%    (with outer braces removed). "\use:n" is defined earlier for bootstrapping.
%    \begin{macrocode}
\cs_set:Npn \use:n   #1    {#1}
\cs_set:Npn \use:nn   #1#2     {#1#2}
\cs_set:Npn \use:nnn  #1#2#3   {#1#2#3}
\cs_set:Npn \use:nnnn #1#2#3#4 {#1#2#3#4}
%    \end{macrocode}
% \end{macro}
% \end{macro}
% \end{macro}
% \end{macro}
%
% \begin{macro}{\use_i:nn}
% \begin{macro}{\use_ii:nn}
%    These macros are needed to provide functions with true and false
%    cases, as introduced by Michael some time ago. By using
%    |\exp_after:wN| |\use_i:nn | |\else:| constructions it
%    is possible to write code where the true or false case is able to
%    access the following tokens from the input stream, which is not
%    possible if the |\c_true_bool| syntax is used.
%    \begin{macrocode}
\cs_set:Npn \use_i:nn  #1#2 {#1}
\cs_set:Npn \use_ii:nn #1#2 {#2}
%    \end{macrocode}
% \end{macro}
% \end{macro}
%
%
%
% \begin{macro}{\use_i:nnn}
% \begin{macro}{\use_ii:nnn}
% \begin{macro}{\use_iii:nnn}
% \begin{macro}{\use_i:nnnn}
% \begin{macro}{\use_ii:nnnn}
% \begin{macro}{\use_iii:nnnn}
% \begin{macro}{\use_iv:nnnn}
% \begin{macro}{\use_i_ii:nnn}
%    We also need something for picking up arguments from a longer
%    list.
%    \begin{macrocode}
\cs_set:Npn \use_i:nnn    #1#2#3{#1}
\cs_set:Npn \use_ii:nnn   #1#2#3{#2}
\cs_set:Npn \use_iii:nnn  #1#2#3{#3}
\cs_set:Npn \use_i:nnnn   #1#2#3#4{#1}
\cs_set:Npn \use_ii:nnnn  #1#2#3#4{#2}
\cs_set:Npn \use_iii:nnnn #1#2#3#4{#3}
\cs_set:Npn \use_iv:nnnn  #1#2#3#4{#4}
\cs_set:Npn \use_i_ii:nnn #1#2#3{#1#2}
%    \end{macrocode}
% \end{macro}
% \end{macro}
% \end{macro}
% \end{macro}
% \end{macro}
% \end{macro}
% \end{macro}
% \end{macro}
%
% \begin{macro}{\use_none_delimit_by_q_nil:w}
% \begin{macro}{\use_none_delimit_by_q_stop:w}
% \begin{macro}{\use_none_delimit_by_q_recursion_stop:w}
%   Functions that gobble everything until they see either |\q_nil| or
%   |\q_stop| resp.
%    \begin{macrocode}
\cs_set:Npn \use_none_delimit_by_q_nil:w #1\q_nil{}
\cs_set:Npn \use_none_delimit_by_q_stop:w #1\q_stop{}
\cs_set:Npn \use_none_delimit_by_q_recursion_stop:w #1 \q_recursion_stop {}
%    \end{macrocode}
% \end{macro}
% \end{macro}
% \end{macro}
%
% \begin{macro}{\use_i_delimit_by_q_nil:nw}
% \begin{macro}{\use_i_delimit_by_q_stop:nw}
% \begin{macro}{\use_i_delimit_by_q_recursion_stop:nw}
%   Same as above but execute first argument after gobbling. Very
%   useful when you need to skip the rest of a mapping sequence but
%   want an easy way to control what should be expanded next.
%    \begin{macrocode}
\cs_set:Npn \use_i_delimit_by_q_nil:nw #1#2\q_nil{#1}
\cs_set:Npn \use_i_delimit_by_q_stop:nw #1#2\q_stop{#1}
\cs_set:Npn \use_i_delimit_by_q_recursion_stop:nw #1#2 \q_recursion_stop {#1}
%    \end{macrocode}
% \end{macro}
% \end{macro}
% \end{macro}
%
%
% \begin{macro}{\use_i_after_fi:nw}
% \begin{macro}{\use_i_after_else:nw}
% \begin{macro}{\use_i_after_or:nw}
% \begin{macro}{\use_i_after_orelse:nw}
%   Returns the first argument after ending the conditional.
%    \begin{macrocode}
\cs_set:Npn \use_i_after_fi:nw #1\fi:{\fi: #1}
\cs_set:Npn \use_i_after_else:nw #1\else:#2\fi:{\fi: #1}
\cs_set:Npn \use_i_after_or:nw #1\or: #2\fi: {\fi:#1}
\cs_set:Npn \use_i_after_orelse:nw #1 #2#3\fi: {\fi:#1}
%    \end{macrocode}
% \end{macro}
% \end{macro}
% \end{macro}
% \end{macro}
%
% \subsection{Gobbling tokens from input}
%
% \begin{macro}{\use_none:n}
% \begin{macro}{\use_none:nn}
% \begin{macro}{\use_none:nnn}
% \begin{macro}{\use_none:nnnn}
% \begin{macro}{\use_none:nnnnn}
% \begin{macro}{\use_none:nnnnnn}
% \begin{macro}{\use_none:nnnnnnn}
% \begin{macro}{\use_none:nnnnnnnn}
% \begin{macro}{\use_none:nnnnnnnnn}
%   To gobble tokens from the input we use a standard naming
%   convention: the number of tokens gobbled is given by the number of
%   |n|'s following the |:| in the name. Although defining
%   |\use_none:nnn| and above as separate calls of |\use_none:n| and
%   |\use_none:nn| is slightly faster, this is very non-intuitive to
%   the programmer who will assume that expanding such a function once
%   will take care of gobbling all the tokens in one go.
%    \begin{macrocode}
\cs_set:Npn \use_none:n #1{}
\cs_set:Npn \use_none:nn #1#2{}
\cs_set:Npn \use_none:nnn #1#2#3{}
\cs_set:Npn \use_none:nnnn #1#2#3#4{}
\cs_set:Npn \use_none:nnnnn #1#2#3#4#5{}
\cs_set:Npn \use_none:nnnnnn #1#2#3#4#5#6{}
\cs_set:Npn \use_none:nnnnnnn #1#2#3#4#5#6#7{}
\cs_set:Npn \use_none:nnnnnnnn #1#2#3#4#5#6#7#8{}
\cs_set:Npn \use_none:nnnnnnnnn #1#2#3#4#5#6#7#8#9{}
%    \end{macrocode}
% \end{macro}
% \end{macro}
% \end{macro}
% \end{macro}
% \end{macro}
% \end{macro}
% \end{macro}
% \end{macro}
% \end{macro}
%
% \subsection{Expansion control from l3expan}
%
%\ExplSyntaxOn 
%  \cs_set_eq:NN \g_saved_doc_macros_seq \g_doc_macros_seq
%\ExplSyntaxOff
% \begin{macro}{\exp_args:Nc}
%   Moved here for now as it is going to be used right away.
%    \begin{macrocode}
\cs_set:Npn \exp_args:Nc #1#2{\exp_after:wN#1\cs:w#2\cs_end:}
%    \end{macrocode}
% \end{macro}
%\ExplSyntaxOn 
%  \cs_set_eq:NN \g_doc_macros_seq \g_saved_doc_macros_seq 
%\ExplSyntaxOff
%
% \subsection{Conditional processing and definitions}
%
% Underneath any predicate function (|_p|) or other conditional forms
% (|TF|, etc.) is a built-in logic saying that it after all of the
% testing and processing must return the \meta{state} this leaves
% \TeX\ in. Therefore, a simple user interface could be something like
% \begin{verbatim}
%   \if_meaning:w #1#2   \prg_return_true:  \else: 
%      \if_meaning:w #1#3  \prg_return_true: \else:
%      \prg_return_false: 
%  \fi: \fi:
% \end{verbatim}
% Usually, a \TeX\ programmer would have to insert a number of
% |\exp_after:wN|s to ensure the state value is returned at exactly
% the point where the last conditional is finished.  However, that
% obscures the code and forces the \TeX\ programmer to prove that
% he/she knows the $2^{n}-1$ table.  We therefore provide the simpler
% interface.
% 
%\ExplSyntaxOn 
%  \cs_set_eq:NN \g_saved_doc_macros_seq \g_doc_macros_seq
%\ExplSyntaxOff
%
% \begin{macro}{\prg_return_true:}
% \begin{macro}{\prg_return_false:}
%   These break statements put \TeX\ in a \m{true} or \m{false} state.
%   The idea is that the expansion of |\tex_romannumeral:D \c_zero| is
%   \m{null} so we set off a |\tex_romannumeral:D|.  It will on its
%   way expand any |\else:| or |\fi:| that are waiting to be discarded
%   anyway before finally arriving at the |\c_zero| we will place
%   right after the conditional.  After this expansion has terminated,
%   we issue either |\if_true:| or |\if_false:| to put \TeX\ in the
%   correct state.
%    \begin{macrocode}
\cs_set:Npn \prg_return_true: { \exp_after:wN\if_true:\tex_romannumeral:D }
\cs_set:Npn \prg_return_false: {\exp_after:wN\if_false:\tex_romannumeral:D }
%    \end{macrocode}
% An extended state space could instead utilize |\tex_ifcase:D|:
% \begin{verbatim}
% \cs_set:Npn \prg_return_true: { 
%   \exp_after:wN\tex_ifcase:D \exp_after:wN \c_zero \tex_romannumeral:D 
% }
% \cs_set:Npn \prg_return_false: { 
%   \exp_after:wN\tex_ifcase:D \exp_after:wN \c_one \tex_romannumeral:D 
% }
% \cs_set:Npn \prg_return_error: { 
%   \exp_after:wN\tex_ifcase:D \exp_after:wN \c_two \tex_romannumeral:D 
% }
% \end{verbatim}
% \end{macro}
% \end{macro}
%
% \begin{macro}{\prg_set_conditional:Npnn,\prg_new_conditional:Npnn,
%     \prg_set_protected_conditional:Npnn,\prg_new_protected_conditional:Npnn}
%   The user functions for the types using parameter text from the
%   programmer. Call aux function to grab parameters, split the base
%   function into name and signature and then use, e.g., |\cs_set:Npn|
%   to define it with.
%    \begin{macrocode}
\cs_set_protected:Npn \prg_set_conditional:Npnn #1{
  \prg_get_parm_aux:nw{
    \cs_split_function:NN #1 \prg_generate_conditional_aux:nnNNnnnn 
    \cs_set:Npn {parm}
  }
}
\cs_set_protected:Npn \prg_new_conditional:Npnn #1{
  \prg_get_parm_aux:nw{
    \cs_split_function:NN #1 \prg_generate_conditional_aux:nnNNnnnn 
    \cs_new:Npn {parm}
  }
}
\cs_set_protected:Npn \prg_set_protected_conditional:Npnn #1{
  \prg_get_parm_aux:nw{
    \cs_split_function:NN #1 \prg_generate_conditional_aux:nnNNnnnn 
    \cs_set_protected:Npn {parm}
  }
}
\cs_set_protected:Npn \prg_new_protected_conditional:Npnn #1{
  \prg_get_parm_aux:nw{
    \cs_split_function:NN #1 \prg_generate_conditional_aux:nnNNnnnn
    \cs_new_protected:Npn {parm}
  }
}
%    \end{macrocode}
% \end{macro} 
%
% \begin{macro}{\prg_set_conditional:Nnn,\prg_new_conditional:Nnn,
%     \prg_set_protected_conditional:Nnn,\prg_new_protected_conditional:Nnn}
%   The user functions for the types automatically inserting the
%   correct parameter text based on the signature. Call aux function
%   after calculating number of arguments, split the base function
%   into name and signature and then use, e.g., |\cs_set:Npn| to
%   define it with.
%    \begin{macrocode}
\cs_set_protected:Npn \prg_set_conditional:Nnn #1{
  \exp_args:Nnf \prg_get_count_aux:nn{
    \cs_split_function:NN #1 \prg_generate_conditional_aux:nnNNnnnn
    \cs_set:Npn {count}
  }{\cs_get_arg_count_from_signature:N #1}
}
\cs_set_protected:Npn \prg_new_conditional:Nnn #1{
  \exp_args:Nnf \prg_get_count_aux:nn{
    \cs_split_function:NN #1 \prg_generate_conditional_aux:nnNNnnnn
    \cs_new:Npn {count}
  }{\cs_get_arg_count_from_signature:N #1}
}

\cs_set_protected:Npn \prg_set_protected_conditional:Nnn #1{
  \exp_args:Nnf \prg_get_count_aux:nn{
    \cs_split_function:NN #1 \prg_generate_conditional_aux:nnNNnnnn
    \cs_set_protected:Npn {count}
  }{\cs_get_arg_count_from_signature:N #1}
}

\cs_set_protected:Npn \prg_new_protected_conditional:Nnn #1{
  \exp_args:Nnf \prg_get_count_aux:nn{
    \cs_split_function:NN #1 \prg_generate_conditional_aux:nnNNnnnn
    \cs_new_protected:Npn {count}
  }{\cs_get_arg_count_from_signature:N #1}
}
%    \end{macrocode}
% \end{macro}
% 
% \begin{macro}{\prg_set_eq_conditional:NNn,\prg_new_eq_conditional:NNn}
%  The obvious setting-equal functions.
%    \begin{macrocode}
\cs_set_protected:Npn \prg_set_eq_conditional:NNn #1#2#3 {
  \prg_set_eq_conditional_aux:NNNn \cs_set_eq:cc #1#2 {#3}
}
\cs_set_protected:Npn \prg_new_eq_conditional:NNn #1#2#3 {
  \prg_set_eq_conditional_aux:NNNn \cs_new_eq:cc #1#2 {#3}
}
%    \end{macrocode}
%    \end{macro}
% 
%\ExplSyntaxOn 
%  \cs_set_eq:NN \g_doc_macros_seq \g_saved_doc_macros_seq 
%\ExplSyntaxOff
%
% \begin{macro}[aux]{\prg_get_parm_aux:nw,\prg_get_count_aux:nn}
% For the "Npnn" type we must grab the parameter text before
% continuing. We make this a very generic function that takes one
% argument before reading everything up to a left brace. Something
% similar for the "Nnn" type.
%    \begin{macrocode}
\cs_set:Npn \prg_get_count_aux:nn #1#2 {#1{#2}}
\cs_set:Npn \prg_get_parm_aux:nw #1#2#{#1{#2}}
%    \end{macrocode}
% \end{macro}
% \begin{macro}[aux]{\prg_generate_conditional_parm_aux:nnNNnnnn,
%     \prg_generate_conditional_parm_aux:nw}
%
%   The workhorse here is going through a list of desired forms, i.e.,
%   p, TF, T and F. The first three arguments come from splitting up
%   the base form of the conditional, which gives the name, signature
%   and a boolean to signal whether or not there was a colon in the
%   name. For the time being, we do not use this piece of information
%   but could well throw an error. The fourth argument is how to
%   define this function, the fifth is the text "parm" or "count" for
%   which version to use to define the functions, the sixth is the
%   parameters to use (possibly empty) or number of arguments, the
%   seventh is the list of forms to define, the eight is the
%   replacement text which we will augment when defining the forms.
%    \begin{macrocode}
\cs_set:Npn \prg_generate_conditional_aux:nnNNnnnn #1#2#3#4#5#6#7#8{
  \prg_generate_conditional_aux:nnw{#5}{
    #4{#1}{#2}{#6}{#8}
  }#7,?, \q_recursion_stop
}
%    \end{macrocode}
% Looping through the list of desired forms. First is the text "parm"
% or "count", second is five arguments packed together and third is
% the form. Use text and form to call the correct type.
%    \begin{macrocode}
\cs_set:Npn \prg_generate_conditional_aux:nnw #1#2#3,{
  \if:w ?#3
    \exp_after:wN \use_none_delimit_by_q_recursion_stop:w
  \fi:
  \use:c{prg_generate_#3_form_#1:Nnnnn} #2
  \prg_generate_conditional_aux:nnw{#1}{#2}
}
%    \end{macrocode}
% \end{macro}
%
% \begin{macro}[aux]{\prg_generate_p_form_parm:Nnnnn,
%     \prg_generate_TF_form_parm:Nnnnn,
%     \prg_generate_T_form_parm:Nnnnn,
%     \prg_generate_F_form_parm:Nnnnn
%   }
%   How to generate the various forms. The "parm" types here takes the
%   following arguments: 1: how to define (an N-type), 2: name, 3:
%   signature, 4: parameter text (or empty), 5: replacement.
%    \begin{macrocode}
\cs_set:Npn \prg_generate_p_form_parm:Nnnnn #1#2#3#4#5{
  \exp_args:Nc #1 {#2_p:#3}#4{#5 \c_zero 
    \exp_after:wN\c_true_bool\else:\exp_after:wN\c_false_bool\fi:
  }
}
\cs_set:Npn \prg_generate_TF_form_parm:Nnnnn #1#2#3#4#5{
  \exp_args:Nc#1 {#2:#3TF}#4{#5 \c_zero 
    \exp_after:wN \use_i:nn \else: \exp_after:wN \use_ii:nn \fi:
  }
}
\cs_set:Npn \prg_generate_T_form_parm:Nnnnn #1#2#3#4#5{
  \exp_args:Nc#1 {#2:#3T}#4{#5 \c_zero 
    \else:\exp_after:wN\use_none:nn\fi:\use:n
  }
}
\cs_set:Npn \prg_generate_F_form_parm:Nnnnn #1#2#3#4#5{
  \exp_args:Nc#1 {#2:#3F}#4{#5 \c_zero 
    \exp_after:wN\use_none:nn\fi:\use:n
  }
}
%    \end{macrocode}
% \end{macro}
%
% \begin{macro}[aux]{\prg_generate_p_form_count:Nnnnn,
%     \prg_generate_TF_form_count:Nnnnn,
%     \prg_generate_T_form_count:Nnnnn,
%     \prg_generate_F_form_count:Nnnnn
%   }
% How to generate the various forms. The "count" types here use a
% number to insert the correct parameter text, otherwise like the
% "parm" functions above.
%    \begin{macrocode}
\cs_set:Npn \prg_generate_p_form_count:Nnnnn #1#2#3#4#5{
  \cs_generate_from_arg_count:cNnn {#2_p:#3} #1 {#4}{#5 \c_zero 
    \exp_after:wN\c_true_bool\else:\exp_after:wN\c_false_bool\fi:
  }
}
\cs_set:Npn \prg_generate_TF_form_count:Nnnnn #1#2#3#4#5{
  \cs_generate_from_arg_count:cNnn {#2:#3TF} #1 {#4}{#5 \c_zero 
    \exp_after:wN\use_i:nn\else:\exp_after:wN\use_ii:nn\fi:
  }
}
\cs_set:Npn \prg_generate_T_form_count:Nnnnn #1#2#3#4#5{
  \cs_generate_from_arg_count:cNnn {#2:#3T} #1 {#4}{#5 \c_zero 
        \else:\exp_after:wN\use_none:nn\fi:\use:n
  }
}
\cs_set:Npn \prg_generate_F_form_count:Nnnnn #1#2#3#4#5{
  \cs_generate_from_arg_count:cNnn {#2:#3F} #1 {#4}{#5 \c_zero 
    \exp_after:wN\use_none:nn\fi:\use:n
  }
}
%    \end{macrocode}
% \end{macro}
% 
% \begin{macro}[aux]{\prg_set_eq_conditional_aux:NNNn,
%                    \prg_set_eq_conditional_aux:NNNw}
%    \begin{macrocode}
\cs_set:Npn \prg_set_eq_conditional_aux:NNNn #1#2#3#4 {
  \prg_set_eq_conditional_aux:NNNw #1#2#3#4,?,\q_recursion_stop
}
%    \end{macrocode}
% Manual clist loop over argument "#4".
%    \begin{macrocode}
\cs_set:Npn \prg_set_eq_conditional_aux:NNNw #1#2#3#4, {
  \if:w ? #4 \scan_stop:
    \exp_after:wN \use_none_delimit_by_q_recursion_stop:w
  \fi:
  #1 {
    \exp_args:NNc \cs_split_function:NN #2 {prg_conditional_form_#4:nnn}
  }{
    \exp_args:NNc \cs_split_function:NN #3 {prg_conditional_form_#4:nnn}
  }
  \prg_set_eq_conditional_aux:NNNw #1{#2}{#3}
}
%    \end{macrocode}
%    \begin{macrocode}
\cs_set:Npn \prg_conditional_form_p:nnn  #1#2#3 {#1_p:#2}
\cs_set:Npn \prg_conditional_form_TF:nnn #1#2#3 {#1:#2TF}
\cs_set:Npn \prg_conditional_form_T:nnn  #1#2#3 {#1:#2T}
\cs_set:Npn \prg_conditional_form_F:nnn  #1#2#3 {#1:#2F}
%    \end{macrocode}
% \end{macro}
%
% All that is left is to define the canonical boolean true and false.
% I think Michael originated the idea of expandable boolean tests.  At
% first these were supposed to expand into either \texttt{TT} or
% \texttt{TF} to be tested using |\if:w| but this was later changed to
% \texttt{00} and \texttt{01}, so they could be used in logical
% operations. Later again they were changed to being numerical
% constants with values of $1$ for true and $0$ for false. We need
% this from the get-go.
%
% \begin{macro}{\c_true_bool}
% \begin{macro}{\c_false_bool}
%    Here are the canonical boolean values.
%    \begin{macrocode}
\tex_chardef:D \c_true_bool = 1~
\tex_chardef:D \c_false_bool = 0~
%    \end{macrocode}
% \end{macro}
% \end{macro}
%
%
% \subsection{Dissecting a control sequence}
%
% \begin{macro}{\cs_to_str:N}
% \begin{macro}[aux]{\cs_to_str_aux:w}
%   This converts a control sequence into the character string of its
%   name, removing the leading escape character. This turns out to be
%   a non-trivial matter as there a different cases:
%   \begin{itemize}
%   \item The usual case of a printable escape character;
%   \item the case of a non-printable escape characters, e.g., when
%   the value of |\tex_escapechar:D| is negative;
%   \item when the escape character is a space.
%   \end{itemize}
%   The route chosen is this: If |\token_to_str:N \a| produces a
%   non-space escape char, then this will produce two tokens. If the
%   escape char is non-printable, only one token is produced. If the
%   escape char is a space, then a space token plus one token
%   character token is produced. If we augment the result of this
%   expansion with the letters |ax| we get the following three
%   scenarios (with \meta{X} being a printable non-space escape
%   character):
%   \begin{itemize}
%   \item \meta{X}"aax" 
%   \item "aax" 
%   \item " aax" 
%   \end{itemize}
% In the second and third case, putting an auxiliary function in front
% reading undelimited arguments will treat them the same, removing the
% space token for us automatically. Therefore, if we test the second
% and third argument of what such a function reads, in case 1 we will
% get true and in cases 2 and 3 we will get false. If we choose to
% optimize for the usual case of a printable escape char, we can do it
% like this (again getting TeX to remove the leading space for us): 
%    \begin{macrocode}
\cs_set_nopar:Npn \cs_to_str:N {
  \if:w \exp_after:wN \cs_str_aux:w\token_to_str:N \a ax\q_nil 
  \else: 
    \exp_after:wN \exp_after:wN\exp_after:wN \use_ii:nn 
  \fi: 
  \exp_after:wN \use_none:n \token_to_str:N 
}
\cs_set:Npn \cs_str_aux:w #1#2#3#4\q_nil{#2#3}
%    \end{macrocode}
% \end{macro}
% \end{macro}
%
%
% \begin{macro}{\cs_split_function:NN}
% \begin{macro}[aux]{\cs_split_function_aux:w}
% \begin{macro}[aux]{\cs_split_function_auxii:w}
%   This function takes a function name and splits it into name with
%   the escape char removed and argument specification. In addition to
%   this, a third argument, a boolean \m{true} or \m{false} is
%   returned with \m{true} for when there is a colon in the function
%   and \m{false} if there is not. Lastly, the second argument of
%   |\cs_split_function:NN| is supposed to be a function
%   taking three variables, one for name, one for signature, and one
%   for the boolean.  For example,
%   "\cs_split_function:NN\foo_bar:cnx\use_i:nnn" as input
%   becomes "\use_i:nnn {foo_bar}{cnx}\c_true_bool".
%
%   Can't use a literal ":" because it has the wrong catcode here, so
%   it's transformed from "@" with "\tex_lowercase:D".
%    \begin{macrocode}
\group_begin:
  \tex_lccode:D  `\@ = `\: \scan_stop:
  \tex_catcode:D `\@ = 12~
\tex_lowercase:D {
  \group_end:
%    \end{macrocode}
% First ensure that we actually get a properly evaluated str as we
% don't know how many expansions |\cs_to_str:N| requires.  Insert
% extra colon to catch the error cases.
%    \begin{macrocode}
\cs_set:Npn \cs_split_function:NN #1#2{
  \exp_after:wN \cs_split_function_aux:w 
    \tex_romannumeral:D -`\q \cs_to_str:N #1 @a \q_nil #2 
}
%    \end{macrocode}
% If no colon in the name, |#2| is |a| with catcode 11 and |#3| is
% empty.  If colon in the name, then either |#2| is a colon or the
% first letter of the signature.  The letters here have catcode 12.
% If a colon was given we need to a) split off the colon and quark at
% the end and b) ensure we return the name, signature and boolean true
% We can't use |\quark_if_no_value:NTF| yet but this is very safe
% anyway as all tokens have catcode~12.
%    \begin{macrocode}
\cs_set:Npn \cs_split_function_aux:w #1@#2#3\q_nil#4{
  \if_meaning:w a#2 
    \exp_after:wN \use_i:nn 
  \else:
    \exp_after:wN\use_ii:nn
  \fi: 
  {#4{#1}{}\c_false_bool}
  {\cs_split_function_auxii:w#2#3\q_nil #4{#1}}
}
\cs_set:Npn \cs_split_function_auxii:w #1@a\q_nil#2#3{
  #2{#3}{#1}\c_true_bool
}
%    \end{macrocode}
% End of lowercase
%    \begin{macrocode}
}
%    \end{macrocode}
% \end{macro}
% \end{macro}
% \end{macro}
%
% \begin{macro}{\cs_get_function_name:N, \cs_get_function_signature:N }
%   Now returning the name is trivial: just discard the last two
%   arguments. Similar for signature.
%    \begin{macrocode}
\cs_set:Npn \cs_get_function_name:N #1 { 
  \cs_split_function:NN #1\use_i:nnn
}
\cs_set:Npn \cs_get_function_signature:N #1 { 
  \cs_split_function:NN #1\use_ii:nnn
}
%    \end{macrocode}
% \end{macro}
%
% \subsection{Exist or free}
%
% A control sequence is said to \emph{exist} (to be used) if has an entry in
% the hash table and its meaning is different from the primitive
% "\tex_relax:D" token. A control sequence is said to be \emph{free}
% (to be defined) if it does not already exist and also meets the
% requirement that it does not contain a "D" signature. The reasoning
% behind this is that most of the time, a check for a free control
% sequence is when we wish to make a new control sequence and we do
% not want to let the user define a new ``do not use'' control
% sequence.
%
% \begin{macro}{\cs_if_exist_p:N,\cs_if_exist_p:c}
% \begin{macro}[TF]{\cs_if_exist:N,\cs_if_exist:c}
%   Two versions for checking existence. For the "N" form we firstly
%   check for "\tex_relax:D" and then if it is in the hash
%   table. There is no problem when inputting something like "\else:"
%   or "\fi:" as \TeX\ will only ever skip input in case the token
%   tested against is "\tex_relax:D".
%    \begin{macrocode}
\prg_set_conditional:Npnn \cs_if_exist:N #1 {p,TF,T,F}{
  \if_meaning:w #1\tex_relax:D 
    \prg_return_false:
  \else:
    \if_cs_exist:N #1
      \prg_return_true:
    \else:
      \prg_return_false:
    \fi:
  \fi:
}
%    \end{macrocode}
% For the "c" form we firstly check if it is in the hash table and
% then for "\tex_relax:D" so that we do not add it to the hash table
% unless it was already there. Here we have to be careful as the text
% to be skipped if the first test is false may contain tokens that
% disturb the scanner. Therefore, we ensure that the second test is
% performed after the first one has concluded completely.
%    \begin{macrocode}
\prg_set_conditional:Npnn \cs_if_exist:c #1 {p,TF,T,F}{
  \if_cs_exist:w #1 \cs_end: 
    \exp_after:wN \use_i:nn
  \else:
    \exp_after:wN \use_ii:nn
  \fi:
  {
    \exp_after:wN \if_meaning:w \cs:w #1\cs_end: \tex_relax:D 
      \prg_return_false:
    \else:   
      \prg_return_true:
    \fi:
  }
  \prg_return_false:
}
%    \end{macrocode}
% \end{macro}
% \end{macro}
%
% \begin{macro}{\cs_if_do_not_use_p:N}
% \begin{macro}[aux]{\cs_if_do_not_use_aux:nnN}
%    \begin{macrocode}
\cs_set:Npn \cs_if_do_not_use_p:N #1{
  \cs_split_function:NN #1 \cs_if_do_not_use_aux:nnN
}
\cs_set:Npn \cs_if_do_not_use_aux:nnN #1#2#3{
  \exp_after:wN\str_if_eq_p:nn \token_to_str:N D {#2}
}
%    \end{macrocode}
% \end{macro}
% \end{macro}
%
% \begin{macro}{\cs_if_free_p:N,\cs_if_free_p:c}
% \begin{macro}[TF]{\cs_if_free:N,\cs_if_free:c}
% The simple implementation is one using the boolean expression
% parser: If it is exists or is do not use, then return false.
%    \begin{verbatim}
% \prg_set_conditional:Npnn \cs_if_free:N #1{p,TF,T,F}{
%   \bool_if:nTF {\cs_if_exist_p:N #1 || \cs_if_do_not_use_p:N #1}
%   {\prg_return_false:}{\prg_return_true:}
% }
%    \end{verbatim}
% However, this functionality may not be available this early on. We
% do something similar: The numerical values of true and false is one
% and zero respectively, which we can use. The problem again here is
% that the token we are checking may in fact be something that can
% disturb the scanner, so we have to be careful. We would like to do
% minimal evaluation so we ensure this.
%    \begin{macrocode}
\prg_set_conditional:Npnn \cs_if_free:N #1{p,TF,T,F}{
  \tex_ifnum:D \cs_if_exist_p:N #1 =\c_zero
    \exp_after:wN \use_i:nn
  \else:
    \exp_after:wN \use_ii:nn
  \fi:
  {   
    \tex_ifnum:D \cs_if_do_not_use_p:N #1 =\c_zero
      \prg_return_true:
    \else:
      \prg_return_false:
    \fi:
  }
  \prg_return_false:
}
\cs_set_nopar:Npn \cs_if_free_p:c{\exp_args:Nc\cs_if_free_p:N}
\cs_set_nopar:Npn \cs_if_free:cTF{\exp_args:Nc\cs_if_free:NTF}
\cs_set_nopar:Npn \cs_if_free:cT{\exp_args:Nc\cs_if_free:NT}
\cs_set_nopar:Npn \cs_if_free:cF{\exp_args:Nc\cs_if_free:NF}
%    \end{macrocode}
% \end{macro}
% \end{macro}
%
%
% \subsection{Defining and checking (new) functions}
%
% \begin{macro}{\c_minus_one}
% \begin{macro}{\c_zero}
% \begin{macro}{\c_sixteen}
%    We need the constants |\c_minus_one| and |\c_sixteen| now for
%    writing information to the log and the terminal and |\c_zero|
%    which is used by some functions in the \textsf{l3alloc} module. The
%    rest are defined in the \textsf{l3int} module -- at least for the
%    ones that can be defined with |\tex_chardef:D| or
%    |\tex_mathchardef:D|. For other constants the \textsf{l3int} module is
%    required but it can't be used until the allocation has been set
%    up properly! The actual allocation mechanism is in
%    \textsf{l3alloc} and as \TeX{} wants to reserve count registers
%    0--9, the first available one is~10 so we use that for
%    |\c_minus_one|.
%    \begin{macrocode}
%<*!initex>
\cs_set_eq:NwN \c_minus_one\m@ne
%</!initex>
%<*!package>
\tex_countdef:D \c_minus_one = 10 ~
\c_minus_one = -1 ~
%</!package>
\tex_chardef:D \c_sixteen = 16~
\tex_chardef:D \c_zero = 0~
%    \end{macrocode}
% \end{macro}
% \end{macro}
% \end{macro}
%
%    We provide two kinds of functions that can be used to define
%    control sequences. On the one hand we have functions that check
%    if their argument doesn't already exist, they are called
%    |\..._new|. The second type of defining functions doesn't check
%    if the argument is already defined.
%
%    Before we can define them, we need some auxiliary macros that
%    allow us to generate error messages. The definitions here are
%    only temporary, they will be redefined later on.
%
% \begin{macro}{\iow_log:x}
% \begin{macro}{\iow_term:x}
%    We define a routine to write only to the log file. And a
%    similar one for writing to both the log file and the terminal.
%
%    \begin{macrocode}
\cs_set_protected_nopar:Npn \iow_log:x {
  \tex_immediate:D \iow_shipout_x:Nn \c_minus_one
}
\cs_set_protected_nopar:Npn \iow_term:x {
  \tex_immediate:D \iow_shipout_x:Nn \c_sixteen
}
%    \end{macrocode}
% \end{macro}
% \end{macro}
%
% \begin{macro}{\msg_kernel_bug:x}
%    This will show internal errors.
%    \begin{macrocode}
\cs_set_protected_nopar:Npn \msg_kernel_bug:x #1 {
  \iow_term:x { This~is~a~LaTeX~bug:~check~coding! }
  \tex_errmessage:D {#1}
}
%    \end{macrocode}
% \end{macro}
%
% \begin{macro}{\cs_record_meaning:N}
%    This macro will be used later on for tracing purposes. But we
%    need some more modules to define it, so we just give some dummy
%    definition here.
%    \begin{macrocode}
%<*trace>
\cs_set:Npn \cs_record_meaning:N #1{}
%</trace>
%    \end{macrocode}
% \end{macro}
%
% \begin{macro}{\chk_if_free_cs:N}
% \begin{macro}{\chk_if_free_cs:c}
%   This command is called by |\cs_new_nopar:Npn| and |\cs_new_eq:NN| etc.\
%   to make sure that the argument sequence is not already in use. If
%   it is, an error is signalled.  It checks if \m{csname} is
%   undefined or |\scan_stop:|. Otherwise an error message is
%   issued. We have to make sure we don't put the argument into the
%   conditional processing since it may be an |\if...| type function!
%    \begin {macrocode}
\cs_set_protected_nopar:Npn \chk_if_free_cs:N #1{
  \cs_if_free:NF #1
  {    
    \msg_kernel_bug:x {Command~name~`\token_to_str:N #1'~
                      already~defined!~
                      Current~meaning:~\token_to_meaning:N #1
                    }
  }
%<*trace>
  \cs_record_meaning:N#1
%     \iow_term:x{Defining~\token_to_str:N #1~on~%}
  \iow_log:x{Defining~\token_to_str:N #1~on~
                line~\tex_the:D \tex_inputlineno:D}
%</trace>
}
\cs_set_protected_nopar:Npn \chk_if_free_cs:c { 
  \exp_args:Nc \chk_if_free_cs:N 
}
%<*package>
\tex_ifodd:D \@l@expl@log@functions@bool \else
  \cs_set_protected_nopar:Npn \chk_if_free_cs:N #1 {
    \cs_if_free:NF #1
      {    
        \msg_kernel_bug:x 
          {
            Command~name~`\token_to_str:N #1'~
            already~defined!~
            Current~meaning:~\token_to_meaning:N #1
          }
      }
  }      
\fi
%</package>
%    \end{macrocode}
% \end{macro}
% \end{macro}
%
% \begin{macro}{\chk_if_exist_cs:N }
% \begin{macro}{\chk_if_exist_cs:c }
%    This function issues a warning message when the control sequence
%    in its argument does not exist.
%    \begin{macrocode}
\cs_set_protected_nopar:Npn \chk_if_exist_cs:N #1 {
  \cs_if_exist:NF #1
  {
    \msg_kernel_bug:x {Command~ `\token_to_str:N #1'~
                     not~ yet~ defined!}
  }
}
\cs_set_protected_nopar:Npn \chk_if_exist_cs:c {
  \exp_args:Nc \chk_if_exist_cs:N 
}
%    \end{macrocode}
% \end{macro}
% \end{macro}
%
%
% \begin{macro}{\str_if_eq_p:nn}
% \begin{macro}[aux]{\str_if_eq_p_aux:w}
%   Takes 2 lists of characters as arguments and expands into
%   |\c_true_bool| if they are equal, and |\c_false_bool| otherwise. Note that
%   in the current implementation spaces in these strings are
%   ignored.\footnote{This is a function which could use
%     \cs{tlist_compare:xx}.}
%    \begin{macrocode}
\prg_set_conditional:Npnn \str_if_eq:nn #1#2{p}{
  \str_if_eq_p_aux:w #1\scan_stop:\\#2\scan_stop:\\
}
\cs_set_nopar:Npn \str_if_eq_p_aux:w #1#2\\#3#4\\{
  \if_meaning:w#1#3
    \if_meaning:w#1\scan_stop:\prg_return_true: \else:
    \if_meaning:w#3\scan_stop:\prg_return_false: \else:
    \str_if_eq_p_aux:w #2\\#4\\\fi:\fi:
  \else:\prg_return_false: \fi:}
%    \end{macrocode}
% \end{macro}
% \end{macro}
%
% \begin{macro}{\cs_if_eq_name_p:NN}
%   An application of the above function, already streamlined for
%   speed, so I put it in here.  It takes two control sequences as
%   arguments and expands into true iff they have the same name.
%   We make it long in case one of them is |\par|!
%    \begin{macrocode}
\prg_set_conditional:Npnn \cs_if_eq_name:NN #1#2{p}{
  \exp_after:wN\exp_after:wN
  \exp_after:wN\str_if_eq_p_aux:w
  \exp_after:wN\token_to_str:N
  \exp_after:wN#1
  \exp_after:wN\scan_stop:
  \exp_after:wN\\
  \token_to_str:N#2\scan_stop:\\}
%    \end{macrocode}
% \end{macro}
%
%
% \begin{macro}{\str_if_eq_var_p:nf}
% \begin{macro}[aux]{\str_if_eq_var_start:nnN}
% \begin{macro}[aux]{\str_if_eq_var_stop:w}
%   A variant of |\str_if_eq_p:nn| which has the advantage of obeying
%   spaces in at least the second argument. See \textsf{l3quark} for
%   an application. From the hand of David Kastrup with slight
%   modifications to make it fit with the remainder of the expl3
%   language.
%
%   The macro builds a string of |\if:w \fi:| pairs from the first
%   argument. The idea is to turn the comparison of |ab| and |cde|
%   into
% \begin{verbatim}
% \tex_number:D
%   \if:w \scan_stop: \if:w b\if:w a cde\scan_stop: '\fi: \fi: \fi:
%  13
% \end{verbatim}
%   The |'| is important here. If all tests are true, the |'| is read
%   as part of the number in which case the returned number is |13| in
%   octal notation so |\tex_number:D| returns |11|. If one test
%   returns false the |'| is never seen and then we get just |13|.  We
%   wrap the whole process in an external |\if:w| in order to make it
%   return either |\c_true_bool| or |\c_false_bool| since some parts of
%   \textsf{l3prg} expect a predicate to return one of these two
%   tokens.
%    \begin{macrocode}
\prg_set_conditional:Npnn \str_if_eq_var:nf #1#2 {p} {
  \if:w \tex_number:D\str_if_eq_var_start:nnN{}{}#1\scan_stop:{#2}
}
\cs_set_nopar:Npn\str_if_eq_var_start:nnN#1#2#3{
  \if:w#3\scan_stop:\exp_after:wN\str_if_eq_var_stop:w\fi:
  \str_if_eq_var_start:nnN{\if:w#3#1}{#2\fi:}
}
\cs_set:Npn\str_if_eq_var_stop:w\str_if_eq_var_start:nnN#1#2#3{
  #1#3\scan_stop:'#213~\prg_return_true:\else:\prg_return_false:\fi:
}
%    \end{macrocode}
% \end{macro}
% \end{macro}
% \end{macro}
%
%
%
%
% \subsection{More new definitions}
%
%
%
%
% \begin{macro}{\cs_new_nopar:Npn}
% \begin{macro}{\cs_new_nopar:Npx}
% \begin{macro}{\cs_new:Npn}
% \begin{macro}{\cs_new:Npx}
% \begin{macro}{\cs_new_protected_nopar:Npn}
% \begin{macro}{\cs_new_protected_nopar:Npx}
% \begin{macro}{\cs_new_protected:Npn}
% \begin{macro}{\cs_new_protected:Npx}
%   Global versions of the above functions.
%     \begin {macrocode}
\cs_set:Npn \cs_tmp:w #1#2 {
  \cs_set_protected_nopar:Npn #1 ##1 
    { 
       \chk_if_free_cs:N ##1
       #2 ##1
    }
}
\cs_tmp:w \cs_new_nopar:Npn           \cs_gset_nopar:Npn
\cs_tmp:w \cs_new_nopar:Npx           \cs_gset_nopar:Npx
\cs_tmp:w \cs_new:Npn                 \cs_gset:Npn
\cs_tmp:w \cs_new:Npx                 \cs_gset:Npx
\cs_tmp:w \cs_new_protected_nopar:Npn \cs_gset_protected_nopar:Npn
\cs_tmp:w \cs_new_protected_nopar:Npx \cs_gset_protected_nopar:Npx
\cs_tmp:w \cs_new_protected:Npn       \cs_gset_protected:Npn
\cs_tmp:w \cs_new_protected:Npx       \cs_gset_protected:Npx
%    \end{macrocode}
% \end{macro}
% \end{macro}
% \end{macro}
% \end{macro}
% \end{macro}
% \end{macro}
% \end{macro}
% \end{macro}
%
%
%
% \begin{macro}{\cs_set_nopar:cpn}
% \begin{macro}{\cs_set_nopar:cpx}
% \begin{macro}{\cs_gset_nopar:cpn}
% \begin{macro}{\cs_gset_nopar:cpx}
% \begin{macro}{\cs_new_nopar:cpn}
% \begin{macro}{\cs_new_nopar:cpx}
%   Like |\cs_set_nopar:Npn| and |\cs_new_nopar:Npn|, except that the
%   first argument consists of the sequence of characters that should
%   be used to form the name of the desired control sequence (the |c|
%   stands for csname argument, see the expansion module). Global
%   versions are also provided.
%
%   |\cs_set_nopar:cpn|\m{string}\m{rep-text} will turn \m{string}
%   into a csname and then assign \m {rep-text} to it by using
%   |\cs_set_nopar:Npn|.  This means that there might be a parameter
%   string between the two arguments.
%    \begin{macrocode}
\cs_set:Npn \cs_tmp:w #1#2{
  \cs_new_protected_nopar:Npn #1 { \exp_args:Nc #2 }
}
\cs_tmp:w  \cs_set_nopar:cpn  \cs_set_nopar:Npn 
\cs_tmp:w \cs_set_nopar:cpx  \cs_set_nopar:Npx 
\cs_tmp:w \cs_gset_nopar:cpn  \cs_gset_nopar:Npn 
\cs_tmp:w \cs_gset_nopar:cpx  \cs_gset_nopar:Npx 
\cs_tmp:w  \cs_new_nopar:cpn        \cs_new_nopar:Npn
\cs_tmp:w  \cs_new_nopar:cpx        \cs_new_nopar:Npx
%    \end{macrocode}
% \end{macro}
% \end{macro}
% \end{macro}
% \end{macro}
% \end{macro}
% \end{macro}
%
%
%
% \begin{macro}{\cs_set:cpn}
% \begin{macro}{\cs_set:cpx}
% \begin{macro}{\cs_gset:cpn}
% \begin{macro}{\cs_gset:cpx}
% \begin{macro}{\cs_new:cpn}
% \begin{macro}{\cs_new:cpx}
%   Variants of the |\cs_set:Npn| versions which make a csname out
%   of the first arguments. We may also do this globally.
%    \begin{macrocode}
\cs_tmp:w \cs_set:cpn  \cs_set:Npn 
\cs_tmp:w \cs_set:cpx  \cs_set:Npx 
\cs_tmp:w \cs_gset:cpn  \cs_gset:Npn 
\cs_tmp:w \cs_gset:cpx  \cs_gset:Npx 
\cs_tmp:w \cs_new:cpn   \cs_new:Npn 
\cs_tmp:w \cs_new:cpx  \cs_new:Npx 
%    \end{macrocode}
% \end{macro}
% \end{macro}
% \end{macro}
% \end{macro}
% \end{macro}
% \end{macro}
%
% \begin{macro}{\cs_set_protected_nopar:cpn}
% \begin{macro}{\cs_set_protected_nopar:cpx}
% \begin{macro}{\cs_gset_protected_nopar:cpn}
% \begin{macro}{\cs_gset_protected_nopar:cpx}
% \begin{macro}{\cs_new_protected_nopar:cpn}
% \begin{macro}{\cs_new_protected_nopar:cpx}
%   Variants of the |\cs_set_protected_nopar:Npn| versions which make a csname
%   out of the first arguments. We may also do this globally.
%    \begin{macrocode}
\cs_tmp:w \cs_set_protected_nopar:cpn \cs_set_protected_nopar:Npn 
\cs_tmp:w \cs_set_protected_nopar:cpx  \cs_set_protected_nopar:Npx  
\cs_tmp:w \cs_gset_protected_nopar:cpn  \cs_gset_protected_nopar:Npn
\cs_tmp:w \cs_gset_protected_nopar:cpx  \cs_gset_protected_nopar:Npx
\cs_tmp:w \cs_new_protected_nopar:cpn   \cs_new_protected_nopar:Npn
\cs_tmp:w \cs_new_protected_nopar:cpx  \cs_new_protected_nopar:Npx 
%    \end{macrocode}
% \end{macro}
% \end{macro}
% \end{macro}
% \end{macro}
% \end{macro}
% \end{macro}
%
% \begin{macro}{\cs_set_protected:cpn}
% \begin{macro}{\cs_set_protected:cpx}
% \begin{macro}{\cs_gset_protected:cpn}
% \begin{macro}{\cs_gset_protected:cpx}
% \begin{macro}{\cs_new_protected:cpn}
% \begin{macro}{\cs_new_protected:cpx}
%   Variants of the |\cs_set_protected:Npn| versions which make a csname
%   out of the first arguments. We may also do this globally.
%    \begin{macrocode}
\cs_tmp:w \cs_set_protected:cpn \cs_set_protected:Npn 
\cs_tmp:w \cs_set_protected:cpx \cs_set_protected:Npx
\cs_tmp:w \cs_gset_protected:cpn  \cs_gset_protected:Npn 
\cs_tmp:w \cs_gset_protected:cpx  \cs_gset_protected:Npx
\cs_tmp:w \cs_new_protected:cpn  \cs_new_protected:Npn 
\cs_tmp:w \cs_new_protected:cpx  \cs_new_protected:Npx 
%    \end{macrocode}
% \end{macro}
% \end{macro}
% \end{macro}
% \end{macro}
% \end{macro}
% \end{macro}
%
% BACKWARDS COMPATIBILITY:
%    \begin{macrocode}
\cs_set_eq:NwN           \cs_gnew_nopar:Npn            \cs_new_nopar:Npn
\cs_set_eq:NwN                 \cs_gnew:Npn                  \cs_new:Npn
\cs_set_eq:NwN \cs_gnew_protected_nopar:Npn  \cs_new_protected_nopar:Npn
\cs_set_eq:NwN       \cs_gnew_protected:Npn        \cs_new_protected:Npn
\cs_set_eq:NwN           \cs_gnew_nopar:Npx            \cs_new_nopar:Npx
\cs_set_eq:NwN                 \cs_gnew:Npx                  \cs_new:Npx
\cs_set_eq:NwN \cs_gnew_protected_nopar:Npx  \cs_new_protected_nopar:Npx
\cs_set_eq:NwN       \cs_gnew_protected:Npx        \cs_new_protected:Npx
\cs_set_eq:NwN           \cs_gnew_nopar:cpn            \cs_new_nopar:cpn
\cs_set_eq:NwN                 \cs_gnew:cpn                  \cs_new:cpn
\cs_set_eq:NwN \cs_gnew_protected_nopar:cpn  \cs_new_protected_nopar:cpn
\cs_set_eq:NwN       \cs_gnew_protected:cpn        \cs_new_protected:cpn
\cs_set_eq:NwN           \cs_gnew_nopar:cpx            \cs_new_nopar:cpx
\cs_set_eq:NwN                 \cs_gnew:cpx                  \cs_new:cpx
\cs_set_eq:NwN \cs_gnew_protected_nopar:cpx  \cs_new_protected_nopar:cpx
\cs_set_eq:NwN       \cs_gnew_protected:cpx        \cs_new_protected:cpx
%    \end{macrocode}
%
%
% \begin{macro}[aux]{\use_0_parameter:}
% \begin{macro}[aux]{\use_1_parameter:}
% \begin{macro}[aux]{\use_2_parameter:}
% \begin{macro}[aux]{\use_3_parameter:}
% \begin{macro}[aux]{\use_4_parameter:}
% \begin{macro}[aux]{\use_5_parameter:}
% \begin{macro}[aux]{\use_6_parameter:}
% \begin{macro}[aux]{\use_7_parameter:}
% \begin{macro}[aux]{\use_8_parameter:}
% \begin{macro}[aux]{\use_9_parameter:}
%   For using parameters, i.e., when you need to define a function to
%   process three parameters. See \textsf{xparse} for an application.
%    \begin{macrocode}
\cs_set_nopar:cpn{use_0_parameter:}{}
\cs_set_nopar:cpn{use_1_parameter:}{{##1}}
\cs_set_nopar:cpn{use_2_parameter:}{{##1}{##2}}
\cs_set_nopar:cpn{use_3_parameter:}{{##1}{##2}{##3}}
\cs_set_nopar:cpn{use_4_parameter:}{{##1}{##2}{##3}{##4}}
\cs_set_nopar:cpn{use_5_parameter:}{{##1}{##2}{##3}{##4}{##5}}
\cs_set_nopar:cpn{use_6_parameter:}{{##1}{##2}{##3}{##4}{##5}{##6}}
\cs_set_nopar:cpn{use_7_parameter:}{{##1}{##2}{##3}{##4}{##5}{##6}{##7}}
\cs_set_nopar:cpn{use_8_parameter:}{
  {##1}{##2}{##3}{##4}{##5}{##6}{##7}{##8}}
\cs_set_nopar:cpn{use_9_parameter:}{
  {##1}{##2}{##3}{##4}{##5}{##6}{##7}{##8}{##9}}
%    \end{macrocode}
% \end{macro}
% \end{macro}
% \end{macro}
% \end{macro}
% \end{macro}
% \end{macro}
% \end{macro}
% \end{macro}
% \end{macro}
% \end{macro}
%
% \subsection{Copying definitions}
%
% \begin{macro}{\cs_set_eq:NN}
% \begin{macro}{\cs_set_eq:cN}
% \begin{macro}{\cs_set_eq:Nc}
% \begin{macro}{\cs_set_eq:cc}
%    These macros allow us to copy the definition of a control sequence
%    to another control sequence.
%
%    The |=| sign allows us to define funny char tokens like |=|
%    itself or \verb*= = with this function. For the definition of
%    |\c_space_chartok{~}| to work we need the |~| after the |=|.
%
%    |\cs_set_eq:NN| is long to avoid problems with a literal argument
%    of |\par|.  While |\cs_new_eq:NN| will probably never be correct
%    with a first argument of |\par|, define it long in order to throw
%    an `already defined' error rather than `runaway argument'.
%
%    The |c| variants are not protected in order for their arguments to
%    be constructed in the correct context.
%
%    \begin{macrocode}
\cs_set_protected:Npn \cs_set_eq:NN #1 { \cs_set_eq:NwN #1=~ }
\cs_set_protected_nopar:Npn \cs_set_eq:cN { \exp_args:Nc \cs_set_eq:NN }
\cs_set_protected_nopar:Npn \cs_set_eq:Nc { \exp_args:NNc \cs_set_eq:NN }
\cs_set_protected_nopar:Npn \cs_set_eq:cc { \exp_args:Ncc \cs_set_eq:NN }
%    \end{macrocode}
% \end{macro}
% \end{macro}
% \end{macro}
% \end{macro}
%
% \begin{macro}{\cs_new_eq:NN}
% \begin{macro}{\cs_new_eq:cN}
% \begin{macro}{\cs_new_eq:Nc}
% \begin{macro}{\cs_new_eq:cc}
%    \begin{macrocode}
\cs_new_protected:Npn \cs_new_eq:NN #1 {
  \chk_if_free_cs:N #1
  \pref_global:D \cs_set_eq:NN #1
}
\cs_new_protected_nopar:Npn \cs_new_eq:cN { \exp_args:Nc  \cs_new_eq:NN }
\cs_new_protected_nopar:Npn \cs_new_eq:Nc { \exp_args:NNc \cs_new_eq:NN }
\cs_new_protected_nopar:Npn \cs_new_eq:cc { \exp_args:Ncc \cs_new_eq:NN }
%    \end{macrocode}
% \end{macro}
% \end{macro}
% \end{macro}
% \end{macro}
%
% \begin{macro}{\cs_gset_eq:NN}
% \begin{macro}{\cs_gset_eq:cN}
% \begin{macro}{\cs_gset_eq:Nc}
% \begin{macro}{\cs_gset_eq:cc}
%    \begin{macrocode}
\cs_new_protected:Npn \cs_gset_eq:NN { \pref_global:D  \cs_set_eq:NN }
\cs_new_protected_nopar:Npn \cs_gset_eq:Nc { \exp_args:NNc  \cs_gset_eq:NN }
\cs_new_protected_nopar:Npn \cs_gset_eq:cN { \exp_args:Nc   \cs_gset_eq:NN }
\cs_new_protected_nopar:Npn \cs_gset_eq:cc { \exp_args:Ncc  \cs_gset_eq:NN }
%    \end{macrocode}
% \end{macro}
% \end{macro}
% \end{macro}
% \end{macro}
%
%
% BACKWARDS COMPATIBILITY
%    \begin{macrocode}
\cs_set_eq:NN   \cs_gnew_eq:NN   \cs_new_eq:NN
\cs_set_eq:NN   \cs_gnew_eq:cN   \cs_new_eq:cN
\cs_set_eq:NN   \cs_gnew_eq:Nc   \cs_new_eq:Nc
\cs_set_eq:NN   \cs_gnew_eq:cc   \cs_new_eq:cc
%    \end{macrocode}
%
% \subsection{Undefining functions}
%
% \begin{macro}{\cs_undefine:N  , \cs_undefine:c }
% \begin{macro}{\cs_gundefine:N , \cs_gundefine:c }
%    The following function is used to free the main memory from the
%    definition of some function that isn't in use any longer.
%    \begin{macrocode}
\cs_new_protected_nopar:Npn \cs_undefine:N  #1 {
  \cs_set_eq:NN #1    \c_undefined:D
}
\cs_new_protected_nopar:Npn \cs_undefine:c  #1 {
  \cs_set_eq:cN {#1}  \c_undefined:D
}
\cs_new_protected_nopar:Npn \cs_gundefine:N #1 {
  \cs_gset_eq:NN #1   \c_undefined:D
}
\cs_new_protected_nopar:Npn \cs_gundefine:c #1 {
  \cs_gset_eq:cN {#1} \c_undefined:D
}
%    \end{macrocode}
% \end{macro}
% \end{macro}
%
%  \subsection{Diagnostic wrapper functions}
%
% \begin{macro}{\kernel_register_show:N,\kernel_register_show:c}
%    \begin{macrocode}
\cs_new_nopar:Npn \kernel_register_show:N #1 {
  \cs_if_exist:NTF #1
   {
    \tex_showthe:D #1
   }
   {
    \msg_kernel_bug:x {Register~ `\token_to_str:N #1'~ is~ not~ defined.}
   }
}
\cs_new_nopar:Npn \kernel_register_show:c { \exp_args:Nc \int_show:N }
%    \end{macrocode}
% \end{macro}
%
%
%  \subsection{Engine specific definitions}
%
% \begin{macro}{\c_xetex_is_engine_bool,\c_luatex_is_engine_bool}
% \begin{macro}[TF]{\xetex_if_engine:,\luatex_if_engine:}
%  In some cases it will be useful to know which engine we're running.
%  Don't provide a "_p" predicate because the "_bool" is used for the
%  same thing.
%    \begin{macrocode}
\cs_if_exist:NTF \xetex_version:D
  { \cs_new_eq:NN \c_xetex_is_engine_bool \c_true_bool  }
  { \cs_new_eq:NN \c_xetex_is_engine_bool \c_false_bool }
\prg_new_conditional:Npnn \xetex_if_engine: {TF,T,F} {
  \if_bool:N \c_xetex_is_engine_bool
    \prg_return_true: \else: \prg_return_false: \fi:
}
%    \end{macrocode}
%
%    \begin{macrocode}
\cs_if_exist:NTF \luatex_directlua:D
  { \cs_new_eq:NN \c_luatex_is_engine_bool \c_true_bool  }
  { \cs_new_eq:NN \c_luatex_is_engine_bool \c_false_bool }
\prg_set_conditional:Npnn \xetex_if_engine: {TF,T,F}{
  \if_bool:N \c_xetex_is_engine_bool \prg_return_true:
  \else: \prg_return_false: \fi:
}
\prg_set_conditional:Npnn \luatex_if_engine: {TF,T,F}{
  \if_bool:N \c_luatex_is_engine_bool \prg_return_true:
  \else: \prg_return_false: \fi:
}
%    \end{macrocode}
% \end{macro}
% \end{macro}
%
%
%
%
% \subsection{Scratch functions}
%
% \begin{macro}{\prg_do_nothing:}
%    I don't think this function belongs here, but one place is as
%    good as any other. I want to use this function when I want to
%    express `no operation'. It is for example used in templates where
%    depending on the users settings we have to either select an function that
%    does something, or one that does nothing.
%    \begin{macrocode}
\cs_new_nopar:Npn \prg_do_nothing: {}
%    \end{macrocode}
% \end{macro}
%
%
% \subsection{Defining functions from a given number of arguments}
%
% \begin{macro}{\cs_get_arg_count_from_signature:N}
% \begin{macro}[aux]{\cs_get_arg_count_from_signature_aux:nnN}
% \begin{macro}[aux]{\cs_get_arg_count_from_signature_auxii:w}
%   Counting the number of tokens in the signature, i.e., the number
%   of arguments the function should take.  If there is no signature,
%   we return that there is $-1$ arguments to signal an error.
%   Otherwise we insert the string |9876543210| after the signature.
%   If the signature is empty, the number we want is $0$ so we remove
%   the first nine tokens and return the tenth.  Similarly, if the
%   signature is |nnn| we want to remove the nine tokens |nnn987654|
%   and return $3$.  Therefore, we simply remove the first nine tokens
%   and then return the tenth.
%    \begin{macrocode}
\cs_set:Npn \cs_get_arg_count_from_signature:N #1{
  \cs_split_function:NN #1 \cs_get_arg_count_from_signature_aux:nnN
}
\cs_set:Npn \cs_get_arg_count_from_signature_aux:nnN #1#2#3{
  \if_predicate:w #3 % \bool_if:NTF here 
    \exp_after:wN \use_i:nn 
  \else:
    \exp_after:wN\use_ii:nn
  \fi: 
  {
    \exp_after:wN \cs_get_arg_count_from_signature_auxii:w 
      \use_none:nnnnnnnnn  #2 9876543210\q_nil 
  }
  {-1}
}
\cs_set:Npn \cs_get_arg_count_from_signature_auxii:w #1#2\q_nil{#1}
%    \end{macrocode}
% A variant form we need right away.
%    \begin{macrocode}
\cs_set_nopar:Npn \cs_get_arg_count_from_signature:c {
  \exp_args:Nc \cs_get_arg_count_from_signature:N
}
%    \end{macrocode}
% \end{macro}
% \end{macro}
% \end{macro}
%
%
%
%
% \begin{macro}[aux]{\cs_generate_from_arg_count:NNnn}
% \begin{macro}[aux]{\cs_generate_from_arg_count_error_msg:Nn}
%   We provide a constructor function for defining functions with a
%   given number of arguments.  For this we need to choose the correct
%   parameter text and then use that when defining.  Since \TeX\
%   supports from zero to nine arguments, we use a simple switch to
%   choose the correct parameter text, ensuring the result is returned
%   after finishing the conditional.  If it is not between zero and
%   nine, we throw an error.
%
%   1: function to define, 2: with what to define it, 3: the number of
%   args it requires and 4: the replacement text
%    \begin{macrocode}
\cs_set:Npn \cs_generate_from_arg_count:NNnn #1#2#3#4{
  \tex_ifcase:D \etex_numexpr:D #3\tex_relax:D
    \use_i_after_orelse:nw{#2#1}
  \or:
    \use_i_after_orelse:nw{#2#1 ##1}
  \or:
    \use_i_after_orelse:nw{#2#1 ##1##2}
  \or:
    \use_i_after_orelse:nw{#2#1 ##1##2##3}
  \or:
    \use_i_after_orelse:nw{#2#1 ##1##2##3##4}
  \or:
    \use_i_after_orelse:nw{#2#1 ##1##2##3##4##5}
  \or:
    \use_i_after_orelse:nw{#2#1 ##1##2##3##4##5##6}
  \or:
    \use_i_after_orelse:nw{#2#1 ##1##2##3##4##5##6##7}
  \or:
    \use_i_after_orelse:nw{#2#1 ##1##2##3##4##5##6##7##8}
  \or:
    \use_i_after_orelse:nw{#2#1 ##1##2##3##4##5##6##7##8##9}
  \else:
    \use_i_after_fi:nw{
      \cs_generate_from_arg_count_error_msg:Nn#1{#3}
      \use_none:n % to remove replacement text
    }
  \fi:
  {#4}
}
%    \end{macrocode}
% A variant form we need right away.
%    \begin{macrocode}
\cs_set_nopar:Npn \cs_generate_from_arg_count:cNnn {
  \exp_args:Nc \cs_generate_from_arg_count:NNnn 
}
%    \end{macrocode}
% The error message. Elsewhere we use the value of $-1$ to signal a
% missing colon in a function, so provide a hint for help on this.
%    \begin{macrocode}
\cs_set:Npn \cs_generate_from_arg_count_error_msg:Nn #1#2 {
  \msg_kernel_bug:x {
    You're~ trying~ to~ define~ the~ command~ `\token_to_str:N #1'~
    with~ \use:n{\tex_the:D\etex_numexpr:D #2\tex_relax:D} ~
    arguments~ but~ I~ only~ allow~ 0-9~arguments.~Perhaps~you~
    forgot~to~use~a~colon~in~the~function~name?~
    I~ can~ probably~ not~ help~ you~ here 
  } 
}
%    \end{macrocode}
% \end{macro}
% \end{macro}
%
%
%
%
% \subsection{Using the signature to define functions}
%
% We can now combine some of the tools we have to provide a simple
% interface for defining functions.  We define some simpler functions
% with user interface |\cs_set:Nn \foo_bar:nn {#1,#2}|, i.e., the
% number of arguments is read from the signature.
%
% 
% \begin{macro}{\cs_set:Nn}
% \begin{macro}{\cs_set:Nx}
% \begin{macro}{\cs_set_nopar:Nn}
% \begin{macro}{\cs_set_nopar:Nx}
% \begin{macro}{\cs_set_protected:Nn}
% \begin{macro}{\cs_set_protected:Nx}
% \begin{macro}{\cs_set_protected_nopar:Nn}
% \begin{macro}{\cs_set_protected_nopar:Nx}
% \begin{macro}{\cs_gset:Nn}
% \begin{macro}{\cs_gset:Nx}
% \begin{macro}{\cs_gset_nopar:Nn}
% \begin{macro}{\cs_gset_nopar:Nx}
% \begin{macro}{\cs_gset_protected:Nn}
% \begin{macro}{\cs_gset_protected:Nx}
% \begin{macro}{\cs_gset_protected_nopar:Nn}
% \begin{macro}{\cs_gset_protected_nopar:Nx}
% We want to define |\cs_set:Nn| as 
% \begin{verbatim}
% \cs_set_protected:Npn \cs_set:Nn #1#2{
%   \cs_generate_from_arg_count:NNnn #1\cs_set:Npn 
%     {\cs_get_arg_count_from_signature:N #1}{#2}
% }
% \end{verbatim}
% In short, to define |\cs_set:Nn| we need just use |\cs_set:Npn|,
% everything else is the same for each variant.  Therefore, we can
% make it simpler by temporarily defining a function to do this for
% us.
%    \begin{macrocode}
\cs_set:Npn \cs_tmp:w #1#2#3{
  \cs_set_protected:cpx {cs_#1:#2}##1##2{
    \exp_not:N \cs_generate_from_arg_count:NNnn ##1
    \exp_after:wN \exp_not:N \cs:w cs_#1:#3 \cs_end: 
      {\exp_not:N\cs_get_arg_count_from_signature:N ##1}{##2}
  }
}
%    \end{macrocode}
% Then we define the 32 variants beginning with |N|. 
%    \begin{macrocode}
\cs_tmp:w {set}{Nn}{Npn}
\cs_tmp:w {set}{Nx}{Npx}
\cs_tmp:w {set_nopar}{Nn}{Npn}
\cs_tmp:w {set_nopar}{Nx}{Npx}
\cs_tmp:w {set_protected}{Nn}{Npn}
\cs_tmp:w {set_protected}{Nx}{Npx}
\cs_tmp:w {set_protected_nopar}{Nn}{Npn}
\cs_tmp:w {set_protected_nopar}{Nx}{Npx}
\cs_tmp:w {gset}{Nn}{Npn}
\cs_tmp:w {gset}{Nx}{Npx}
\cs_tmp:w {gset_nopar}{Nn}{Npn}
\cs_tmp:w {gset_nopar}{Nx}{Npx}
\cs_tmp:w {gset_protected}{Nn}{Npn}
\cs_tmp:w {gset_protected}{Nx}{Npx}
\cs_tmp:w {gset_protected_nopar}{Nn}{Npn}
\cs_tmp:w {gset_protected_nopar}{Nx}{Npx}
%    \end{macrocode}
% \end{macro}
% \end{macro}
% \end{macro}
% \end{macro}
% \end{macro}
% \end{macro}
% \end{macro}
% \end{macro}
% \end{macro}
% \end{macro}
% \end{macro}
% \end{macro}
% \end{macro}
% \end{macro}
% \end{macro}
% \end{macro}
%
% \begin{macro}{\cs_new:Nn}
% \begin{macro}{\cs_new:Nx}
% \begin{macro}{\cs_new_nopar:Nn}
% \begin{macro}{\cs_new_nopar:Nx}
% \begin{macro}{\cs_new_protected:Nn}
% \begin{macro}{\cs_new_protected:Nx}
% \begin{macro}{\cs_new_protected_nopar:Nn}
% \begin{macro}{\cs_new_protected_nopar:Nx}
%    \begin{macrocode}
\cs_tmp:w {new}{Nn}{Npn}
\cs_tmp:w {new}{Nx}{Npx}
\cs_tmp:w {new_nopar}{Nn}{Npn}
\cs_tmp:w {new_nopar}{Nx}{Npx}
\cs_tmp:w {new_protected}{Nn}{Npn}
\cs_tmp:w {new_protected}{Nx}{Npx}
\cs_tmp:w {new_protected_nopar}{Nn}{Npn}
\cs_tmp:w {new_protected_nopar}{Nx}{Npx}
%    \end{macrocode}
% \end{macro}
% \end{macro}
% \end{macro}
% \end{macro}
% \end{macro}
% \end{macro}
% \end{macro}
% \end{macro}
%
% Then something similar for the |c| variants. 
% \begin{verbatim}
% \cs_set_protected:Npn \cs_set:cn #1#2{
%   \cs_generate_from_arg_count:cNnn {#1}\cs_set:Npn 
%     {\cs_get_arg_count_from_signature:c {#1}}{#2}
% }
% \end{verbatim}
%    \begin{macrocode}
\cs_set:Npn \cs_tmp:w #1#2#3{
  \cs_set_protected:cpx {cs_#1:#2}##1##2{
    \exp_not:N\cs_generate_from_arg_count:cNnn {##1}
    \exp_after:wN \exp_not:N \cs:w cs_#1:#3 \cs_end: 
      {\exp_not:N\cs_get_arg_count_from_signature:c {##1}}{##2}
  }
}
%    \end{macrocode}
% \begin{macro}{\cs_set:cn}
% \begin{macro}{\cs_set:cx}
% \begin{macro}{\cs_set_nopar:cn}
% \begin{macro}{\cs_set_nopar:cx}
% \begin{macro}{\cs_set_protected:cn}
% \begin{macro}{\cs_set_protected:cx}
% \begin{macro}{\cs_set_protected_nopar:cn}
% \begin{macro}{\cs_set_protected_nopar:cx}
% \begin{macro}{\cs_gset:cn}
% \begin{macro}{\cs_gset:cx}
% \begin{macro}{\cs_gset_nopar:cn}
% \begin{macro}{\cs_gset_nopar:cx}
% \begin{macro}{\cs_gset_protected:cn}
% \begin{macro}{\cs_gset_protected:cx}
% \begin{macro}{\cs_gset_protected_nopar:cn}
% \begin{macro}{\cs_gset_protected_nopar:cx}
% The 32 |c| variants.
%    \begin{macrocode}
\cs_tmp:w {set}{cn}{Npn}
\cs_tmp:w {set}{cx}{Npx}
\cs_tmp:w {set_nopar}{cn}{Npn}
\cs_tmp:w {set_nopar}{cx}{Npx}
\cs_tmp:w {set_protected}{cn}{Npn}
\cs_tmp:w {set_protected}{cx}{Npx}
\cs_tmp:w {set_protected_nopar}{cn}{Npn}
\cs_tmp:w {set_protected_nopar}{cx}{Npx}
\cs_tmp:w {gset}{cn}{Npn}
\cs_tmp:w {gset}{cx}{Npx}
\cs_tmp:w {gset_nopar}{cn}{Npn}
\cs_tmp:w {gset_nopar}{cx}{Npx}
\cs_tmp:w {gset_protected}{cn}{Npn}
\cs_tmp:w {gset_protected}{cx}{Npx}
\cs_tmp:w {gset_protected_nopar}{cn}{Npn}
\cs_tmp:w {gset_protected_nopar}{cx}{Npx}
%    \end{macrocode}
% \end{macro}
% \end{macro}
% \end{macro}
% \end{macro}
% \end{macro}
% \end{macro}
% \end{macro}
% \end{macro}
% \end{macro}
% \end{macro}
% \end{macro}
% \end{macro}
% \end{macro}
% \end{macro}
% \end{macro}
% \end{macro}
%
% \begin{macro}{\cs_new:cn}
% \begin{macro}{\cs_new:cx}
% \begin{macro}{\cs_new_nopar:cn}
% \begin{macro}{\cs_new_nopar:cx}
% \begin{macro}{\cs_new_protected:cn}
% \begin{macro}{\cs_new_protected:cx}
% \begin{macro}{\cs_new_protected_nopar:cn}
% \begin{macro}{\cs_new_protected_nopar:cx}
%    \begin{macrocode}
\cs_tmp:w {new}{cn}{Npn}
\cs_tmp:w {new}{cx}{Npx}
\cs_tmp:w {new_nopar}{cn}{Npn}
\cs_tmp:w {new_nopar}{cx}{Npx}
\cs_tmp:w {new_protected}{cn}{Npn}
\cs_tmp:w {new_protected}{cx}{Npx}
\cs_tmp:w {new_protected_nopar}{cn}{Npn}
\cs_tmp:w {new_protected_nopar}{cx}{Npx}
%    \end{macrocode}
% \end{macro}
% \end{macro}
% \end{macro}
% \end{macro}
% \end{macro}
% \end{macro}
% \end{macro}
% \end{macro}
%
%
%
%
%
%
% \begin{macro}{\cs_if_eq_p:NN,\cs_if_eq_p:cN,\cs_if_eq_p:Nc,\cs_if_eq_p:cc}
% \begin{macro}[TF]{\cs_if_eq:NN,\cs_if_eq:cN,\cs_if_eq:Nc,\cs_if_eq:cc}
% Check if two control sequences are identical.
%    \begin{macrocode}
\prg_set_conditional:Npnn \cs_if_eq:NN #1#2{p,TF,T,F}{
  \if_meaning:w #1#2
  \prg_return_true: \else: \prg_return_false: \fi:
}
\cs_new_nopar:Npn \cs_if_eq_p:cN {\exp_args:Nc  \cs_if_eq_p:NN}
\cs_new_nopar:Npn \cs_if_eq:cNTF {\exp_args:Nc  \cs_if_eq:NNTF}
\cs_new_nopar:Npn \cs_if_eq:cNT  {\exp_args:Nc  \cs_if_eq:NNT}
\cs_new_nopar:Npn \cs_if_eq:cNF  {\exp_args:Nc  \cs_if_eq:NNF}
\cs_new_nopar:Npn \cs_if_eq_p:Nc {\exp_args:NNc \cs_if_eq_p:NN}
\cs_new_nopar:Npn \cs_if_eq:NcTF {\exp_args:NNc \cs_if_eq:NNTF}
\cs_new_nopar:Npn \cs_if_eq:NcT  {\exp_args:NNc \cs_if_eq:NNT}
\cs_new_nopar:Npn \cs_if_eq:NcF  {\exp_args:NNc \cs_if_eq:NNF}
\cs_new_nopar:Npn \cs_if_eq_p:cc {\exp_args:Ncc \cs_if_eq_p:NN}
\cs_new_nopar:Npn \cs_if_eq:ccTF {\exp_args:Ncc \cs_if_eq:NNTF}
\cs_new_nopar:Npn \cs_if_eq:ccT  {\exp_args:Ncc \cs_if_eq:NNT}
\cs_new_nopar:Npn \cs_if_eq:ccF  {\exp_args:Ncc \cs_if_eq:NNF}
%    \end{macrocode}
% \end{macro}
% \end{macro}
%
%    \begin{macrocode}
%</initex|package>
%    \end{macrocode}
%
%    \begin{macrocode}
%<*showmemory>
\showMemUsage
%</showmemory>
%    \end{macrocode}
%
% \end{implementation}
%
% \PrintIndex
%
% \endinput

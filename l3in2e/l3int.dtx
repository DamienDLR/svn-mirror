% \iffalse
%% File: l3int.dtx Copyright (C) 1990-2010 LaTeX3 project
%%
%% It may be distributed and/or modified under the conditions of the
%% LaTeX Project Public License (LPPL), either version 1.3c of this
%% license or (at your option) any later version.  The latest version
%% of this license is in the file
%%
%%    http://www.latex-project.org/lppl.txt
%%
%% This file is part of the ``expl3 bundle'' (The Work in LPPL)
%% and all files in that bundle must be distributed together.
%%
%% The released version of this bundle is available from CTAN.
%%
%% -----------------------------------------------------------------------
%%
%% The development version of the bundle can be found at
%%
%%    http://www.latex-project.org/svnroot/experimental/trunk/
%%
%% for those people who are interested.
%%
%%%%%%%%%%%
%% NOTE: %%
%%%%%%%%%%%
%%
%%   Snapshots taken from the repository represent work in progress and may
%%   not work or may contain conflicting material!  We therefore ask
%%   people _not_ to put them into distributions, archives, etc. without
%%   prior consultation with the LaTeX Project Team.
%%
%% -----------------------------------------------------------------------
%
%<*driver|package>
\RequirePackage{l3names}
%</driver|package>
%\fi
\GetIdInfo$Id$
          {L3 Experimental Integer module}
%\iffalse
%<*driver>
%\fi
\ProvidesFile{\filename.\filenameext}
  [\filedate\space v\fileversion\space\filedescription]
%\iffalse
\documentclass[full]{l3doc}
\begin{document}
\DocInput{l3int.dtx}
\end{document}
%</driver>
% \fi
%
%
% \title{The \textsf{l3int} package\thanks{This file
%         has version number \fileversion, last
%         revised \filedate.}\\
% Integers/counters}
% \author{\Team}
% \date{\filedate}
% \maketitle
%
% \begin{documentation}
% 
%\section{Integer values}
%
%
% Calculation and comparison of integer values can be carried out
% using literal numbers, \texttt{int} registers, constants and
% integers stored in token list variables. The standard operators
% \texttt{+}, \texttt{-}, \texttt{/} and \texttt{*} and
% parentheses can be used within such expressions to carry
% arithmetic operations. This module carries out these functions
% on \emph{integer expressions} (`\texttt{int expr}').
%
%\subsection{Integer expressions}
%
%\begin{function}{ \int_eval:n / (EXP) }
%  \begin{syntax}
%    \cs{int_eval:n} \Arg{integer expression}
%  \end{syntax}
%  Evaluates the \meta{integer expression}, expanding any
%  integer and token list variables within the \meta{expression}
%  to their content (without requiring \cs{int_use:N}/\cs{tl_use:N})
%  and applying the standard mathematical rules. The result of the
%  calculation is left in the input stream as a number. For example
%  both
%  \begin{verbatim}
%    \int_eval:n { 5 +  4 * 3 - ( 3 + 4 * 5 ) }
%  \end{verbatim}   
%  and
%  \begin{verbatim}
%    \tl_new:N  \l_my_tl
%    \tl_set:Nn \l_my_tl { 5 }
%    \int_new:N  \l_my_int
%    \int\set:Nn \l_my_int { 4 }
%    \int_eval:n { \l_my_tl +  \l_my_int * 3 - ( 3 + 4 * 5 ) }
%  \end{verbatim}   
%  both evaluate to \( -6 \). The  \Arg{integer expression} may
%  contain the operators \texttt{+}, \texttt{-}, \texttt{*} and
%  \texttt{/}, along with parenthesis \texttt{(} and \texttt{)}.
%\end{function}
%
%\begin{function}{ \int_abs:n / (EXP) }
%  \begin{syntax}
%    \cs{int_abs:n} \Arg{integer expression}
%  \end{syntax}
%  Evaluates the \meta{integer expression} as described for
%  \cs{int_eval:n} and leaves the absolute value of the result in
%  the input stream.
%\end{function}
%
%\begin{function}{ \int_div_round:nn / (EXP) }
%  \begin{syntax}
%    \cs{int_div_round:nn} \Arg{intexpr1} \Arg{intexpr2}
%  \end{syntax}
%  Evaluates the two \meta{integer expressions} as described earlier,
%  then calculates the result of dividing the first value by the
%  second, rounding any remainder. Note that division using "/"
%  is identical to this function.
%\end{function}
%
%\begin{function}{ \int_div_truncate:nn / (EXP) }
%  \begin{syntax}
%    \cs{int_div_truncate:nn} \Arg{intexpr1} \Arg{intexpr2}
%  \end{syntax}
%  Evaluates the two \meta{integer expressions} as described earlier,
%  then calculates the result of dividing the first value by the
%  second, truncating any remainder. Note that division using "/"
%  rounds the result.
%\end{function}
%
%\begin{function}{ 
%  \int_max:nn / (EXP) | 
%  \int_min:nn / (EXP) |
%}
%  \begin{syntax}
%    \cs{int_max:nn} \Arg{intexpr1} \Arg{intexpr2}
%    \cs{int_min:nn} \Arg{intexpr1} \Arg{intexpr2}
%  \end{syntax}
%  Evaluates the \meta{integer expressions} as described for
%  \cs{int_eval:n} and leaves either the larger or smaller value
%  in the input stream, depending on the function name.
%\end{function}
%
%\begin{function}{ \int_mod:nn / (EXP) }
%  \begin{syntax}
%    \cs{int_mod:nn} \Arg{intexpr1} \Arg{intexpr2}
%  \end{syntax}
%  Evaluates the two \meta{integer expressions} as described earlier,
%  then calculates the integer remainder of dividing the first
%  expression by the second. This is left in the input stream.
%\end{function}
%
%\subsection{Integer variables}
%
%\begin{function}{ 
%  \int_new:N |
%  \int_new:c |
%}
%  \begin{syntax}
%    \cs{int_new:N} \meta{integer}
%  \end{syntax}
%  Creates a new \meta{inter} or raises an error if the name is
%  already taken. The declaration is global. The \meta{integer} will 
%  initially be equal to \( 0 \).
%\end{function}
%
%\begin{function}{ 
%  \int_set_eq:NN |
%  \int_set_eq:cN |
%  \int_set_eq:Nc |
%  \int_set_eq:cc |
%}
%  \begin{syntax}
%    \cs{int_set_eq:NN} \meta{integer1} \meta{integer 2}
%  \end{syntax}
%  Sets the content of \meta{integer1} equal to that of 
%  \meta{integer 2}. This assignment is restricted to the current
%  \TeX\ group level.
%\end{function}
%
%\begin{function}{ 
%  \int_gset_eq:NN |
%  \int_gset_eq:cN |
%  \int_gset_eq:Nc |
%  \int_gset_eq:cc |
%}
%  \begin{syntax}
%    \cs{int_gset_eq:NN} \meta{integer1} \meta{integer2}
%  \end{syntax}
%  Sets the content of \meta{integer1} equal to that of \meta{integer2}. 
%  This assignment is global and so is not limited by the current 
%  \TeX\ group level.
%\end{function}
%
%\begin{function}{ 
%  \int_add:Nn |
%  \int_add:cn |
%}
%  \begin{syntax}
%    \cs{int_add:Nn} \meta{integer} \Arg{integer expression}
%  \end{syntax}
%  Adds the result of the \meta{integer expression} to the current 
%  content of the \meta{integer}. This assignment is local.
%\end{function}
%
%\begin{function}{ 
%  \int_gadd:Nn |
%  \int_gadd:cn |
%}
%  \begin{syntax}
%    \cs{int_gadd:Nn} \meta{integer} \Arg{integer expression}
%  \end{syntax}
%  Adds the result of the \meta{integer expression} to the current 
%  content of the \meta{integer}. This assignment is global.
%\end{function}
% 
%\begin{function}{ 
%  \int_decr:N |
%  \int_decr:c |
%}
%  \begin{syntax}
%    \cs{int_decr:N} \meta{integer}
%  \end{syntax}
%  Decreases the value stored in \meta{integer} by \( 1 \) within
%  the scope of the current \TeX\ group.
%\end{function}
%
%\begin{function}{ 
%  \int_gdecr:N |
%  \int_gdecr:c |
%}
%  \begin{syntax}
%    \cs{int_incr:N} \meta{integer}
%  \end{syntax}
%  Decreases the value stored in \meta{integer} by \( 1 \) globally
%  (\emph{i.e}.~not limited by the current group level).
%\end{function}
% 
%\begin{function}{ 
%  \int_incr:N |
%  \int_incr:c |
%}
%  \begin{syntax}
%    \cs{int_incr:N} \meta{integer}
%  \end{syntax}
%  Increases the value stored in \meta{integer} by \( 1 \) within
%  the scope of the current \TeX\ group.
%\end{function}
%
%\begin{function}{ 
%  \int_gincr:N |
%  \int_gincr:c |
%}
%  \begin{syntax}
%    \cs{int_incr:N} \meta{integer}
%  \end{syntax}
%  Increases the value stored in \meta{integer} by \( 1 \) globally
%  (\emph{i.e}.~not limited by the current group level).
%\end{function}
%
%\begin{function}{ 
%  \int_set:Nn |
%  \int_set:cn |
%}
%  \begin{syntax}
%    \cs{int_set:Nn} \meta{integer} \Arg{integer expression}
%  \end{syntax}
%  Sets \meta{integer} to the value of \meta{integer expression}, 
%  which must evaluate to an integer (as described for 
%  \cs{int_eval:n}).  This assignment is restricted to the 
%  current \TeX\ group.
%\end{function}
%
%\begin{function}{ 
%  \int_gset:Nn |
%  \int_gset:cn |
%}
%  \begin{syntax}
%    \cs{int_gset:Nn} \meta{integer} \Arg{integer expression}
%  \end{syntax}
%  Sets \meta{integer} to the value of \meta{integer expression}, 
%  which must evaluate to an integer (as described for 
%  \cs{int_eval:n}). This assignment is global and is not limited 
%  to the current \TeX\ group level.
%\end{function} 
%
%\begin{function}{ 
%  \int_sub:Nn |
%  \int_sub:cn |
%}
%  \begin{syntax}
%    \cs{int_sub:Nn} \meta{integer} \Arg{integer expression}
%  \end{syntax}
%  Subtracts the result of the \meta{integer expression} to the 
%  current content of the \meta{integer}. This assignment is local.
%\end{function}
%
%\begin{function}{ 
%  \int_gsub:Nn |
%  \int_gsub:cn |
%}
%  \begin{syntax}
%    \cs{int_gsub:Nn} \meta{integer} \Arg{integer expression}
%  \end{syntax}
%  Subtracts the result of the \meta{integer expression} to the 
%  current content of the \meta{integer}. This assignment is global.
%\end{function}
%
%\begin{function}{ 
%  \int_zero:N |
%  \int_zero:c |
%}
%  \begin{syntax}
%    \cs{int_zero:N} \meta{integer}
%  \end{syntax}
%  Sets \meta{integer} to \( 0 \) within the scope of the current
%  \TeX\ group.
%\end{function}
%
%\begin{function}{ 
%  \int_gzero:N |
%  \int_gzero:c |
%}
%  \begin{syntax}
%    \cs{int_gzero:N} \meta{integer}
%  \end{syntax}
%  Sets \meta{integer} to \( 0 \) globally, \emph{i.e}.~not
%  restricted by the current \TeX\ group level.
%\end{function}
%
%\begin{function}{
%  \int_show:N |
%  \int_show:c |
%}
%  \begin{syntax}
%    \cs{int_show:N} \meta{integer}
%  \end{syntax}
%  Displays the value of the \meta{integer} on the terminal.
%\end{function}
%
%\begin{function}{ 
%  \int_use:N / (EXP) |
%  \int_use:c / (EXP) |
%}
%  \begin{syntax}
%    \cs{int_use:N} \meta{integer}
%  \end{syntax}
%  Recovers the content of a \meta{integer} and places it directly 
%  in the input stream. An error will be raised if the variable does 
%  not exist or if it is invalid. Can be omitted in places where a
%  \meta{integer} is required (such as in the first and third arguments
%  of \cs{int_compare:nNnTF}).
%\end{function}
%
%\subsection{Comparing integer expressions}
%
%\begin{function}{ 
%  \int_compare_p:nNn / (EXP)      |
%  \int_compare:nNn   / (EXP) (TF) |
%}
%  \begin{syntax}
%    \cs{int_compare_p:nNn} 
%    ~~\Arg{intexpr1} \meta{relation} \Arg{intexpr2}
%    \cs{int_compare:nNnTF} 
%    ~~\Arg{intexpr1} \meta{relation} \Arg{intexpr2}
%    ~~\Arg{true code} \Arg{false code}
%  \end{syntax}
%  This function first evaluates each of the \meta{integer expressions}
%  as described for \cs{int_eval:n}. The two results are then
%  compared using the \meta{relation}:
%  \begin{center}
%    \begin{tabular}{ll}
%      Equal                 & "=" \\
%      Greater than          & ">" \\
%      Less than             & "<" \\
%    \end{tabular}
%  \end{center}
%  The branching versions then leave either \meta{true code} or 
%  \meta{false code} in the input stream, as appropriate to the truth 
%  of the test and the variant of the function chosen. The logical 
%  truth of the test is left in the input stream by the predicate
%  version.
%\end{function}
%
%\begin{function}{ 
%  \int_compare_p:n / (EXP)      |
%  \int_compare:n   / (EXP) (TF) |
%}
%  \begin{syntax}
%    \cs{int_compare_p:n} 
%    ~~\{ \meta{intexpr1} \meta{relation} \meta{intexpr2} \}
%    \cs{int_compare:nTF} 
%    ~~\{ \meta{intexpr1} \meta{relation} \meta{intexpr2} \}
%    ~~\Arg{true code} \Arg{false code}
%  \end{syntax}
%  This function first evaluates each of the \meta{integer expressions}
%  as described for \cs{int_eval:n}. The two results are then
%  compared using the \meta{relation}:
%  \begin{center}
%    \begin{tabular}{ll}
%      Equal                    & "=" or "==" \\
%      Greater than or equal to & "=>"        \\
%      Greater than             & ">"         \\
%      Less than or equal to    & "=<"        \\
%      Less than                & "<"         \\
%      Not equal                & "!="        \\
%    \end{tabular}
%  \end{center}
%  The branching versions then leave either \meta{true code} or 
%  \meta{false code} in the input stream, as appropriate to the truth 
%  of the test and the variant of the function chosen. The logical 
%  truth of the test is left in the input stream by the predicate
%  version.
%\end{function}
%
%\begin{function}{ 
%  \int_if_even_p:n / (EXP)      |
%  \int_if_even:n   / (EXP) (TF) |
%  \int_if_odd_p:n  / (EXP)      |
%  \int_if_odd:n    / (EXP) (TF) |
%}
%  \begin{syntax}
%    \cs{int_if_odd_p:n} \Arg{integer expression}
%    \cs{int_if_odd:nTF} \Arg{integer expression}
%    ~~\Arg{true code} \Arg{false code}
%  \end{syntax}
%  This function first evaluates the \meta{integer expression}
%  as described for \cs{int_eval:n}. It then evaluates if this
%  is odd or even, as appropriate. The branching versions then leave 
%  either \meta{true code} or \meta{false code} in the input stream, 
%  as appropriate to the truth of the test and the variant of the 
%  function chosen. The logical truth of the test is left in the input
%  stream by the predicate version.
%\end{function}
%
%\begin{function}{ \int_do_while:nNnn / (EXP) }
%  \begin{syntax}
%     \cs{int_do_while:nNnn} 
%     ~~\Arg{intexpr1} \meta{relation} \Arg{intexpr2} \Arg{code}
%  \end{syntax}
%  Evaluates the relationship between the two \meta{integer expressions}
%  as described for \cs{int_compare:nNnTF}, and then places the
%  \meta{code} in the input stream if the \meta{relation} is
%  \texttt{true}. After the \meta{code} has been processed by \TeX\ the
%  test will be repeated, and a loop will occur until the test is
%  \texttt{false}.
% \end{function}
% 
%\begin{function}{ \int_do_until:nNnn / (EXP) }
%  \begin{syntax}
%     \cs{int_do_until:nNnn} 
%     ~~\Arg{intexpr1} \meta{relation} \Arg{intexpr2} \Arg{code}
%  \end{syntax}
%  Evaluates the relationship between the two \meta{integer expressions}
%  as described for \cs{int_compare:nNnTF}, and then places the
%  \meta{code} in the input stream if the \meta{relation} is
%  \texttt{false}. After the \meta{code} has been processed by \TeX\ the
%  test will be repeated, and a loop will occur until the test is
%  \texttt{true}.
% \end{function}
% 
%\begin{function}{ \int_until_do:nNnn / (EXP) }
%  \begin{syntax}
%     \cs{int_until_do:nNnn} 
%     ~~\Arg{intexpr1} \meta{relation} \Arg{intexpr2} \Arg{code}
%  \end{syntax}
%  Places the \meta{code} in the input stream for \TeX\ to process, and
%  then evaluates the relationship between the two 
%  \meta{integer expressions} as described for \cs{int_compare:nNnTF}.
%  If the test is \texttt{false} then the \meta{code} will be inserted
%  into the input stream again and a loop will occur until the 
%  \meta{relation} is \texttt{true}.
% \end{function}
% 
%\begin{function}{ \int_while_do:nNnn / (EXP) }
%  \begin{syntax}
%     \cs{int_while_do:nNnn} \
%     ~~\Arg{intexpr1} \meta{relation} \Arg{intexpr2} \Arg{code}
%  \end{syntax}
%  Places the \meta{code} in the input stream for \TeX\ to process, and
%  then evaluates the relationship between the two 
%  \meta{integer expressions} as described for \cs{int_compare:nNnTF}.
%  If the test is \texttt{true} then the \meta{code} will be inserted
%  into the input stream again and a loop will occur until the 
%  \meta{relation} is \texttt{false}.
% \end{function}
%
%\begin{function}{ \int_do_while:nn / (EXP) }
%  \begin{syntax}
%     \cs{int_do_while:nNnn} 
%     ~~\{ \meta{intexpr1} \meta{relation} \meta{intexpr2} \} \Arg{code}
%  \end{syntax}
%  Evaluates the relationship between the two \meta{integer expressions}
%  as described for \cs{int_compare:nTF}, and then places the
%  \meta{code} in the input stream if the \meta{relation} is
%  \texttt{true}. After the \meta{code} has been processed by \TeX\ the
%  test will be repeated, and a loop will occur until the test is
%  \texttt{false}.
% \end{function}
% 
%\begin{function}{ \int_do_until:nn / (EXP) }
%  \begin{syntax}
%     \cs{int_do_until:nn} 
%     ~~\{ \meta{intexpr1} \meta{relation} \meta{intexpr2} \} \Arg{code}
%  \end{syntax}
%  Evaluates the relationship between the two \meta{integer expressions}
%  as described for \cs{int_compare:nTF}, and then places the
%  \meta{code} in the input stream if the \meta{relation} is
%  \texttt{false}. After the \meta{code} has been processed by \TeX\ the
%  test will be repeated, and a loop will occur until the test is
%  \texttt{true}.
% \end{function}
% 
%\begin{function}{ \int_until_do:nn / (EXP) }
%  \begin{syntax}
%     \cs{int_until_do:nn} 
%     ~~\{ \meta{intexpr1} \meta{relation} \meta{intexpr2} \} \Arg{code}
%  \end{syntax}
%  Places the \meta{code} in the input stream for \TeX\ to process, and
%  then evaluates the relationship between the two 
%  \meta{integer expressions} as described for \cs{int_compare:nTF}.
%  If the test is \texttt{false} then the \meta{code} will be inserted
%  into the input stream again and a loop will occur until the 
%  \meta{relation} is \texttt{true}.
% \end{function}
% 
%\begin{function}{ \int_while_do:nn / (EXP) }
%  \begin{syntax}
%     \cs{int_while_do:nn} \
%     ~~\{ \meta{intexpr1} \meta{relation} \meta{intexpr2} \} \Arg{code}
%  \end{syntax}
%  Places the \meta{code} in the input stream for \TeX\ to process, and
%  then evaluates the relationship between the two 
%  \meta{integer expressions} as described for \cs{int_compare:nTF}.
%  If the test is \texttt{true} then the \meta{code} will be inserted
%  into the input stream again and a loop will occur until the 
%  \meta{relation} is \texttt{false}.
% \end{function}
%
%\subsection{Formatting integers}
%
% Integers can be placed into the output stream with formatting. These
% conversions apply to any integer expressions.
% 
%\begin{function}{ \int_to_arabic:n / (EXP) }
%  \begin{syntax}
%    \cs{int_to_arabic:n} \Arg{integer expression}
%  \end{syntax}
%  Places the value of the \meta{integer expression} in the input
%  stream as digits, with category code \( 12 \) (other).
%\end{function}
%
%\begin{function}{ 
%  \int_to_alph:n / (EXP) |
%  \int_to_Alph:n / (EXP) |
%}
%  \begin{syntax}
%    \cs{int_to_alph:n} \Arg{integer expression}
%  \end{syntax}
%  Evaluates the \meta{integer expression} and converts the result 
%  into a series of letters, which are then left in the input stream.
%  The conversion rule uses the \( 26 \) letters of the English 
%  alphabet, in order. Thus 
%  \begin{verbatim}
%    \int_to_alph:n { 1 }
%  \end{verbatim}
%  places "a" in the input stream, 
%  \begin{verbatim}
%    \int_to_alph:n { 26 }
%  \end{verbatim}
%  is represented as "z" and
%  \begin{verbatim}
%    \int_to_alph:n { 27 }
%  \end{verbatim}
%  is converted to `aa'. For conversions using other alphabets, use
%  \cs{int_convert_to_symbols:nnn} to define an alphabet-specific
%  function. The basic \cs{int_to_alph:n} and \cs{int_to_Alph:n} 
%  functions should not be modified.
%\end{function}
%
%\begin{function}{ \int_to_binary:n / (EXP) }
%  \begin{syntax}
%    \cs{int_to_binary:n} \Arg{integer expression}
%  \end{syntax}
%  Calculates the value of the \meta{integer expression} and places
%  the binary representation of the result in the input stream.
%\end{function}
%
%\begin{function}{ \int_to_hexadecimal:n / (EXP) }
%  \begin{syntax}
%    \cs{int_to_binary:n} \Arg{integer expression}
%  \end{syntax}
%  Calculates the value of the \meta{integer expression} and places
%  the hexadecimal (base~\( 16 \)) representation of the result in the
%  input stream. Upper case letters are used for digits beyond \( 9 \).
%\end{function}
%
%\begin{function}{ \int_to_octal:n / (EXP) }
%  \begin{syntax}
%    \cs{int_to_octal:n} \Arg{integer expression}
%  \end{syntax}
%  Calculates the value of the \meta{integer expression} and places
%  the octal (base~\( 8 \)) representation of the result in the input
%  stream.
%\end{function}
%
%\begin{function}{ 
%  \int_to_roman:n / (EXP) |
%  \int_to_Roman:n / (EXP) |
%}
%  \begin{syntax}
%    \cs{int_to_roman:n} \Arg{integer expression}
%  \end{syntax}
%  Places the value of the \meta{integer expression} in the input
%  stream as Roman numerals, either lower case (\cs{int_to_roman:n})
%  or upper case (\cs{int_to_Roman:n}). The numerals are letters 
%  with category code \( 11 \) (letter). 
%\end{function}
%
%\begin{function}{ \int_to_symbol:n / (EXP) }
%  \begin{syntax}
%    \cs{int_to_symbol:n} \Arg{integer expression}
%  \end{syntax}
%  Calculates the value of the \meta{integer expression} and places
%  the symbol representation of the result in the input stream. The
%  list of symbols used is equivalent to \LaTeXe's \cs{@fnsymbol}
%  set.
%\end{function}
%
%\subsection{Converting from other formats}
%
%\begin{function}{ \int_from_alph:n / (EXP) }
%  \begin{syntax}
%    \cs{int_from_alpa:n} \Arg{letters}
%  \end{syntax}
%  Converts the \meta{letters} into the integer (base~\( 10 \))
%  representation and leaves this in the input stream. The 
%  \meta{letters} are treated using the English alphabet only, with 
%  `a' equal to \( 1 \) through to `z' equal to \( 26 \). Either lower
%  or upper case letters may be used. This is the inverse function of
%  \cs{int_to_alph:n}.
%\end{function}
%
%\begin{function}{ \int_from_binary:n / (EXP) }
%  \begin{syntax}
%    \cs{int_from_binary:n} \Arg{binary number}
%  \end{syntax}
%  Converts the \meta{binary number} into the integer (base~\( 10 \))
%  representation and leaves this in the input stream.
%\end{function}
%
%\begin{function}{ \int_from_hexadecimal:n / (EXP) }
%  \begin{syntax}
%    \cs{int_from_binary:n} \Arg{hexadecimal number}
%  \end{syntax}
%  Converts the \meta{hexadecimal number} into the integer 
%  (base~\( 10 \)) representation and leaves this in the input stream.
%  Digits greater than \( 9 \) may be represented in the 
%  \meta{hexadecimal number} by upper or lower case letters.
%\end{function}
%
%\begin{function}{ \int_from_octal:n / (EXP) }
%  \begin{syntax}
%    \cs{int_from_octal:n} \Arg{octal number}
%  \end{syntax}
%  Converts the \meta{octal number} into the integer (base~\( 10 \))
%  representation and leaves this in the input stream.
%\end{function}
%
%\begin{function}{ \int_from_roman:n / (EXP) }
%  \begin{syntax}
%    \cs{int_from_roman:n} \Arg{roman numeral}
%  \end{syntax}
%  Converts the \meta{roman numeral} into the integer (base~\( 10 \))
%  representation and leaves this in the input stream. The 
%  \meta{roman numeral} may be in upper or lower case; if the numeral
%  is not valid then the resulting value will be \( -1 \).
%\end{function}
%
%\subsection{Low-level conversion functions}
%
% As well as the higher-level functions already documented, there
% are a series of lower-level functions which can be used to carry out
% generic conversions. These are used to create the higher-level
% versions documented above.
%
%\begin{function}{ \int_convert_from_base_ten:nn / (EXP) }
%  \begin{syntax}
%    \cs{int_convert_from_base_ten:nn} \Arg{integer expression}
%    ~~\Arg{base}
%  \end{syntax}
%  Calculates the value of the \meta{integer expression} and 
%  converts it into the appropriate representation in the \meta{base};
%  the later may be given as an integer expression. For bases greater
%  than \( 10 \) the higher `digits' are represented by the upper case 
%  letters from the English alphabet (with normal category codes). The
%  maximum \meta{base} value is \( 36 \).
%\end{function}
%
%\begin{function}{ \int_convert_to_base_ten:nn / (EXP) }
%  \begin{syntax}
%    \cs{int_convert_to_base_ten:nn} \Arg{number}
%    ~~\Arg{base}
%  \end{syntax}
%  Converts the \meta{number} in \meta{base} into the appropriate
%  value in base \( 10 \). The \meta{number} should consist of 
%  digits and letters (either lower or upper case), plus optionally
%  a leading sign. The maximum \meta{base} value is \( 36 \).
%\end{function}
% 
%\begin{function}{ \int_convert_to_symbols:nnn / (EXP) }
%  \begin{syntax}
%    \cs{int_convert_to_symbols:nnn} 
%    ~~\Arg{integer expression} \Arg{total symbols} 
%    ~~\meta{value to symbol mapping}
%  \end{syntax}
%  This is the low-level function for conversion of an 
%  \meta{integer expression} into a symbolic form (which will often
%  be letters). The \meta{total symbols} available should be given
%  as an integer expression. Values are actually converted to symbols
%  according to the \meta{value to symbol mapping}. This should be given
%  as \meta{total symbols} pairs of entries, a number and the
%  appropriate symbol. Thus the \cs{int_to_alph:n} function is defined
%  as
%  \begin{verbatim}
%    \cs_new:Npn \int_to_alph:n #1 { 
%      \int_convert_to_sybols:nnn {#1} { 26 }
%        {
%          {  1 } { a }
%          {  2 } { b }
%          {  3 } { c }
%          {  4 } { d }
%          {  5 } { e }
%          {  6 } { f }
%          {  7 } { g }
%          {  8 } { h }
%          {  9 } { i }
%          { 10 } { j }
%          { 11 } { k }
%          { 12 } { l }
%          { 13 } { m }
%          { 14 } { n }
%          { 15 } { o }
%          { 16 } { p }
%          { 17 } { q }
%          { 18 } { r }
%          { 19 } { s }
%          { 20 } { t }
%          { 21 } { u }
%          { 22 } { v }
%          { 23 } { w }
%          { 24 } { x }
%          { 25 } { y }
%          { 26 } { z }
%        }
%    }
%  \end{verbatim}
%\end{function}
%
%\section{Variables and constants}
%
% \begin{variable}{%
%                  \l_tmpa_int |
%                  \l_tmpb_int |
%                  \l_tmpc_int |
%                  \g_tmpa_int |
%                  \g_tmpb_int |
% }
% Scratch register for immediate use. They are not used by conditionals
% or predicate functions.
% \end{variable}
%
%\begin{function}{ 
%  \int_const:Nn |
%  \int_const:cn |
%}
%  \begin{syntax}
%    \cs{int_const:Nn} \meta{integer} \Arg{integer expression}
%  \end{syntax}
%  Creates a new constant \meta{integer} or raises an error if the name
%  is already taken. The value of the \meta{integer} will be set 
%  globally to the \meta{integer expression}.
%\end{function}
%
%\begin{variable}{ \c_max_int }
%  The maximum value that can be stored as an integer.
%\end{variable}
%
% \begin{variable}{ \c_minus_one | \c_zero | \c_one | \c_two | \c_three |
%                   \c_four | \c_five | \c_six | \c_seven | \c_eight | 
%                   \c_nine | \c_ten | \c_eleven | \c_twelve | \c_thirteen | 
%                   \c_fourteen | \c_fifteen | \c_sixteen | \c_thirty_two |
%                   \c_hundred_one |
%                   \c_twohundred_fifty_five | \c_twohundred_fifty_six |
%                   \c_thousand | 
%                   \c_ten_thousand | \c_ten_thousand_one |
%                   \c_ten_thousand_two | \c_ten_thousand_three | 
%                   \c_ten_thousand_four | \c_twenty_thousand }
% Set of constants denoting useful values.
% \begin{texnote}
% Some of these constants have been available under \LaTeX2 under names
% like \tn{m@ne}, \tn{z@}, \tn{@ne},\tn{tw@}, \tn{thr@@}, etc.
% \end{texnote}
% \end{variable}
%
% \begin{variable}{\c_max_register_int}
% Maximum number of registers.
% \end{variable}
%
%\subsection{Internal functions}
%
% \begin{function}{\int_to_roman:w / (EXP)}
% \begin{syntax}
%   "\int_to_roman:w" <integer> <space> \textit{or} <non-expandable token>
% \end{syntax}
% Converts <integer> to it lowercase roman representation. Note that
% it produces a string of letters with catcode 12.
% \begin{texnote}
%   This is the \TeX{} primitive \tn{romannumeral} renamed.
% \end{texnote}
% \end{function}
%
% \begin{function}{
%                   \int_roman_lcuc_mapping:Nnn |
%                   \int_to_roman_lcuc:NN |
% }
% \begin{syntax}
%   "\int_roman_lcuc_mapping:Nnn"  <roman_char> \Arg{licr} \Arg{LICR}
%   "\int_to_roman_lcuc:NN"  <roman_char> <char>
% \end{syntax}
% "\int_roman_lcuc_mapping:Nnn" specifies how the roman
% numeral <roman\_ char> (i, v, x, l, c, d, or m) should be
% interpreted when converting the number. <licr> is the lower case and
% <LICR> is the uppercase mapping. "\int_to_roman_lcuc:NN" is a
% recursive function converting the roman numerals.
% \end{function}
%
%
% \begin{function}{
%                   \int_convert_number_with_rule:nnN |
%                   \int_symbol_math_conversion_rule:n |
%                   \int_symbol_text_conversion_rule:n |
% }
% \begin{syntax}
%   "\int_convert_number_with_rule:nnN" \Arg{int1} \Arg{int2} <function>
% \end{syntax}
% "\int_convert_number_with_rule:nnN" converts <int1> into letters,
% symbols, whatever as defined by <function>. <int2> denotes the base
% number for the conversion.
% \end{function}
% 
%\begin{function}{
%  \if_num:w            / (EXP) |
%  \if_inexpr_compare:w / (EXP)
%}
%  \begin{syntax}
%    "\if_num:w" <number1> <rel> <number2> <true> "\else:" <false> "\fi:"
%  \end{syntax}
%  Compare two integers using <rel>, which must be one of
%  \texttt{=}, "<" or ">" with category code \(12\).
%  The \cs{else:} branch is optional. 
%  \begin{texnote}
%   These are both names for the \TeX\ primitive \cs{ifnum}.
%  \end{texnote}
%\end{function}
%
%\begin{function}{
%  \if_case:w         / (EXP) |
%  \or:               / (EXP)
%}
%  \begin{syntax}
%    "\if_case:w" <number> <case0> "\or:" <case1> "\or:" "..." "\else:"
%    <default> "\fi:"
%  \end{syntax}
%  Selects a case to execute based on the value of <number>. The first
%  case (<case0>) is executed if <number> is \(0\), the second
%  (<case1>) if the <number> is \(1\), \emph{etc}. The
%  <number> may be a literal, a constant or an integer
%  expression (\emph{e.g}.~using \cs{int_eval:n}).
%  \begin{texnote}
%    These are the \TeX\ primitives \cs{ifcase}  and \cs{or}.
%  \end{texnote}
%\end{function}
%
%\begin{function}{\int_value:w / (EXP)}
%  \begin{syntax}
%    "\int_value:w" <integer> 
%    "\int_value:w" <tokens>  <optional space>
%  \end{syntax}
%  Expands <tokens> until an <integer> is formed. One space may be
%  gobbled in the process. 
%  \begin{texnote}
%    This is the \TeX\ primitive \tn{number}.
%  \end{texnote}
%\end{function}
%
%\begin{function}{
%  \int_eval:w   / (EXP) |
%  \int_eval_end:
%}
%  \begin{syntax}
%    "\int_eval:w" <int expr> "\int_eval_end:"
%  \end{syntax}
%  Evaluates <integer expression> as described for \cs{int_eval:n}.
%  The evalution stops when an unexpandable token with category code
%  other than \(12\) is read or when \cs{int_eval_end:} is
%  reached. The latter is gobbled by the scanner mechanism:
%  \cs{int_eval_end:} itself is unexpandable but used correctly
%  the entire construct is expandable.
%  \begin{texnote}
%   This is the \eTeX\ primitive \cs{numexpr}.
%  \end{texnote}
%\end{function}
%
%\begin{function}{\if_int_odd:w / (EXP)}
%  \begin{syntax}
%    "\if_int_odd:w" <tokens>  <true> "\else:" <false> "\fi:"
%    "\if_int_odd:w" <number>  <true> "\else:" <false> "\fi:"
%  \end{syntax}
%  Expands <tokens> until a non-numeric tokens is found, and 
%  tests whether the resulting <number> is odd. If so, <true code>
%  is executed. The \cs{else:} branch is optional.
%  \begin{texnote}
%   This is the \TeX\ primitive \cs{ifodd}.
%  \end{texnote}
%\end{function}
%
% \end{documentation}
%
% \begin{implementation}
%
% \section{\pkg{l3int} implementation}
%
% \subsection{Internal functions and variables}
%
% \begin{function}{\int_advance:w}
% \begin{syntax}
% "\int_advance:w" <int register> <optional `\texttt{by}'> <number> <space>
% \end{syntax}
% Increments the count register by the specified amount.
% \begin{texnote}
% This is \TeX's \tn{advance}.
% \end{texnote}
% \end{function}
%
%
% \begin{function}{\int_convert_number_to_letter:n / (EXP)}
% \begin{syntax}
% "\int_convert_number_to_letter:n" \Arg{integer expression}
% \end{syntax}
% Internal function for turning a number for a different base into a letter or digit.
% \end{function}
%
% \begin{function}{\int_pre_eval_one_arg:Nn | \int_pre_eval_two_args:Nnn}
% \begin{syntax}
% "\int_pre_eval_one_arg:Nn" <function> \Arg{integer expression}
% "\int_pre_eval_one_arg:Nnn" <function> \Arg{int~expr~1} \Arg{int~expr~2}
% \end{syntax}
% These are expansion helpers; they evaluate their integer expressions
% before handing them off to the specified <function>.
% \end{function}
%
% \begin{function}{ \int_get_sign_and_digits:n / (EXP) | 
%                   \int_get_sign:n / (EXP )           | 
%                   \int_get_digits:n / (EXP)          }
% \begin{syntax}
% "\int_get_sign_and_digits:n" \Arg{number}
% \end{syntax}
% From an argument that may or may not include a "+" or "-" sign, these
% functions expand to the respective components of the number.
% \end{function}
%
% \subsection{Module loading and primitives definitions}
%
%    We start by ensuring that the required packages are loaded.
%    \begin{macrocode}
%<*package>
\ProvidesExplPackage
  {\filename}{\filedate}{\fileversion}{\filedescription}
\package_check_loaded_expl:
%</package>
%<*initex|package>
%    \end{macrocode}
%
% \begin{macro}{\int_value:w}
% \begin{macro}{\int_eval:n,\int_eval:w,\int_eval_end:}
% \begin{macro}{\if_int_compare:w}
% \begin{macro}{\if_int_odd:w}
% \begin{macro}{\if_num:w}
% \begin{macro}{\if_case:w}
% \begin{macro}{\int_to_roman:w}
% \begin{macro}{\int_advance:w}
%   Here are the remaining primitives for number comparisons and
%   expressions.
%    \begin{macrocode}
\cs_set_eq:NN \int_value:w \tex_number:D
\cs_set_eq:NN \int_eval:w \etex_numexpr:D
\cs_set_protected:Npn \int_eval_end: {\tex_relax:D}
\cs_set_eq:NN \if_int_compare:w \tex_ifnum:D
\cs_new_eq:NN \if_num:w             \tex_ifnum:D
\cs_set_eq:NN \if_int_odd:w \tex_ifodd:D
\cs_new_eq:NN \if_case:w          \tex_ifcase:D
\cs_new_eq:NN \int_to_roman:w \tex_romannumeral:D
\cs_new_eq:NN \int_advance:w \tex_advance:D
%    \end{macrocode}
% \end{macro}
% \end{macro}
% \end{macro}
% \end{macro}
% \end{macro}
% \end{macro}
% \end{macro}
% \end{macro}
%
%
% \begin{macro}{\int_eval:n}
% Wrapper for \cs{int_eval:w}. Can be used in an integer expression
% or directly in the input stream.
%    \begin{macrocode}
\cs_set:Npn \int_eval:n #1{
  \int_value:w \int_eval:w #1\int_eval_end:
}
%    \end{macrocode}
% \end{macro}
%
%
%
% \subsection{Allocation and setting}
%
% \begin{macro}{\int_new:N,\int_new:c}
%  For the \LaTeX3 format:
%    \begin{macrocode}
%<*initex>
\alloc_new:nnnN {int} {11} {\c_max_register_int} \tex_countdef:D
%</initex>
%    \end{macrocode}
%  For `l3in2e':
%    \begin{macrocode}
%<*package>
\cs_new_protected_nopar:Npn \int_new:N #1 {
  \chk_if_free_cs:N #1
  \newcount #1
}
%</package>
%    \end{macrocode}
%    \begin{macrocode}
\cs_generate_variant:Nn \int_new:N {c}
%    \end{macrocode}
% \end{macro}
%
%
% \begin{macro}{\int_set:Nn}
% \begin{macro}{\int_set:cn}
% \begin{macro}{\int_gset:Nn}
% \begin{macro}{\int_gset:cn}
%    Setting counters is again something that I would like to make
%    uniform at the moment to get a better overview.
%    \begin{macrocode}
\cs_new_protected_nopar:Npn \int_set:Nn #1#2{#1 \int_eval:w #2\int_eval_end:
%<*check>
\chk_local_or_pref_global:N #1
%</check>
}
\cs_new_protected_nopar:Npn \int_gset:Nn {
%<*check>
   \pref_global_chk:
%</check>
%<-check> \pref_global:D
   \int_set:Nn }
\cs_generate_variant:Nn\int_set:Nn  {cn}
\cs_generate_variant:Nn\int_gset:Nn {cn}
%    \end{macrocode}
% \end{macro}
% \end{macro}
% \end{macro}
% \end{macro}
% 
%\begin{macro}{\int_set_eq:NN}
%\begin{macro}{\int_set_eq:cN}
%\begin{macro}{\int_set_eq:Nc}
%\begin{macro}{\int_set_eq:cc}
%\begin{macro}{\int_gset_eq:NN}
%\begin{macro}{\int_gset_eq:cN}
%\begin{macro}{\int_gset_eq:Nc}
%\begin{macro}{\int_gset_eq:cc}
% Setting equal means using one integer inside the set function of
% another.
%    \begin{macrocode}
\cs_new_protected_nopar:Npn \int_set_eq:NN #1#2 {
  \int_set:Nn #1 {#2}
}
\cs_generate_variant:Nn \int_set_eq:NN { c }
\cs_generate_variant:Nn \int_set_eq:NN { Nc }
\cs_generate_variant:Nn \int_set_eq:NN { cc }
\cs_new_protected_nopar:Npn \int_gset_eq:NN #1#2 {
  \int_gset:Nn #1  {#2}
}
\cs_generate_variant:Nn \int_gset_eq:NN { c }
\cs_generate_variant:Nn \int_gset_eq:NN { Nc }
\cs_generate_variant:Nn \int_gset_eq:NN { cc }
%    \end{macrocode}
%\end{macro}
%\end{macro}
%\end{macro}
%\end{macro}
%\end{macro}
%\end{macro}
%\end{macro}
%\end{macro}
%
% \begin{macro}{\int_incr:N}
% \begin{macro}{\int_decr:N}
% \begin{macro}{\int_gincr:N}
% \begin{macro}{\int_gdecr:N}
% \begin{macro}{\int_incr:c}
% \begin{macro}{\int_decr:c}
% \begin{macro}{\int_gincr:c}
% \begin{macro}{\int_gdecr:c}
%    Incrementing and decrementing of integer registers is done with
%    the following functions.
%    \begin{macrocode}
\cs_new_protected_nopar:Npn \int_incr:N #1{\int_advance:w#1\c_one
%<*check>
        \chk_local_or_pref_global:N #1
%</check>
}
\cs_new_protected_nopar:Npn \int_decr:N #1{\int_advance:w#1\c_minus_one
%<*check>
        \chk_local_or_pref_global:N #1
%</check>
}
\cs_new_protected_nopar:Npn \int_gincr:N {
%    \end{macrocode}
%    We make sure that a local variable is not updated globally by
%    changing the internal test (i.e.\ |\chk_local_or_pref_global:N|) before
%    making the assignment. This is done by |\pref_global_chk:| which also
%    issues the necessary |\pref_global:D|. This is not very efficient, but
%    this code will be only included for debugging purposes. Using
%    |\pref_global:D| in front of the local function is better in the
%    production versions.
%    \begin{macrocode}
%<*check>
   \pref_global_chk:
%</check>
%<-check> \pref_global:D
   \int_incr:N}
\cs_new_protected_nopar:Npn \int_gdecr:N {
%<*check>
   \pref_global_chk:
%</check>
%<-check> \pref_global:D
   \int_decr:N}
%    \end{macrocode}
%    With the |\int_add:Nn| functions we can shorten the above code.
%    If this makes it too slow \ldots
%    \begin{macrocode}
\cs_set_protected_nopar:Npn \int_incr:N #1{\int_add:Nn#1\c_one}
\cs_set_protected_nopar:Npn \int_decr:N #1{\int_add:Nn#1\c_minus_one}
\cs_set_protected_nopar:Npn \int_gincr:N #1{\int_gadd:Nn#1\c_one}
\cs_set_protected_nopar:Npn \int_gdecr:N #1{\int_gadd:Nn#1\c_minus_one}
%    \end{macrocode}
%    
%    \begin{macrocode}
\cs_generate_variant:Nn \int_incr:N {c}
\cs_generate_variant:Nn \int_decr:N {c}
\cs_generate_variant:Nn \int_gincr:N {c}
\cs_generate_variant:Nn \int_gdecr:N {c}
%    \end{macrocode}
% \end{macro}
% \end{macro}
% \end{macro}
% \end{macro}
% \end{macro}
% \end{macro}
% \end{macro}
% \end{macro}
%
%
%  \begin{macro}{\int_zero:N}
%  \begin{macro}{\int_zero:c}
%  \begin{macro}{\int_gzero:N}
%  \begin{macro}{\int_gzero:c}
%  Functions that reset an \m{int} register to zero.
%    \begin{macrocode}
\cs_new_protected_nopar:Npn \int_zero:N  #1 {#1=\c_zero}
\cs_generate_variant:Nn \int_zero:N {c}
%    \end{macrocode}
%    
%    \begin{macrocode}
\cs_new_protected_nopar:Npn \int_gzero:N #1 {\pref_global:D #1=\c_zero}
\cs_generate_variant:Nn \int_gzero:N {c}
%    \end{macrocode}
%  \end{macro}
%  \end{macro}
%  \end{macro}
%  \end{macro}
%
% \begin{macro}{\int_add:Nn}
% \begin{macro}{\int_add:cn}
% \begin{macro}{\int_gadd:Nn}
% \begin{macro}{\int_gadd:cn}
% \begin{macro}{\int_sub:Nn}
% \begin{macro}{\int_sub:cn}
% \begin{macro}{\int_gsub:Nn}
% \begin{macro}{\int_gsub:cn}
%    Adding and substracting to and from a counter \ldots
%    We should think of using these functions
%    \begin{macrocode}
\cs_new_protected_nopar:Npn \int_add:Nn #1#2{
%    \end{macrocode}
%    We need to say |by| in case the first argument is a register
%    accessed by its number, e.g., |\count23|. Not that it should
%    ever happen but\dots
%    \begin{macrocode}
    \int_advance:w #1 by \int_eval:w #2\int_eval_end:
%<*check>
    \chk_local_or_pref_global:N #1
%</check>
}
\cs_new_nopar:Npn \int_sub:Nn #1#2{
    \int_advance:w #1-\int_eval:w #2\int_eval_end:
%<*check>
\chk_local_or_pref_global:N #1
%</check>
}
\cs_new_protected_nopar:Npn \int_gadd:Nn {
%<*check>
   \pref_global_chk:
%</check>
%<-check> \pref_global:D
   \int_add:Nn }
\cs_new_protected_nopar:Npn \int_gsub:Nn {
%<*check>
   \pref_global_chk:
%</check>
%<-check> \pref_global:D
   \int_sub:Nn }
\cs_generate_variant:Nn \int_add:Nn  {cn}
\cs_generate_variant:Nn \int_gadd:Nn {cn}
\cs_generate_variant:Nn \int_sub:Nn  {cn}
\cs_generate_variant:Nn \int_gsub:Nn {cn}
%    \end{macrocode}
% \end{macro}
% \end{macro}
% \end{macro}
% \end{macro}
% \end{macro}
% \end{macro}
% \end{macro}
% \end{macro}
%
%
%  \begin{macro}{\int_use:N}
%  \begin{macro}{\int_use:c}
%    Here is how counters are accessed:
%    \begin{macrocode}
\cs_new_eq:NN \int_use:N \tex_the:D
\cs_new_nopar:Npn \int_use:c #1{\int_use:N \cs:w#1\cs_end:}
%    \end{macrocode}
%  \end{macro}
%  \end{macro}
%
%  \begin{macro}{\int_show:N}
%  \begin{macro}{\int_show:c}
%    \begin{macrocode}
\cs_new_eq:NN  \int_show:N  \kernel_register_show:N
\cs_new_eq:NN  \int_show:c  \kernel_register_show:c
%    \end{macrocode}
%  \end{macro}
%  \end{macro}
%
%
%
%
%  \begin{macro}{\int_to_arabic:n}
%  Nothing exciting here.
%    \begin{macrocode}
\cs_new_nopar:Npn \int_to_arabic:n #1{ \int_eval:n{#1}}
%    \end{macrocode}
%  \end{macro}
%
%
%
%  \begin{macro}{\int_roman_lcuc_mapping:Nnn}
%  Using \TeX's built-in feature for producing roman numerals has some
%  surprising features. One is the the characters resulting from
%  |\int_to_roman:w| have category code~12 so they may fail in
%  certain comparison tests. Therefore we use a mapping from the
%  character \TeX{} produces to the character we actually want which
%  will give us letters with category code~11.%
%    \begin{macrocode}
\cs_new_protected_nopar:Npn \int_roman_lcuc_mapping:Nnn #1#2#3{
  \cs_set_nopar:cpn {int_to_lc_roman_#1:}{#2}
  \cs_set_nopar:cpn {int_to_uc_roman_#1:}{#3}
}
%    \end{macrocode}
%  \end{macro}
%  Here are the default mappings. I haven't found any examples of say
%  Turkish doing the mapping |i \i I| but at least there is a
%  possibility for it if needed. Note: I have now asked a Turkish
%  person and he tells me they do the |i I| mapping.
%    \begin{macrocode}
\int_roman_lcuc_mapping:Nnn i i I
\int_roman_lcuc_mapping:Nnn v v V
\int_roman_lcuc_mapping:Nnn x x X
\int_roman_lcuc_mapping:Nnn l l L
\int_roman_lcuc_mapping:Nnn c c C
\int_roman_lcuc_mapping:Nnn d d D
\int_roman_lcuc_mapping:Nnn m m M
%    \end{macrocode}
%  For the delimiter we cheat and let it gobble its arguments instead.
%    \begin{macrocode}
\int_roman_lcuc_mapping:Nnn Q \use_none:nn \use_none:nn
%    \end{macrocode}
%
%  \begin{macro}{\int_to_roman:n}
%  \begin{macro}{\int_to_Roman:n}
%  \begin{macro}{\int_to_roman_lcuc:NN}
%  The commands for producing the lower and upper case roman numerals
%  run a loop on one character at a time and also carries some
%  information for upper or lower case with it. We put it through
%  |\int_eval:n| first which is safer and more flexible.
%    \begin{macrocode}
\cs_new_nopar:Npn \int_to_roman:n #1 {
  \exp_after:wN \int_to_roman_lcuc:NN \exp_after:wN l
    \int_to_roman:w \int_eval:n {#1} Q
}
\cs_new_nopar:Npn \int_to_Roman:n #1 {
  \exp_after:wN \int_to_roman_lcuc:NN \exp_after:wN u
    \int_to_roman:w \int_eval:n {#1} Q
}
\cs_new_nopar:Npn \int_to_roman_lcuc:NN #1#2{
  \use:c {int_to_#1c_roman_#2:}
  \int_to_roman_lcuc:NN #1
}
%    \end{macrocode}
%  \end{macro}
%  \end{macro}
%  \end{macro}
%  
%\begin{macro}{\int_convert_to_symbols:nnn}
% For conversion of integers to arbitrary symbols the method is in
% general as follows. The input number ("#1") is compared to the total
% number of symbols available at each place ("#2"). If the input is
% larger than the total number of symbols available then the modulus
% is needed, with one added so that the positions don't have to number
% from zero. Using an \texttt{f}-type expansion, this is done so that
% the system is recursive. The actual conversion function therefore
% gets a `nice' number at each stage. Of course, if the initial input 
% was small enough then there is no problem and everything is easy. This
% is more or less the same as \cs{int_convert_number_with_rule:nnN} but
% `pre-packaged'.
%    \begin{macrocode}
\cs_new_nopar:Npn \int_convert_to_symbols:nnn #1#2#3 {
  \int_compare:nNnTF {#1} > {#2}
    {
      \exp_args:Nf \int_convert_to_symbols:nnn
        { \int_div_truncate:nn { #1 - 1 } {#2} } {#2} {#3}
      \exp_args:Nf \prg_case_int:nnn
        { \int_eval:n { 1 + \int_mod:nn { #1 - 1 } {#2} } }
        {#3} { }
    }
    { \exp_args:Nf \prg_case_int:nnn { \int_eval:n {#1} } {#3} { } }
}
%    \end{macrocode}
%\end{macro}
%
%
%
%  \begin{macro}{\int_convert_number_with_rule:nnN}
%  This is our major workhorse for conversions. |#1| is the number we
%  want converted, |#2| is the base number, and |#3| is the function
%  converting the number. This function expects to receive a
%  non-negative integer and as such is ideal for something using
%  |\if_case:w| internally.
%
%  The basic example is this: We want to convert the number 50 (|#1|)
%  into an alphabetic equivalent |ax|. For the English language our
%  list contains 26 elements so this is our argument |#2| while the
%  function |#3| just turns |1| into |a|, |2| into |b|, etc. Hence our
%  goal is to turn 50 into the sequence |#3{1}#1{24}| so what we do is
%  to first divide 50 by 26 and truncating the result returning 1.
%  Then before we execute this we call the function again but this time
%  on the result of the remainder of the division. This goes on until
%  the remainder is less than or equal to the base number where we just
%  call the function |#3| directly on the number.
%
%  We do a little pre-expansion of the arguments below as they
%  otherwise have a tendency to grow quite large.
%    \begin{macrocode}
\cs_set_nopar:Npn \int_convert_number_with_rule:nnN #1#2#3{
  \int_compare:nNnTF {#1}>{#2}
  {
    \exp_args:Nf \int_convert_number_with_rule:nnN
      { \int_div_truncate:nn {#1-1}{#2} }{#2}
      #3
%    \end{macrocode}
%  Note that we have to nudge our modulus function so it won't
%  return~$0$ as that wouldn't work with |\if_case:w| when that
%  expects a positive number to produce a letter.
%    \begin{macrocode}
    \exp_args:Nf #3 { \int_eval:n{1+\int_mod:nn {#1-1}{#2}} }
  }
  { \exp_args:Nf #3{ \int_eval:n{#1} } }
}
%    \end{macrocode}
%    As can be seen it is even simpler to convert to number systems
%    that contain 0, since then we don't have to add or subtract 1
%    here and there.
%  \end{macro}
%
%
%\begin{macro}{\int_to_alph:n}
%\begin{macro}{\int_to_Alpha:n}
% These both use the above function with input functions that make sense
% for the alphabet in English.
%    \begin{macrocode}
\cs_new_nopar:Npn \int_to_alph:n #1 {
  \int_convert_to_symbols:nnn {#1} { 26 }
    {
      {  1 } { a }
      {  2 } { b }
      {  3 } { c }
      {  4 } { d }
      {  5 } { e }
      {  6 } { f }
      {  7 } { g }
      {  8 } { h }
      {  9 } { i }
      { 10 } { j }
      { 11 } { k }
      { 12 } { l }
      { 13 } { m }
      { 14 } { n }
      { 15 } { o }
      { 16 } { p }
      { 17 } { q }
      { 18 } { r }
      { 19 } { s }
      { 20 } { t }
      { 21 } { u }
      { 22 } { v }
      { 23 } { w }
      { 24 } { x }
      { 25 } { y }
      { 26 } { z }
    }
}
\cs_new_nopar:Npn \int_to_Alph:n #1 {
  \int_convert_to_symbols:nnn {#1} { 26 } 
    {
      {  1 } { A }
      {  2 } { B }
      {  3 } { C }
      {  4 } { D }
      {  5 } { E }
      {  6 } { F }
      {  7 } { G }
      {  8 } { H }
      {  9 } { I }
      { 10 } { J }
      { 11 } { K }
      { 12 } { L }
      { 13 } { M }
      { 14 } { N }
      { 15 } { O }
      { 16 } { P }
      { 17 } { Q }
      { 18 } { R }
      { 19 } { S }
      { 20 } { T }
      { 21 } { U }
      { 22 } { V }
      { 23 } { W }
      { 24 } { X }
      { 25 } { Y }
      { 26 } { Z }
    }
}
%    \end{macrocode}
%\end{macro}
%\end{macro}
%
%  \begin{macro}{\int_to_symbol:n}
%  Turning a number into a symbol is also easy enough.
%    \begin{macrocode}
\cs_new_nopar:Npn \int_to_symbol:n #1{
  \mode_if_math:TF
  {
    \int_convert_number_with_rule:nnN {#1}{9}
      \int_symbol_math_conversion_rule:n
  }
  {
    \int_convert_number_with_rule:nnN {#1}{9}
      \int_symbol_text_conversion_rule:n
  }
}
%    \end{macrocode}
%  \end{macro}
%  \begin{macro}{\int_symbol_math_conversion_rule:n}
%  \begin{macro}{\int_symbol_text_conversion_rule:n}
%  Nothing spectacular here.
%    \begin{macrocode}
\cs_new_nopar:Npn \int_symbol_math_conversion_rule:n #1 {
  \if_case:w #1
    \or: *
    \or: \dagger
    \or: \ddagger
    \or: \mathsection
    \or: \mathparagraph
    \or: \|
    \or: **
    \or: \dagger\dagger
    \or: \ddagger\ddagger
  \fi:
}
\cs_new_nopar:Npn \int_symbol_text_conversion_rule:n #1 {
  \if_case:w #1
    \or: \textasteriskcentered
    \or: \textdagger
    \or: \textdaggerdbl
    \or: \textsection
    \or: \textparagraph
    \or: \textbardbl
    \or: \textasteriskcentered\textasteriskcentered
    \or: \textdagger\textdagger
    \or: \textdaggerdbl\textdaggerdbl
  \fi:
}
%    \end{macrocode}
%  \end{macro}
%  \end{macro}
%
% \begin{macro}{\l_tmpa_int}
% \begin{macro}{\l_tmpb_int}
% \begin{macro}{\l_tmpc_int}
% \begin{macro}{\g_tmpa_int}
% \begin{macro}{\g_tmpb_int}
%    We provide four local and two global scratch counters, maybe we
%    need more or less.
%    \begin{macrocode}
\int_new:N \l_tmpa_int
\int_new:N \l_tmpb_int
\int_new:N \l_tmpc_int
\int_new:N \g_tmpa_int
\int_new:N \g_tmpb_int
%    \end{macrocode}
% \end{macro}
% \end{macro}
% \end{macro}
% \end{macro}
% \end{macro}
%
%
%
%
% \begin{macro}{\int_pre_eval_one_arg:Nn}
% \begin{macro}{\int_pre_eval_two_args:Nnn}
%   These are handy when handing down values to other
%   functions. All they do is evaluate the number in advance.
%    \begin{macrocode}
\cs_set_nopar:Npn \int_pre_eval_one_arg:Nn #1#2{
  \exp_args:Nf#1{\int_eval:n{#2}}}
\cs_set_nopar:Npn \int_pre_eval_two_args:Nnn #1#2#3{
  \exp_args:Nff#1{\int_eval:n{#2}}{\int_eval:n{#3}}
}
%    \end{macrocode}
% \end{macro}
% \end{macro}
%
%
%
%
%
%
% \subsection{Scanning and conversion}
%
%
%\begin{macro}{\int_from_roman:n}
%\begin{macro}[aux]{\int_from_roman_aux:NN}
%\begin{macro}[aux]{\int_from_roman_end:w}
%\begin{macro}[aux]{\int_from_roman_clean_up:w}
% The method here is to iterate through the input, finding the
% appropriate value for each letter and building up a sum. This is
% then evaluated by \TeX.
%    \begin{macrocode}
\cs_new_nopar:Npn \int_from_roman:n #1 {
  \tl_if_blank:nF {#1}
    {
      \tex_expandafter:D \int_from_roman_end:w 
        \tex_number:D \etex_numexpr:D 
          \int_from_roman_aux:NN #1 Q \q_stop
    }
}
\cs_new_nopar:Npn \int_from_roman_aux:NN #1#2 {
  \str_if_eq:nnTF {#1} { Q }
    {#1#2}
    {
      \str_if_eq:nnTF {#2} { Q }
        {
          \cs_if_exist:cF { c_int_from_roman_ #1 _int }
            { \int_from_roman_clean_up:w }
          +
          \use:c { c_int_from_roman_ #1 _int }
          #2
        }
        {
          \cs_if_exist:cF { c_int_from_roman_ #1 _int }
            { \int_from_roman_clean_up:w }
          \cs_if_exist:cF { c_int_from_roman_ #2 _int }
            { \int_from_roman_clean_up:w }
          \int_compare:nNnTF 
            { \use:c { c_int_from_roman_ #1 _int } }
            < 
            { \use:c { c_int_from_roman_ #2 _int  } }
            {
              + \use:c { c_int_from_roman_ #2 _int }
              - \use:c { c_int_from_roman_ #1 _int }
              \int_from_roman_aux:NN
            }
            {
              + \use:c { c_int_from_roman_ #1 _int }
              \int_from_roman_aux:NN #2
            }
        }
    }
}
\cs_new_nopar:Npn \int_from_roman_end:w #1 Q #2 \q_stop {
  \tl_if_empty:nTF {#2} {#1} {#2}
}
\cs_new_nopar:Npn \int_from_roman_clean_up:w #1 Q { + 0 Q -1 }
%    \end{macrocode}
%\end{macro}
%\end{macro}
%\end{macro}
%\end{macro}
%
%\begin{macro}{\int_convert_from_base_ten:nn}
%\begin{macro}[aux]{\int_convert_from_base_ten_aux:nnn}
%\begin{macro}[aux]{\int_convert_number_to_letter:n}
% Converting from base ten ("#1") to a second base ("#2") starts with
% a simple sign check. As the input is base \( 10 \) \TeX\ can then
% do the actual work with the sign itself.
%    \begin{macrocode}
\cs_new:Npn \int_convert_from_base_ten:nn #1#2 {
  \int_compare:nNnTF {#1} < { 0 }
    {
      - 
      \exp_args:Nnf \int_convert_from_base_ten_aux:nnn
        { } { \int_eval:n { 0 - ( #1 ) } } {#2}
    }
    {
      \exp_args:Nnf \int_convert_from_base_ten_aux:nnn
        { } { \int_eval:n {#1} } {#2}
    }
}
%    \end{macrocode}
% Here, the idea is to provide a recursive system to deal with the
% input. The output is build up as argument "#1", which is why it 
% starts off empty in the above. At each pass, the value in "#2" is
% checked to see if it is less than the new base ("#3"). If it is
% the it is converted directly and the rest of the output is added in.
% On the other hand, if the value to convert is greater than or equal
% to the new base then the modulus and remainder values are found. The
% modulus is converted to a symbol and the remainder is carried forward
% to the next round.S
%    \begin{macrocode}
\cs_new:Npn \int_convert_from_base_ten_aux:nnn #1#2#3 {
  \int_compare:nNnTF {#2} < {#3}
    {
      \int_convert_number_to_letter:n {#2}
      #1
    }
    {
      \exp_args:Nff \int_convert_from_base_ten_aux:nnn
        {
          \int_convert_number_to_letter:n
            { \int_mod:nn {#2} {#3} }
          #1   
        }
        { \int_div_truncate:nn {#2} {#3} }
        {#3}
    }
}
%    \end{macrocode}
% Convert to a letter only if necessary, otherwise simply return the
% value unchanged.
%    \begin{macrocode}
\cs_new:Npn \int_convert_number_to_letter:n #1 {
  \prg_case_int:nnn { #1 - 9 }
    {
      {  1 } { A }
      {  2 } { B }
      {  3 } { C }
      {  4 } { D }
      {  5 } { E }
      {  6 } { F }
      {  7 } { G }
      {  8 } { H }
      {  9 } { I }
      { 10 } { J }
      { 11 } { K }
      { 12 } { L }
      { 13 } { M }
      { 14 } { N }
      { 15 } { O }
      { 16 } { P }
      { 17 } { Q }
      { 18 } { R }
      { 19 } { S }
      { 20 } { T }
      { 21 } { U }
      { 22 } { V }
      { 23 } { W }
      { 24 } { X }
      { 25 } { Y }
      { 26 } { Z }
    }
    {#1}
}
%    \end{macrocode}
%\end{macro}
%\end{macro}
%\end{macro}
%
%\begin{macro}{\int_convert_to_base_ten:nn}
%\begin{macro}[aux]{\int_convert_to_base_ten_aux:nn}
%\begin{macro}[aux]{\int_convert_to_base_ten_aux:nnN}
%\begin{macro}[aux]{\int_convert_to_base_ten_aux:N}
%\begin{macro}[aux]{\int_get_sign_and_digits:n}
%\begin{macro}[aux]{\int_get_sign:n}
%\begin{macro}[aux]{\int_get_digits:n}
%\begin{macro}[aux]{\int_get_sign_and_digits_aux:nNNN}
%\begin{macro}[aux]{\int_get_sign_and_digits_aux:oNNN}
% Conversion to base ten means stripping off the sign then iterating
% through the input one token at a time. The total number is then added
% up as the code loops.
%    \begin{macrocode}
\cs_new:Npn \int_convert_to_base_ten:nn #1#2 {
  \int_eval:n
    {
      \int_get_sign:n {#1}
      \exp_args:Nf \int_convert_to_base_ten_aux:nn
        { \int_get_digits:n {#1} } {#2}
    }
}
\cs_new:Npn \int_convert_to_base_ten_aux:nn #1#2 {
  \int_convert_to_base_ten_aux:nnN { 0 } { #2 } #1 \q_nil
}
\cs_new:Npn \int_convert_to_base_ten_aux:nnN #1#2#3 {
  \quark_if_nil:NTF #3
    {#1}
    {
      \exp_args:Nf \int_convert_to_base_ten_aux:nnN
        { \int_eval:n { #1 * #2 + \int_convert_to_base_ten_aux:N #3 } }
        {#2}
    } 
}
%    \end{macrocode}
% The conversion here will take lower or upper case letters and turn
% them into the appropriate number, hence the two-part nature of the 
% function.
%    \begin{macrocode}
\cs_new:Npn \int_convert_to_base_ten_aux:N #1 {
  \int_compare:nNnTF { `#1 } < { 58 }
    {#1}
    {
      \int_eval:n 
        { `#1 - \int_compare:nNnTF { `#1 } < { 91 } { 55 } { 87 } }
    }
}
%    \end{macrocode}
% Finding a number and its sign requires dealing with an arbitrary 
% list of "+" and "-" symbols. This is done by working through token
% by token until there is something else at the start of the input.
% The sign of the input is tracked by the first Boolean used by the
% auxiliary function. 
%    \begin{macrocode}
\cs_new:Npn \int_get_sign_and_digits:n #1 {
  \int_get_sign_and_digits_aux:nNNN {#1}
    \c_true_bool \c_true_bool \c_true_bool
}
\cs_new:Npn \int_get_sign:n #1 {
  \int_get_sign_and_digits_aux:nNNN {#1}
    \c_true_bool \c_true_bool \c_false_bool
}
\cs_new:Npn \int_get_digits:n #1 {
  \int_get_sign_and_digits_aux:nNNN {#1}
    \c_true_bool \c_false_bool \c_true_bool
}
%    \end{macrocode}
% The auxiliary loops through, finding sign tokens and removing them.
% The sign itself is carried through as a flag.
%    \begin{macrocode}
\cs_new:Npn \int_get_sign_and_digits_aux:nNNN #1#2#3#4 {
  \tl_if_head_eq_charcode:fNTF {#1} -
    {
      \bool_if:NTF #2
        { 
          \int_get_sign_and_digits_aux:oNNN 
            { \use_none:n #1 } \c_false_bool #3#4 
        }    
        { 
          \int_get_sign_and_digits_aux:oNNN 
            { \use_none:n #1 } \c_true_bool #3#4 
        }    
    }
    {
      \tl_if_head_eq_charcode:fNTF {#1} +
        { \int_get_sign_and_digits_aux:oNNN { \use_none:n #1 } #2#3#4 }
        {
          \bool_if:NT #3 { \bool_if:NF #2 - }
          \bool_if:NT #4 {#1}
        }
    }
}
\cs_generate_variant:Nn \int_get_sign_and_digits_aux:nNNN { o }
%    \end{macrocode}
%\end{macro}
%\end{macro}
%\end{macro}
%\end{macro}
%\end{macro}
%\end{macro}
%\end{macro}
%\end{macro}
%\end{macro}
%
%\begin{macro}{\int_from_binary:n}
%\begin{macro}{\int_from_hexadecimal:n}
%\begin{macro}{\int_from_octal:n}
%\begin{macro}{\int_to_binary:n}
%\begin{macro}{\int_to_hexadecimal:n}
%\begin{macro}{\int_to_octal:n}
% Wrappers around the generic function.
%    \begin{macrocode}
\cs_new:Npn \int_from_binary:n #1 { 
  \int_convert_to_base_ten:nn {#1} { 2 }
}
\cs_new:Npn \int_from_hexadecimal:n #1 { 
  \int_convert_to_base_ten:nn {#1} { 16 }
}
\cs_new:Npn \int_from_octal:n #1 { 
  \int_convert_to_base_ten:nn {#1} { 8 }
}
\cs_new:Npn \int_to_binary:n #1 { 
  \int_convert_from_base_ten:nn {#1} { 2 }
}
\cs_new:Npn \int_to_hexadecimal:n #1 { 
  \int_convert_from_base_ten:nn {#1} { 16 }
}
\cs_new:Npn \int_to_octal:n #1 { 
  \int_convert_from_base_ten:nn {#1} { 8 }
}
%    \end{macrocode}
%\end{macro}
%\end{macro}
%\end{macro}
%\end{macro}
%\end{macro}
%\end{macro}
%
%\begin{macro}{\int_from_alph:n}
%\begin{macro}[aux]{\int_from_alph_aux:n}
%\begin{macro}[aux]{\int_from_alph_aux:nN}
%\begin{macro}[aux]{\int_from_alph_aux:N}
% The aim here is to iterate through the input, converting one letter at
% a time to a number. The same approach is also used for base
% conversion, but this needs a different final auxiliary.
%    \begin{macrocode}
\cs_new:Npn \int_from_alph:n #1 {
  \int_eval:n
    {
      \int_get_sign:n {#1}
      \exp_args:Nf \int_from_alph_aux:n
        { \int_get_digits:n {#1} } 
    }
}
\cs_new:Npn \int_from_alph_aux:n #1 {
  \int_from_alph_aux:nN { 0 } #1 \q_nil
}
\cs_new:Npn \int_from_alph_aux:nN #1#2 {
  \quark_if_nil:NTF #2
    {#1}
    {
      \exp_args:Nf \int_from_alph_aux:nN
        { \int_eval:n { #1 * 26 + \int_from_alph_aux:N #2 } }
    } 
}
\cs_new:Npn \int_from_alph_aux:N #1 {
  \int_eval:n 
    { `#1 - \int_compare:nNnTF { `#1 } < { 91 } { 64 } { 96 } }
}
%    \end{macrocode}
%\end{macro}
%\end{macro}
%\end{macro}
%\end{macro}
%
%
% \begin{macro}{\int_compare_p:n}
% \begin{macro}[TF]{\int_compare:n}
% Comparison tests using a simple syntax where only one set of braces
% is required and additional operators such as "!=" and ">=" are
% supported. First some notes on the idea behind this. We wish to
% support writing code like
% \begin{verbatim}
% \int_compare_p:n { 5 + \l_tmpa_int != 4 - \l_tmpb_int }
% \end{verbatim}
% In other words, we want to somehow add the missing "\int_eval:w"
% where required.  We can start evaluating from the left using
% "\int_eval:w", and we know that since the relation symbols "<", ">",
% "=" and "!" are not allowed in such expressions, they will terminate
% the expression. Therefore, we first let \TeX\ evaluate this left
% hand side of the (in)equality.
%    \begin{macrocode}
\prg_set_conditional:Npnn \int_compare:n #1{p,TF,T,F}{
  \exp_after:wN \int_compare_auxi:w \int_value:w
    \int_eval:w #1\q_stop
}
%    \end{macrocode}
% Then the next step is to figure out which relation we should use, so
% we have to somehow get rid of the first evaluation so that we can
% see what stopped it. "\tex_romannumeral:D" is handy here since its
% expansion given a non-positive number is \m{null}. We therefore
% simply check if the first token of the left hand side evaluation is
% a minus. If not, we insert it and issue "\tex_romannumeral:D",
% thereby ridding us of the left hand side evaluation. We do however
% save it for later.
%    \begin{macrocode}
\cs_set:Npn \int_compare_auxi:w #1#2\q_stop{
   \exp_after:wN   \int_compare_auxii:w \tex_romannumeral:D
   \if:w #1- \else: -\fi: #1#2 \q_mark #1#2 \q_stop
}
%    \end{macrocode}
% This leaves the first relation symbol in front and assuming the
% right hand side has been input, at least one other token as well. We
% support the following forms: |=|, |<|, |>| and the extended |!=|,
% |==|, |<=| and |>=|. All the extended forms have an extra |=| so we
% check if that is present as well. Then use specific function to
% perform the test.
%    \begin{macrocode}
\cs_set:Npn \int_compare_auxii:w #1#2#3\q_mark{ 
   \use:c{
     int_compare_ 
     #1  \if_meaning:w =#2 =  \fi:
     :w}  
}
%    \end{macrocode}
% The actual comparisons are then simple function calls, using the
% relation as delimiter for a delimited argument.
% Equality is easy:
%    \begin{macrocode}
\cs_set:cpn {int_compare_=:w} #1=#2\q_stop{
  \if_int_compare:w #1=\int_eval:w #2 \int_eval_end: 
    \prg_return_true: \else: \prg_return_false: \fi:
}
%    \end{macrocode}
% So is the one using |==| -- we just have to use |==| in the
% parameter text.
%    \begin{macrocode}
\cs_set:cpn {int_compare_==:w} #1==#2\q_stop{
  \if_int_compare:w #1=\int_eval:w #2 \int_eval_end: 
  \prg_return_true: \else: \prg_return_false: \fi:
}
%    \end{macrocode}
% Not equal is just about reversing the truth value.
%    \begin{macrocode}
\cs_set:cpn {int_compare_!=:w} #1!=#2\q_stop{
  \if_int_compare:w #1=\int_eval:w #2 \int_eval_end: 
  \prg_return_false: \else: \prg_return_true: \fi:
}
%    \end{macrocode}
% Less than and greater than are also straight forward.
%    \begin{macrocode}
\cs_set:cpn {int_compare_<:w} #1<#2\q_stop{
  \if_int_compare:w #1<\int_eval:w #2 \int_eval_end: 
  \prg_return_true: \else: \prg_return_false: \fi:
}
\cs_set:cpn {int_compare_>:w} #1>#2\q_stop{
  \if_int_compare:w #1>\int_eval:w #2 \int_eval_end:
  \prg_return_true: \else: \prg_return_false: \fi:
}
%    \end{macrocode}
% The less than or equal operation is just the opposite of the greater
% than operation. Vice versa for less than or equal.
%    \begin{macrocode}
\cs_set:cpn {int_compare_<=:w} #1<=#2\q_stop{
  \if_int_compare:w #1>\int_eval:w #2 \int_eval_end: 
  \prg_return_false: \else: \prg_return_true: \fi:
}
\cs_set:cpn {int_compare_>=:w} #1>=#2\q_stop{
  \if_int_compare:w #1<\int_eval:w #2 \int_eval_end:
  \prg_return_false: \else: \prg_return_true: \fi:
}
%    \end{macrocode}
% \end{macro}
% \end{macro}
%
% \begin{macro}{\int_compare_p:nNn}
% \begin{macro}[TF]{\int_compare:nNn}
% More efficient but less natural in typing.
%    \begin{macrocode}
\prg_set_conditional:Npnn \int_compare:nNn #1#2#3{p}{
  \if_int_compare:w \int_eval:w #1 #2 \int_eval:w #3
  \int_eval_end:
  \prg_return_true: \else: \prg_return_false: \fi:
}
\cs_set_nopar:Npn \int_compare:nNnT #1#2#3 {
  \tex_ifnum:D \etex_numexpr:D #1 #2 \etex_numexpr:D #3 \scan_stop:
    \tex_expandafter:D \use:n
  \tex_else:D
    \tex_expandafter:D \use_none:n
  \tex_fi:D
}
\cs_set_nopar:Npn \int_compare:nNnF #1#2#3 {
  \tex_ifnum:D \etex_numexpr:D #1 #2 \etex_numexpr:D #3 \scan_stop:
    \tex_expandafter:D \use_none:n
  \tex_else:D
    \tex_expandafter:D \use:n
  \tex_fi:D
}
\cs_set_nopar:Npn \int_compare:nNnTF #1#2#3 {
  \tex_ifnum:D \etex_numexpr:D #1 #2 \etex_numexpr:D #3 \scan_stop:
    \tex_expandafter:D \use_i:nn
  \tex_else:D
    \tex_expandafter:D \use_ii:nn
  \tex_fi:D
}
%    \end{macrocode}
% \end{macro}
% \end{macro}
%
% \begin{macro}{\int_max:nn}
% \begin{macro}{\int_min:nn}
% \begin{macro}{\int_abs:n}
% Functions for $\min$, $\max$, and absolute value.
%    \begin{macrocode}
\cs_set:Npn \int_abs:n #1{
  \int_value:w 
  \if_int_compare:w \int_eval:w #1<\c_zero 
    -
  \fi:
  \int_eval:w #1\int_eval_end:
}
\cs_set:Npn \int_max:nn #1#2{
  \int_value:w \int_eval:w 
    \if_int_compare:w 
      \int_eval:w #1>\int_eval:w #2\int_eval_end: 
      #1
    \else:
      #2
    \fi:
  \int_eval_end:
}
\cs_set:Npn \int_min:nn #1#2{
  \int_value:w \int_eval:w 
    \if_int_compare:w 
      \int_eval:w #1<\int_eval:w #2\int_eval_end: 
      #1
    \else:
      #2
    \fi:
  \int_eval_end:
}
%    \end{macrocode}
% \end{macro}
% \end{macro}
% \end{macro}
%
%
% \begin{macro}{\int_div_truncate:nn}
% \begin{macro}{\int_div_round:nn}
% \begin{macro}{\int_mod:nn}
% As "\int_eval:w" rounds the result of a division we also
% provide a version that truncates the result.
%    \begin{macrocode}
%    \end{macrocode}
%    Initial version didn't work correctly with e\TeX's implementation.    
%    \begin{macrocode}
%\cs_set:Npn \int_div_truncate_raw:nn #1#2 {
%  \int_eval:n{ (2*#1 - #2) / (2* #2) }
%}
%    \end{macrocode}
%    New version by Heiko:
%    \begin{macrocode}
\cs_set:Npn \int_div_truncate:nn #1#2 {
  \int_value:w \int_eval:w
    \if_int_compare:w \int_eval:w #1 = \c_zero
      0
    \else:
      (#1
      \if_int_compare:w \int_eval:w #1 < \c_zero
        \if_int_compare:w \int_eval:w #2 < \c_zero
          -( #2 +
        \else:   
          +( #2 -
        \fi:
      \else:
        \if_int_compare:w \int_eval:w #2 < \c_zero
          +( #2 + 
        \else:   
          -( #2 -
        \fi:
      \fi:  
      1)/2)
    \fi:
    /(#2)
  \int_eval_end:  
}
%    \end{macrocode}
% For the sake of completeness:
%    \begin{macrocode}
\cs_set:Npn \int_div_round:nn #1#2 {\int_eval:n{(#1)/(#2)}}
%    \end{macrocode}
% Finally there's the modulus operation.
%    \begin{macrocode}
\cs_set:Npn \int_mod:nn #1#2 {
  \int_value:w 
    \int_eval:w  
    #1 - \int_div_truncate:nn {#1}{#2} * (#2) 
    \int_eval_end:
}
%    \end{macrocode}
% \end{macro}
% \end{macro}
% \end{macro}
%
% \begin{macro}{\int_if_odd_p:n}
% \begin{macro}[TF]{\int_if_odd:n}
% \begin{macro}{\int_if_even_p:n}
% \begin{macro}[TF]{\int_if_even:n}
% A predicate function.
%    \begin{macrocode}
\prg_set_conditional:Npnn \int_if_odd:n #1 {p,TF,T,F} {
  \if_int_odd:w \int_eval:w #1\int_eval_end:
    \prg_return_true: \else: \prg_return_false: \fi:
}
\prg_set_conditional:Npnn \int_if_even:n #1 {p,TF,T,F} {
  \if_int_odd:w \int_eval:w #1\int_eval_end:
    \prg_return_false: \else: \prg_return_true: \fi:
}
%    \end{macrocode}
% \end{macro}
% \end{macro}
% \end{macro}
% \end{macro}
%
% \begin{macro}{\int_while_do:nn}
% \begin{macro}{\int_until_do:nn}
% \begin{macro}{\int_do_while:nn}
% \begin{macro}{\int_do_until:nn}
%  These are quite easy given the above functions. The "while" versions
%  test first and then execute the body. The "do_while" does it the
%  other way round. 
%    \begin{macrocode}
\cs_set:Npn \int_while_do:nn #1#2{
  \int_compare:nT {#1}{#2 \int_while_do:nn {#1}{#2}}
}
\cs_set:Npn \int_until_do:nn #1#2{
  \int_compare:nF {#1}{#2 \int_until_do:nn {#1}{#2}}
}
\cs_set:Npn \int_do_while:nn #1#2{
  #2 \int_compare:nT {#1}{\int_do_while:nNnn {#1}{#2}}
}
\cs_set:Npn \int_do_until:nn #1#2{
  #2 \int_compare:nF {#1}{\int_do_until:nn {#1}{#2}}
}
%    \end{macrocode}
% \end{macro}
% \end{macro}
% \end{macro}
% \end{macro}
%
% \begin{macro}{\int_while_do:nNnn}
% \begin{macro}{\int_until_do:nNnn}
% \begin{macro}{\int_do_while:nNnn}
% \begin{macro}{\int_do_until:nNnn}
%  As above but not using the more natural syntax. 
%    \begin{macrocode}
\cs_set:Npn \int_while_do:nNnn #1#2#3#4{
  \int_compare:nNnT {#1}#2{#3}{#4 \int_while_do:nNnn {#1}#2{#3}{#4}}
}
\cs_set:Npn \int_until_do:nNnn #1#2#3#4{
  \int_compare:nNnF {#1}#2{#3}{#4 \int_until_do:nNnn {#1}#2{#3}{#4}}
}
\cs_set:Npn \int_do_while:nNnn #1#2#3#4{
  #4 \int_compare:nNnT {#1}#2{#3}{\int_do_while:nNnn {#1}#2{#3}{#4}}
}
\cs_set:Npn \int_do_until:nNnn #1#2#3#4{
  #4 \int_compare:nNnF {#1}#2{#3}{\int_do_until:nNnn {#1}#2{#3}{#4}}
}
%    \end{macrocode}
% \end{macro}
% \end{macro}
% \end{macro}
% \end{macro}
%

%  \subsection{Defining constants}
%  \begin{macro}{\int_const:Nn}
%  As stated, most constants can be defined as |\tex_chardef:D| or
%  |\tex_mathchardef:D| but that's engine dependent.
%    \begin{macrocode}
\cs_new_protected_nopar:Npn \int_const:Nn #1#2 {
  \int_compare:nTF { #2 > \c_minus_one }
    {
      \int_compare:nTF { #2 > \c_max_register_int }
        {
          \int_new:N #1 
          \int_gset:Nn #1 {#2}
        }
        {
          \chk_if_free_cs:N #1 
            \tex_global:D \tex_mathchardef:D #1 = \int_eval:n {#2}
        }
    } 
    { 
      \int_new:N #1 
      \int_gset:Nn #1 {#2} 
    }
}
\cs_generate_variant:Nn \int_const:Nn { c }
%    \end{macrocode}
%  \end{macro}
%
%\begin{macro}{\c_max_register_int}
% This is here as this particular integer is needed both in package
% mode and to bootstrap \pkg{l3alloc}
%    \begin{macrocode}
\tex_mathchardef:D \c_max_register_int = 32767 \scan_stop:
%    \end{macrocode}
%  \end{macro}
%
% \begin{macro}{\c_minus_one, 
%               \c_zero, \c_one, \c_two, \c_three, \c_four, \c_five, \c_six, 
%               \c_seven, \c_eight, \c_nine, \c_ten, 
%               \c_eleven, \c_twelve, \c_thirteen, \c_fourteen, \c_fifteen, 
%               \c_sixteen, \c_thirty_two, 
%               \c_hundred_one, 
%               \c_twohundred_fifty_five, \c_twohundred_fifty_six, 
%               \c_thousand, 
%               \c_ten_thousand, 
%               \c_ten_thousand_one, \c_ten_thousand_two, 
%               \c_ten_thousand_three, \c_ten_thousand_four, 
%               \c_twenty_thousand}
%    And the usual constants, others are still missing. Please, make
%    every constant a real constant at least for the moment. We can
%    easily convert things in the end when we have found what
%    constants are used in critical places and what not.
%    \begin{macrocode}
 %% \tex_countdef:D \c_minus_one = 10 \scan_stop:
 %% \c_minus_one = -1 \scan_stop:        %% in l3basics
%\int_const:Nn \c_zero   {0}
\int_const:Nn \c_one    {1}
\int_const:Nn \c_two    {2}
\int_const:Nn \c_three  {3}
\int_const:Nn \c_four   {4}
\int_const:Nn \c_five   {5}
\int_const:Nn \c_six    {6}
\int_const:Nn \c_seven  {7}
\int_const:Nn \c_eight  {8}
\int_const:Nn \c_nine   {9}
\int_const:Nn \c_ten      {10}
\int_const:Nn \c_eleven   {11}
\int_const:Nn \c_twelve   {12}
\int_const:Nn \c_thirteen {13}
\int_const:Nn \c_fourteen {14}
\int_const:Nn \c_fifteen  {15}
 %% \tex_chardef:D \c_sixteen    = 16\scan_stop: %% in l3basics
\int_const:Nn \c_thirty_two {32}
%    \end{macrocode}
% The next one may seem a little odd (obviously!) but is useful when
% dealing with logical operators.
%    \begin{macrocode}
\int_const:Nn \c_hundred_one          {101}
\int_const:Nn \c_twohundred_fifty_five{255}
\int_const:Nn \c_twohundred_fifty_six {256}
\int_const:Nn \c_thousand             {1000}
\int_const:Nn \c_ten_thousand         {10000}
\int_const:Nn \c_ten_thousand_one     {10001}
\int_const:Nn \c_ten_thousand_two     {10002}
\int_const:Nn \c_ten_thousand_three   {10003}
\int_const:Nn \c_ten_thousand_four    {10004}
\int_const:Nn \c_twenty_thousand      {20000}
%    \end{macrocode}
% \end{macro}
%
% \begin{macro}{\c_max_int}
%    The largest number allowed is $2^{31}-1$
%    \begin{macrocode}
\int_const:Nn \c_max_int {2147483647}
%    \end{macrocode}
% \end{macro}
% 
%\begin{macro}[aux]{\c_int_from_roman_i_int}
%\begin{macro}[aux]{\c_int_from_roman_v_int}
%\begin{macro}[aux]{\c_int_from_roman_x_int}
%\begin{macro}[aux]{\l_int_from_roman_l_int}
%\begin{macro}[aux]{\c_int_from_roman_c_int}
%\begin{macro}[aux]{\c_int_from_roman_d_int}
%\begin{macro}[aux]{\c_int_from_roman_m_int}
%\begin{macro}[aux]{\c_int_from_roman_I_int}
%\begin{macro}[aux]{\c_int_from_roman_V_int}
%\begin{macro}[aux]{\c_int_from_roman_X_int}
%\begin{macro}[aux]{\c_int_from_roman_L_int}
%\begin{macro}[aux]{\c_int_from_roman_C_int}
%\begin{macro}[aux]{\c_int_from_roman_D_int}
%\begin{macro}[aux]{\c_int_from_roman_M_int}
% Delayed from earlier.
%    \begin{macrocode}
\int_const:cn { c_int_from_roman_i_int } { 1 }
\int_const:cn { c_int_from_roman_v_int } { 5 }
\int_const:cn { c_int_from_roman_x_int } { 10 }
\int_const:cn { c_int_from_roman_l_int } { 50 }
\int_const:cn { c_int_from_roman_c_int } { 100 }
\int_const:cn { c_int_from_roman_d_int } { 500 }
\int_const:cn { c_int_from_roman_m_int } { 1000 }
\int_const:cn { c_int_from_roman_I_int } { 1 }
\int_const:cn { c_int_from_roman_V_int } { 5 }
\int_const:cn { c_int_from_roman_X_int } { 10 }
\int_const:cn { c_int_from_roman_L_int } { 50 }
\int_const:cn { c_int_from_roman_C_int } { 100 }
\int_const:cn { c_int_from_roman_D_int } { 500 }
\int_const:cn { c_int_from_roman_M_int } { 1000 }
%    \end{macrocode}
%\end{macro}
%\end{macro}
%\end{macro}
%\end{macro}
%\end{macro}
%\end{macro}
%\end{macro}
%\end{macro}
%\end{macro}
%\end{macro}
%\end{macro}
%\end{macro}
%\end{macro}
%\end{macro}
%
% Needed from the tl module:
%    \begin{macrocode}
\int_new:N \g_tl_inline_level_int
%    \end{macrocode}
%
% \subsection{Backwards compatibility}
%    \begin{macrocode}
\cs_set_eq:NN \intexpr_value:w \int_value:w
\cs_set_eq:NN \intexpr_eval:w \int_eval:w
\cs_set_eq:NN \intexpr_eval_end: \int_eval_end:
\cs_set_eq:NN \if_intexpr_compare:w \if_int_compare:w
\cs_set_eq:NN \if_intexpr_odd:w \if_int_odd:w
\cs_set_eq:NN \if_intexpr_case:w \if_case:w
\cs_set_eq:NN \intexpr_eval:n \int_eval:n

\cs_set_eq:NN \intexpr_compare_p:n \int_compare_p:n
\cs_set_eq:NN \intexpr_compare:nTF \int_compare:nTF
\cs_set_eq:NN \intexpr_compare:nT  \int_compare:nT
\cs_set_eq:NN \intexpr_compare:nF  \int_compare:nF

\cs_set_eq:NN \intexpr_compare_p:nNn \int_compare_p:nNn
\cs_set_eq:NN \intexpr_compare:nNnTF \int_compare:nNnTF
\cs_set_eq:NN \intexpr_compare:nNnT  \int_compare:nNnT
\cs_set_eq:NN \intexpr_compare:nNnF  \int_compare:nNnF

\cs_set_eq:NN \intexpr_abs:n  \int_abs:n
\cs_set_eq:NN \intexpr_max:nn \int_max:nn
\cs_set_eq:NN \intexpr_min:nn \int_min:nn

\cs_set_eq:NN \intexpr_div_truncate:nn \int_div_truncate:nn
\cs_set_eq:NN \intexpr_div_round:nn    \int_div_round:nn
\cs_set_eq:NN \intexpr_mod:nn          \int_mod:nn

\cs_set_eq:NN \intexpr_if_odd_p:n \int_if_odd_p:n
\cs_set_eq:NN \intexpr_if_odd:nTF \int_if_odd:nTF
\cs_set_eq:NN \intexpr_if_odd:nT  \int_if_odd:nT
\cs_set_eq:NN \intexpr_if_odd:nF  \int_if_odd:nF

\cs_set_eq:NN \intexpr_while_do:nn \int_while_do:nn
\cs_set_eq:NN \intexpr_until_do:nn \int_until_do:nn
\cs_set_eq:NN \intexpr_do_while:nn \int_do_while:nn
\cs_set_eq:NN \intexpr_do_until:nn \int_do_until:nn

\cs_set_eq:NN \intexpr_while_do:nNnn \int_while_do:nNnn
\cs_set_eq:NN \intexpr_until_do:nNnn \int_until_do:nNnn
\cs_set_eq:NN \intexpr_do_while:nNnn \int_do_while:nNnn
\cs_set_eq:NN \intexpr_do_until:nNnn \int_do_until:nNnn
%    \end{macrocode}
%
%    \begin{macrocode}
%</initex|package>
%    \end{macrocode}
%
%    Show token usage:
%    \begin{macrocode}
%<*showmemory>
\showMemUsage
%</showmemory>
%    \end{macrocode}
%
% \end{implementation}
% \PrintIndex
%
%
% \endinput
